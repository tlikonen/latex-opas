\chapter{Dokumentin asetukset}
\section{Dokumenttiluokat}
\label{luku:dokumenttiluokat}
\section{Fontit}
\label{luku:kirjaintyypit}

Fontit ja niiden asettaminen on Latexissa melko monimutkainen
kokonaisuus, koska fonteilla on paljon ominaisuuksia ja niihin
vaikutetaan monilla eri asetuksilla ja abstraktiotasoilla. Aika monta
asiaa pitää ymmärtää, jotta voi tehokkaasti työskennellä fonttien
kanssa.

Fontti jo itsessään on moniselitteinen käsite, joka vaatii typografiassa
usein täsmentäviä ilmauksia. Sana \emph{fontti} voi tarkoittaa
kokonaista kirjainperhettä, eli muutaman yhteensopivan
kirjainleikkauksen muodostamaa kokonaisuutta. Samaan kirjainperheeseen
kuuluu yleensä neljä eri leikkausta: tavallinen, kursiivi, lihavoitu ja
lihavoitu kursiivi. Joihinkin perheisiin kuuluu leikkauksia paljon
enemmänkin, kuten useita eri vahvuuksia. Joissakin puheissa sana
\emph{fontti} tarkoittaa vain yhtä kirjainleikkausta, ja silloin koko
perheeseen viitataan sanalla fonttiperhe.

Tässä oppaassa käytän \emph{fontti}\-/sanaa yleisnimityksenä Latexin
kirjaimiin liittyville asetuksille. Se tarkoittaa kirjainperhettä tai
siihen kuuluvaa yksittäistä leikkausta. Silloin kun merkitystä pitää
täsmentää, käytän suomenkielisiä nimiä kirjainperhe ja kirjainleikkaus.
Sen sijaan sanan \emph{kirjasin} jätän kokonaan pois, koska se
tarkoittaa vanhassa metalliladonnassa ja mekaanisissa kirjoituskoneissa
metallisen kirjakkeen päähän muotoiltua kirjaimen kohokuviota, joka
painaa mustejäljen paperille.

Kuten Latexissa yleensäkin myös fonttien kanssa kannattaa käyttää
korkean abstraktiotason komentoja, jotka piilottavat yksityiskohdat ja
teknisen toteutuksen. Latexin fonttitoiminnot on suunniteltu juuri
siihen: matalan tason fontti\-asetukset määritellään mieluiten vain
kerran dokumentin esittely\-osassa, ja sen jälkeen käytetään pelkästään
korkean tason komentoja.

\subsection{Fontin valinta}

Latexin fonttien perus\-toiminnot rakentuvat kolmen erityyppisen
kirjainperheen varaan: antiikva eli pääteviivallinen
(\textenglish{serif, roman}), groteski eli pääteviivaton
(\textenglish{sans serif}) ja tasalevyinen kirjoituskoneen kaltainen
perhe (\textenglish{typewriter, monospace}). Kuvassa
\ref{kuva:kirjainperhetyypit} on tässä oppaassa käytetyt kolme eri
kirjainperhetyyppiä. Leipätekstissä käytetään antiikvaa, otsikoissa
groteskia ja koodi\-esi\-mer\-keis\-sä tasalevyistä.

\leijukuva{
  {\rmfamily\addfontfeatures{Scale=5}Amf}
  \hfill
  {\sffamily\addfontfeatures{Scale=5}Amf}
  \hfill
  {\ttfamily\addfontfeatures{Scale=5}Amf}
}{
  \caption{Vasemmalla antiikva (Libertinus Serif), keskellä groteski
    (Libertinus Sans) ja oikealla tasalevyinen antiikvakirjainperhe
    (Libertinus Mono)}
  \label{kuva:kirjainperhetyypit}
}

Kirjoituskoneen kaltainen tasalevyinen kirjainperhe on tässä tapauksessa
tyypiltään antiikva eli pääteviivallinen, mutta se voisi olla muutakin.
Tasalevyisyys on sen kirjainperheen tärkein määrittävä tekijä Latexin
asetusten näkökulmasta.

Kirjainperheet otetaan käyttöön Fontspec\-/paketin komennoilla seuraavan
esimerkin mukaisesti.

\begin{koodilohkosis}
  \usepackage{fontspec}
  \setmainfont{TeX Gyre Termes}[Scale=1]
  \setsansfont{TeX Gyre Heros} [Scale=MatchLowercase]
  \setmonofont{TeX Gyre Cursor}[Scale=MatchLowercase]
\end{koodilohkosis}

Samalla voi määritellä lukuisia kirjainperheeseen sisältyviä asetuksia
kuten ligatuureja ja optisia kokoja. Tässä esimerkissä käytetään vain
\koodi{Scale}\-/valitsinta, jolla fontin voi skaalata haluttuun kokoon.
\koodi{Scale}\-/kerroin on desimaaliluku, ja oletus\-arvo on 1.

Esimerkissä peruskirjainperheen (\koodi{\keno setmainfont}) skaalaus on
1, eli sille ei tehdä mitään, ja koko valitsimen voisi jättää pois. Sen
sijaan kahdella muulla kirjainperheellä (\koodi{\keno setsansfont},
\koodi{\keno setmonofont}) käytetään ker\-roin\-ase\-tus\-ta
\koodi{MatchLowercase}, joka skaalaa fontin siten, että gemenakirjaimet
eli pienet kirjaimet ovat yhtä korkeita kuin peruskirjainperheessä.
Mikäli skaa\-laus\-ase\-tus \koodi{MatchLowercase} ei tuota ihan
toivottua tulosta, voi kirjainperheen skaalausta hienosäätää vielä
\koodi{ScaleAgain}\-/valitsimella seuraavaan tapaan:

\begin{koodilohkosis}
  \setmonofont{TeX Gyre Cursor}
  [Scale=MatchLowercase, ScaleAgain=.97]
\end{koodilohkosis}

Jos edellä kuvatut kolme kirjainperhettä eivät riitä, on
Fontspec\-/paketissa komennot lisäperheiden ja \=/leikkausten
määrittämiseen. Uusi perhe määritellään seuraavasti:

\begin{koodilohkosis}
  \newfontfamily{\hienoperhe}{TeX Gyre Schola}[...]
\end{koodilohkosis}

Komento \koodi{\keno newfontfamily} toimii samalla tavalla kuin aiemmin
esitellyt \koodi{\keno setmainfont} ym. komennot, mutta lisäksi
ensimmäinen parametri määrittää komennon, jolla kirjainperhe otetaan
käyttöön. Edellisessä esimerkissä luodaan komento \koodi{\keno
  hieno\-perhe}, joka kytkee päälle TeX Gyre Scho\-la \=/nimisen
kirjainperheen.
          
Jos ei tarvita kokonaista perhettä vaan yksi leikkaus riittää, käytetään
komentoa \koodi{\keno newfontface}:

\begin{koodilohkosis}
  \newfontface{\hienoleikkaus}{TeX Gyre Schola Bold}[...]
\end{koodilohkosis}

Edellä määriteltävä komento \koodi{\keno hieno\-leikkaus} ottaa käyttöön
lihavoidun (bold) kirjainleikkauksen perheestä TeX Gyre Scho\-la. Tässä
vaih\-to\-eh\-dos\-sa on se etu, että tietokoneen muistiin ladataan vain
yksi leikkaus, ei koko perhettä.

\subsection{Fontin koko ja rivikorkeus}

Fonttien koot on totuttu valitsemaan typografisen pistemitan avulla.
Esimerkiksi 10--12 pistettä on tyypillinen leipätekstin oletuskoko
teks\-tin\-kä\-sit\-tely\-ohjel\-mis\-sa. Piste on mitta\-yksikkö, jonka
pituus on 1/72 tuumaa eli 0,3528 millimetriä. Kirjainleikkauksen
pistekoko mitataan kirjaimiston ylimmän ja alimman kohdan välillä, ja
siihen luetaan mukaan ylä- ja alapuolella oleva pieni tyhjä tila, jonka
fontin suunnittelija on määritellyt.

Myös Latexissa koot voi määritellä pistemittojen (lyhenne pt) avulla,
mutta halutessaan ne voi unohtaa lähes kokonaan ja käyttää niin sanottua
suhteellista tapaa koko\-asetuksiin. Käsittelen suhteellisia ja
absoluuttisia koko\-ase\-tuk\-sia luvuissa
\ref{luku:fontti_suhteellinen} ja \ref{luku:fontti_absoluuttinen}.

Matalalla tasolla fonttien kokoon vaikuttaa Latexissa eräs yllättävä
asia. Nimittäin dokumenttiluokalle (luku \ref{luku:dokumenttiluokat})
voi antaa valitsimen, jolla koko asetetaan. Vaihtoehtoja on Latexin
normaaleissa dokumenttiluokissa vain kolme: \koodi{10pt} (oletus),
\koodi{11pt} ja \koodi{12pt}. Dokumenttiluokan koko\-asetus vaikuttaa
myös sivun marginaaleihin, koska Latex pyrkii pitämään rivin
merkkimäärän lukijalle sopivana: yhdelle riville ei kannata latoa ihan
mahdottomasti merkkejä, koska luettavuus heikkenee.

Fontin koon määrittäminen dokumenttiluokan valitsimella kuuluu jo vähän
menneisyyteen, mutta voi sitä edelleen käyttää, jos se riittää ja sillä
saa halutun lopputuloksen. Yleensä kyllä jättäisin dokumenttiluokan
fontti\-asetuksen oletukseksi (\koodi{10pt}) ja käyttäisin koon
asettamiseen luvussa \ref{luku:fontti_suhteellinen} tai
\ref{luku:fontti_absoluuttinen} kerrottuja tapoja. Sivun marginaalien ja
muiden mittojen määrittämiseen on ohjeita luvussa
\ref{luku:sivuasetukset}.

Fontti\-asetuksiin kuuluu fontin koon lisäksi toinenkin mitta:
rivikorkeus (\koodi{\keno baselineskip}). Se on mitta rivin
peruslinjalta seuraavan rivin peruslinjalle. Fontin koko ja rivikorkeus
määritellään saman\-aikaisesti, koska ne ovat saman \koodi{\keno
  fontsize}\-/komennon parametreja. Esimerkki:

\begin{koodilohkosis}
  \fontsize{10pt}{12pt} \selectfont
\end{koodilohkosis}

Ensimmäinen parametri on fontin kokomitta ja toinen on rivikorkeus.
Mitta\-yksiköt voivat olla mitä tahansa Latexin mittoja, ja oletuksena
käytetään pistemittaa (pt), jos yksikköä ei ole mainittu. Komento
\koodi{\keno selectfont} on mukana, koska vasta sen myötä matalan tason
fonttikomennot tulevat voimaan. Korkean tason fonttikomennot (luku
\ref{luku:fontit_korkea}) suorittavat sen automaattisesti.

Rivikorkeus on vähintään sama kuin fontin koko, mutta yleensä se
asetetaan hieman suuremmaksi, jotta rivit eivät olisi liian lähellä
toisiaan. Esimerkissä \ref{esim:rivikorkeus} on kaksi erilaista
\koodi{\keno fontsize}\-/komentoa ja ladottu lopputulos.

\begin{esimerkki}
\begin{koodilohko}
  \fontsize{10pt}{12pt}\selectfont Tässä on pienehkö leipätekstin
  fonttikoko ja sitä hieman suurempi rivikorkeus. Rivikorkeus on
  liian pieni näin pitkille riveille.

  \fontsize{16pt}{25pt}\selectfont Tässä on melko suuri fontti ja
  reilu rivikorkeus. Rivit eivät tunnu kuuluvan enää yhteen.
\end{koodilohko}
\centering%
\parbox{.9\textwidth}{%
  \linespread{1}\addfontfeatures{Scale=1}
  \fontsize{10pt}{12pt}\selectfont Tässä on pienehkö leipätekstin
  fonttikoko ja sitä hieman suurempi rivikorkeus. Rivikorkeus on liian
  pieni näin pitkille riveille.

  \fontsize{16pt}{25pt}\selectfont Tässä on melko suuri fontti ja reilu
  rivikorkeus. Rivit eivät tunnu kuuluvan enää yhteen. }

\vspace{1ex}
\hrulefill
\vspace{2ex}

\caption{Fontin koon ja rivikorkeuden asettaminen ja vaikutus}
\label{esim:rivikorkeus}
\end{esimerkki}

Toinen tekstirivien peruslinjojen väliseen etäisyyteen vaikuttava asetus
on \koodi{\keno baselinestretch}. Se on desimaalilukukerroin, jolla
nykyinen rivikorkeus kerrotaan. Kerroin asetetaan helpoimmin komennolla
\koodi{\keno linespread}.\footnote{Toinen tapa: \koodi{\keno
    renewcommand\{\keno baselinestretch\}\{kerroin\}}}

\begin{koodilohkosis}
  \fontsize{10pt}{12pt} \linespread{1.3} \selectfont
\end{koodilohkosis}

Edellä oleva esimerkki asettaa fontin kooksi 10 pistettä ja
rivikorkeudeksi 12 pistettä. \koodi{\keno linespread}\-/komennolla
asetetun kertoimen vuoksi rivien peruslinjojen väliseksi etäisyydeksi
tulee lopulta 1,3 kertaa 12 pistettä eli 15,6 pistettä. Ei ole väliä,
kummassa järjestyksessä \koodi{\keno fontsize}- ja \koodi{\keno
  linespread}\-/komennot annetaan. Asetukset tulevat voimaan vasta
\koodi{\keno selectfont}\-/komennon jälkeen.

Käytännössä \koodi{\keno linespread} sopii rivikorkeuden yleistason
hienosäätöön, esimerkiksi dokumentin esittely\-osassa. Sitä ei
kannattane kovin paljon muutella, koska se vaikuttaa kaikkialla. Sen
sijaan \koodi{\keno fontsize}\-/komennolla määritetään rivikorkeus
tietylle fonttikoolle ja tiettyyn tilanteeseen.

\subsection{Korkean tason komennot}
\label{luku:fontit_korkea}

Latexissa on joukko korkean tason fontti\-komentoja, jotka on
tarkoitettu käytettäväksi sen jälkeen, kun matalan tason asetukset on
kerran määritetty. Taulukossa \ref{tlk:fonttimallikomennot} on komennot
kirjainperheen ja kirjainleikkauksen valintaan.

Komennot fontin koon valintaan ovat taulukossa
\ref{tlk:fonttikokokomennot}. Taulukko kertoo myös, mitä fontin
pistekokoa mikäkin komento tarkoittaa oletuksena. Oletus riippuu Latexin
dokumenttiluokkien (luku \ref{luku:dokumenttiluokat})
fonttikokovalitsimista \koodi{10pt}, \koodi{11pt} ja \koodi{12pt}.

Kaikille korkean tason fonttikomennoille on olemassa myös samanniminen
ympäristönsä. Alla olevassa esimerkissä on kaksi fontteihin vaikuttavaa
ympäristöä sisäkkäin.

\begin{koodilohkosis}
  \begin{LARGE}
    \begin{scshape}
      Tämä teksti on isoa (LARGE) kapiteelia (scshape).
      Typografisesti varmaan aika typerää...
    \end{scshape}
  \end{LARGE}
\end{koodilohkosis}

\leijutlk{
  \begin{tabular}{llll}
    \toprule
    \multicolumn{2}{l}{\ots{Komento}}
    & \multicolumn{2}{l}{\ots{Merkitys}} \\
    \midrule
    \koodi{\keno rmfamily} & \koodi{\keno textrm\{\}}
    & {\rmfamily perhe} & antiikva, serif, roman \\
    \koodi{\keno sffamily} & \koodi{\keno textsf\{\}}
    & {\sffamily perhe} & groteski, sans serif \\
    \koodi{\keno ttfamily} & \koodi{\keno texttt\{\}}
    & {\ttfamily perhe} & tasalevyinen, typewriter \\
    \midrule
    \koodi{\keno mdseries} & \koodi{\keno textmd\{\}}
    & {\mdseries leikkaus} & lihavoimaton, tavallinen \\
    \koodi{\keno bfseries} & \koodi{\keno textbf\{\}}
    & {\bfseries leikkaus} & lihavoitu, bold \\
    \midrule
    \koodi{\keno upshape} & \koodi{\keno textup\{\}}
    & {\upshape leikkaus} & pystyasento, tavallinen \\
    \koodi{\keno slshape} & \koodi{\keno textsl\{\}}
    & {\slshape leikkaus} & kallistettu, slanted, oblique \\
    \koodi{\keno itshape} & \koodi{\keno textit\{\}}
    & {\itshape leikkaus} & kursiivi, italic \\
    \koodi{\keno scshape} & \koodi{\keno textsc\{\}}
    & {\scshape leikkaus} & kapiteeli, small caps \\
    \bottomrule
  \end{tabular}
}{
  \caption{Komennot kirjainperheen ja kirjainleikkauksen valintaan}
  \label{tlk:fonttimallikomennot}
}

\leijutlk{
  \begin{tabular}{lr@{}lr@{}lr@{}l}
    \toprule
    \ots{Komento}
    & \multicolumn{2}{c}{\ots{10pt}}
    & \multicolumn{2}{c}{\ots{11pt}}
    & \multicolumn{2}{c}{\ots{12pt}} \\
    \midrule
    \koodi{\keno tiny} & 5 && 6 && 6 \\
    \koodi{\keno scriptsize} & 7 && 8 && 8 \\
    \koodi{\keno footnotesize} & 8 && 9 && 10 \\
    \koodi{\keno small} & 9 && 10 && 10&,95 \\
    \koodi{\keno normalsize} & 10 && 10&,95 & 12 \\
    \koodi{\keno large} & 12 && 12 && 14&,4 \\
    \koodi{\keno Large} & 14&,4 & 14&,4 & 17&,28 \\
    \koodi{\keno LARGE} & 17&,28 & 17&,28 & 20&,74 \\
    \koodi{\keno huge} & 20&,74 & 20&,74 & 24&,88 \\
    \koodi{\keno Huge} & 24&,88 & 24&,88 & 24&,88 \\
    \bottomrule
  \end{tabular}
}{
  \caption{Fonttien oletuspistekoot dokumenttiluokkien valitsimilla
    \koodi{10pt}, \koodi{11pt} ja \koodi{12pt}}
  \label{tlk:fonttikokokomennot}
}

\subsection{Koot suhteellisesti}
\label{luku:fontti_suhteellinen}

Dokumentin fonttien koot on helpointa määrittää siten, että asettaa
ensin peruskirjainperheen koon ja antaa muiden fonttien määräytyä
suhteessa siihen. Esimerkki \ref{esim:fontti_suhteellinen} selventää,
kuinka se tapahtuu. Alussa otetaan käyttöön dokumenttiluokka
\koodi{article} ja annetaan sille valitsin \koodi{10pt}, joka määrittää
fonttikooksi 10 pistettä. Se on dokumenttiluokan ole\-tus\-ase\-tus,
jota ei tarvitsisi edes kirjoittaa näkyviin. Esimerkin toisella rivillä
otetaan Fontspec\-/paketti käyttöön.

Peruskirjainperheen (rivi~4) koko skaalataan 1,4\-/kertaiseksi, eli
pistekooksi tulee 1,4 kertaa 10 pistettä eli 14 pistettä.
Normaalikokoinen peruskirjainperhe on ainoa, jonka pistekoko tiedetään.
Kaikkien muiden koot täytyisi selvittää laskemalla.

Groteski eli pääteviivaton kirjainperhe (rivi~5) ja tasalevyinen perhe
(rivi~6) skaalataan samankorkuiseksi kuin perusperhe. Vertailukohtana
ovat pienet kirjaimet (\koodi{MatchLowercase}). Näiden kahden
kirjainperheen pistekokoa ei tiedetä. Se ei välttämättä ole sama kuin
perusfontissa, koska fonttien pistekoko mitataan ylimmän ja alimman
kohdan välillä ja koska fonttien mittasuhteet ovat erilaisia.

\begin{esimerkki}
\begin{koodilohko}
  \documentclass[10pt]{article} % 10pt on oletus
  \usepackage{fontspec}

  \setmainfont{TeX Gyre Termes}[Scale=1.4]
  \setsansfont{TeX Gyre Heros} [Scale=MatchLowercase]
  \setmonofont{TeX Gyre Cursor}[Scale=MatchLowercase]
  \linespread{1.45}
\end{koodilohko}    
\caption{Fonttikokojen määrittäminen suhteessa peruskirjainperheeseen}
\label{esim:fontti_suhteellinen}
\end{esimerkki}

Viimeisellä rivillä oleva \koodi{\keno linespread}\-/komento on tärkeä.
Se asettaa rivikorkeuden kertoimeksi 1,45. Kertoimen täytyy olla
vähintään yhtä suuri kuin peruskirjainperheen skaalauskerroin (1,4),
jotta rivivälit ovat riittävän suuret. Näiden asetusten jälkeen
dokumentissa käytetään korkeamman tason komentoja fonttien valintaan,
esimerkiksi fonttikoon valintakomentoja \koodi{\keno small},
\koodi{\keno normalsize}, \koodi{\keno large} (taulukko
\ref{tlk:fonttikokokomennot}).

Edellä kuvatussa suhteellisessa kirjainperheiden koon määrittelyssä on
sellainen ongelma tai kummallisuus, että Latex koko ajan luulee, että
peruskirjainperhe on normaalikokoisena 10 pistettä. Latexin matalan
tason fonttikomennot eivät tiedä kirjainperheen skaalauskertoimesta, ja
siksi esimerkiksi komentojen

\begin{koodilohkosis}
  \fontsize{10pt}{12pt} \selectfont
\end{koodilohkosis}

tuloksena ei todellisuudessa ole 10 pisteen fontti, vaan mukaan
lasketaan myös kirjainperheen skaalauskerroin. Tämän vuoksi \koodi{\keno
  fontsize}\-/komennon käyttö menee aika oudoksi. Parametrina annetut
mitat eivät pidä paikkaansa.

Jos korkean tason font\-ti\-koko\-komen\-to\-jen (taulukko
\ref{tlk:fonttikokokomennot}) lisäksi tarvitaan jotakin muuta kokoa,
voisi mahdollisesti \koodi{\keno fontsize}\-/komennon sijasta käyttää
Fontspec\-/paketin tarjoamaa komentoa ja tilanteeseen sopivaa
skaalauskerrointa esimerkiksi seuraavalla tavalla:

\begin{koodilohkosis}
  \addfontfeatures{Scale=3.2} Poikkeuksellisen isoa tekstiä
\end{koodilohkosis}

Jos edellä mainitut kummallisuudet eivät häiritse eikä ole tarvetta
määritellä fontteja tarkasti tietyn pistekoon mukaiseksi, on
suhteellinen määrittelytapa todella helppo. Kaikki dokumentin fontit
määräytyvät perusfontin skaalauskertoimen kautta. Tämä tapa sopii hyvin
varsinkin dokumentin sisällön kirjoittamisvaiheeseen, jossa ehkä
halutaan vain nopeasti asettaa dokumentti suurin piirtein järkevän
näköiseksi. Myöhemmin voi määrittää koot absoluuttisesti.

\subsection{Koot absoluuttisesti}
\label{luku:fontti_absoluuttinen}

Absoluuttinen fonttien koonmääritystapa tarkoittaa sitä, että koot
asetetaan tietyn kokoiseksi käyttämällä esimerkiksi pistemittoja ja että
kirjaimet myös päätyvät lopulliseen dokumenttiin juuri sen kokoisena.
Tämä tapa on myös teknisesti eheä, eli Latexin eri osat ovat samaa
mieltä siitä, minkäkokoisesta fontista on kyse. Näin ei ollut
suhteellisen tavan kanssa (luku \ref{luku:fontti_suhteellinen}).

Joskus yrityksen, oppilaitoksen tai muun tahon julkaisu\-ohjeissa
määritellään tarkasti, mitä fontteja käytetään ja mikä on leipätekstin
ja otsikoiden fonttikoko. Silloin tarvitaan tässä luvussa kuvattua tapaa
fonttien asettamiseen.

\begin{esimerkki}
\begin{koodilohko}
  \documentclass{article}
  \usepackage{fontspec}

  % Leipätekstiin samankokoiset fontit
  \setmainfont{TeX Gyre Termes}
  \setsansfont{TeX Gyre Heros} [Scale=MatchLowercase]
  \setmonofont{TeX Gyre Cursor}[Scale=MatchLowercase]
  
  % Muualle sans ja mono ilman skaalausta
  \newfontfamily{\sffamilyabs}{TeX Gyre Heros}
  \newfontfamily{\ttfamilyabs}{TeX Gyre Cursor}

  \linespread{1} % ei välttämättä tarvita

  % Kaikki tarvittavat fonttikoot ja komennot
  \renewcommand{\footnotesize}{\fontsize{10pt}{12pt}\selectfont}
  \renewcommand{\small}       {\fontsize{12pt}{14pt}\selectfont}
  \renewcommand{\normalsize}  {\fontsize{14pt}{17pt}\selectfont}
  \renewcommand{\large}       {\fontsize{17pt}{19pt}\selectfont}
  \renewcommand{\Large}       {\fontsize{20pt}{22pt}\selectfont}
  \normalsize % jotta tulee heti voimaan eikä vasta tekstiosassa
\end{koodilohko}    
\caption{Fonttikokojen määrittäminen pistekoon avulla}
\label{esim:fontti_absoluuttinen}
\end{esimerkki}

Esimerkistä \ref{esim:fontti_absoluuttinen} selviää perus\-ajatus.
Peruskirjainperhe (rivi~5) otetaan käyttöön ilman skaalausta (eli
\koodi{Scale=1}), minkä vuoksi koon voi jatkossa asettaa täsmälleen
kohdalleen \koodi{\keno fontsize}\-/komennolla. Samaa ei tehdä groteskin
eikä tasalevyisen fontin kanssa (rivit 6--7), vaan käytetään skaalausta
\koodi{MatchLowercase}, jotta tekstikappaleessa kaikki kirjainperheet
näyttävät samankokoisilta. Tässä menetetään mahdollisuus määrittää
näiden kirjainperheiden koko täsmällisesti pistemitan avulla. Jos siihen
on tarvetta esimerkiksi otsikoissa, voidaan käyttää rivien 10--11
komentoja. Niillä luodaan uudet kirjainperheet, jotka ovat käytännössä
samoja mutta ilman skaalausta.

Uusien kirjainperheiden komentojen nimiksi on valittu \koodi{\keno
  sffamilyabs} ja \koodi{\keno ttfamilyabs} (vrt. \koodi{\keno sffamily}
ja \koodi{\keno ttfamily}), ja näillä komennoilla kirjainperheet
kytketään päälle. Jos esimerkiksi jonkin julkaisun vaatimuksiin kuuluu,
että otsikossa täytyy olla 16 pisteen lihavoitu TeX Gyre Heros
\=/kirjainleikkaus, voi esimerkissä \ref{esim:fontti_absoluuttinen}
olevien asetusten pohjalta antaa otsikolle seuraavat komennot:

\begin{koodilohkosis}
  \sffamilyabs\fontsize{16pt}{18pt}\bfseries
\end{koodilohkosis}

Esimerkin \ref{esim:fontti_absoluuttinen} riveillä 16--20 määritellään
uudelleen Latexin korkean tason komennot, joilla fonttikoot asetetaan.
Vähintäänkin täytyy määritellä komento \koodi{\keno normalsize} mutta
sen lisäksi kaikki ne, joita omassa dokumentissa tarvitaan. Tässä
esimerkissä normaali koko asetetaan 14 pisteen kokoiseksi ja muut koot
on ajateltu suhteessa siihen.

Jokaiselle fonttikoolle määritetään riveillä 16--20 myös oma
rivikorkeus, ja se on tarkoitus asettaa sopivaksi juuri kyseiselle
koolle. Rivikorkeuteen vaikuttaa myös kerroin \koodi{\keno
  baselinestretch}, joka asetetaan komennolla \koodi{\keno linespread}.
Sitä ei välttämättä tarvitse käyttää, koska kirjainperheitä ei ole
skaalattu ja koska rivikorkeus asetetaan aina \koodi{\keno
  fontsize}\-/komennolla. \koodi{\keno linespread} on kuitenkin kätevä
komento rivikorkeuden säätämiseen yleisesti kaikkialla.

Fonttikokojen määrittelyn lopuksi rivillä 21 suoritetaan komento
\koodi{\keno normalsize}, jotta se tulee heti voimaan. Dokumentin
esittely\-osassa voidaan käyttää fonttikokoon viittaavia mittoja
\koodi{em} ja \koodi{ex}, ja ne viittaavat nyt tähän kokoon.
\koodi{\keno normalsize}\-/komento suoritetaan kyllä myöhemmin
automaattisesti dokumentin teksti\-osan eli
\koodi{document}\-/ympäristön alussa.

Edellä kuvatun absoluuttisen koonmääritystavan etuna on se, että
kirjoittaja hallitsee fonttien kokoa ja rivikorkeuksia tarkasti ja että
julkaisuun saadaan juuri ne mitat, jotka halutaan tai vaaditaan. Tapa on
myös teknisesti eheä eli toimii Latexin sisäisen logiikan näkökulmasta
oikein. Haittana voi pitää sitä, että kaikki koot täytyy määritellä
erikseen (esimerkki \ref{esim:fontti_absoluuttinen}, rivit 16--20).

\section{Kieli}
\label{luku:kieliasetukset}

\subsection{Polyglossia}
\subsection{Babel}

\section{Sivu}
\label{luku:sivuasetukset}
\subsection{Marginaalit ja mitat}
\subsection{Ylä- ja alatunnisteet}

