\begin{esimerkki*}
\begin{koodilohko}
  \begin{document}

  \pagestyle{empty} % sivunumerot pois näkyvistä
  % nimiösivu yms.

  \cleardoublepage
  \setcounter{tocdepth}{2} % sisällysluettelon syvyys
  \pagestyle{plain}        % sivunumerot näkyviin
  \tableofcontents         % sisällysluettelo

  \setcounter{secnumdepth}{-1} % lukujen numerointi pois

  \chapter{Esipuhe}
  % esipuheen teksti
  
  \setcounter{secnumdepth}{2} % aloitetaan lukujen numerointi

  \chapter{Ensimmäinen pääluku}
  % ...
  \chapter{Toinen pääluku}
  % ...

  % mahdollisesti liitteet
  \appendix
  \chapter{Tärkeä liite}
  % ...
  \chapter{Tosi tärkeä liite}
  % ...

  % Kirjallisuusluettelot, asiahakemistot yms.
  \setcounter{secnumdepth}{-1} % lukujen numerointi pois

  \chapter{Kirjallisuutta}
  \printbibliography

  \chapter{Asiahakemisto}
  \printindex

  \end{document}
\end{koodilohko}
\caption{Tyypillinen suomalainen tietokirjojen rakenne sekä sivujen ja
  lukujen numerointikäytännöt}
\label{esim:tietokirjojen_rakenne}
\end{esimerkki*}
