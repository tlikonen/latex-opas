\chapter*{Esipuhe}
\addcontentsline{toc}{chapter}{Esipuhe}
\phantomsection

% Miksi Latex?

Tekstidokumenttien toteuttaminen Latexilla ei ole läheskään samanlaista
kuin työskentely tietotekniikassa yleensä. Yleensähän käynnistetään
jokin sovellus\-ohjelma, joka sisältää suunnilleen kaikki tarvittavat
toiminnot. Samalla sovelluksella työ viedään alusta loppuun.

Sen sijaan Latexin kanssa työskentely on lähempänä ohjelmoijan
työskentelyä. Lähdedokumentti (''koodi'') on muodoltaan täysin erilainen
kuin lopullinen tuotos. Kirjoittaminen vaatii omanlaisensa kielen
osaamista. Lopulliseen toteutukseen tarvitaan yleensä eri tekijöiden
tuottamia makropaketteja (vrt. ohjelmakirjastot), ja niiden ohjekirjoja
täytyy joskus vilkaista. Lähdedokumentit käännetään erillisellä
kääntäjäohjelmalla lopulliseksi tuotokseksi, eikä käännettyä tuotosta
voi enää palauttaa alkuperäiseksi lähdedokumentiksi.

Latexin käyttö ei silti ole mitään ohjelmointia. Merkintäkieli on eri
asia kuin ohjelmointikieli. Työskentelyn luonteessa on kuitenkin useita
yhtäläisyyksiä, ja
