\chapter{Dokumentin rakenne}
\section{Kappaleet}
\label{luku:kappale}

..

\subsection{Tasaus}
%\sloppy, \fussy, \newline \\ \\*

...

\subsection{Sisennykset}

...

\subsection{Välit}

% \bigskip
% \medskip
% \smallskip
% \bigbreak
% \medbreak
% \smallbreak

...

\subsection{Lesket ja orvot}

...

\subsection{Marginaalihuomautukset}
\label{luku:marginaalihuomautukset}

...

\section{Korostus}

...

\section{Sivun vaihto}
% newpage, clearpage, minipage, nopagebreak

...

\section{Otsikot ja jäsennys}
\label{luku:otsikot}

% \sectionbreak ym. ovat titlesec-paketin ominaisuuksia. titlesec
% kannattaa täällä käsitellä.

...

\subsection{Kansilehti ja dokumentin perustiedot}
\label{luku:kansilehti}

...

\subsection{Esittely, pääluvut ja liitteet}
\label{luku:frontmainbackmatter}
% \frontmatter, \mainmatter, \backmatter, \appendix

% Alussa käytetään komentoa \koodi{\keno frontmatter}, joka asettaa
% sivunumeroiden tyyliksi roomalaiset numerot (i, ii, iii jne.). Tässä
% osassa ovat ainakin sisällysluettelo ja tietokirjan tai tutkimuksen
% tiivistelmä. Varsinainen sisältö aloitetaan komennolla \koodi{\keno
%   mainmatter}

...

\section{Luetelmat}
\label{luku:luetelmat}

...

\section{Taulukot}
\label{luku:taulukot}

...

\section{Leijuvat osat}
\label{luku:leijuosat}

...

\section{Ristiviitteet}
\label{luku:ristiviitteet}

...

\section{Alaviitteet}
\label{luku:alaviitteet}

...

\section{Palstat}
\label{luku:palstat}

...

\section{Lähdeluettelo ja lähdeviitteet}
\label{luku:lähteet}

Samoin kuin ristiviitteissäkin (luku \ref{luku:ristiviitteet}) Latex
sisältää komennot, joilla lähdeluettelossa mainittuihin teoksiin voidaan
viitata muualta tekstistä. Ajatus on se, että lähdeluettelo laaditaan
tiettyjen komentojen avulla niin, että jokainen lähdeteos saa jonkin
yksilöllisen tunnisteen. Muualta tekstistä viitataan lähdeteoksiin
käyttämällä samoja tunnisteita, ja Latex osaa automaattisesti poimia
lähdeluettelosta esimerkiksi teoksen tekijöiden nimet ja vuosiluvun.

Lähdeviittauksiin ja lähdemerkintöihin on useita käytäntöjä, jotka
vaihtelevat eri ammatti- ja tieteen\-aloilla, oppilaitoksilla tai
julkaisijoilla. Tässä yhteydessä käsitellään melko vakiintuneita
suomalaisia käytäntöjä, jotka kuvataan \emph{Kielitoimiston
  oikein\-kir\-joi\-tus\-oppaas\-sa} \parencite{kt_oik}. Samalla
käsitellään joitakin asetuksia, joilla kukin voi muokata viittausten
ulkoasua omiin tarpeisiin sopivaksi.

Latex sisältää lähdeluetteloiden ja \=/viittausten perustoiminnot, mutta
niiden avulla ei saa yleisen suomalaisen käytännön mukaisia
lähdeviittauksia. Siksi käytämme apuna makropakettia, jolla viittausten
ja lähdeluettelon ulkoasuun voi vaikuttaa. Ensin käsiteltävä paketti
\paketti{natbib} (luku \ref{luku:natbib}) soveltuu perustarpeisiin ja
lienee sopivin valinta useimmille kirjoittajille. Laajoja tieteellisiä
teoksia kirjoittavan kannattanee opetella käyttämään monipuolista
\paketti{bib\-latex}\-/pakettia (luku \ref{luku:biblatex}) ja ylläpitää
yhteistä lähdeteosten tietokantaa, josta tarvittavat teokset poimitaan
kunkin dokumentin lähdeluetteloon automaattisesti.

\subsection{Peruskuviot (natbib)}
\label{luku:natbib}

Makropaketti \paketti{natbib}\avctan{natbib} laajentaa Latexin
lähdeviittausten perustoimintoja sen verran, että lähteisiin voidaan
viitata teoksen tekijöiden nimen ja vuosiluvun avulla. Seuraava
esimerkki havainnollistaa paketin käyttöönottoa ja asetuksia.

\begin{koodilohkosis}
  \usepackage{natbib}
  \setcitestyle{authoryear,aysep={},notesep={: }}
\end{koodilohkosis}

Edellisessä esimerkissä viittaustyyli valitaan \koodi{\keno
  set\-cite\-style}\-/komennon argumentissa valitsimella
\koodi{author\-year} (tekijä\--vuosi). Valitsimella \koodi{aysep}
määritetään, mikä välimerkki ladotaan tekijän nimen ja vuosiluvun
väliin. Tässä se jätetään tyhjäksi \koodi{\{\}}.
\koodi{note\-sep}\-/valitsimella asetetaan merkit, jotka ladotaan
vuosiluvun ja sitä seuraavan huomautuksen kuten sivunumeroiden väliin;
tässä tapauksessa määritettiin kaksoispiste ja väli \koodi{\{:~\}}.
\koodi{\keno set\-cite\-style}\-/komennon valitsimet erotetaan
toisistaan pilkulla, eikä erotinpilkkujen ympärillä saa olla
välilyöntejä. Lopputuloksena lähdemerkinnät näyttävät esimerkiksi
seuraavanlaisilta.

\pagebreak[3]

\begin{koodilohkosis}
  \citet*[27--29]{johdatus} % Viittaus lähteeseen ”johdatus”.
\end{koodilohkosis}

\begin{tulossis}
  Meikäläinen \& Teikäläinen (2020: 27--29)
\end{tulossis}

Lähdeluettelo kirjoitetaan \koodi{thebibliography}\-/ympäristön ja
\koodi{\keno bib\-item}\-/komentojen avulla esimerkin
\ref{esim:thebibliography} tavoin. Ympäristön aloittavan komennon
(rivi~1) yhteydessä on argumentti \koodi{00}, jolla ei ole tässä
yhteydessä merkitystä. Jos lähdeviittauksen tyylinä olisi
\koodi{numbers} (eikä \koodi{author\-year}), lähdeluettelon teokset
numeroitaisiin, ja silloin \koodi{thebibliography}\-/ympäristön
argumentti ilmaisee, kuinka leveän sisennyksen numeroidut teokset
tarvitsevat. Argumentiksi voi kirjoittaa mitä tahansa merkkejä, ja Latex
mittaa niiden leveyden. Kannattaa kirjoittaa leveitä numeroita kuten
nollia (\koodi{0}) niin monta kappaletta kuin on numeroita suurimmassa
lähdemerkinnän luvussa. Yksi nolla riittää, jos lähteitä on 1--9
kappaletta, kaksi jos lähteitä on kaksinumeroinen määrä eli 10--99
kappaletta jne.

\begin{esimerkki}
\begin{koodilohko}
  \begin{thebibliography}{00}

  \bibitem[Meikäläinen ym.(2020)Meikäläinen \& Teikäläinen]{johdatus}
    Meikäläinen, Matti \& Teikäläinen, Teija (2020): Johdatus alkeiden
    perusteisiin. Toinen painos. Kustantaja Oy.

  \bibitem[Itkonen(2019)]{typografia} Itkonen, Markus (2019):
    Typografian käsikirja. Viides, tarkistettu painos. Typoteekki.
    Graafinen suunnittelu Markus Itkonen Oy.

  \end{thebibliography}
\end{koodilohko}
\caption{Lähdeluettelon kirjoittaminen
  \koodi{thebibliography}\-/ympäristön ja \koodi{\keno
    bib\-item}\-/komentojen avulla.}
\label{esim:thebibliography}
\end{esimerkki}

Komennolla \koodi{\keno bib\-item} tehdään varsinaiset teosmerkinnät.
Samalla määritetään teoksen yksilöllinen tunniste ja mitä tietoja
lähdeviittauksissa näytetään. Yleinen muoto on seuraavanlainen.

\begin{koodilohkosis}
  \bibitem[lyhyt(vuosi)pitkä]{tunniste} Lähdeluettelon tekstit.
\end{koodilohkosis}

Valinnaisen argumentin aluksi kirjoitetaan lähdeviittauksen lyhyt
merkintä, joka tulisi näkymään lähdeviittauksissa esimerkiksi muodossa
''Meikäläinen ym.''. Heti sen perään kirjoitetaan sulkeissa teoksen
vuosiluku ja sen perään vapaavalintainen lähdeviittauksen pitkä
merkintä, joka näkyisi esimerkiksi tekstinä ''Meikäläinen \&
Teikäläinen''. Vuosiluvun sulkeiden ympärillä ei saa olla välilyöntejä.

Komennon pakollinen argumentti on kyseisen lähdeteoksen yksilöllinen
tunniste, jonka avulla kyseiseen teokseen viitataan. Komennon
argumenttien jälkeen kirjoitetaan samaan tekstikappaleeseen teksti, joka
tulee näkymään lähdeluettelossa.

Lähdeteoksiin viittaamiseen on useita eri komentoja, jotka eroavat
toisistaan siinä, mitä tietoa lähdeviittauksessa näytetään ja onko
lähdeviittaus tai sen osa sulkeissa vai ei. Taulukossa
\ref{tlk:natbib-cite} on joitakin \paketti{natbib}\-/paketin
viittauskomentoja sekä esimerkki viittauksen ulkoasusta. Kuviteltu
esimerkkiteos on peräisin esimerkistä \ref{esim:thebibliography}.

\providecommand{\rivi}{}
\renewcommand{\rivi}[2]{\koodi{\keno #1\{\ldots\}} & #2 \\}

\leijutlk{
  \begin{tabular}{ll}
    \toprule
    \ots{Komento} & \ots{Esimerkki} \\
    \midrule
    \rivi{citet}{Meikäläinen ym. (2020)}
    \rivi{citet*}{Meikäläinen \& Teikäläinen (2020)}
    \rivi{citep}{(Meikäläinen ym. 2020)}
    \rivi{citep*}{(Meikäläinen \& Teikäläinen 2020)}
    \rivi{citealt}{Meikäläinen ym. 2020}
    \rivi{citealt*}{Meikäläinen \& Teikäläinen 2020}
    \rivi{citeauthor}{Meikäläinen ym.}
    \rivi{citeauthor*}{Meikäläinen \& Teikäläinen}
    \rivi{citeyear}{2020}
    \rivi{citeyearpar}{(2020)}
    \bottomrule
  \end{tabular}
}{
  \caption{\paketti{natbib}\-/paketin lähdeviittauskomentoja}
  \label{tlk:natbib-cite}
}

Lähdeluettelon ulkoasuun voi vaikuttaa mittojen \koodi{\keno bibhang} ja
\koodi{\keno bibsep} avulla. Ensin mainittu on lähdemerkinnän
vaakasuuntaisen riippuvan sisennyksen suuruus, ja jälkimmäinen on
lähdemerkintöjen välinen pystysuuntainen tila. Mitat asetetaan
tavalliseen tapaan \koodi{\keno set\-length}\-/komennolla (luku
\ref{luku:mitat}).

\begin{koodilohkosis}
  \setlength{\parindent}{1.1em} % tekstikappaleiden sisennys
  \setlength{\bibhang}{\parindent}
  \setlength{\bibsep}{.5ex plus .1ex minus .1ex}
\end{koodilohkosis}

Lähdemerkintöjen fonttiin voi vaikuttaa määrittelemällä uudelleen
komennon \koodi{\keno bibfont} ja sijoittamalla halutut fontti- tai muut
komennot kyseisen komennon määritelmään.

\begin{koodilohkosis}
  \renewcommand{\bibfont}{\sffamily\small}
\end{koodilohkosis}

Oletuksena \koodi{thebibliography}\-/ympäristö latoo lähdeluettelolle
otsikon, ja otsikon teksti määräytyy kieli\-ase\-tus\-ten (luku
\ref{luku:kieliasetukset}) ja dokumenttiluokan perusteella (luku
\ref{luku:dokumenttiluokat}). Suomenkielisen lähdeluettelon otsikon voi
määrittää dokumentin esittelyosassa seuraavan esimerkin tavoin.
Esimerkissä hyödynnetään \koodi{\keno add\-to}\-/komentoa, joka sisältyy
\paketti{polyglossia}\-/{} ja \paketti{babel}\-/paketteihin.

\begin{koodilohkosis}
  \addto{\captionsfinnish}{%
    \renewcommand{\refname}{Lähteet} % article-luokka
    \renewcommand{\bibname}{Lähteet} % report- ja book-luokka
  }
\end{koodilohkosis}


On myös mahdollista määritellä koko komentosarja, joka suoritetaan
lähdeluettelon otsikoinnin yhteydessä. Se tehdään määrittelemällä
uudelleen komento \koodi{\keno bibsection}.

\begin{koodilohkosis}
  \renewcommand{\bibsection}{%
    \setcounter{secnumdepth}{-1}
    \section{Lähteet}
  }
\end{koodilohkosis}

Edellisessä esimerkissä komennolla \koodi{\keno set\-counter}
määritetään, mille otsikkotasolle dokumentin otsikoiden eli lukujen
numerointi yltää. Pieni arvo (\koodi{-1}) käytännössä tarkoittaa, että
seuraaviin otsikoihin ei tule numerointia. Lähdeluettelon otsikkoon ei
numerointia välttämättä haluta. Komento \koodi{\keno section} tekee itse
otsikon.

Jos ei halua, että \koodi{thebibliography}\-/ympäristö tekee otsikon
automaattisesti, voi \koodi{\keno bibsection}\-/komennon määrittää
tyhjäksi.

\begin{koodilohkosis}
  \renewcommand{\bibsection}{}
\end{koodilohkosis}

Tässä alaluvussa on käsitelty lähdeluettelon ja lähdeviitteiden
tekemistä \paketti{natbib}\-/paketin toimintojen avulla. Paketti
sisältää muitakin ominaisuuksia, joihin kannattaa tutustua paketin
ohjekirjan avulla. On muun muassa mahdollista tehdä lähdeteoksista
tietokanta Bibtex\-/järjestelmän avulla. Jos kuitenkin siihen suuntaan
haluaa edetä, ei kannata käyttää \paketti{natbib}\-/pakettia eikä
Bibtexiä vaan monipuolisempaa pakettia \paketti{bib\-latex}, jota
käsitellään seuraavassa alaluvussa.

\subsection{Koko työkalupakki (biblatex)}
\label{luku:biblatex}

...

% \bibitemsep
% \bibhang
% \bibfont

\section{Grafiikka}
\label{luku:grafiikka}

...

\section{Laatikot}

...

\section{Diaesitykset}
\label{luku:diaesitykset}

...

\section{Kirjeet}
\label{luku:kirjeet}

...
