\chapter{Dokumentin rakenne}
\section{Kappaleet}
\label{luku:kappale}

\subsection{Tasaus}
%\sloppy, \fussy, \newline \\ \\*
\subsection{Sisennykset}
\subsection{Välit}

% \bigskip
% \medskip
% \smallskip
% \bigbreak
% \medbreak
% \smallbreak

\subsection{Lesket ja orvot}
\subsection{Marginaalihuomautukset}
\label{luku:marginaalihuomautukset}

\section{Korostus}
\section{Sivun vaihto}
% newpage, clearpage, minipage, nopagebreak
\section{Otsikot ja jäsennys}
\label{luku:otsikot}

\subsection{Kansilehti ja dokumentin perustiedot}
\label{luku:kansilehti}

% \sectionbreak ym. ovat titlesec-paketin ominaisuuksia. titlesec
% kannattaa täällä käsitellä.
%

\subsection{Esittely, pääluvut ja liitteet}
\label{luku:frontmainbackmatter}
% \frontmatter, \mainmatter, \backmatter, \appendix

% Alussa käytetään komentoa \koodi{\keno frontmatter}, joka asettaa
% sivunumeroiden tyyliksi roomalaiset numerot (i, ii, iii jne.). Tässä
% osassa ovat ainakin sisällysluettelo ja tietokirjan tai tutkimuksen
% tiivistelmä. Varsinainen sisältö aloitetaan komennolla \koodi{\keno
%   mainmatter}


\section{Luetelmat}
\label{luku:luetelmat}
\section{Taulukot}
\label{luku:taulukot}
\section{Leijuvat osat}
\label{luku:leijuosat}
\section{Ristiviitteet}
\label{luku:ristiviitteet}
\section{Alaviitteet}
\label{luku:alaviitteet}
\section{Palstat}
\label{luku:palstat}
\section{Lähdeluettelo ja -viitteet}
\label{luku:lähteet}
\section{Grafiikka}
\label{luku:grafiikka}
\section{Laatikot}
\section{Diaesitykset}
\label{luku:diaesitykset}
\section{Kirjeet}
\label{luku:kirjeet}
