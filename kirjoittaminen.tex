\chapter{Dokumentin rakenne}
\section{Kappaleet}
\label{luku:kappale}

\subsection{Tasaus}
%\sloppy, \fussy, \newline \\ \\*
\subsection{Sisennykset}
\subsection{Välit}

% \bigskip
% \medskip
% \smallskip
% \bigbreak
% \medbreak
% \smallbreak

\subsection{Lesket ja orvot}
\section{Korostus}
\section{Sivun vaihto}
% newpage, clearpage, minipage, nopagebreak
\section{Otsikot}

% \sectionbreak ym. ovat titlesec-paketin ominaisuuksia. titlesec
% kannattaa täällä käsitellä.

\section{Luetelmat}
\label{luku:luetelmat}
\section{Taulukot}
\label{luku:taulukot}
\section{Kelluvat osat}
\section{Ristiviitteet}
\section{Alaviitteet}
\section{Tavutus}
\label{luku:tavutus}

\subsection{Perustoiminta}

Latex\-/järjestelmä tavuttaa eli katkaisee sanat automaattisesti rivien
lopussa, jotta rivit ja kappaleet saisi näyttämään tasapainoisilta.
Tavutukseen vaikuttavat kielikohtaiset tavutussäännöt ja \=/asetukset
sekä \koodi{\keno hyphenation}\-/komennolla määritellyt säännöt.
Yksittäisen sanan tavutukseen voi vaikuttaa myös tiettyjen sanassa
mukana olevien merkkien avulla. Niitä käsitellään seuraavissa
alaluvuissa.

Mitat \koodi{\keno lefthyphenmin=N} ja \koodi{\keno righthyphenmin=N}
vaikuttavat tavutukseen sanan reunoissa. Parametri \koodi{N} on
positiivinen kokonaisluku, ja se määrittää, kuinka monta merkkiä
vähintään sanan vasemmasta tai oikeasta reunasta pidetään yhdessä.
Oletus\-arvot ovat kielikohtaisia ja määritellään
\paketti{polyglossia}\-/{} ja \paketti{babel}\-/paketeissa. Suomen
kielessä kumpikin mitta on 2~merkkiä.

Mitat alustetaan kielikohtaisiin oletus\-arvoihin aina, kun
kie\-li\-ase\-tuk\-set tulevat voimaan. Näin on esimerkiksi dokumentin
aloittavan \koodi{document}\-/ympäristön alussa tai kielen vaihtuessa.
Tästä seuraa, että jos mittoja ha\-luaa muuttaa, täytyy ne asettaa
\koodi{document}\-/ympäristön alussa tai aina kielen vaihtamisen
jälkeen.

Omat kieli\-asetukset saa mukaan automaattiseksi, kun lisää muutokset
kielikohtaisiin alustuskomentoihin alla olevan esimerkin mukaisesti.
Tällä tavoin omat asetukset tulevat voimaan samalla kuin muutkin
kie\-li\-ase\-tuk\-set. Esimerkissä oleva komento \koodi{\keno add\-to}
on \paketti{polyglossia}\-/{} ja \paketti{babel}\-/paketin ominaisuus.

\begin{koodilohkosis}
  \addto{\captionsfinnish}{
    \lefthyphenmin=3
    \righthyphenmin=3
  }
\end{koodilohkosis}

Tavutukseen voi vaikuttaa myös \koodi{\keno hyphenation}\-/komennolla,
jolla määritellään yksittäisten sanojen tavutuskohdat kaikkialla
dokumentissa, riippumatta kielestä. Seuraava esimerkki selventää
\koodi{\keno hyphenation}\-/komennon käyttöä.

\begin{koodilohkosis}
  \hyphenation{
    nimen-omaan
    poh-jois-euroop-pa-lai-nen
    inter-net-ope-raat-to-ri
  }
\end{koodilohkosis}

\subsection{Tavutukseen vaikuttavat merkit}

Tietyt sanassa olevat merkit kytkevät muut tavutussäännöt pois päältä ja
muuttavat sanan tavutuksen yksilölliseksi. Jos sanassa on mukana yksikin
tavutusvihje (\koodi{\keno-}), yhdysmerkki (\koodi{-}), lyhyt
ajatusviiva (\koodi{--}) tai pitkä ajatusviiva (\koodi{---}), sana
katkaistaan vain näiden kohdalta (\koodi{\keno-}) tai jälkeen.

\leijutlk{
  \begin{tabular}{lll}
    \toprule
    \ots{Lähde}
    & \ots{Tavutus}
    & \ots{Selitys} \\
    \midrule
    \koodi{matkustaa}
    & mat\tavukohta kus\tavukohta taa
    & tavutus kaikista kohdista \\
    \koodi{matkus\keno-taa}
    & matkus\tavukohta taa
    & vain tavutusvihjeen kohdalta \\
    \koodi{matka-aika}
    & matka-\tavukohta aika
    & vain yhdysmerkin jälkeen \\
    \koodi{matka-ai\keno-ka}
    & matka-\tavukohta ai\tavukohta ka
    & vain yhdysmerkki ja tavutusvihje \\
    \koodi{Oulu--Rovaniemi}
    & Oulu--\tavukohta Rovaniemi
    & vain ajatusviivan jälkeen \\
    \koodi{Oulu--Rova\keno-niemi}
    & Oulu--\tavukohta Rova\tavukohta niemi
    & vain ajatusviiva ja tavuvihje \\
    \koodi{matka-}
    & matka-
    & ei tavutuskohtia \\
    \koodi{-aika}
    & -\tavukohta aika
    & vain yhdysmerkin jälkeen \\
    \bottomrule
  \end{tabular}
}{
  \caption{Tavutusvihjeen, yhdysmerkin ja ajatusviivan vaikutus
    tavutukseen}
  \label{tlk:tex-tavutus}
}

Taulukossa \ref{tlk:tex-tavutus} on esimerkkejä tavutusvihjeiden,
yhdysmerkin ja ajatusviivan vaikutuksesta. Ensimmäisessä sarakkeessa on
esimerkkisana siinä muodossa kuin se kirjoitetaan lähdetiedostoon.
Toisessa sarakkeessa on ladottu sana, johon on punaisella pystyviivalla
merkittynä mahdolliset tavutuskohdat.

Viimeinen rivi paljastaa suomen kielen kannalta ongelmallisen tilanteen.
Esimerkiksi ilmauksessa \emph{matkasuunnitelma ja \mbox{-aika}} ei sanaa
saa katkaista yhdysmerkin jälkeen, koska rivin loppuun jäisi yksinäinen
yhdysmerkki. Latexin peruskeinoilla sana täytyy laittaa näkymättömään
laatikkoon, joka pitää merkit yhdessä: \koodi{\keno mbox\{-aika\}}.
Toinen vaihtoehto on sitovan yhdysmerkin käyttö, johon onkin monenlaisia
toteutuksia. Niistä kerrotaan seuraavissa luvuissa.

\subsection{Unicoden sitova yhdysmerkki}
\label{luku:unicode_sitova}

Unicode-merkistön sitova yhdysmerkki \uctunnus{u+2011 non\-/breaking
  hyphen} näkyy tavallisena yhdysmerkkinä, mutta se estää sanan
katkaisemisen yhdysmerkin vierestä. Sitovaa yhdysmerkkiä voi käyttää
Xelatex\-/kääntäjän kanssa. Sen sijaan Lualatex\-/kääntäjä vain kadottaa
koko merkin.

Tavallinen yhdysmerkki \uctunnus{u+002d hyphen\-/minus} ja Unicoden
sitova yhdysmerkki \uctunnus{u+2011} vaikuttavat sanan tavuttamiseen eri
tavalla. Tavallinen yhdysmerkki estää sanan tavuttamisen muualta kuin
yhdysmerkin jälkeen. Unicoden sitova yhdysmerkki puolestaan sallii sanan
tavuttamisen kaikista normaaleista tavutuskohdista. Ainoastaan sitovan
yhdysmerkin vierestä ei katkaista.

\leijutlk{%
  \begin{tabular}{ll}
    \toprule
    \ots{Lähde}
    & \ots{Tavutus} \\
    \midrule
    \koodi{Latex-järjestelmä}
    & Latex-\tavukohta järjestelmä \\
    \koodi{Latex\sitovaym järjestelmä}
    & La\tavukohta tex-jär\tavukohta jes\tavukohta tel\tavukohta mä \\
    \koodi{-järjestelmä}
    & -\tavukohta järjestelmä \\
    \koodi{\sitovaym järjestelmä}
    & -jär\tavukohta jes\tavukohta tel\tavukohta mä \\
    \bottomrule
  \end{tabular}
}{%
  \caption{Tavallisen ja Unicoden sitovan yhdysmerkin (punaisella)
    vaikutus tavutukseen}
  \label{tlk:unicode-yhdysmerkki}
}

Taulukossa \ref{tlk:unicode-yhdysmerkki} vertaillaan yhdysmerkkejä.
Viimeinen rivi osoittaa, että Unicoden sitovan yhdysmerkin avulla voi
estää sen, että yhdysmerkillä alkava sana katkaistaisiin heti
yhdysmerkin jälkeen eli merkki jäisi yksinään rivin loppuun.

\subsection{Pakettiratkaisuja}

Makropaketti \paketti{extdash} tuo uusia komentoja ja mahdollisuuksia
tavutuksen hallintaan. Komennot ovat sellaisia kuin \koodi{\keno
  Hyph\-dash} ja \koodi{\keno En\-dash}, mutta niille on saatavilla myös
lyhemmät muodot, jos paketin lataa käyttämällä
\koodi{shortcuts}\-/valitsinta.

\begin{koodilohkosis}
  \usepackage[shortcuts]{extdash}
\end{koodilohkosis}

Paketti sisältää kaksi lisä\-vaihto\-ehtoa kolmelle viivavälimerkille
eli yhdysmerkille, lyhyelle ajatusviivalle ja pitkälle ajatusviivalle.
Kun Latexin viivavälimerkit aina estävät tavutuksen muualta kuin
välimerkin jälkeen, \paketti{extdash}\-/paketin perus\-vaihto\-ehdot
sallivat tavuttamisen. Lisäksi kaikille kolmelle viivavälimerkille on
sitova versio, joka estää tavutuksen viivavälimerkin jälkeen. Taulukossa
\ref{tlk:extdash} on \paketti{extdash}\-/paketin tärkeimmät komennot ja
niiden merkitykset. Taulukossa \ref{tlk:extdash-vertailu} vertaillaan
\paketti{extdash}\-/paketin komentoja Latexin vastaaviin.

\leijutlk{
  \begin{tabular}{ll}
    \toprule
    \ots{Komento} & \ots{Merkitys} \\
    \midrule
    \koodi{\keno -/} & tavutuksen salliva yhdysmerkki \\
    \koodi{\keno =/} & sitova, tavutuksen salliva yhdysmerkki \\
    \koodi{\keno --} & tavutuksen salliva lyhyt ajatusviiva \\
    \koodi{\keno ==} & sitova, tavutuksen salliva lyhyt ajatusviiva \\
    \koodi{\keno ---} & tavutuksen salliva pitkä ajatusviiva \\
    \koodi{\keno ===} & sitova, tavutuksen salliva pitkä ajatusviiva \\
    \bottomrule
  \end{tabular}
}{
  \caption{\paketti{extdash}-paketin komentoja}
  \label{tlk:extdash}
}

\leijutlk{
  \begin{tabular}{llll}
    \toprule
    \ots{Lähde} & \ots{Tavutus}
    & \ots{Lähde} & \ots{Tavutus} \\
    \midrule
    \koodi{matka-aika}
                & matka-\tavukohta aika
                & \koodi{Oulu--Rovaniemi}
                & Oulu--\tavukohta Rovaniemi \\
    \koodi{matka\keno -/aika}
                & mat\tavukohta ka-\tavukohta ai\tavukohta ka
                & \koodi{Oulu\keno --Rovaniemi}
                & Ou\tavukohta lu--\tavukohta Ro\tavukohta va\tavukohta
                  nie\tavukohta mi \\
    \koodi{matka\keno =/aika}
                & mat\tavukohta ka-ai\tavukohta ka
                & \koodi{Oulu\keno ==Rovaniemi}
                & Ou\tavukohta lu--Ro\tavukohta va\tavukohta
                  nie\tavukohta mi \\
    \koodi{matka-}
                & matka-
                & \koodi{-aika}
                & -\tavukohta aika \\
    \koodi{matka\keno -/}
                & mat\tavukohta ka-
                & \koodi{\keno =/aika}
                & -ai\tavukohta ka \\
    \bottomrule
  \end{tabular}
}{
  \caption{Latexin ja \paketti{extdash}-paketin komentojen vertailua}
  \label{tlk:extdash-vertailu}
}

Kieli\-asetuksia hoitavan \paketti{polyglossia}\-/paketin uudehkot
versiot (2020\==) tukevat suomen kielelle \koodi{babelshorthands}\-/
valitsinta, joka lisää keinovalikoimaan \paketti{babel}\-/paketista
peräisin olevia tavutuksen hallintakomentoja. On esimerkiksi tavutuksen
salliva yhdysmerkki (\koodi{\textquotedbl-}), joka toimii samoin kuin
\paketti{extdash}\-/paketin \koodi{\keno-/}.

Mukana on myös sitova, kaiken tavutuksen estävä yhdysmerkki
(\koodi{\textquotedbl\textasciitilde}), jolle ei ole vastinetta Texissä
eikä \paketti{extdash}\-/paketissa -- muuten kuin \koodi{\keno
  mbox}\-/laatikon avulla. \paketti{polyglossia}\-/paketin ''Baabelin
lyhenteet'' sisältävät eräitä muitakin hyödyllistä sanan katkaisemiseen
liittyviä komentoja.

Dokumentin otsikoissa (\koodi{\keno section} ym.) täytyy olla varovainen
\paketti{extdash}\-/paketin komentojen kanssa. Ne toimivat ladotussa
dokumentissa, mutta \paketti{hyperref}\-/paketin avulla syntyvässä
pdf\-/tiedoston sisällysluettelossa ne eivät näy oikein. Ongelman voi
kiertää käyttämällä Latexin otsikkokomentojen valinnaista parametria,
jolla määritellään erikseen sisällysluettelossa näytettävä otsikko. Sen
avulla varsinainen otsikko voi sisältää monimutkaisempia komentoja ja
sisällysluetteloon tuleva versio vain tekstiä ja tavutusvihjeitä.

\begin{koodilohkosis}
  \section[Latex-jär\-jes\-tel\-mä]{\LaTeX\-/järjestelmä}
\end{koodilohkosis}

Otsikot voivat vaatia tavutusvihjeitä silloin, kun niihin viitataan
tekstistä komennolla \koodi{\keno nameref} tai kun otsikko itsessään on
pitkä ja täytyy jakaa useammalle riville.

\subsection{Johtopäätökset}

Suomenkielisen tekstin kirjoittamiseen tarvittaneen välillä
\paketti{extdash}\-/paketin tavutus\-apuja, koska kielessä on pitkiä
yhdysmerkin tai ajatusviivan sisältäviä yhdyssanoja kuten
\emph{Molotov\--Ribben\-trop\-/sopimus} tai
\emph{kognitiivis\-/behavioristinen}. Rivejä ei välttämättä saa
siisteiksi, ellei pitkissä yhdyssanoissa salli tavuttamista muualtakin
kuin yhdysmerkin jälkeen. Tavutusvihjeitä voi toki lisäillä käsin, mutta
se voi käydä joskus työlääksi.\footnote{Esimerkiksi:
  \koodi{kog\keno-ni\keno-tii\keno-vis-%
    be\keno-ha\keno-vio\keno-ris\keno-ti\keno-nen}} Tässä oppaassa ja
edellä mainituissa esimerkkisanoissa on käytetty
\paketti{extdash}\-/paketin yhdysmerkkiä \koodi{\keno -/} ja
ajatusviivaa \koodi{\keno --}, joten tavutus hoituu itsestään.

Unicoden sitovaa yhdysmerkkiä (luku \ref{luku:unicode_sitova}) ei ehkä
kannata käyttää, koska Lualatex\-/kääntäjä kadottaa koko merkin. Merkki
kyllä toimii Xelatex\-/kääntäjällä, mutta \paketti{extdash}\-/paketti
tarjoaa kaikkialla toimivan vaihtoehdon sitovalle yhdysmerkille
(\koodi{\keno =/}).

Pdf-tiedostojen sisäinen sisällysluettelo syntyy
\paketti{hyperref}\-/paketin avustuksella itsestään dokumentin
otsikoista (\koodi{\keno section} ym.). Otsikoissa kannattanee käyttää
tavutuksessa vain Latexin perustoimintoja eli tavallisia yhdysmerkkejä
ja tavutusvihjeitä (\koodi{\keno-}), koska erikoisemmat komennot eivät
näy pdf:n sisällysluettelossa oikein.

Jos \paketti{extdash}\-/paketin toimintojen lisäksi tarvitsee sitovaa,
kaiken tavutuksen estävää yhdysmerkkiä (esimerkiksi
\koodi{\textquotedbl\textasciitilde aika}), täytyy käyttää myös
\paketti{polyglossia}\-/paketin ''Baabelin lyhenteitä''. Seuraava
esimerkki näyttää, kuinka kaikki mainitut toiminnot kytketään päälle.

\begin{koodilohkosis}
  \usepackage{polyglossia}
  \usepackage[shortcuts]{extdash}
  \setdefaultlanguage[babelshorthands]{finnish}
\end{koodilohkosis}

\section{Lähdeluettelo ja -viitteet}
\section{Grafiikka}
\section{Laatikot}
