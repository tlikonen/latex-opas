% Tekijä:   Teemu Likonen <tlikonen@iki.fi>
% Lisenssi: Creative Commons Nimeä-JaaSamoin 4.0 Kansainvälinen (CC BY-SA 4.0)
% https://creativecommons.org/licenses/by-sa/4.0/legalcode.fi

\chapter{Asetukset}

Dokumentin yleiset asetukset koostuvat Latexissa dokumenttiluokan
valinnasta, sivun koon ja marginaalien määrittelystä, fonttien
määrittelystä sekä kieliasetuksista. Näitä kaikkia käsitellään tässä
luvussa.

Kirjoittajan ei tarvitse tehdä kaikkia dokumentin asetuksia kerralla. On
ehkä jopa suotavaakin keskittyä aluksi lähinnä dokumentin sisällön ja
rakenteen suunnitteluun ja tuottamiseen. Ulkoasuun kyllä ehtii tulla
kirjoittamisen aikana monenlaisia ajatuksia, joita ei olisi välttämättä
alussa osannut huomioida. Kirjoittamaan pääsee hyvinkin yksinkertaisen
dokumenttirungon avulla (esimerkki \ref{esim/ensimmäinen},
s.~\pageref{esim/ensimmäinen}), mutta tässä luvussa käsitellään
asetusten määrittelyä ja mahdollisuuksia melko perusteellisesti.

\section{Dokumenttiluokat}
\label{luku/dokumenttiluokat}

Latexin lähdedokumenttien alussa on aina samankaltainen rivi, joka
määrittelee käytettävän dokumenttiluokan ja mahdollisesti dokumentin
perusasetuksia. Dokumenttiluokka määritellään komennolla
\komento{documentclass} ja sen argumentiksi annetaan dokumenttiluokan
nimi. Valinnaisilla argumenteilla vaikutetaan asetuksiin.

\komentoi{documentclass}
\luokkai{article}
\begin{koodilohkosis}
\documentclass[a4paper, 12pt]{article}
\end{koodilohkosis}

\noindent
Dokumenttiluokka on eräänlainen pohjadokumentti eli ominaisuuksien ja
asetusten kokoelma, jonka varaan oma dokumentti kirjoitetaan. Eri
dokumenttiluokat sisältävät erilaisia ominaisuuksia ja erilaiset
oletusasetukset. Edellä olevassa esimerkissä käytettiin
\luokka{article}\-/ luokkaa, joka on yleiskäyttöinen luokka monenlaisten
dokumenttien kirjoittamiseen. Seuraavissa alaluvuissa käsitellään
tavallisimpia dokumenttiluokkia ja niiden asetuksia.

% Dokumenttiluokkia voi tehdä itsekin, ja sitä aihetta käsitellään
% luvussa \ref{luku/xxxx}.

\subsection{Perusdokumenttiluokat}
\label{luku/perusdokumenttiluokat}

Muutama dokumenttiluokka kuuluu Latexin perusvalikoimaan, eli ne ovat
aina saatavilla ja asennettuna. Niitä kutsutaan välillä standardeiksi
dokumenttiluokiksi, ja ne ovat ikään kuin muuttumattomia perusluokkia,
joiden varaan on turvallista rakentaa omia dokumentteja tai muita
dokumenttiluokkia. Muita dokumenttiluokkia kehitetään Latexin perusosien
ulkopuolella, ja niiden ominaisuudet voivat muuttua ja kehittyä
nopeammin ajan myötä.

Normaalit paperisivuihin perustuvat dokumentit tehdään luokkien
\luokka{article}, \luokka{report} tai \luokka{book} avulla. Ne ovat
keskenään hyvin samanlaisia, mutta niiden oletusasetukset poikkeavat
toisistaan. Karkeasti jaoteltuna \luokka{article}\-/ luokka on
tarkoitettu lyhyehköille artikkeleille ja yleiseksi perustaksi
monenlaisille dokumenteille. Sen sijaan \luokka{report} ja \luokka{book}
on tarkoitettu laajoihin dokumentteihin, ja niissä on kirjatypografian
piirteitä.

Suurin edellä mainittujen perusluokkien ero on otsikoinnissa ja
dokumentin jäsentämisessä: \luokka{report} ja \luokka{book} sisältävät
suuret pääluvut (\komento{chapter}) eli otsikot, jotka alkavat uudelta
sivulta; \luokka{article} ei sisällä samanlaisia päälukuja vaan
ainoastaan tavalliset leipätekstin mukana kulkevat väliotsikot
(\komento{section}, \komento{subsection} jne.). Tekstin jäsentämistä
käsitellään luvussa \ref{luku/jäsennys} ja otsikointia tarkemmin luvussa
\ref{luku/otsikot}.

Toinen ero perusluokkien välillä on se, että vain \luokka{book}\-/
luokka sisältää komennot \komento{frontmatter}, \komento{mainmatter} ja
\komento{backmatter}, joita voi käyttää ilmaisemaan tietokirjan
erityyppiset osat: esittely\-/, sisältö\-/{} ja liitesivut. Näitä
käsitellään luvussa \ref{luku/frontmainbackmatter}. Lisäksi
\luokka{report}\-/ luokka sisältää ympäristön \ymparisto{abstract}, joka
on tarkoitettu tutkimusraportin tai vastaavan dokumentin tiivistelmäosan
tekemiseen; \luokka{book}\-/ luokassa sitä ei ole.

Muut erot koskevat lähinnä oletusasetuksia, mutta asetukset ovat
tietenkin muutettavissa, eivätkä ne siten ole määräävä tekijä
valittaessa luokkien \luokka{article}, \luokka{report} ja \luokka{book}
välillä. Dokumenttiluokkien asetuksia käsitellään luvussa
\ref{luku/perusdokumenttiluokat-asetukset}.

Muita perusdokumenttiluokkia ovat \luokka{letter} ja \luokka{slides}.
Nimensä mukaisesti \luokka{letter} on kirjeiden kirjoittamista varten.
Se sisältää kirjeen tyypilliseen rakenteeseen liittyviä komentoja ja
soveltuu varsinkin tarpeisiin, joissa täytyy automaattisesti tuottaa
samanmuotoisia kirjeitä usealle vastaanottajalle. Sama
Latex\-/dokumentti voi sisältää useita kirjeitä, ja sivunumerointi ja
alaviitteiden numerointi alkaa joka kirjeessä alusta -- kuten on
tietysti toivottavaa. \luokka{letter}\-/ luokkaa käsitellään tarkemmin
luvussa \ref{luku/kirjeet}.

Piirtoheittimien läpinäkyvien kalvojen ja sen kaltaisten dokumenttien
tekemiseen on dokumenttiluokka \luokka{slides}. Piirtoheittimet ovat
jääneet menneisyyteen, ja nykyään jokseenkin vastaavanlainen rooli on
esitysgrafiikkaohjelmilla, joilla tehdään diaesityksiä esimerkiksi
esitelmien havaintomateriaaliksi. \luokka{slides}\-/ luokkaa ei varmaan
tarvita enää mihinkään, mutta esitysgrafiikkaan on olemassa erittäin
monipuolinen luokka \luokka{beamer}, jonka perusasioita käsitellään
luvussa \ref{luku/esitysgrafiikka}.

\subsection{Perusdokumenttiluokkien asetukset}
\label{luku/perusdokumenttiluokat-asetukset}

Taulukkoon \ref{tlk/luokkavalitsimet} on koottu perusdokumenttiluokkien
asetusten valitsimet. Sivun koko asetetaan valitsimilla
\koodi{a4\-paper} (210 × 297\,mm), \koodi{a5\-paper} (148 × 210\,mm),
\koodi{b5\-paper} (176 × 250\,mm), \koodi{execu\-tive\-paper} (7,25 ×
10,5 tuumaa), \koodi{legal\-paper} (8,5 × 14 tuumaa) tai
\koodi{letter\-paper} (8,5 × 11 tuumaa). Latexin oletuksena on
\koodi{letterpaper}, mutta oletusta voi olla muutettu Latexin
jakelupaketissa. Varminta on aina itse kirjoittaa haluttu sivukoko
dokumentin asetuksiin.

\leijutlk{
  \ttfamily
  \begin{tabular}{llll}
    \toprule
    a4paper & landscape & openright & 10pt \\
    a5paper & final & openany & 11pt \\
    b5paper & draft & onecolumn & 12pt \\
    executivepaper & oneside & twocolumn & titlepage \\
    legalpaper & twoside & fleqn & notitlepage \\
    letterpaper & openbib & leqno & clock \\
    \bottomrule
  \end{tabular}
}{
  \caption{Perusdokumenttiluokkien valitsimet}
  \label{tlk/luokkavalitsimet}
}

Sivu on oletuksena pystyasennossa, mutta valitsin \koodi{landscape}
asettaa sen vaaka\-/asentoon. Tämä asetus koskee dokumentin kaikkia
sivuja, mutta jos haluaa asettaa vain yksittäisiä sivuja
vaaka\-/asentoon, täytyy käyttää \pakettictan{pdflscape}\-/ pakettia ja
sen tarjoamaa ympäristöä.

Sivun koon ja muitakin mittoja voi määritellä monipuolisemmin
\paketti{geometry}\-/paketin toimintojen avulla (luku
\ref{luku/sivuasetukset}). Jos sitä pakettia käyttää, ei näitä
dokumenttiluokan sivukokoasetuksia tarvita välttämättä lainkaan.

Leipätekstin fontin koon voi määrittää valitsimilla \koodi{10pt},
\koodi{11pt} ja \koodi{12pt}, joista ensin mainittu on oletus. Lyhenne
pt tarkoittaa typografista pistemittaa, joita käsitellään luvussa
\ref{luku/mitat}. Nämä asetukset vaikuttavat myös sivun marginaaleihin,
koska fontin koko vaikuttaa tekstiriville mahtuvien merkkien määrään ja
Latex pyrkii pitämään rivit sopivan mittaisina lukemisen kannalta.

Vain kolme eri fonttikokoa leipätekstille on kovin vähän ja peräisin
ajalta, jolloin Latexin fontit olivat pikseligrafiikkaa eli muodostuivat
erillisistä kuvapisteistä. Nykyaikana fontit ovat vektorigrafiikkaa eli
matemaattisten kaavojen avulla muodostettavia kuvia, ja ne voi venyttää
mihin kokoon tahansa. Sen vuoksi nämä dokumenttiluokkien
fonttikokovalitsimet eivät ole kovin tarpeellisia eivätkä riitä kaikkiin
tarpeisiin nykyaikana. Voi niitä silti käyttää, jos niiden avulla saa
halutun lopputuloksen. Fontteja käsitellään tarkemmin luvussa
\ref{luku/kirjaintyypit}.

Dokumentista voi latoa luonnosversion käyttämällä dokumenttiluokan
valitsinta \koodi{draft}. Luonnoksena ladottuun dokumenttiin tulevat
merkityksi (musta suorakulmio) ainakin tekstipalstan yli pursuavat
rivit, mikä voi auttaa tekstikappaleiden ulkoasun korjailussa (luku
\ref{luku/kappale}). Eri paketit voivat hyödyntää \koodi{draft}\-/
valitsinta omalla tavallaan eli muuttaa toimintaansa sen perusteella.
Esimerkiksi grafiikkaan liittyvä paketti \paketti{graphicx} (luku
\ref{luku/grafiikka}) jättää luonnosversiosta kuvat pois, ja niiden
paikalla on vain suorakulmio. Pdf\-/tiedoston ristiviitteistä huolehtiva
\paketti{hyperref}\-/paketti jättää pdf:n sisäiset ristiviitteet ja
sisällysluettelon tekemättä (luku \ref{luku/ristiviitteet}). Latominen
voi nopeutua huomattavasti.

Lopullinen ladontatila on \koodi{final}, joka tarkoittaa, että
dokumenttiin ladotaan kaikki sisältö ja ominaisuudet niin kuin se on
tarkoitettu julkaistavaksi. Tätä valitsinta ei tarvitse käyttää, koska
se on oletustila.

Sivujen yksipuolisuus (\koodi{oneside}) ja kaksipuolisuus
(\koodi{twoside}) ovat asetuksia, jotka vaikuttavat sivun marginaalien
asetuksiin ja mahdollisesti uuden pääluvun (\komento{chapter})
aloitukseen. Oletus on yksipuolinen dokumentti kaikissa muissa
perusdokumenttiluokissa paitsi \luokka{book}\-/ luokassa, jonka oletus
on kaksipuolinen.

Yksipuolisessa dokumentissa kaikkien sivujen marginaalit ovat
lähtökohtaisesti samanlaisia ja asetuksissa puhutaan esimerkiksi
vasemmasta (\englanti{left}) ja oikeasta (\englanti{right})
marginaalista. Kaksipuolisessa dokumentissa sivut ajatellaan pareittain:
kaksi sivua muodostaa kirjan aukeaman, jonka vasemmalla puolella on
parillinen sivunumero ja oikealla pariton. Marginaalien asetuksissa
puhutaan sisämarginaaleista (\englanti{inner}) ja ulkomarginaaleista
(\englanti{outer}). Sisämarginaalit ovat oletuksena pienemmät, koska
niitä on aukeamalla aina kaksi vierekkäin. Ulkomarginaaleissa on tilaa
marginaalihuomautuksille (luku \ref{luku/marginaalihuomautukset}).
Marginaaleja ja muita sivun asetuksia käsitellään tarkemmin luvussa
\ref{luku/sivuasetukset}.

Jos dokumentti on määritetty kaksipuoliseksi (\koodi{twoside}), voidaan
\koodi{open\-right}\-/ valitsimella määrittää pääluvut
(\komento{chapter}) alkamaan aina oikeanpuoleiselta sivulta. Se onkin
oletus dokumenttiluokassa \luokka{book}. Valitsin \koodi{open\-any}
asettaa pääluvut alkamaan miltä tahansa seuraavalta sivulta.

Valitsin \koodi{open\-bib} liittyy automaattiseen lähdeluettelon
muotoiluun. Jos tämä valitsin on mukana, automaattisessa lähdeluettelon
muotoilussa lisätään rivinvaihtoja lähdemerkinnän eri osien kuten
teoksen tekijöiden ja teoksen nimen jälkeen. Lähdemerkintöjä ja
lähdeluettelon laatimista käsitellään luvussa \ref{luku/lähteet}.

Matemaattisten kaavojen ladontaan liittyvässä \ymparisto{equation}\-/
ympäristössä kaavat ladotaan normaalisti vaakasuunnassa sivun keskelle.
Dokumenttiluokan valitsinta \koodi{fleqn} käyttämällä kaavat ladotaan
kuitenkin sivun vasempaan reunaan. Kaavojen numerot sijoitetaan
oletuksena sivun oikeaan reunaan, mutta valitsimella \koodi{leqno} ne
saa ladottua sivun vasempaan reunaan. Matematiikkatilaa käsitellään
luvussa \ref{luku/matematiikka}.

Nimiösivun tai dokumentin perustietojen esitystapaan vaikutetaan
valitsimilla \koodi{title\-page} ja \koodi{no\-title\-page}. Latexissa
on yksinkertaiset komennot dokumentin perustietojen eli pääotsikon,
tekijöiden nimien ja päiväyksen latomiseen (luku
\ref{luku/dokumentin-perustiedot}), ja näillä valitsimilla vaikutetaan
siihen, ladotaanko perustiedot omalle sivulleen (\koodi{title\-page})
vaiko ensimmäisen sisältösivun alkuun (\koodi{no\-title\-page}).
Oletus\-asetus vaihtelee eri dokumenttiluokissa.

Oletuksena sivun tekstit ladotaan yhdelle palstalle
(\koodi{onecolumn}), mutta valitsimella \koodi{twocolumn} teksti
ladotaan kahdelle palstalle. Nämä valitsimet vaikuttavat myös sivun
marginaaleihin. Latexin perusosat eivät hallitse useampaa kuin kaksi
palstaa, mutta \paketti{multicol}\-/paketin avulla saa enemmänkin.
Palstoja käsitellään luvussa \ref{luku/palstat}.

Dokumenttiluokassa \luokka{slides} toimii \koodi{clock}\-/ valitsin,
joka latoo kellonajan piirtoheitinkalvon loppuun \komento{note}\-/
komennon yhteydessä. Tämä valitsin kuulunee samaan paikkaan kuin
piirtoheittimet ja \luokka{slides}\-/luokkakin: kierrätykseen tai
museoon.

\subsection{Kehittyneempi dokumentti: memoir}

Luokka \luokkactan{memoir} on Latexin \luokka{book}\-/ luokkaan
perustuva, ominaisuuksiltaan laajennettu luokka, joka sopii varsinkin
laajoihin dokumentteihin. Tähän luokkaan on sisällytetty monia
dokumentin rakenteeseen ja ulkoasuun liittyviä toimintoja, joihin
muutoin tarvitaan erillisiä paketteja tai lisäkoodia.

Saattaa siis olla helpompaa käyttää lähes pelkästään \luokka{memoir}\-/
luokkaa ja opiskella sen ohjekirjaa, kuin että ratkoisi dokumentin
rakenteen ja asettelun kysymyksiä erillisten pakettien avulla. Jos
päädyt käyttämään \luokka{memoir}\-/ luokkaa, kannattaa aina ensin
tutkia sen ohjekirjasta, saako tarvittavat asiat toteutettua luokan
omien ominaisuuksien avulla. Vasta sen jälkeen voi hakea neuvoa tästä
tai muista oppaista.

\section{Sivu}
\label{luku/sivuasetukset}

Latex\-/dokumentit muodostuvat aina peräkkäisistä sivuista, eli
taustalla kummittelee paperiarkkeihin perustuva julkaisumuoto, vaikka
dokumenttia ei varsinaisesti tulostettaisi tai painettaisi paperille.
Tämän vuoksi melkein aina täytyy määrittää sivun asetukset kuten
paperikoko ja marginaalit. Ne tehdään \paketti{geometry}\-/ paketin
avulla, jota käsitellään luvussa \ref{luku/sivun-mitat}. Joskus halutaan
muokata sivun ylä- tai alatunnistetta eli verrattain pysyviä
tunnistetietoja, jotka toistuvat joka sivulla ylä- tai alareunassa.
Niiden muokkaamiseen käytetään \paketti{fancyhdr}\-/ pakettia, jonka
ohjeita on luvussa \ref{luku/ylä-ala-tunnisteet}. Sivun taustavärin
asettamista käsitellään luvussa \ref{luku/korostus-värit}.

\subsection{Sivun koko ja marginaalit}
\label{luku/sivun-mitat}

Paperin eli sivun kokoon ja marginaaleihin pääsee vaikuttamaan
\pakettictan{geometry}\-/ paketin avulla. Halutut asetukset voi kertoa
paketin lataamisen yhteydessä. Seuraavassa esimerkissä asetetaan
paperikoko (\koodi{a4\-paper} eli \textsc{a4}-koko), ylä- ja
alamarginaalin pituus (\koodi{top}, \koodi{bottom}) sekä vasemman ja
oikean marginaalin pituus (\koodi{left}, \koodi{right}).

\komentoi{usepackage}
\begin{koodilohkosis}
\usepackage[a4paper, top=20mm, bottom=30mm,
  left=20mm, right=20mm]{geometry}
\end{koodilohkosis}

\noindent
Vaihtoehtoisesti asetukset voi ilmaista erikseen \komento{geometry}\-/
komennon avulla seuraavalla tavalla:

\komentoi{usepackage}
\komentoi{geometry}
\begin{koodilohkosis}
\usepackage{geometry}
\geometry{a4paper, top=20mm, bottom=30mm, left=20mm, right=20mm}
\end{koodilohkosis}

\noindent
Mikäli myöhemmin dokumentissa täytyy vaihtaa sivun asettelua, käytetään
komentoa \komento{newgeometry}. Alkuperäiset asetukset palautetaan
komennolla \komento{restoregeometry}.

\komentoi{newgeometry}
\komentoi{restoregeometry}
\begin{koodilohkosis}
\newgeometry{top=…, bottom=…, …} % Uudet asetukset.
\restoregeometry     % Palautetaan alkuperäiset asetukset.
\end{koodilohkosis}

\noindent
Valmiiksi määriteltyjä standardipaperikokoja on useita.
\textsc{iso}\-/standardin mukaiset koot \textsc{a0}--\textsc{a6}
valitaan valitsimilla kuten \koodi{a3paper}, \koodi{a4paper} tai
\koodi{a5paper}. Samoin käytetään myös kokoja \textsc{b0}--\textsc{b6}
ja \textsc{c0}--\textsc{c6}, esimerkiksi \koodi{b2paper} tai
\koodi{c6paper}. Lisäksi on valittavissa useita japanilaisia ja
yhdysvaltalaisia standardeja käyttämällä valitsimia kuten
\koodi{b0j}--\koodi{b6j}, \koodi{ansi\-a\-paper} tai
\koodi{letter\-paper}.

Mikäli standardit koot eivät riitä, voi sivun mitat määrittää vapaasti
\koodi{paper\-size}\-/ valitsimella, jolle annetaan arvoksi leveys- ja
korkeusmitta.

\komentoi{geometry}
\begin{koodilohkosis}
\geometry{papersize={10cm, 16cm}}
\end{koodilohkosis}

\noindent
Sivu on oletuksena pystyasennossa (\koodi{portrait}), mutta sen voi
asettaa vaaka\-/asentoon valitsimella \koodi{landscape}. Tämä asetus on
voimassa koko dokumentin ajan. Jos sen sijaan haluaa asettaa vain
yksittäisiä sivuja vaaka\-/asentoon, täytyy käyttää
\pakettictan{pdflscape}\-/ pakettia ja sen tarjoamaa
\ymparisto{landscape}\-/ ympäristöä. Ympäristön sisältö ladotaan
vaakasuuntaisille sivuille.

\ymparistoi{landscape}
\begin{koodilohkosis}
\begin{landscape}
  ...
\end{landscape}
\end{koodilohkosis}

\noindent
Sivun asetusten suunnittelussa voi olla avuksi valitsin
\koodi{show\-frame}, joka piirtää viivat marginaalien kohdalle ja
merkitsee myös ylä- ja alatunnisteiden sekä marginaalihuomautusten
rajoja.

Taulukkoon \ref{tlk/sivun-marginaalit} on koottu tärkeimpiä valitsimia
marginaalien mittojen asettamiseksi. Valitsimille annetaan arvoksi Texin
mittayksikkö, jotka ovat taulukossa \ref{tlk/mittayksiköt}
(s.~\pageref{tlk/mittayksiköt}). Kaksipuolisessa asettelussa
(\koodi{twoside=\katk true}) sivut muodostavat aukeaman eli on erikseen
vasemmanpuoleinen ja oikeanpuoleinen sivu. Tällöin vasen ja oikea
marginaali vuorottelevat, ja niitä on havainnollisempaa kutsua sisä- ja
ulkomarginaaliksi. Marginaalien määrittelyssä voi käyttää valitsimen
\koodi{left} sijasta valitsinta \koodi{inner}, ja vastaavasti
\koodi{right}\-/ valitsin voidaan korvata \koodi{outer}\-/ valitsimella.
Teknisesti näillä ei ole mitään eroa.

\leijutlk{
  \providecommand{\rivi}{}
  \renewcommand{\rivi}[3]{\koodi{#1} & \koodi{#2} & #3 \\}
  \begin{tabular}{lll}
    \toprule
    \multicolumn{2}{l}{\ots{Valitsin}} & \ots{Merkitys} \\
    \midrule
    \rivi{left}{inner}{vasen marginaali tai sisämarginaali}
    \rivi{right}{outer}{oikea marginaali tai ulkomarginaali}
    \rivi{top}{}{ylämarginaali}
    \rivi{bottom}{}{alamarginaali}
    \bottomrule
  \end{tabular}
}{
  \caption{\paketti{geometry}\-/paketin valitsimia sivun marginaalien
    määrittelemiseen}
  \label{tlk/sivun-marginaalit}
}

Sivun tekstialueen koon ja marginaalit voi määrittää myös
suhteellisesti. Voisi esimerkiksi määrittää, että leveyssuunnassa
tekstialue täyttää 0,7\=/kertaisesti (70\,\%) sivun leveyden ja loput
jää marginaaleille. Marginaalien keskinäiset suhteetkin voi ilmaista
suhdelukuna: esimerkiksi vasemman (sisä) ja oikean (ulko) marginaalin
suhde voisi olla 2:3. Suhteellisessa tavassa ei tarvitse ottaa kantaa
sivun kokoon eikä muihinkaan varsinaisiin mittoihin, vaan samat
tekstialueen ja marginaalien suhteet säilyvät, vaikka sivukokoa
muuttaisikin. Suhteellisia mittoja koskevia valitsimia on koottu
taulukkoon \ref{tlk/sivun-marginaalit-suhd}.

\leijutlk{
  \providecommand{\rivi}{}
  \renewcommand{\rivi}[2]{\koodi{#1} & #2 \\}
  \begin{tabular}{ll}
    \toprule
    \ots{Valitsin} & \ots{Merkitys} \\
    \midrule
    \rivi{hscale}{tekstialueen osuus sivun leveydestä}
    \rivi{vscale}{tekstialueen osuus sivun korkeudesta}
    \rivi{hmarginratio}{vasemman (sisä) ja oikean (ulko) marginaalin suhde}
    \rivi{vmarginratio}{ylä- ja alamarginaalin suhde}
    \bottomrule
  \end{tabular}
}{
  \caption{\paketti{geometry}\-/paketin valitsimia sivun tekstitilan ja
    marginaalien suhteiden määrittämiseen. Osuudet (\koodi{*scale})
    ilmaistaan prosenttikertoimella (esim. \koodi{0.7}). Suhteet
    (\koodi{*marginratio}) ilmaistaan suhdelukuna (esim. \koodi{2:3})}
  \label{tlk/sivun-marginaalit-suhd}
}

Klassisessa kirjatypografiassa, jossa teksti ladotaan yhdelle palstalle,
marginaalien suuruusjärjestys on suurimmasta pienimpään seuraavanlainen:
ala-, ulko-, ylä- ja sisämarginaali. Marginaalien suhdeluvut ovat
samassa järjestyksessä: 32, 28, 20, 17. \paketti{geometry}\-/paketin
valitsimilla tämä ilmaistaan seuraavasti:

\komentoi{geometry}
\begin{koodilohkosis}
\geometry{hmarginratio=17:28, vmarginratio=20:32}
\end{koodilohkosis}

\noindent
Klassisia suhteita ei nykyaikana yleensä noudateta kovin tarkasti, mutta
niistä kannattaa ymmärtää yleinen ajatus. Alamarginaalin pitäisi olla
hieman suurempi kuin ylämarginaali, koska muuten tekstialue tuntuu
pudonneen sivulla alas. Alamarginaalissa (alatunnisteessa) on yleensä
sivunumero. Kirjan sivuja katsotaan pareittain eli aukeamina, joten
sisämarginaaleja on kaksi vierekkäin. Siksi sisämarginaalit yksittäin
ajateltuna ovat pienemmät kuin ulkomarginaalit. Lisäksi ulkomarginaaleja
voidaan käyttää huomautusten kirjoittamiseen, mikä on melko yleinen
käytäntö tietokirjoissa.

Marginaalihuomautukset (luku \ref{luku/marginaalihuomautukset})
sijaitsevat oletuksena sivujen ulkomarginaalissa tai oikeanpuoleisessa
marginaalissa. Huomautuspalstan leveys asetetaan valitsimella
\koodi{marginparwidth}, ja palstan etäisyys sivun varsinaisesta
tekstialueesta määritellään valitsimella \koodi{marginparsep}. Kumpikin
valitsin tarvitsee argumentiksi mitan. Jos haluaa vaihtaa huomautukset
sivun vastakkaiseen marginaaliin, lisätään mukaan valitsin
\koodi{reversemarginpar}.

Perus Latex osaa latoa tekstin yhdelle tai kahdelle palstalle, ja
\paketti{geometry}\-/ paketin valitsimella \koodi{onecolumn} tai
\koodi{twocolumn} asetetaan, kumpi tila on oletuksena päällä.
Useampikin palsta on mahdollista laajennuspaketin avulla. Valitsimen
\koodi{columnsep} avulla asetetaan palstojen välinen etäisyys.
Käytännössä tämä valitsin on asettaa mitan \mitta{columnsep}, jota voi
muokata myös komennolla \komento{setlength}, kuten muitakin mittoja
(luku \ref{luku/mitat}). Tarkempaa tietoa palstoista on luvussa
\ref{luku/palstat}.

\subsection{Sivun mittoja}

\leijutlk{
  \providecommand{\rivi}{}
  \renewcommand{\rivi}[2]{\mitta{#1} & #2 \\}
  \begin{tabular}{ll}
    \toprule
    \ots{Mitta} & \ots {Merkitys} \\
    \midrule
    \rivi{paperwidth}{paperin eli sivun leveys}
    \rivi{paperheight}{paperin eli sivun korkeus}
    \rivi{textwidth}{tekstialueen leveys}
    \rivi{columnwidth}{nykyisen palstan leveys}
    \rivi{linewidth}{nykyisen rivin leveys}
    \rivi{textheight}{tekstialueen korkeus}
    \bottomrule
  \end{tabular}
}{
  \caption{Sivun mittoja}
  \label{tlk/sivun-mittoja}
}

Sivun kokoasetusten määrittämisen jälkeenkin voi dokumentissa olla
tarpeen hyödyntää joitakin sivun mittoja. Usein esimerkiksi halutaan
piirtää taulukko tai kuva, joka on sivun tekstialueen levyinen tai
siihen suhteutettu. Silloin on kätevää käyttää mittaa, jonka arvona on
juuri tekstialueen leveys. Tärkeimmät sivun mitat on koottu taulukkoon
\ref{tlk/sivun-mittoja}, mutta perusteellisemmin niitä käsitellään
\pakettictan{geometry}\-/ paketin ohjekirjassa.

Taulukossa \ref{tlk/sivun-mittoja} mainittu mitta \mitta{linewidth}
eroaa tekstialueen (\mitta{textwidth}) tai palstan (\mitta{columnwidth})
leveysmitasta esimerkiksi silloin, kun tekstikappaletta on sisennetty.
Kappaleiden ensimmäinen rivi voi olla sisennetty, ja sen vuoksi rivi ei
ole täysilevyinen. Muutkin sisennykset kuten lohkolainaukset tai
luetelmat vaikuttavat rivin leveysmittaan. Sisennysasiat liittyvät
tekstikappaleiden muotoiluun, jota käsitellään
luvussa~\ref{luku/kappale}. Luetelmia puolestaan käsitellään
luvussa~\ref{luku/luetelmat}.

\subsection{Leikkuuvarat}

Sivu voi olla ulkoisesti erikokoinen kuin sisäisesti. On siis
mahdollista asettaa sivu esimerkiksi \textsc{a4}\-/kokoiseksi ja
käsitellä marginaalit ja muut sivun mitat \textsc{a4}-koon mukaan, mutta
ulkoisesti tai fyysisesti sivu onkin osana suurempaa sivua tai
paperiarkkia. Tällaista tarvitaan ainakin silloin, kun halutaan
merkitä leikkuuvarat dokumentin painamista varten.

Painokoneen paperiarkkien leikkauskohta ei välttämättä osu täsmälleen
samaan kohtaan pdf\-/tiedoston sivun reunan kanssa, ja siksi
dokumentissa reunaan saakka yltävät kuvat asetetaan varmuuden vuoksi
hieman ylikokoiseksi. Sivun reunaan saakka aiotut kuvat siis yltävät
lähde-pdf:ssä pari millimetriä varsinaisen sivualueen ulkopuolelle eli
leikkuuvaran puolelle. Tällä varmistetaan, että painamisen jälkeen
leikatuissa paperiarkeissa kuva varmasti yltää reunaan saakka.

Jos dokumenttiin tarvitaan sivun ulkopuoliset leikkuuvarat, määritellään
dokumentin ulommaiset mitat edelleen samalla tavalla kuin
tavallisestikin eli esimerkiksi valitsimella \koodi{paper\-size} (luku
\ref{luku/sivun-mitat}). Sen sijaan sivun sisäiset mitat täytyy
määritellä toisella tavalla, käyttämällä valitsinta \koodi{lay\-out} tai
\koodi{layout\-size}.

\begin{esimerkki*}
  \komentoi{geometry}
\begin{koodilohko}
\geometry{
  papersize={220mm, 307mm},
  layout=a4paper,           % tai: layoutsize={210mm, 297mm}
  layoutoffset={5mm, 5mm},
  showcrop
}
\end{koodilohko}
  \caption{Sivun ulkoisten ja sisäisten mittojen sekä leikkuvaarojen
    määrittäminen}
  \label{esim/leikkuuvarat}
\end{esimerkki*}

Esimerkissä \ref{esim/leikkuuvarat} käytetään sisäisesti \textsc{a4}\-/
kokoista (210 × 297\,mm) sivua, mutta sivulle on määritetty joka
puolelle 5\,mm:n leikkuuvarat. Niinpä ulkoisesti sivu on 10\,mm leveämpi
ja korkeampi, eli ulkoiset mitat ovat 220 × 307\,mm. Valitsimella
\koodi{layout\-offset} asetetaan sisäisen sivun etäisyys ulkoisen sivun
vasemmasta ylänurkasta. Esimerkissä on mukana myös valitsin
\koodi{show\-crop}, joka merkitsee sisäisen ja ulkoisen sivun rajakohdan
eli leikkuuvaran rajan. Merkinnät näkyvät vain sivun nurkissa
leikkuuvaran puolella, joten ne eivät päädy lopulliseen
painotuotteeseen.

\subsection{Ylä- ja alatunnisteet}
\label{luku/ylä-ala-tunnisteet}

\paketti{geometry}\-/paketin asetuksiin kuuluu pari valitsinta, joilla
vaikutetaan ylä- ja alatunnisteiden mittoihin. Valitsimella \koodi{head}
ilmaistaan ylätunnisteen korkeus ja valitsimella \koodi{headsep} sen
etäisyys sivun tekstipalstasta. Alatunnisteen peruslinjan etäisyys
tekstipalstasta säädetään valitsimella \koodi{footskip}. Taulukkoon
\ref{tlk/ylä-ala-tunnistemitat} on koottu näiden valitsimien merkitys,
ja seuraavassa on niiden käyttämisestä esimerkki. Mukana on myös
valitsin \koodi{show\-frame}, joka piirtää sivulle apuviivoja. Se auttaa
sivun mittojen suunnittelussa.

\komentoi{geometry}
\begin{koodilohkosis}
\geometry{head=24bp, headsep=8bp, footskip=12mm, showframe}
\end{koodilohkosis}

\leijutlk{
  \begin{tabular}{ll}
    \toprule
    \ots{Valitsin} & \ots{Merkitys} \\
    \midrule
    \koodi{head} & ylätunnisteen korkeusmitta \\
    \koodi{headsep} & ylätunnisteen etäisyys tekstipalstasta \\
    \koodi{footskip} & alatunnisteen peruslinjan etäisyys tekstipalstasta \\
    \bottomrule
  \end{tabular}
}{
  \caption{\paketti{geometry}\-/paketin valitsimet ylä- ja
    alatunnisteiden mittojen asettamiseen}
  \label{tlk/ylä-ala-tunnistemitat}
}

\noindent
Latexin perusosat eivät sisällä kovin kummoista keinovalikoimaa ylä- ja
alatunnisteiden muokkaamiseen, mutta pari hyödyllistä sivutyyliä on
kuitenkin mukana. Ylä- ja alatunnisteet määräytyvät sivutyylin
perusteella, ja haluttu tyyli asetetaan voimaan komennolla
\komento{pagestyle}:

\komentoi{pagestyle}
\begin{koodilohkosis}
\pagestyle{plain}
\end{koodilohkosis}

\noindent
Edellä mainittu sivutyyli \koodi{plain} latoo alatunnisteeseen
sivunumeron. Se on yleensä oletustyyli. Sivunumero on peräisin
laskurista \laskuri{page} ja sen arvon tulostavasta komennosta
\komento{thepage} (luku \ref{luku/laskurit}). Toinen hyödyllinen tyyli
on \koodi{empty}, joka tarkoittaa tyhjää, eli ylä- eikä
alatunnisteeseen ei ladota mitään.

Yksittäiselle sivulle voi asettaa muusta dokumentista poikkeavan
sivutyylin komennolla \komento{thispagestyle}. Komento siis vaikuttaa
vain sillä hetkellä ladottavan sivun tyyliin, ja sen jälkeen palataan
taas voimassa olevaan tyyliin, joka on aiemmin määritelty komennolla
\komento{pagestyle}.

\komentoi{thispagestyle}
\begin{koodilohkosis}
\thispagestyle{empty}
\end{koodilohkosis}

\noindent
Sivutyyli \koodi{headings} latoo ylätunnisteeseen aukeaman
vasemmanpuoleisille sivuille esimerkiksi kirjan pääluvun nimen ja
oikeanpuoleisille sivuille meneillään olevan alaluvun nimen. Mainitut
lukujen nimet tulevat sivun sisäreunaan; ulkoreunaan ladotaan
sivunumero.

Teknisesti ja sisäisesti tämä on toteutettu siten, että pääluvun
aloittava otsikkokomento (esim. \komento{chapter}) automaattisesti
määrittelee joka kerta uudelleen komennon \komento{leftmark}, niin että
se sisältää pääluvun nimen. Sivuja ladottaessa Latex sitten latoo
ylätunnisteeseen sen, mitä \komento{leftmark}\-/komento sattuu
tulostamaan. Vastaavasti alaluvun otsikkokomento (esim.
\komento{section}) määrittelee uudelleen komennon \komento{rightmark},
niin että se sisältää alaluvun nimen. Tämän komennon tulostama teksti
ladotaan aukeaman oikeanpuoleisille sivuille.

Jos haluaa itse vaikuttaa ylätunnisteen tekstiin, voi käyttää sivutyyliä
\koodi{myheadings} sekä komentoa \komento{markboth}, jolla määritellään
aukeaman vasemmanpuoleisen ja oikeanpuoleisen sivun ylätunnisteen
teksti. Tämä komento on tarkoitettu suoritettavaksi päälukujen
yhteydessä. Komennolla \komento{markright} määritellään pelkästään
oikeanpuoleisen sivun teksti, ja komento on tarkoitettu suoritettavaksi
aina alalukujen yhteydessä.

\komentoi{markboth}
\komentoi{markright}
\begin{koodilohkosis}
\markboth{vasen}{oikea}  % määrittelee: \leftmark ja \rightmark
\markright{oikea}        % määrittelee: \rightmark
\end{koodilohkosis}

\noindent
Sivunumeroinnin tyyliin voi vaikuttaa esimerkiksi komennolla
\komento{pagenumbering}, jonka argumentiksi annetaan numerointityylin
nimi. Ne on koottu taulukkoon \ref{tlk/sivu-numerointityylit}.

\komentoi{pagenumbering}
\begin{koodilohkosis}
\pagenumbering{roman}
\end{koodilohkosis}

\noindent
Edellä mainittu komento määrittelee käytännössä uudelleen komennon
\komento{thepage}, joka on tarkoitettu juuri sivunumerolaskurin
latomiseen. Lisätietoa sivunumeroista ja muista laskureista on luvussa
\ref{luku/laskurit}.

\leijutlk{
  \providecommand{\rivi}{}
  \renewcommand{\rivi}[2]{\koodi{#1} & #2 \\}
  \begin{tabular}{ll}
    \toprule
    \ots{Tyyli} & \ots{Merkitys} \\
    \midrule
    \rivi{arabic}{arabialaiset luvut: 1, 2, 3\dots}
    \rivi{roman}{roomalaiset luvut: i, ii, iii\dots}
    \rivi{Roman}{roomalaiset luvut: I, II, III\dots}
    \rivi{alph}{kirjaimet: a, b, c\dots\ (vain 1--26)}
    \rivi{Alph}{kirjaimet: A, B, C\dots\ (vain 1--26)}
    \bottomrule
  \end{tabular}
}{
  \caption{Sivunumerointityylit \komento{pagenumbering}\-/ komennon
    argumentiksi}
  \label{tlk/sivu-numerointityylit}
}

Monipuolisemmin ylä- ja alatunnisteita voi muokata paketin
\pakettictan{fancyhdr} toimintojen avulla. Silloin sivutyylinä voi olla
myös \koodi{fancy}:

\komentoi{usepackage}
\komentoi{pagestyle}
\begin{koodilohkosis}
\usepackage{fancyhdr}
\pagestyle{fancy}
\end{koodilohkosis}

\noindent
Kun käytössä on sivutyyli \koodi{fancy}, voi ylä- ja alatunnisteiden
sisällön asettaa vapaasti komennolla \komento{fancyhf}. Komennon
argumenttien merkitys on seuraavanlainen:

\komentoi{fancyhf}
\begin{koodilohkosis}
\fancyhf[paikka]{sisältö}
\end{koodilohkosis}

\noindent
Valinnainen argumentti \koodi{paikka} kertoo, mihin paikkaan tai
paikkoihin \koodi{sisältö} sijoitetaan. Vaihtoehtoina on ylä- tai
alatunniste, pariton tai parillinen sivu, sivun vasen reuna, keskiosa
tai oikea reuna. Nämä vaihtoehdot ja niitä vastaavat valitsimet on
koottu taulukkoon \ref{tlk/fancyhf-paikat}. Saman komennon avulla voi
määrittää useitakin paikkoja, kun ne erottaa pilkulla, esimerkiksi
seuraavalla tavalla:

\leijutlk{
  \providecommand{\rivi}{}
  \renewcommand{\rivi}[2]{\koodi{#1} & #2 \\}
  \begin{tabular}{cl}
    \toprule
    \ots{Valitsin} & \ots{Merkitys} \\
    \midrule
    \rivi{H}{ylätunniste (header)}
    \rivi{F}{alatunniste (footer)}
    \rivi{E}{parillinen sivu, vasen (even)}
    \rivi{O}{pariton sivu, oikea (odd)}
    \rivi{L}{sivun vasen reuna (left)}
    \rivi{C}{sivun keskelle (center)}
    \rivi{R}{sivun oikea reuna (right)}
    \bottomrule
  \end{tabular}
}{
  \caption{\komento{fancyhf}\-/ komennon valitsimia ylä- tai
    alatunnisteen paikan määrittämiseen}
  \label{tlk/fancyhf-paikat}
}

\komentoi{fancyhf}
\begin{koodilohkosis}
\fancyhf{}  % Tyhjennetään ylä- ja alatunnisteet.
\fancyhf[HEL,HOR]{\thepage}
\end{koodilohkosis}

\noindent
Edellä mainittu komento sijoittaa sivunumeron (\komento{thepage})
ylätunnisteeseen (\koodi{H}) parillisten sivujen (\koodi{E}) vasempaan
reunaan (\koodi{L}) ja parittomien sivujen (\koodi{O}) oikeaan reunaan
(\koodi{R}). Käytännössä siis aukeaman ulkoreunoihin ladotaan
sivunumerot. Seuraava esimerkki sijoittaa ajatusviivoilla (\==)
reunustetun sivunumeron kaikille sivuille alatunnisteeseen (\koodi{F})
sivun keskelle (\koodi{C}):

\komentoi{fancyhf}
\komentoi{thepage}
\begin{koodilohkosis}
\fancyhf[FC]{-- \thepage\ --}
\end{koodilohkosis}

\noindent
Paketin \paketti{fancyhdr} avulla voi myös määritellä Latexin
sivutyylejä toisenlaiseksi tai luoda kokonaan omia sivutyylejä. Nämä
tehdään komennolla \komento{fancypagestyle}, jonka ensimmäinen
argumentti on sivutyylin nimi ja toinen argumentti on sivutyylin
määritelmä. Määritelmä sisältää tarvittavat \komento{fancyhf}\-/
komennot, joilla ylä- ja alatunnisteet määritellään.

Seuraava esimerkki tekee Latexin \koodi{plain}\-/ sivutyylistä saman
kuin \koodi{fancy}\-/tyylin, eli sekin noudattaa \komento{fancyhf}\-/
komennolla määriteltyjä ylä- ja alatunnisteita:

\komentoi{fancypagestyle}
\begin{koodilohkosis}
\fancypagestyle{plain}{}
\end{koodilohkosis}

\noindent
Seuraava esimerkki määrittelee kokonaan oman sivutyylin:

\komentoi{fancypagestyle}
\komentoi{fancyhf}
\begin{koodilohkosis}
\fancypagestyle{omatyyli}{
  \fancyhf{}
  \fancyhf[FEL,FOR]{\thepage}
}
\end{koodilohkosis}

\noindent
Ylä- ja alatunniste voidaan erottaa tekstipalstasta vaakasuuntaisella
viivalla, jonka leveyttä on mahdollista muuttaa määrittelemällä
uudelleen komennot \komento{headrulewidth} ja \komento{footrulewidth}.
Komennon määritelmäksi kirjoitetaan Texin mitta. Seuraavassa esimerkissä
asetetaan yhden typografisen pisteen (1\,bp) levyiset viivat. Mitan
voisi asettaa myös nollaksi (0\,bp), jolloin erotinviiva katoaa kokonaan
näkyvistä.

\komentoi{renewcommand}
\komentoi{headrulewidth}
\komentoi{footrulewidth}
\begin{koodilohkosis}
\renewcommand{\headrulewidth}{1bp} % ylätunnisteen erotinviiva
\renewcommand{\footrulewidth}{1bp} % alatunnisteen erotinviiva
\end{koodilohkosis}

\noindent
Mikäli haluaa omiin ylätunnisteisiin esimerkiksi päälukujen ja
alalukujen nimiä, täytyy tunnisteisiin sisällyttää aiemmin kuvatut
\komento{leftmark}- ja \komento{rightmark}\-/komennot. Ensin mainittu
sisältää pääluvun nimen ja jälkimmäinen alaluvun nimen.

\komentoi{fancyhf}
\komentoi{leftmark}
\komentoi{rightmark}
\begin{koodilohkosis}
\fancyhf[HEL]{\leftmark}
\fancyhf[HOR]{\rightmark}
\end{koodilohkosis}

\noindent
Latex latoo pää- ja alalukujen nimet oletuksena versaalikirjaimilla eli
isoilla kirjaimilla. Jos ne haluaa johonkin toiseen muotoon, täytyy itse
määritellä uudelleen komennot \komento{chaptermark},
\komento{sectionmark} tai \komento{subsectionmark} (vain kaksi näistä)
ja käyttää määritelmässä komentoja \komento{markboth} ja
\komento{markright}. Esimerkkiin \ref{esim/fancyhdr-koko} on koottu
varsin kokonaisvaltainen koodi omien ylä- ja alatunnisteiden
toteutukseen.

\begin{esimerkki*}
  \komentoi{documentclass}
  \luokkai{book}
  \komentoi{usepackage}
  \pakettii{fancyhdr}
  \komentoi{fancypagestyle}
  \komentoi{fancyhf}
  \komentoi{renewcommand}
  \komentoi{headrulewidth}
  \komentoi{footrulewidth}
  \komentoi{leftmark}
  \komentoi{rightmark}
  \komentoi{thepage}
  \komentoi{pagestyle}
  \komentoi{chaptermark}
  \komentoi{markboth}
  \komentoi{chaptername}
  \komentoi{thechapter}
  \komentoi{sectionmark}
  \komentoi{thesection}

\begin{koodilohko}
\documentclass{book}
\usepackage{fancyhdr}

% Päälukujen (\chapter) aloitussivu käyttää plain-sivutyyliä.
% Tässä määritellään se uudestaan.
\fancypagestyle{plain}{
  \fancyhf{}
  \fancyhf[FC]{-- \thepage\ --}
  \renewcommand{\headrulewidth}{0bp}
  \renewcommand{\footrulewidth}{0bp}
}

% fancy-sivutyylin asetukset:
\fancyhf{}
\fancyhf[HEL]{\leftmark}
\fancyhf[HOR]{\rightmark}
\fancyhf[FC]{-- \thepage\ --}
\renewcommand{\headrulewidth}{1bp}
\renewcommand{\footrulewidth}{0bp}

\begin{document}

\pagestyle{fancy}

% Päälukujen (\chapter) yhteydessä komento \markboth{…}{…} määrittää
% sekä \leftmark- että \rightmark-komennot. Tässä jälkimmäinen
% määritellään tyhjäksi.
\renewcommand{\chaptermark}[1]{%
  \markboth{\chaptername\ \thechapter: #1}{}}

% Alalukujen (\section) yhteydessä komento \markright{…} määrittää
% vain \rightmark-komennon.
\renewcommand{\sectionmark}[1]{\markright{\thesection\ #1}}
\end{koodilohko}
  \caption{Omien ylä- ja alatunnisteiden toteuttaminen}
  \label{esim/fancyhdr-koko}
\end{esimerkki*}

\section{Pdf-tiedosto}
\label{luku/pdf-asetukset}

Pdf-tiedostot voivat sisältää metatietoja kuten dokumentin nimen,
aiheen, tekijän ja päiväyksen. Tiedostoille voi määrittää myös erilaisia
asetuksia kuten ristiviitteiden ja linkkien ulkoasun tai sisäisen
sisällysluettelon ominaisuuksia. Pdf\-/tiedoston asetukset toteutetaan
\pakettictan{hyperref}\-/ paketin avulla. Paketti neuvotaan lataamaan
muiden pakettien jälkeen, koska se lisää ominaisuuksia muihin
komentoihin. Paketin voisi ladata esimerkiksi seuraavalla tavalla:

\komentoi{usepackage}
\pakettii{hyperref}
\begin{koodilohkosis}
% Muiden \usepackage-komentojen jälkeen.
\usepackage[unicode]{hyperref}
\end{koodilohkosis}

\noindent
Esimerkissä käytetty valitsin \koodi{unicode} aiheuttaa sen, että pdf\-/
tiedoston sisäisissä merkkijonoissa käytetään Unicode\-/ merkistöä ja
sen \textsc{utf\=/8}\-/ koodausta. Ilman tätä valitsinta pdf\-/
tiedoston sisäisen sisällysluettelon merkistö ei välttämättä näy oikein.

\paketti{hyperref}\-/ paketin asetuksia voi määritellä valitsimien
avulla \komento{usepackage}\-/ komennon yhteydessä mutta myös
erillisellä \komento{hypersetup}\-/ komennolla. Komennolle annetaan
yksi argumentti, joka sisältää pilkuilla erotettuna erilaisia valitsimia
ja niiden arvoja.

\begin{esimerkki*}
  \komentoi{hypersetup}

\begin{koodilohko}
\hypersetup{
  hidelinks, bookmarksopen, bookmarksnumbered,
  pdfinfo={
    Title={Laatikollinen lateksia},
    Subject={Opas lateksiin ja liisteriin},
    Author={Lauri Liisteri}
  }
}
\end{koodilohko}
  \caption{\komento{hypersetup}\-/ komennolla asetetaan
    \paketti{hyperref}\-/ paketin asetuksia, esimerkiksi pdf:n
    metatietoja}
  \label{esim/hypersetup}
\end{esimerkki*}

Esimerkki \ref{esim/hypersetup} havainnollistaa \komento{hypersetup}\-/
komennon käyttöä. Komennon argumentissa valitsin \koodi{hidelinks} saa
aikaan sen, että pdf:ssä olevia linkkejä ja ristiviitteitä ei merkitä
millään tavalla. Oletuksena linkit kehystetään eri väreillä riippuen
linkin tyypistä. Valitsin \koodi{bookmarksopen} näyttää pdf:n
sisällysluettelon kokonaan avattuna. Oletuksena alaluvut on
piilotettuna, ja ne joutuu avaamaan napsauttamalla hiirellä
ylemmäntasoisen otsikon avauspainiketta. Valitsin
\koodi{bookmarksnumbered} näyttää pdf:n sisällysluettelossa lukujen
numeroinnin (kuten 1.1, 1.2, 2.1 tms.). Valitsimella \koodi{pdfinfo}
määritellään pdf:n metatietoja kuten otsikko, aihe ja tekijä mutta
omiakin metatietoja voi lisätä. Paljon muitakin asetuksia on olemassa,
ja niistä voi lukea lisää \paketti{hyperref}\-/ paketin ohjekirjasta.

Pdf\-/ tiedoston sisäiseen sisällysluetteloon tulevat automaattisesti
mukaan samat otsikot (luku \ref{luku/otsikot}) kuin ladottavaan
sisällysluetteloonkin (luku \ref{luku/sisällysluettelo}). Pdf\-/
tiedoston luetteloon voi kuitenkin itse lisäillä omia otsikoita, jotka
eivät näy missään muualla. Se tehdään \paketti{hyperref}\-/ paketin
komennolla \komento{pdfbookmark}:

\komentoi{pdfbookmark}
\begin{koodilohkosis}
\pdfbookmark[taso]{Teksti}{tunniste}
\end{koodilohkosis}

\noindent
Komennon valinnainen argumentti \koodi{taso} on kokonaisluku, joka
tarkoittaa otsikon tasoa. Se on samankaltainen tasonumero kuin
otsikkotasojen taulukossa \ref{tlk/otsikkotasot}
(s.~\pageref{tlk/otsikkotasot}). Argumentti \koodi{Teksti} on pdf:n
sisällysluettelomerkinnän teksti, ja \koodi{tunniste} on mikä tahansa
yksilöllinen tekstimuotoinen tunniste kyseiselle luettelomerkinnälle.
Tunniste ei näy missään, mutta pdf tarvitsee sisäiseen toimintaansa
jonkin yksilöllisen tunnisteen.

\section{Fontit}
\label{luku/kirjaintyypit}

Fontit ja niiden asettaminen on Latexissa melko monimutkainen
kokonaisuus, koska fonteilla on paljon ominaisuuksia ja niihin
vaikutetaan monilla eri asetuksilla ja abstraktiotasoilla. Aika monta
asiaa pitää ymmärtää, jotta voi tehokkaasti työskennellä Latexin
fonttien kanssa.

Fontti jo itsessään on moniselitteinen käsite, joka vaatii typografiassa
usein täsmentäviä ilmauksia. Sana \emph{fontti} voi tarkoittaa
kokonaista kirjainperhettä eli yhteensopivien kirjainleikkausten
muodostamaa kokonaisuutta. Samaan kirjainperheeseen kuuluu yleensä
ainakin neljä eri leikkausta: tavallinen, \textit{kursiivi},
\textbf{lihavoitu} ja \textbf{\textit{lihavoitu kursiivi}}. Joihinkin
perheisiin kuuluu leikkauksia paljon enemmänkin, kuten useita eri
vahvuuksia. Joissakin puheissa sana \emph{fontti} tarkoittaa vain yhtä
kirjainleikkausta, ja silloin koko perheeseen viitataan ehkä sanalla
fonttiperhe.

Tässä oppaassa käytetään \emph{fontti}\-/sanaa yleisnimityksenä Latexin
kirjaintyyppeihin liittyville asetuksille. Se tarkoittaa kirjainperhettä
tai siihen kuuluvaa yksittäistä leikkausta sekä asetuksia, jotka
liittyvät niihin. Silloin kun merkitystä pitää täsmentää, käytetään
suomenkielisiä nimiä kirjainperhe ja kirjainleikkaus. Sen sijaan sana
\emph{kirjasin} on jätetty kokonaan pois. Se tarkoittaa vanhassa
metalliladonnassa ja mekaanisissa kirjoituskoneissa metallisen
ladontakappaleen eli kirjakkeen päähän valettua kirjaimen tai muun
merkin kohokuviota, joka painaa mustejäljen paperille.

Kuten Latexissa yleensäkin myös fonttien kanssa kannattaa käyttää
korkean abstraktiotason komentoja, jotka piilottavat yksityiskohdat ja
teknisen toteutuksen. Latexin fonttitoiminnot on suunniteltu juuri
siihen: matalan tason fonttiasetukset määritellään mieluiten vain kerran
dokumentin esittelyosassa, ja sen jälkeen käytetään pelkästään korkean
tason komentoja.

Latexin fonttitekniikka rakentuu eri\-/ikäisistä kerroksista ja
tekniikoista. Fontteja on aikoinaan tehty
\englanti{Metafont}\avctan{metafont}\-/ nimisellä kielellä, jolla
kuvataan merkkien muodot. \englanti{Metafont} on myös tietokoneohjelma,
joka tuottaa kuvauskielen perusteella bittikarttafontteja eli
pikseleistä koostuvia fontteja. On käytetty myös kehittyneempää
\englanti{Metapost}\avctan{metapost}\-/ kuvauskieltä ja \=/ohjelmaa,
joilla on tuotettu vektorigrafiikkafontteja \textsc{eps}- eli
\englanti{Encapsulated Post Script} \=/muodossa ja muutettu niitä
edelleen \englanti{Post Script Type~1} \=/fonteiksi. Myöhemmin mukaan
ovat tulleet nykyaikaiset \englanti{True Type}- ja \englanti{Open Type}
\=/fontit, ja niihin tämä opas keskittyy.

\subsection{Fonttien määrittäminen}
\label{luku/fontin-valinta}

Latexin fonttien perustoiminnot rakentuvat kolmen erityyppisen
kirjainperheen varaan:

\begin{nluetelma}
\item peruskirjainperhe eli dokumentin pääasiallinen kirjainperhe, joka
  on kirjatypografiassa usein antiikva eli pääteviivallinen
  (\englanti{serif, roman})%
  \footnote{Antiikva (lat. \emph{antiquus} 'vanha') perustuu antiikin
    Roomassa käytettyihin kirjainmuotoihin. Niissä on pääteviivat, ja
    viivojen vahvuus vaihtelee.}
\item groteski eli pääteviivaton (\englanti{sans serif, gothic})%
  \footnote{Groteskiin (ransk. \emph{grotesque} 'kummallinen') kuuluu
    pääteviivojen puuttumisen lisäksi lähes tasavahvuiset kirjainten
    viivat. Tämän oppaan groteskifontissa on kuitenkin selvästi
    antiikvamaiset kaksivahvuiset viivat, joten se on eräänlainen
    antiikvan ja groteskin välimuoto.}
\item tasalevyinen kirjoituskoneen kaltainen perhe
  (\englanti{type\-writer, mono\-spaced, tele\-type}).
\end{nluetelma}

\noindent
Kuvassa \ref{kuva/kirjainperhetyypit} ovat tässä oppaassa käytetyt kolme
eri kirjainperhettä. Leipätekstissä käytetään antiikvaa, otsikoissa ja
kuvateksteissä groteskia ja koodiesimerkeissä tasalevyistä.
Kirjoituskoneen kaltainen tasalevyinen kirjainperhe on tässä tapauksessa
tyypiltään antiikva eli pääteviivallinen, mutta se voisi olla muutakin.
Tasalevyisyys on sen kirjainperheen tärkein määrittävä tekijä Latexin
asetusten näkökulmasta.

\leijukuva{
  {\rmfamily\addfontfeatures{ScaleAgain=5}Amf}
  \hfill
  {\sffamily\addfontfeatures{Scale=5}Amf}
  \hfill
  {\ttfamily\addfontfeatures{FakeStretch=1, Scale=4.5}Amf}
}{
  \caption{Vasemmalla pääteviivallinen, keskellä pääteviivaton ja
    oikealla tasalevyinen pääteviivallinen kirjainperhe}
  \label{kuva/kirjainperhetyypit}
}

Joidenkin fonttien käyttöönottoon on tehty oma pakettinsa, joten
sellaiset fontit voi ladata dokumentin esittelyosassa komennolla
\komento{usepackage}. Fonttikohtaisia paketteja on olemassa
varsinkin vanhalle fonttitekniikalle (\englanti{Metafont, Post Script
  Type~1}) mutta myös matematiikkatilan fonttiasetuksille (luku
\ref{luku/matematiikka-fontit}) ja joillekin
kir\-jain\-perhe\-koko\-nai\-suuk\-sille.

Latexissa pisimmälle ''tuotteistettu'' kokonaisuus taitaa olla
Libertinus\-/ kirjainperhe, joka sisältää antiikvan, groteskin ja
tasalevyisen kirjainperheen sekä matematiikkatilan symboleita.
Libertinus\-/ kirjainperheet saa käyttöön lataamalla paketin
\pakettictan{libertinus}:

\komentoi{usepackage}
\pakettii{libertinus}
\begin{koodilohkosis}
\usepackage{libertinus}
\end{koodilohkosis}

\noindent
Valmiita paketteja on kuitenkin vain harvoille fonteille, ja käytännössä
lähes aina \englanti{True Type}- ja \englanti{Open Type} \=/muodossa
olevat fontit otetaan käyttöön \pakettictan{fontspec}\-/ paketin
komennoilla\footnote{Vaihtoehtoisesti fonttien asettamiseen voi käyttää
  \paketti{babel}\-/ paketin komentoja (luku \ref{luku/babel}).}
seuraavan esimerkin mukaisesti:

\komentoi{setmainfont}
\komentoi{setsansfont}
\komentoi{setmonofont}
\begin{koodilohkosis}
\setmainfont{TeX Gyre Termes}[Scale=1]
\setsansfont{TeX Gyre Heros} [Scale=MatchLowercase]
\setmonofont{TeX Gyre Cursor}[Scale=MatchLowercase]
\end{koodilohkosis}

\noindent
Edellisessä esimerkissä \englanti{TeX Gyre Termes, Heros} ja
\englanti{Cursor} ovat kirjainperheiden nimiä. Fonttitiedostojen
tulee olla asennettuna käyttöjärjestelmän normaalien käytäntöjen
mukaisesti tai Latex\-/jakelun käytäntöjen mukaisesti.

Kirjainperheiden käyttöönoton yhteydessä voi määritellä lukuisia
asetuksia kuten ligatuureja, gemenanumeroita, optisia kokoja ja muita
fontin ominaisuuksia. Edellisessä esimerkissä käytetään vain
\koodi{Scale}\-/ valitsinta, jolla fontin voi skaalata haluttuun kokoon.

Peruskirjainperheen (\komento{setmainfont}) skaalaukseksi asetetaan
esimerkissä \koodi{Scale=1}, eli sille ei tehdä mitään, ja koko
valitsimen voisi jättää pois. Sen sijaan kahdella muulla
kirjainperheellä (\komento{setsansfont}, \komento{setmonofont})
käytetään kerroinasetusta \koodi{MatchLowercase}, joka skaalaa fontin
siten, että gemenakirjaimet eli pienet kirjaimet ovat yhtä korkeita kuin
peruskirjainperheessä. Mikäli skaalausasetus \koodi{MatchLowercase} ei
tuota ihan toivottua tulosta, voi kirjainperheen skaalausta hienosäätää
vielä \koodi{Scale\-Again}\-/ valitsimella seuraavalla tavalla:

\komentoi{setmonofont}
\begin{koodilohkosis}
\setmonofont{TeX Gyre Cursor}
[Scale=MatchLowercase, ScaleAgain=.97]
\end{koodilohkosis}

\noindent
Kirjainperheen määrittelyn yhteydessä ei yleensä tarvitse antaa kuin
kirjainperheen nimi, sillä \paketti{fontspec}\-/paketti ja kääntäjät
osaavat automaattisesti ladata perheeseen sisältyviä eri
fonttitiedostoja kuten pystyasentoisen leikkauksen, kursiivin ja
lihavoinnin. On kuitenkin mahdollista määritellä kirjainperheeseen
kuuluvia leikkausten nimiä tai fonttitiedostoja erikseen. Tällainen on
tarpeen esimerkiksi silloin, kun kirjainperhe sisältää useita eri
vahvuuksia ja halutaan itse määritellä, mikä niistä tulee
perusvahvuudeksi ja mikä lihavoiduksi.

Esimerkki \ref{esim/fontit-leik-omin} selventää, kuinka kirjainperheen
eri leikkausten nimet tai fonttitiedostot määritellään. Kullekin
leikkaukselle voi määrittää myös omat asetuksensa \koodi{Fea\-tures}\-/
sanaan päättyvällä valitsimella. Samaa asiaa havainnollistetaan myös
konkreettisemmin esimerkissä \ref{esim/fontit-leik-omin-käyt}.
Leikkauksen nimessä voi käyttää tähteä (\koodi{*}), joka korvautuu
kirjainperheen nimellä (\englanti{Macklin Text}).

\begin{esimerkki*}
  \komentoi{setmainfont}
\begin{koodilohko}
\setmainfont{…}[
  UprightFont={…},     UprightFeatures={…},
  ItalicFont={…},      ItalicFeatures={…},
  BoldFont={…},        BoldFeatures={…},
  BoldItalicFont={…},  BoldItalicFeatures={…},
  SlantedFont={…},     SlantedFeatures={…},
  BoldSlantedFont={…}, BoldSlantedFeatures={…},
  SmallCapsFont={…},   SmallCapsFeatures={…}]
\end{koodilohko}
  \caption{Kirjainperheeseen sisältyvien leikkausten nimien ja
    kirjainleikkauskohtaisten ominaisuuksien määrittely}
  \label{esim/fontit-leik-omin}
\end{esimerkki*}

\begin{esimerkki*}
  \komentoi{setmainfont}
\begin{koodilohko}
\setmainfont{Macklin Text}[
  UprightFont    = {* Light},
  ItalicFont     = {* Light Italic},
  BoldFont       = {* Medium},
  BoldItalicFont = {* Medium Italic}]
\end{koodilohko}
  \caption{Eri leikkausten nimien määrittely \englanti{Macklin Text}
    \=/kirjainperheelle. Leikkauksen nimessä tähti (\koodi{*}) korvautuu
    automaattisesti koko perheen nimellä}
  \label{esim/fontit-leik-omin-käyt}
\end{esimerkki*}

Jos edellä kuvatut kolme kirjainperhettä (\komento{setmainfont},
\komento{setsansfont} ja \komento{setmonofont}) eivät riitä, on
\paketti{fontspec}\-/paketissa komennot lisäperheiden ja \=/leikkausten
määrittämiseen. Uusi perhe määritellään seu\-raa\-vasti:

\komentoi{newfontfamily}
\begin{koodilohkosis}
\newfontfamily{\hienoperhe}{TeX Gyre Schola}[…]
\end{koodilohkosis}

\noindent
Komento \komento{newfontfamily} toimii samalla tavalla kuin aiemmin
esitellyt \komento{setmainfont} ym. komennot, mutta lisäksi
ensimmäisellä argumentilla nimetään komento, jolla kirjainperhe otetaan
käyttöön. Edellisessä esimerkissä luodaan komento
\komentox{hieno\-perhe}, joka kytkee päälle \englanti{TeX Gyre Schola}
\=/nimisen kirjainperheen.

Jos ei tarvita kokonaista perhettä vaan yksi leikkaus riittää, käytetään
komentoa \komento{newfontface}. Seuraavassa esimerkissä määriteltävä
komento \komentox{hieno\-leikkaus} ottaa käyttöön lihavoidun (bold)
kirjainleikkauksen perheestä \englanti{TeX Gyre Schola}.

\komentoi{newfontface}
\begin{koodilohkosis}
\newfontface{\hienoleikkaus}{TeX Gyre Schola Bold}[…]
\end{koodilohkosis}

\subsection{Fontin koko ja rivikorkeus}

Fonttien koot on tapana valita ja ilmaista typografisen pistemitan
avulla. Esimerkiksi 10--12 pistettä on tyypillinen leipätekstin
oletuskoko tekstinkäsittelyohjelmissa. Piste on typografiassa
mittayksikkö, jonka pituus on määritelty eri tavoin eri aikoina ja eri
kulttuureissa.

Myös Latexissa fonttien koot voi määritellä pistemittojen avulla. Niitä
ja muitakin Latexin mittayksiköitä käsitellään tarkemmin luvussa
\ref{luku/mitat}. Fonteissa oletusmittayksikkönä on vanha pica\-/
järjestelmän piste, jonka pituus on noin 0,3515 millimetriä. Sen lyhenne
Latexissa on~pt. Tämän oppaan esimerkeissä käytetään kuitenkin
\englanti{Post Script} \=/standardin mukaista, julkaisuohjelmiin
vakiintunutta uudempaa pica\-/ pistettä, joka on hieman edellistä
pidempi: noin 0,3528 millimetriä. Latexissa sen lyhenne on~bp. Ero
näiden kahden pistemitan välillä on hyvin pieni, tavallisilla
fonttiko'oilla käytännössä merkityksetön.

Kirjainleikkauksen koko mitataan merkistön ylimmän ja alimman kohdan
välillä, esimerkiksi k\=/kirjaimen ylimmän pisteen ja y\=/kirjaimen
alimman pisteen välillä. Lisäksi mittaan luetaan mukaan merkistön ylä-
ja alapuolella oleva pieni tyhjä tila, jonka fontin suunnittelija on
määritellyt.

Matalalla tasolla fonttien kokoon vaikuttaa Latexissa eräs yllättävä
asia. Nimittäin dokumenttiluokalle (luku \ref{luku/dokumenttiluokat})
voi antaa valitsimen, jolla fontin koko asetetaan. Vaihtoehtoja on
Latexin normaaleissa dokumenttiluokissa vain kolme: \koodi{10pt}
(oletus), \koodi{11pt} ja \koodi{12pt}. Dokumenttiluokan kokoasetus
vaikuttaa myös sivun marginaaleihin, koska Latex pyrkii pitämään rivin
merkkimäärän lukijalle sopivana: yhdelle riville ei kannata latoa ihan
mahdottomasti merkkejä, koska kovin pitkän rivin seuraaminen rasittaa
lukijaa ja vaatii enemmän keskittymistä.

Fontin koon määrittäminen dokumenttiluokan valitsimella ehkä kuuluu jo
vähän menneisyyteen, mutta voi sitä edelleen käyttää, jos se riittää ja
sillä saa halutun lopputuloksen. Yleensä lienee järkevää jättää
dokumenttiluokan fonttiasetus oletukseksi (\koodi{10pt}) ja käyttää koon
asettamiseen luvuissa \ref{luku/fontti-suhteellinen} ja
\ref{luku/fontti-absoluuttinen} kerrottuja tapoja. Sivun marginaalien ja
muiden mittojen määrittämiseen on ohjeita luvussa
\ref{luku/sivuasetukset}.

Fonttiasetuksiin kuuluu fontin koon lisäksi toinenkin mitta: rivikorkeus
(\mitta{baselineskip}). Se on peräkkäisten rivien peruslinjojen välinen
etäisyys. Fontin koko ja rivikorkeus määritellään samanaikaisesti, koska
ne ovat saman \komento{fontsize}\-/ komennon argumentteja. Esimerkki:

\komentoi{fontsize}
\komentoi{selectfont}
\begin{koodilohkosis}
\fontsize{10bp}{12bp} \selectfont
\end{koodilohkosis}

\noindent
Ensimmäinen argumentti on fontin kokomitta ja toinen on rivikorkeus.
Mittayksiköt voivat olla mitä tahansa Texin mittoja, ja oletuksena
käytetään pt\-/pistemittaa, jos yksikköä ei ole mainittu. Komento
\komento{selectfont} on mukana, koska vasta sen myötä matalan tason
fonttikomennot tulevat voimaan. Korkean tason fonttikomennot (luku
\ref{luku/fontit-korkea}) suorittavat sen automaattisesti.

Rivikorkeus on vähintään sama kuin fontin koko, mutta yleensä se
asetetaan pari pistettä suuremmaksi, jotta rivit eivät olisi liian
lähellä toisiaan. Esimerkissä \ref{esim/rivikorkeus} on kaksi erilaista
\komento{fontsize}\-/komentoa ja ladottu lopputulos.

\begin{esimerkki*}
  \komentoi{fontsize}
  \komentoi{selectfont}

\begin{koodilohko}
\fontsize{8bp}{11bp}\selectfont Tässä on pienehkö leipätekstin
fonttikoko ja suhteellisen suuri rivikorkeus. Pitkät rivit vaativat
suuremman rivikorkeuden kuin lyhyet rivit.

\fontsize{16bp}{17bp}\selectfont Tässä on melko suuri fontti ja
suhteellisen pieni rivikorkeus. Suuri fontti ja lyhyet rivit eivät
tarvitse kovin suurta rivikorkeutta.
\end{koodilohko}
  \begin{tulos}
    \fontsize{8bp}{11bp}\selectfont Tässä on pienehkö leipätekstin
    fonttikoko ja suhteellisen suuri rivikorkeus. Pitkät rivit vaativat
    suuremman rivikorkeuden kuin lyhyet rivit.

    \fontsize{16bp}{17bp}\selectfont Tässä on melko suuri fontti ja
    suhteellisen pieni rivikorkeus. Suuri fontti ja lyhyet rivit eivät
    tarvitse kovin suurta rivikorkeutta.
  \end{tulos}
  \caption{Fontin koon ja rivikorkeuden asettaminen ja vaikutus}
  \label{esim/rivikorkeus}
\end{esimerkki*}

Toinen tekstirivien peruslinjojen väliseen etäisyyteen vaikuttava asetus
on \komento{baselinestretch}. Se on desimaalilukukerroin, jolla nykyinen
rivikorkeus kerrotaan. Kerroin asetetaan helpoimmin komennolla
\komento{linespread}.\footnote{Toinen tapa: \koodi{\keno
    renewcommand\{\keno baselinestretch\}\{kerroin\}}}

\komentoi{fontsize}
\komentoi{linespread}
\komentoi{selectfont}
\begin{koodilohkosis}
\fontsize{10bp}{12bp} \linespread{1.3} \selectfont
\end{koodilohkosis}

\noindent
Edellä oleva esimerkki asettaa fontin kooksi 10 pistettä ja
rivikorkeudeksi 12 pistettä. \komento{linespread}\-/ komennolla asetetun
kertoimen vuoksi rivien peruslinjojen väliseksi etäisyydeksi tulee
lopulta 1,3 kertaa 12 pistettä eli 15,6 pistettä. Ei ole väliä, kummassa
järjestyksessä \komento{fontsize}- ja \komento{linespread}\-/ komennot
annetaan. Asetukset tulevat voimaan vasta \komento{selectfont}\-/
komennon jälkeen.

Käytännössä \komento{linespread} sopii rivikorkeuden yleistason
hienosäätöön esimerkiksi dokumentin esittelyosassa. Tilannekohtainen
rivikorkeus on parasta asettaa \komento{fontsize}\-/komennolla.

\subsection{Kirjainperheen ja -leikkauksen valitseminen}
\label{luku/fontit-korkea}

Latexissa on joukko korkean tason fonttikomentoja, jotka on tarkoitettu
käytettäväksi sen jälkeen, kun matalan tason asetukset on kerran
määritetty. Taulukossa \ref{tlk/fonttimallikomennot} on komennot
kirjainperheen ja kirjainleikkauksen valintaan. Joka rivillä ensin
mainittu komento (esim. \komento{rmfamily}) vaikuttaa tekstiin, joka
tulee komennon jälkeen. Vaikutusalue rajoittuu nykyisen ympäristön (luku
\ref{luku/ympäristöt}) sisään tai aaltosulkeilla (luku
\ref{luku/aaltosulkeet}) rajatun alueen sisään. Rivillä toisena olevalle
komennolle (esim. \komentox{textrm}) annetaan yksi argumentti, ja
komennon vaikutus koskee vain argumenttina olevaa tekstiä.

\leijutlk{
  \providecommand{\rivi}{}
  \renewcommand{\rivi}[4]{%
    \komento{#1} & \komento{#2}\komentoarg{\dots} & #3 & #4 \\}
  \begin{tabular}{llll}
    \toprule
    \multicolumn{2}{l}{\ots{Komento}}
    & \multicolumn{2}{l}{\ots{Merkitys}} \\

    \midrule
    \rivi {rmfamily} {textrm} {\rmfamily perhe}
    {perus, yl. antiikva, serif, roman}
    \rivi {sffamily} {textsf} {\sffamily perhe}
    {groteski, sans serif, gothic}
    \rivi {ttfamily} {texttt} {\ttfamily perhe}
    {tasalevyinen, typewriter}

    \midrule
    \rivi {mdseries} {textmd} {\mdseries leikkaus}
    {tavallinen vahvuus, medium}
    \rivi {bfseries} {textbf} {\bfseries leikkaus}
    {lihavoitu, bold}

    \midrule
    \rivi {upshape} {textup} {\upshape leikkaus}
    {pystyasento, tavallinen}
    \rivi {itshape} {textit} {\itshape leikkaus}
    {kursiivi, italic}
    \rivi {slshape} {textsl} {\slshape leikkaus}
    {kalteva, slanted, oblique}
    \rivi {scshape} {textsc} {\scshape leikkaus}
    {pienversaali, kapiteeli, small caps}
    \bottomrule
  \end{tabular}
}{
  \caption{Komennot kirjainperheen ja kirjainleikkauksen valintaan.
    Perustila on \komento{rmfamily}, \komento{mdseries} ja
    \komento{upshape}}
  \label{tlk/fonttimallikomennot}
}

Taulukossa on ensimmäisillä riveillä kirjainperhekomennot (3\,kpl),
jotka vaihtavat koko perheen kaikkine leikkauksineen. Niiden jälkeen
tulevat kirjainleikkauskomennot, jotka valitsevat toisen leikkauksen
samasta perheestä. Kirjainleikkauksen asetukset jaetaan kahteen ryhmään:
\englantik{series} 'sarja' ja \englantik{shape} 'muoto'. Kummastakin
ryhmästä on valittuna aina yksi ominaisuus, eli samanaikaisesti voi olla
voimassa esimerkiksi \komento{bfseries} (\textbf{lihavoitu}) ja
\komento{itshape} (\textit{kursiivi}), ja tuloksena on \textbf{\itshape
  lihavoitua kursiivia}.

Useimmissa fonteissa kursiivileikkaus (\komento{itshape}) ja kalteva
leikkaus (\komento{slshape}) tuottavat saman lopputuloksen, mutta
käsitteellisesti ne ovat eri asia. Kursiivi on aina muodoltaan erilainen
leikkaus, joka hieman mukailee käsialakirjoitusta, joskaan kirjaimia ei
ole sidottu toisiinsa. Sen sijaan kalteva leikkaus on tavallisen eli
pystyasentoisen leikkauksen kallistettu versio. Kallistus voi olla
kirjainmuotoilijan piirtämä leikkaus, tai kallistus on voitu toteuttaa
mekaanisesti. Kallistuksen voi tehdä myös Latexissa (luku
\ref{luku/fontit-venytys}).

Komennot fontin koon valintaan ovat taulukossa
\ref{tlk/fonttikokokomennot}. Taulukko kertoo myös, mitä fontin
pistekokoa (pt) mikäkin komento tarkoittaa oletuksena. Oletus riippuu
Latexin dokumenttiluokkien (luku \ref{luku/perusdokumenttiluokat})
fonttikokovalitsimista \koodi{10pt}, \koodi{11pt} ja \koodi{12pt}.

\leijutlk{
  \begin{tabular}{lr@{}lr@{}lr@{}l}
    \toprule
    \ots{Komento}
    & \multicolumn{2}{c}{\ots{10pt}}
    & \multicolumn{2}{c}{\ots{11pt}}
    & \multicolumn{2}{c}{\ots{12pt}} \\
    \midrule
    \komento{tiny} & 5 && 6 && 6 \\
    \komento{scriptsize} & 7 && 8 && 8 \\
    \komento{footnotesize} & 8 && 9 && 10 \\
    \komento{small} & 9 && 10 && 10&,95 \\
    \komento{normalsize} & 10 && 10&,95 & 12 \\
    \komento{large} & 12 && 12 && 14&,4 \\
    \komento{Large} & 14&,4 & 14&,4 & 17&,28 \\
    \komento{LARGE} & 17&,28 & 17&,28 & 20&,74 \\
    \komento{huge} & 20&,74 & 20&,74 & 24&,88 \\
    \komento{Huge} & 24&,88 & 24&,88 & 24&,88 \\
    \bottomrule
  \end{tabular}
}{
  \caption{Fonttien oletuspistekoot dokumenttiluokkien valitsimilla
    \koodi{10pt}, \koodi{11pt} ja \koodi{12pt}}
  \label{tlk/fonttikokokomennot}
}

\ymparistoi{rmfamily}
\ymparistoi{sffamily}
\ymparistoi{ttfamily}
\ymparistoi{mdseries}
\ymparistoi{bfseries}
\ymparistoi{upshape}
\ymparistoi{itshape}
\ymparistoi{slshape}
\ymparistoi{scshape}
\ymparistoi{tiny}
\ymparistoi{scriptsize}
\ymparistoi{footnotesize}
\ymparistoi{small}
\ymparistoi{normalsize}
\ymparistoi{large}
\ymparistoi{Large}
\ymparistoi{LARGE}
\ymparistoi{huge}
\ymparistoi{Huge}%

Kaikille korkean tason fonttikomennoille on olemassa myös samanniminen
ympäristönsä, esimerkiksi \ymparisto{rmfamily}, \ymparisto{bfseries},
\ymparisto{itshape} tai \ymparisto{small}. Seuraavassa esimerkissä on
kaksi fontteihin vaikuttavaa ympäristöä sisäkkäin.

\ymparistoi{footnotesize}
\ymparistoi{scshape}
\begin{koodilohkosis}
\begin{footnotesize}
  \begin{scshape}
    Tämä teksti on pientä pienversaalia.
  \end{scshape}
\end{footnotesize}
\end{koodilohkosis}

\begin{tulossis}
  \begin{footnotesize}
    \begin{scshape}
      Tämä teksti on pientä pienversaalia.
    \end{scshape}
  \end{footnotesize}
\end{tulossis}

\subsection{Fonttikoon määrittely suhteellisesti}
\label{luku/fontti-suhteellinen}

Dokumentin fonttien koot on helpointa määrittää siten, että asettaa
ensin peruskirjainperheen koon ja antaa muiden fonttien määräytyä
suhteessa siihen. Esimerkki \ref{esim/fontti-suhteellinen} selventää,
kuinka se tapahtuu. Alussa otetaan käyttöön dokumenttiluokka
\luokka{article} ja annetaan sille valitsin \koodi{10pt}, joka määrittää
fonttikooksi 10 pistettä. Se on dokumenttiluokan oletusasetus, jota ei
tarvitsisi edes kirjoittaa näkyviin. Esimerkin toisella rivillä otetaan
\paketti{fontspec}\-/paketti käyttöön.

Peruskirjainperheen (rivi~4) koko skaalataan 1,4\-/kertaiseksi, eli
pistekooksi tulee 1,4 kertaa 10 pistettä eli 14 pistettä (pt).
Normaalikokoinen peruskirjainperhe on ainoa, jonka pistekoko tiedetään.
Kaikkien muiden koot täytyisi selvittää laskemalla.

Groteski eli pääteviivaton kirjainperhe (rivi~5) ja tasalevyinen perhe
(rivi~6) skaalataan samankorkuiseksi kuin perusperhe. Vertailukohtana
ovat gemenat eli pienaakkoset (\koodi{MatchLowercase}). Näiden kahden
kirjainperheen pistekokoa ei tiedetä. Se ei välttämättä ole sama kuin
perusfontissa, koska fonttien pistekoko mitataan ylimmän ja alimman
kohdan välillä ja koska fonttien mittasuhteet ovat erilaisia.

\begin{esimerkki*}
  \komentoi{documentclass}
  \komentoi{usepackage}
  \pakettii{fontspec}
  \komentoi{setmainfont}
  \komentoi{setsansfont}
  \komentoi{setmonofont}
  \komentoi{linespread}
  \luokkai{article}

\begin{koodilohko}
\documentclass[10pt]{article} % 10pt on oletus
\usepackage{fontspec}

\setmainfont{TeX Gyre Termes}[Scale=1.4]
\setsansfont{TeX Gyre Heros} [Scale=MatchLowercase]
\setmonofont{TeX Gyre Cursor}[Scale=MatchLowercase]
\linespread{1.45}
\end{koodilohko}
  \caption{Fonttikokojen määrittäminen suhteessa peruskirjainperheeseen}
  \label{esim/fontti-suhteellinen}
\end{esimerkki*}

Viimeisellä rivillä oleva \komento{linespread}\-/ komento on tärkeä. Se
asettaa rivikorkeuden kertoimeksi 1,45. Kertoimen täytyy olla vähintään
yhtä suuri kuin peruskirjainperheen skaalauskerroin (1,4), jotta
rivivälit ovat riittävän suuret. Näiden asetusten jälkeen dokumentissa
käytetään korkeamman tason komentoja fonttien valintaan, esimerkiksi
fonttikoon valintakomentoja \komento{small}, \komento{normalsize},
\komento{large} (taulukko \ref{tlk/fonttikokokomennot}).

Edellä kuvatussa suhteellisessa kirjainperheiden koon määrittelyssä on
sellainen ongelma tai kummallisuus, että Latex koko ajan luulee, että
peruskirjainperhe on normaalikokoisena 10 pistettä (pt). Latexin matalan
tason fonttikomennot eivät tiedä kirjainperheen skaalauskertoimesta, ja
siksi esimerkiksi komentojen

\komentoi{fontsize}
\komentoi{selectfont}
\begin{koodilohkosis}
\fontsize{10bp}{12bp} \selectfont
\end{koodilohkosis}

\noindent
tuloksena ei todellisuudessa ole 10 pisteen (bp) fontti, vaan mukaan
lasketaan myös kirjainperheen skaalauskerroin. Tämän vuoksi
\komento{fontsize}\-/ komennon käyttö menee aika oudoksi. Argumenttina
annettu kokomitta ei pidä paikkaansa.

Jos korkean tason fonttikokokomentojen (taulukko
\ref{tlk/fonttikokokomennot}) lisäksi tarvitaan jotakin muuta kokoa,
voisi mahdollisesti \komento{fontsize}\-/ komennon sijasta käyttää
\paketti{fontspec}\-/paketin tarjoamaa komentoa ja tilanteeseen sopivaa
skaalauskerrointa esimerkiksi seuraavalla tavalla:

\komentoi{addfontfeatures}
\begin{koodilohkosis}
{\addfontfeatures{Scale=3.2} Poikkeuksellisen isoa tekstiä}
\end{koodilohkosis}

\noindent
Jos edellä mainitut kummallisuudet eivät häiritse eikä ole tarvetta
määritellä fontteja tarkasti tietyn pistekoon mukaiseksi, on
suhteellinen määrittelytapa todella helppo. Kaikki dokumentin fontit
määräytyvät perusfontin skaalauskertoimen kautta. Tämä tapa sopii hyvin
varsinkin dokumentin sisällön kirjoittamisvaiheeseen, jossa ehkä
halutaan vain nopeasti asettaa dokumentti suurin piirtein järkevän
näköiseksi. Myöhemmin voi määrittää koot tarkemmin niin sanotun
absoluuttisen menetelmän avulla, jota käsitellään seuraavassa
alaluvussa.

\subsection{Fonttikoon määrittely absoluuttisesti}
\label{luku/fontti-absoluuttinen}

Absoluuttinen fonttien koonmääritystapa tarkoittaa sitä, että koot
asetetaan tietyn kokoiseksi käyttämällä esimerkiksi pistemittoja ja että
kirjaimet myös päätyvät lopulliseen dokumenttiin juuri sen kokoisena.
Tämä tapa on myös teknisesti eheä, eli Latexin eri osat ovat samaa
mieltä siitä, minkäkokoisesta fontista on kyse. Näin ei ollut
suhteellisen tavan kanssa (luku \ref{luku/fontti-suhteellinen}).

Joskus oppilaitoksen, yhtiön tai muun julkaisijan ohjeissa määritellään
tarkasti, mitä fontteja käytetään ja mikä on leipätekstin ja otsikoiden
fonttikoko. Silloin tarvitaan tässä luvussa kuvattua tapaa fonttien
asettamiseen.

\begin{esimerkki*}
  \komentoi{documentclass}
  \komentoi{usepackage}
  \pakettii{fontspec}
  \komentoi{setmainfont}
  \komentoi{setsansfont}
  \komentoi{setmonofont}
  \komentoi{newfontfamily}
  \komentoi{linespread}
  \komentoi{renewcommand}
  \komentoi{footnotesize}
  \komentoi{small}
  \komentoi{normalsize}
  \komentoi{large}
  \komentoi{Large}

\begin{koodilohko}
\documentclass{article}
\usepackage{fontspec}

% Leipätekstiin samankokoiset fontit
\setmainfont{TeX Gyre Termes}
\setsansfont{TeX Gyre Heros} [Scale=MatchLowercase]
\setmonofont{TeX Gyre Cursor}[Scale=MatchLowercase]

% Muualle sans ja mono ilman skaalausta
\newfontfamily{\sffamilyabs}{TeX Gyre Heros}
\newfontfamily{\ttfamilyabs}{TeX Gyre Cursor}

\linespread{1} % ei välttämättä tarvita

% Kaikki tarvittavat fonttikoot ja komennot
\renewcommand{\footnotesize}{\fontsize{10bp}{12bp}\selectfont}
\renewcommand{\small}       {\fontsize{12bp}{14bp}\selectfont}
\renewcommand{\normalsize}  {\fontsize{14bp}{17bp}\selectfont}
\renewcommand{\large}       {\fontsize{17bp}{19bp}\selectfont}
\renewcommand{\Large}       {\fontsize{20bp}{22bp}\selectfont}

\normalsize % jotta tulee heti voimaan eikä vasta tekstiosassa
\end{koodilohko}
  \caption{Fonttikokojen määrittäminen pistekoon avulla}
  \label{esim/fontti-absoluuttinen}
\end{esimerkki*}

Esimerkistä \ref{esim/fontti-absoluuttinen} selviää perusajatus.
Peruskirjainperhe (rivi~5) otetaan käyttöön ilman skaalausta
(\koodi{Scale=1}), minkä vuoksi koon voi jatkossa asettaa täsmälleen
kohdalleen \komento{fontsize}\-/komennolla. Samaa ei tehdä groteskin
eikä tasalevyisen fontin kanssa (rivit 6--7), vaan käytetään skaalausta
\koodi{MatchLowercase}, jotta tekstikappaleessa kaikki kirjainperheet
näyttävät samankokoisilta. Tässä menetetään mahdollisuus määrittää
näiden kirjainperheiden koko täsmällisesti pistemitan avulla. Jos siihen
on tarvetta esimerkiksi otsikoissa, voidaan käyttää rivien 10--11
komentoja. Niillä luodaan uudet kirjainperheet, jotka ovat käytännössä
samoja mutta ilman skaalausta.

Uusien skaalaamattomien kirjainperheiden komentojen nimiksi on valittu
\komentox{sf\-fam\-i\-ly\-abs} ja \komentox{tt\-fam\-i\-ly\-abs} (vrt.
\komento{sffamily} ja \komento{ttfamily}, taulukko
\ref{tlk/fonttimallikomennot}), ja näillä komennoilla kirjainperheet
kytketään päälle. Jos esimerkiksi jonkin julkaisun vaatimuksiin kuuluu,
että otsikossa täytyy olla 20 pisteen lihavoitu TeX Gyre Heros
\=/kirjainleikkaus, voi esimerkissä \ref{esim/fontti-absoluuttinen}
olevien asetusten pohjalta antaa otsikon ulkoasun (luku
\ref{luku/otsikot-ulkoasu}) määrittelyn yhteydessä seuraavat komennot:

\komentoi{Large}
\komentoi{bfseries}
\begin{koodilohkosis}
\sffamilyabs\Large\bfseries
\end{koodilohkosis}

\noindent
Esimerkin \ref{esim/fontti-absoluuttinen} riveillä 16--20 määritellään
uudelleen Latexin korkean tason komennot, joilla fonttikoot asetetaan.
Oletusarvot tulevat dokumenttiluokasta (luku
\ref{luku/dokumenttiluokat}), mutta jos ne eivät ole sopivia, täytyy
vähintäänkin määritellä komento \komento{normalsize} mutta sen lisäksi
kaikki ne koot, joita omassa dokumentissa tarvitaan. Tässä esimerkissä
normaali koko asetetaan 14 pisteen kokoiseksi.

Jokaiselle fonttikoolle määritetään riveillä 16--20 myös oma
rivikorkeus, ja se on tarkoitus asettaa sopivaksi juuri kyseiselle
koolle. Rivikorkeuteen vaikuttaa myös kerroin \komento{baselinestretch},
joka asetetaan komennolla \komento{linespread}. Sitä ei välttämättä
tarvitse käyttää, koska kirjainperheitä ei ole skaalattu ja koska
rivikorkeus asetetaan aina \komento{fontsize}\-/komennolla.
\komento{linespread} on kuitenkin kätevä komento rivikorkeuden
säätämiseen yleisesti kaikkialla.

Fonttikokojen määrittelyn lopuksi rivillä 22 suoritetaan komento
\komento{normalsize}, jotta se tulee heti voimaan. Dokumentin
esittelyosassa voidaan käyttää fonttikokoon viittaavia mittayksiköitä em
ja ex, ja ne viittaavat nyt tähän kokoon. \komento{normalsize}\-/
komento suoritetaan kyllä myöhemmin automaattisesti dokumentin
tekstiosan eli \ymparistox{document}\-/ ympäristön alussa.

Edellä kuvatun absoluuttisen koonmääritystavan etuna on se, että
kirjoittaja hallitsee fonttien kokoa ja rivikorkeuksia tarkasti ja että
julkaisuun saadaan juuri ne mitat, jotka halutaan tai vaaditaan. Tapa on
myös teknisesti eheä eli toimii Latexin sisäisen logiikan näkökulmasta
oikein. Haittana voi pitää sitä, että kaikki koot täytyy määritellä
erikseen.

\subsection{Fonttien oletusasetuksia}
\label{luku/fontit-oletusasetukset}

Oletuksena Latex\-/dokumentin peruskirjainperheessä
(\komento{setmainfont}) ja pääteviivattomassa kirjainperheessä
(\komento{setsansfont}) ovat päällä muun muassa seuraavat asetukset:%

\begin{koodilohkosis}
Ligatures={TeX, Common}
HyphenChar=-
\end{koodilohkosis}

\noindent
\koodi{TeX}\-/ligatuurit tarkoittavat lainausmerkkien ja ajatusviivojen
tuottamiseen tarkoitettuja Texin merkintätapoja kuten \koodi{''} ja
\koodi{\==}, joita käsitellään luvuissa \ref{luku/lainausmerkit} ja
\ref{luku/yhdys-ajatus-miinus}. Ne eivät ole varsinaisia luonnollisen
kielen eivätkä typografisia ligatuureja vaan kuuluvat ainoastaan
Tex\-/kielen merkintätapoihin. \koodi{Common}\-/ ligatuurit sen sijaan
ovat oikeita typografisia ligatuureja kuten fi, ff ja fl, ja niitä
käsitellään tarkemmin luvussa \ref{luku/typo-liga}. Molemmat edellä
mainitut ligatuurityypit saa pois päältä seuraavalla asetuksella:

\begin{koodilohkosis}
Ligatures={TeXReset, NoCommon}
\end{koodilohkosis}

\noindent
Valitsin \koodi{Hyphen\-Char} asettaa tavutusmerkin kyseiselle
kirjainperheelle. Oletuksena se on yhdysmerkki (\=/), mutta merkin
vaihtaminen toimii vain Xelatex\-/kääntäjän kanssa, ei esimerkiksi
Lualatexilla. Sen sijaan kaikilla kääntäjillä tavutuksen voi kytkeä
kirjainperheeltä pois, kun antaa arvoksi \koodi{None}. Tavutusta
käsitellään perusteellisemmin luvussa \ref{luku/tavutus}.

Tasalevyinen kirjainperhe (\komento{setmonofont}) toimii oletuksena
hieman eri tavalla. Niissä muun muassa ei ole lainausmerkkien ja
ajatusviivojen kirjoittamiseen tarkoitetut Tex\-/ligatuurit päällä.
Oletusasetuksia ovat muun muassa seuraavat:

\begin{koodilohkosis}
Ligatures=Common
HyphenChar=None
\end{koodilohkosis}

\noindent
Yleensä tasalevyisestä kirjainperheestä kannattaa kytkeä typografiset
ligatuurit pois päältä asetuksella \koodi{Liga\-tures=\katk No\-Common}.
Tasalevyisen fontin ajatukseen nimenomaan kuuluu, että merkit ovat
samanlevyisiä eikä yhden merkin tilaan sovi sulloa useampaa kirjainta.
Kaikki tasalevyiset kirjainperheet eivät edes sisällä ligatuurimerkkejä,
joten ominaisuus näyttää olevan automaattisesti pois päältä. Varmuuden
vuoksi on kuitenkin hyvä lisätä asetus \koodi{Liga\-tures=\katk
  No\-Common} tasalevyiselle perheelle.

Oletuksena tavutus on kytketty pois päältä tasalevyiseltä
kirjainperheeltä, koska sitä käytetään tavallisesti tietokoneisiin
liittyvien koodien tai vastaavien ilmausten latomiseen, eikä niitä
haluta yleensä tavuttaa. Tavutuksen saa kuitenkin päälle Xelatexissa
määrittämällä tavutusmerkiksi esimerkiksi yhdysmerkin
(\koodi{Hyphen\-Char=-}) tai kaikilla kääntäjillä poistamalla
tasalevyisen kirjainperheen oletusasetukset ennen sen määrittelyä
seuraavasti:

\komentoi{defaultfontfeatures}
\komentoi{ttfamily}
\komentoi{setmonofont}
\begin{koodilohkosis}
\defaultfontfeatures[\ttfamily]{}
\setmonofont{…}[…]
\end{koodilohkosis}

\noindent
Komennolla \komento{defaultfontfeatures} voi asettaa (joidenkin)
kirjainperheiden oletusasetukset, kun komennon suorittaa ennen
kirjainperheiden määrittelyä.

\komentoi{defaultfontfeatures}
\komentoi{rmfamily}
\komentoi{sffamily}
\begin{koodilohkosis}
\defaultfontfeatures[\rmfamily,\sffamily]{Ligatures={TeX, Common},
  Numbers=Lowercase}
\end{koodilohkosis}

\noindent
Komennon valinnaisella argumentilla voi rajata, mitä kirjainperheitä
oletusasetukset koskevat. Valinnaiseen argumenttiin kirjoitetaan yksi
tai useampia pilkulla erotettuja komentoja, joilla kirjainperheet
kytketään päälle: \komento{rmfamily}, \komento{sffamily},
\komento{ttfamily} (taulukko \ref{tlk/fonttimallikomennot},
s.~\pageref{tlk/fonttimallikomennot}) tai komennolla
\komento{newfontfamily} määritelty fonttikomento (luku
\ref{luku/fontin-valinta}).

\subsection{Typografiset ligatuurit}
\label{luku/typo-liga}

Typografiset ligatuurit ovat fontissa olevia yhdistelmämerkkejä, joissa
on typografisista syistä yhdistetty kaksi tai useampia kirjaimia yhteen
merkkiin.\footnote{Typografisten ligatuurien lisäksi on olemassa myös
  luonnollisten kielten ligatuureja, jotka muodostavat kieleen kuuluvan
  kirjaimen, kuten æ norjan kielessä. Niillä on jokin luonnolliseen
  kieleen liittyvä merkitys, eikä eri merkkejä ei ole yhdistetty
  typografisista syistä.} Tällaisia ligatuureja käytettiin jo
metalliladonnassa, eli samassa metallikirjakkeessa saattoi olla enemmän
kuin yksi kirjain.

Typografisten ligatuurien tarkoituksena on tuottaa tyylikkäämpi
lopputulos kuin saataisiin erillisten kirjainten avulla. Syynä on
esimerkiksi se, että erilliset peräkkäiset kirjaimet eivät ole aina
keskenään yhteensopivia. Kirjainten välistys voi olla ongelmallista:
sopivalle etäisyydelle asetettaessa kirjainten osat voivat mennä
rumannäköisesti päällekkäin. Riippuu fontista, mitkä ligatuurit ovat
tarpeellisia ja mitkä ligatuurimerkit on ylipäätään toteutettu fontin
merkistöön. Tavallisia antiikvojen ligatuureja ovat fi, ff, ffi, fl,
ffl, fj ja ffj, mutta jotkin fontit sisältävät muitakin. Kuvaan
\ref{kuva/ligatuurit} on koottu esimerkkejä.

\leijukuva{
  \rmfamily
  \addfontfeatures{Scale=3.4, Ligatures={Common, Historic,
      Discretionary}}
  fi ff ffi fl ffl fj ffj Th st ct
}{
  \caption{Tavalliset f\=/alkuiset typografiset ligatuurit, vähän
    harvinaisempi Th sekä historialliset ligatuurit st ja ct}
  \label{kuva/ligatuurit}
}

Latex hoitaa tavallisten typografisten ligatuurien latomisen
automaattisesti, eli lähdedokumenttiin ei kannata kirjoittaa
Unicode\-/merkistön typografisia ligatuurimerkkejä (esimerkiksi
\uctunnus{u+fb01 latin small ligature fi}) vaan ihan tavallisia
erillisiä kirjaimia. \englanti{Open Type} \=/fontin suunnittelija on
päättänyt meidän puolestamme, mitkä kirjainyhdistelmät on parasta latoa
ligatuurin avulla, ja oletusasetuksilla Latex noudattaa niitä ohjeita.

Jos kuitenkin yksittäisen ligatuurin muodostumisen haluaa estää, voi
kirjainten väliin kirjoittaa komennon \komento{textcompwordmark}, joka
tekee näkymättömän, juuri tähän tarkoitukseen olevan merkin
\uctunnus{u+200c zero width non-joiner}.

\komentoi{textcompwordmark}
\begin{koodilohkosis}
fi f\textcompwordmark i
\end{koodilohkosis}

\begin{tulossis}
  fi f\textcompwordmark i
\end{tulossis}

\noindent
Mikäli tavallisia ligatuureja ei halua käyttöön lainkaan, on parasta
valita jo kirjainperheen tai \=/leikkauksen käyttöönotossa asetus
\koodi{Liga\-tures=\katk No\-Com\-mon}. Väliaikaisesti fontin asetuksiin
voi vaikuttaa komennolla \komento{addfontfeatures}, joka sekin on
\paketti{fontspec}\-/pakettiin sisältyvä komento.

\komentoi{addfontfeatures}
\begin{koodilohkosis}
{\addfontfeatures{Ligatures=NoCommon} fi fl}
\end{koodilohkosis}

\noindent
Tavallisten ligatuurien lisäksi \englanti{Open Type} \=/fontit voivat
sisältää myös harvinaisempia ligatuureja, jotka täytyy erikseen kytkeä
päälle. Niiden tarkoituksena on esimerkiksi erityinen koristeellisuus
tai historiallisen kirjainleikkauksen jäljittely. Harvinaisia
ligatuureja ei ole tarkoitus käyttää joka tilanteessa vaan ainoastaan
erityisestä syystä. Kuvassa \ref{kuva/ligatuurit} ja tässäkin
tekstikappaleessa olevat {%
  \newcommand{\hlig}[1]{{\addfontfeatures{Ligatures=Historic}#1}}%
  \hlig{st}- ja \hlig{ct}\-/ligatuurit%
} on saatu fontin asetuksella \koodi{Liga\-tures=\katk His\-toric}.
Joissakin fonteissa sama tehdään asetuksella \koodi{Liga\-tures=\katk
  Discretionary}. Nämä ligatuurit sopivat niin sanotun humanistisen
antiikvaperheen (mm. \englanti{Adobe Jenson}) ja renessanssityylisen
sisällön kanssa käytettäväksi.

Teknisesti on mahdollista kytkeä päälle useitakin ligatuurityyppejä
samanaikaisesti:

\begin{koodilohkosis}
Ligatures={TeX, Required, Common, Historic, Discretionary,
  Contextual}
\end{koodilohkosis}

\noindent
Tuskin mikään fontti sisältää kaikkia ligatuurityyppejä, eikä se ole
tarkoituskaan. Jotkin ligatuurit kuuluvat vain tiettyyn typografian
aikakauteen tai kirjaintyyliin. Open Type \=/fontin ominaisuuksia voi
selvittää käyttöjärjestelmän komentotulkissa komennolla \koodi{otfinfo}.
Komennon argumentiksi annetaan muun muassa fonttitiedoston nimi.

\subsection{Numeroiden muoto}
\label{luku/fontit-numerot}

Varsinkin kirjatypografiassa on tavallista käyttää pääasiassa
gemenanumeroita ({\gemenanum 1967}) eikä versaalinumeroita
({\versaalinum 1967}), koska gemenanumerot sopivat leipätekstin
gemenakirjainten kanssa paremmin yhteen. Versaalinumerot puolestaan
erottuvat paremmin ja sopivat esimerkiksi taulukoihin, joissa on paljon
lukuja. Näihin asetuksiin vaikutetaan fontin asetuksella
\koodi{Numbers}. Tosin kaikissa fonteissa ei gemenanumeroita edes ole.

\begin{koodilohkosis}
Numbers=Lowercase % gemenanumerot
Numbers=Uppercase % versaalinumerot (oletus)
\end{koodilohkosis}

\noindent
Jotkin koodi\-/ilmaukset sisältävät versaalikirjaimia ja numeroita
sekaisin. Gemenanumerot eivät kuitenkaan sovi yhteen versaalikirjainten
kanssa, koska merkkien kokoero on häiritsevän suuri. Ei siis näin:
RJ45, R2D2. Gemenanumeroiden kanssa täytyy käyttää pienversaalia eli
kapiteelia: \textsc{rj45}, \textsc{r2d2}. Toinen vaihtoehto on käyttää
pelkästään versaaleja: {\versaalinum RJ45, R2D2}.

Silloin kun käytetään numeroita teknisten koodien ilmaisemiseen, voi
olla tarpeen merkitä numero nolla poikkiviivalla
({\addfontfeatures{Numbers={Uppercase, SlashedZero}} 0}), jottei se
sekoitu O\=/kirjaimeen. Nollaan saadaan poikkiviiva seuraavalla
asetuksella, jos vain fontissa on tämä ominaisuus:

\begin{koodilohkosis}
Numbers=SlashedZero
\end{koodilohkosis}

\noindent
Kun numeroita ladotaan taulukkoon, voidaan haluta käyttää tasalevyisiä
numeroita, jotta ne sijoittuvat allekkain samalle linjalle. Joissakin
fonteissa on tarjolla tavallisten vaihtelevan levyisten numeroiden
lisäksi myös tasalevyiset. Numeroiden leveyteen vaikutetaan seuraavilla
asetuksilla:

\begin{koodilohkosis}
Numbers=Monospaced   % tasalevyiset numerot
Numbers=Proportional % vaihtelevan levyiset numerot (oletus)
\end{koodilohkosis}

\subsection{Välistykset: harvennus ja tiivistys}
\label{luku/fontit-välistys}

Fontin kanssa käytettäviä sanavälejä voi säätää \koodi{Word\-Space}\-/
valitsimella, jonka arvoksi annetaan desimaalilukukerroin eli suhdeluku
normaaliin verrattuna. Arvona voi olla myös kolme eri kerrointa, jolloin
mukana on lisäksi sanavälin venymisen rajat. Tämä asetus on tehtävä
kirjainperheen tai \=/leikkauksen määrittelyn yhteydessä, eli se ei
toimi esimerkiksi \komento{addfontfeatures}\-/ komennon kanssa. Katso
tietoa sanaväleistä myös luvusta \ref{luku/sanaväli} tai
tekstikappaleiden latomista käsittelevästä luvusta \ref{luku/kappale}.
\noclub[2]

Fonttiasetusten valitsimella \koodi{Letter\-Space} säädetään merkkien
välistystä eli niiden välistä tyhjää tilaa. Arvoksi annetaan
positiivinen tai negatiivinen prosenttiluku, joka ilmaisee alkuperäiseen
lisättävän osuuden. Asetus \koodi{Letter\-Space=\katk 6} tuottaa siis
kuusi prosenttia leveämmät merkkien välit. Välistyksen tiivistäminen ja
harvennuksen typografiaa käsitellään luvussa
\ref{luku/korostus-harvennus}, mutta seuraavassa on esimerkki, kuinka ne
teknisesti toteutetaan:

\komentoi{scshape}
\komentoi{addfontfeatures}
\begin{koodilohkosis}
\scshape {\addfontfeatures{LetterSpace=-2} tiivistys} \\
normaali \\ {\addfontfeatures{LetterSpace=6} harvennus}
\end{koodilohkosis}

\begin{tulossis}
  \scshape {\addfontfeatures{LetterSpace=-2} tiivistys} \\
  normaali \\ {\addfontfeatures{LetterSpace=6} harvennus}
\end{tulossis}

\noindent
Jos esimerkiksi harvennuksen haluaa automaattisesti mukaan
kirjainperheeseen kuuluvaan pienversaaliin, käytetään valitsinta
\koodi{Small\-Caps\-Features}:

\komentoi{setmainfont}
\begin{koodilohkosis}
\setmainfont{…}[SmallCapsFeatures={LetterSpace=6}]
\end{koodilohkosis}

\subsection{Keinotekoinen venytys, lihavointi ja kallistus}
\label{luku/fontit-venytys}

Valitsimella \koodi{Fake\-Stretch} voi venyttää tai kutistaa merkkejä
leveyssuunnassa. Arvoksi annetaan desimaalilukukerroin, joka on
suhdeluku alkuperäiseen leveyteen nähden. Seuraavassa havainnollistava
esimerkki:

\komentoi{addfontfeatures}
\begin{koodilohkosis}
esimerkki \\
{\addfontfeatures{FakeStretch=1.3} esimerkki}
\end{koodilohkosis}

\begin{tulossis}
  esimerkki \\
  {\addfontfeatures{FakeStretch=1.3} esimerkki}
\end{tulossis}

\noindent
Kirjainleikkauksen leventäminen tekee tekstistä samalla lihavampaa,
koska kirjainten viivat tulevat vahvemmiksi leveyssuunnassa.
Korkeussuunnassa vahvuus säilyy ennallaan, joten kirjainten
viivakontrasti muuttuu. Kirjainleikkauksen kutistaminen vaikuttaa
lihavuuteen päinvastaisesti. Tällainen mekaaninen fontin korjaileminen
ei välttämättä tuota typografisesti kovin hyvää jälkeä mutta voi sopia
lievästi käytettynä esimerkiksi otsikoihin. Tämän oppaan tasalevyistä
kirjainperhettä on hieman kutistettu, koska alkuperäinen on kohtuuttoman
leveä ({\ttfamily\addfontfeatures{FakeStretch=1} esimerkki}) verrattuna
muiden kirjainperheiden leveyteen.

Jos kirjainperheeseen ei sisälly sopivaa lihavoitua tai kaltevaa
leikkausta, voi sellaiset koettaa tehdä myös keinotekoisesti
\koodi{Fake\-Bold}- ja \koodi{Fake\-Slant}\-/ valitsimien avulla.
Niille annetaan arvoksi desimaalilukukerroin. Valitettavasti
\koodi{Fake\-Bold} toimii vain Xelatex\-/kääntäjän kanssa.

\komentoi{addfontfeatures}
\begin{koodilohkosis}
pysty {\addfontfeatures{FakeSlant=.2} kalteva}
\end{koodilohkosis}

\begin{tulossis}
  pysty {\addfontfeatures{FakeSlant=.2} kalteva}
\end{tulossis}

\noindent
Mekaanisen lihavoinnin tai kallistuksen saa osaksi kirjainperhettä, kun
asettaa kirjainperheen määrittelyn yhteydessä halutun kirjainleikkauksen
valitsimella \koodi{Bold\-Font} tai \koodi{Slanted\-Font} ja määrittää
niille erityisiä ominaisuuksia valitsimella \koodi{Bold\-Fea\-tures} tai
\koodi{Slanted\-Fea\-tures}. Seuraava esimerkki asettaa dokumentin
perusfontille kaltevan leikkauksen. Tämän jälkeen komento
\komento{slshape} valitsee \textsl{kaltevan} leikkauksen, joka on siis
eri asia kuin \textit{kursiivi} (\komento{itshape}).

\komentoi{setmainfont}
\begin{koodilohkosis}
\setmainfont{…}[
  SlantedFont={…}, % tähän pystyasentoinen leikkaus
  SlantedFeatures={FakeSlant=.2}]
\end{koodilohkosis}

\subsection{Keinotekoinen pienversaali}
\label{luku/fontit-keinopienversaali}

Moniin fontteihin ei sisälly lainkaan pienversaalia. Jos sellaisen silti
haluaa omaan korostusvalikoimaan, voi yrittää pienentää
versaalikirjaimet sopivaan kokoon. Seuraavassa esimerkissä versaaleja
ensin pienennetään ja sitten venytetään hieman leveyssuunnassa. Lisäksi
merkkivälejä harvennetaan lievästi. Käytännössä versaaleja pienentämällä
ei saada tyylikästä pienversaalia aikaan, koska merkkien viivoista tulee
liian ohuita.

\komentoi{textsc}
\komentoi{addfontfeatures}
\begin{koodilohkosis}
\textsc{pienversaali} oikea \\
{\addfontfeatures{ScaleAgain=.68, FakeStretch=1.17, LetterSpace=4}
  PIENVERSAALI} keinotekoinen
\end{koodilohkosis}

\begin{tulossis}
  \textsc{pienversaali} oikea \\
  {\addfontfeatures{ScaleAgain=.68, FakeStretch=1.17, LetterSpace=4}
    PIENVERSAALI} keinotekoinen
\end{tulossis}

\subsection{Matematiikkatilan fontti}
\label{luku/matematiikka-fontit}

Latexin matematiikkatilan (luku \ref{luku/matematiikka}) fontti
asetetaan eri tavalla kuin normaalin tilan eli tekstitilan fontit.
Matematiikkatilaa varten tarvitaan paketti \pakettictan{unicode-math},
joka täytyy ladata muiden matematiikkaan tai fontteihin liittyvien
pakettien jälkeen. Paketti tuo komennon \komento{setmathfont}, joka on
matematiikkatilan vastine luvussa \ref{luku/fontin-valinta} esitellyille
fontinmäärittelykomennoille \komento{setmainfont}, \komento{setsansfont}
ja \komento{setmonofont}.

\pakettii{fontspec}
\pakettii{unicode-math}
\komentoi{setmainfont}
\komentoi{setmathfont}
\begin{koodilohkosis}
\usepackage{fontspec}
\usepackage{unicode-math}
\setmainfont{TeX Gyre Pagella}       % tekstitilan perusfontti
\setmathfont{TeX Gyre Pagella Math}  % matematiikkatilan fontti
\end{koodilohkosis}

\leijutlk{
  \providecommand{\rivi}{}
  \renewcommand{\rivi}[3]{\koodi{#1} & \koodi{#2} & \koodi{#3} \\}

  \begin{tabular}{lll}
    \toprule
    \ots{Yleistyyli} & \ots{Lihavointi} & \ots{Pääteviivaton} \\
    \midrule
    \rivi{math-style=ISO}    {bold-style=ISO}    {sans-style=upright}
    \rivi{math-style=TeX}    {bold-style=TeX}    {sans-style=italic}
    \rivi{math-style=french} {bold-style=upright}{sans-style=literal}
    \rivi{math-style=upright}{}{}
    \rivi{math-style=literal}{}{}
    \bottomrule
  \end{tabular}
}{
  \caption{Matematiikkatilan tyyliasetuksia (\paketti{unicode-math}\-/
    paketti)}
  \label{tlk/unicode-math-style}
}

\noindent
Paketin \paketti{unicode-math} lataamisen yhteydessä voi
\komento{usepackage}\-/ komennon valinnaisen argumentin avulla vaikuttaa
matematiikkatilassa käytettyihin kirjainleikkauksiin. Argumenttiin
sopivia valitsimia ovat \koodi{math-style}, \koodi{bold-style} ja
\koodi{sans-style}, ja niille sopivia arvoja on koottu taulukkoon
\ref{tlk/unicode-math-style}. Asetukset vaikuttavat esimerkiksi siihen,
mitä kirjainleikkausta käytetään, kun ladotaan matematiikkatilan
latinalaisia tai kreikkalaisia kirjaimia.

\begin{koodilohkosis}
\usepackage[math-style=ISO]{unicode-math}
\end{koodilohkosis}

\noindent
Matematiikkatilan fontiksi ei kelpaa mikä tahansa, koska tarvittavien
matemaattisten symbolien täytyy sisältyä \englanti{Open Type} \=/fontin
merkkivalikoimaan. Matematiikkatukea ei todennäköisesti ole, jos sitä ei
fontin ominaisuuksien yhteydessä mainita. Fonttitiedoston ominaisuuksia
voi tutkia esimerkiksi käyttöjärjestelmän komentotulkissa
\koodi{otfinfo}\-/ komennolla:

\begin{koodilohkosis}
otfinfo -s texgyrepagella-math.otf
\end{koodilohkosis}

\noindent
Tex Live \=/jakelun mukana tulee muutama \englanti{Open Type}
\=/kirjainperhe, joihin kuuluvat myös matematiikan symbolit. Niitä ovat
ainakin Libertinus, eri TeX Gyre \=/perheet sekä \englanti{Latin Modern
  Roman}, jonka kanssa sopii matematiikkafontti \englanti{Latin Modern
  Math}. Libertinus\-/ perheen voi ottaa käyttöön hyvin
yksinkertaisesti: ladataan vain paketti \paketti{libertinus}, joka
määrittää kerralla antiikvan, groteskin, tasalevyisen ja
matematiikkafontin. Toisaalta Libertinus\-/ perhettä voi käyttää myös
tavallisten fontinmäärityskomentojen avulla.

\begin{esimerkki*}
  \komentoi{setmathfont}

\begin{koodilohko}
\setmathfont{Libertinus Math Regular}
\setmathfont{TeX Gyre Bonum Math}
\setmathfont{TeX Gyre Dejavu Math}
\setmathfont{TeX Gyre Pagella Math}
\setmathfont{TeX Gyre Schola Math}
\setmathfont{TeX Gyre Termes Math}
\setmathfont{Latin Modern Math}
\end{koodilohko}

  \caption{Matematiikan symboleja sisältäviä vapaita fontteja. Komento
    \komento{setmathfont} sisältyy pakettiin \paketti{unicode-math}}
  \label{esim/matematiikka-fontteja}
\end{esimerkki*}

Esimerkissä \ref{esim/matematiikka-fontteja} on komentoja joidenkin
matematiikkafonttien lataamiseen. Lähes aina on järkevää käyttää
asetusta \koodi{Scale=\katk MatchLowercase}, joka asettaa fontin
samankokoiseksi kuin vastaava peruskirjainperhe.

\begin{koodilohkosis}
\setmathfont{TeX Gyre Pagella Math}[Scale=MatchLowercase]
\end{koodilohkosis}

\section{Kieli}
\label{luku/kieliasetukset}

Melkein aina Latex\-/dokumenttiin täytyy ladata kielipaketti ja sen
mukana asetukset tiettyjä kieliä varten. Kieliasetukset sisältävät
ainakin tavutussäännöt (luku \ref{luku/tavutus}) sekä kielelle
mukautettuja nimiä dokumentin eri osille. Esimerkiksi sisällysluettelon
ja kirjallisuusluettelon otsikot tulevat kieliasetuksista, samoin
leijuvien taulukoiden ja kuvien nimet ''\tablename'' ja ''\figurename''.
Myös muita asetuksia tai komentoja saattaa tulla kieliasetusten mukana,
mutta ne vaihtelevat eri kielissä.

Kielipaketteja on kaksi -- \paketti{babel} ja \paketti{polyglossia} \==,
ja kirjoittajan täytyy valita niistä jompikumpi. Pakettien historia ja
kehitys on suunnilleen seuraavanlainen: \paketti{babel} on paljon
vanhempi, ja monet vanhat oppaat ja esimerkit käsittelevät pelkästään
sitä. Kun Latex siirtyi Unicode\-/aikaan Xelatex\-/kääntäjän ja
\paketti{fontspec}\-/paketin myötä, \paketti{babel} ei pysynyt
kehityksessä mukana. Syntyi \paketti{polyglossia}, joka hallitsi
Unicoden sekä muitakin kuin latinalaiseen kirjaimistoon perustuvia
kieliä ja kirjoitusjärjestelmiä. \paketti{babel}\-/paketin kehitys ei
kuitenkaan pysähtynyt, ja sittemmin myös se on kehittynyt Unicode\-/
aikakaudelle.

Suomen kielen kannalta ei ole merkitystä, kumpaa kielipakettia käyttää,
mutta meidän näkökulmastamme ''erikoisemmat'' kielet ja
kirjoitusjärjestelmät voivat vaatia selvittämistä, kumpi kielipaketti
soveltuu paremmin. \paketti{babel} sisältää enemmän ominaisuuksia,
esimerkiksi omien komentojen ja muiden viritysten tekemiseen;
\paketti{polyglossia} on yksinkertaisempi paketti, jonka kehitys
tuntuisi keskittyvän vain ydintehtävään.

Seuraavissa alaluvuissa käsitellään kummankin kielipaketin tärkeimmät
toiminnot eli kielen valintaan liittyvät asiat. Yleinen ajatus on se,
että Latex\-/dokumentin esittelyosassa ladataan kielipaketti ja
määritellään dokumentin pääasiallinen kieli ja mahdolliset muut kielet.
Jos dokumentin tekstiosa sisältää muita kuin pääasiallista kieltä,
täytyy käyttää erityisiä komentoja tai ympäristöjä, joilla kerrotaan
Latexille, mistä kielestä on kyse. Tavutusta käsitellään kummankin
kielipaketin osalta luvussa \ref{luku/tavutus}.

\subsection{Polyglossia}
\label{luku/polyglossia}

Kielipaketti \pakettictan{polyglossia} vaatii toimiakseen Lualatex\-/\
tai Xelatex\-/ kääntäjän. Se ei siis toimi perinteisillä Latexin
kääntäjillä. Kielipaketin käyttöönotto dokumentin esittelyosassa näyttää
esimerkiksi seuraavanlaiselta:

\komentoi{usepackage}
\komentoi{setdefaultlanguage}
\komentoi{setotherlanguage}
\pakettii{polyglossia}
\begin{koodilohkosis}
\usepackage{polyglossia}
\setdefaultlanguage{finnish}
\setotherlanguage{english}
\setotherlanguage{greek}
\end{koodilohkosis}

\noindent
Eri kielillä on erilaisia valinnaisia asetuksia, jotka täytyy selvittää
\paketti{polyglossia}\-/paketin ohjekirjasta. Asetukset liittyvät
esimerkiksi kielen kirjoitusjärjestelmän alueelliseen tai
historialliseen vaihteluun.

Dokumentin tekstiosa käyttää kieltä, joka määriteltiin komennolla
\komento{setdefaultlanguage}. Muita dokumentin esittelyosassa
määriteltyjä kieliä voi käyttää väliaikaisesti komennolla, joka alkaa
kirjaimilla \komentox{text} ja jatkuu kielen nimellä, esimerkiksi
\komento{textenglish} tai \komento{textgreek}.

\komentoi{textenglish}
\begin{koodilohkosis}
Englannin sana \textenglish{shorthand} tarkoittaa 'pikakirjoitusta'.
\end{koodilohkosis}

\noindent
Toinen vaihtoehto on käyttää kielen nimen mukaista ympäristöä:

\begin{koodilohkosis}
\begin{greek}
  ...
\end{greek}
\end{koodilohkosis}

\noindent
Tietynkieliseksi merkitty teksti voi näyttää ladotussa dokumentissa ihan
samalta kuin muukin teksti. Tavutuksessa kuitenkin käytetään eri
kielissä eri asetuksia, ja joissakin kielissä voi olla myös pieniä
typografisia yksityiskohtia eri tavalla.

Oletuksena kaikilla kielillä käytetään samoja fontteja, mutta
kirjoittaja voi määritellä fontin myös kielikohtaisesti, niin että
kielen vaihtuessa fonttikin vaihtuu automaattisesti. Tietyn kielen
fontti pitää määrittää komentoon, jonka nimessä on ensin kielen nimi ja
lopuksi sana \koodi{font} (peruskirjainperhe), \koodi{fontsf}
(pääteviivaton) tai \koodi{fonttt} (tasalevyinen). Esimerkiksi
fonttikomennot englannin kielelle ja kolmelle eri kirjainperheelle ovat
\komento{englishfont}, \komento{englishfontsf} ja
\komento{englishfonttt}.

Näiden fonttikomentojen ja kirjainperheiden määrittämiseen kannattaa
käyttää \paketti{fontspec}\-/ paketin komentoa \komento{newfontfamily}
(luku \ref{luku/fontin-valinta}). Seuraavassa esimerkissä asetetaan
kaikki kolme kirjainperhettä kreikan kielelle:

\komentoi{newfontfamily}
\komentoi{greekfont}
\komentoi{greekfontsf}
\komentoi{greekfonttt}
\begin{koodilohkosis}
\newfontfamily{\greekfont}  {GFS Artemisia}  [Scale=MatchLowercase]
\newfontfamily{\greekfontsf}{GFS Neohellenic}[Scale=MatchLowercase]
\newfontfamily{\greekfonttt}{TeX Gyre Cursor}[Scale=MatchLowercase]
\end{koodilohkosis}

\noindent
Kirjoittajan ei tarvitse itse käyttää edellä mainittuja komentoja kuten
\komento{englishfont}, \komento{greekfont} jne. Kirjoittaja käyttää vain
kielenvaihtokomentoja, ja fonttikin vaihtuu samalla itsestään, jos vain
kohdekielelle ja \=/kirjainperheelle on määritelty sopiva fonttikomento.

\subsection{Babel}
\label{luku/babel}

Vanha kunnon \pakettictan{babel}\yipilkku\avctan{babel-finnish} toimii
useiden eri Latex\-/ kääntäjien kanssa. Ennen dokumentit kirjoitettiin
rajallisilla merkistöillä (kuten \textsc{iso-8859\=/1}), jotka
sisältävät vain reilut kaksisataa ihmiskielten kirjoitusmerkkiä.
\paketti{babel} kuitenkin toimii myös Lualatex\-/\ ja Xelatex\-/
kääntäjillä eli Unicode\-/ merkistön kanssa.

Kieliasetukset otetaan käyttöön \paketti{babel}\-/paketissa seuraavan
esimerkin tavoin. Paketin lataamisessa valinnaiseen argumenttiin
kirjoitetaan ladattavien kielten nimet ja viimeisenä mainitaan se kieli,
joka halutaan pääasialliseksi kieleksi. Toisaalta pääasiallisen kielen
voi valita myös \koodi{main}\-/ valitsimella. Seuraavassa on esimerkki
kummastakin kieltenvalintatavasta:

\komentoi{usepackage}
\pakettii{babel}
\begin{koodilohkosis}
\usepackage[english,greek,finnish]{babel}
\usepackage[main=finnish,english,greek]{babel}
\end{koodilohkosis}

\noindent
Dokumentin tekstissä yksittäiset vieraskieliset sanat tai ilmaukset
merkitään komennolla \komento{foreignlanguage}. Komennon ensimmäinen
argumentti on kielen nimi ja toinen on sillä kielellä ladottava teksti.

\komentoi{foreignlanguage}
\begin{koodilohkosis}
Englannin sana \foreignlanguage{english}{shorthand} tarkoittaa
'pikakirjoitusta'.
\end{koodilohkosis}

\noindent
Komennon sijasta voi käyttää ympäristöä \ymparisto{otherlanguage}, joka
vaihtaa kieltä ympäristön ajaksi.

\ymparistoi{otherlanguage}
\begin{koodilohkosis}
\begin{otherlanguage}{greek}
  ...
\end{otherlanguage}
\end{koodilohkosis}

\noindent
Kielen vaihtamista voi helpottaa \komento{babeltags}\-/ komennolla, joka
määrittelee lyhempiä komentoja ja ympäristöjä kielen vaihtamiseen.
Komentoa käytetään seuraavan esimerkin tavoin. Sen jälkeen voi käyttää
englanninkielisen tekstin merkitsemiseen komentoa \komentox{text\-eng}
tai ympäristöä \ymparistox{eng}.

\komentoi{babeltags}
\begin{koodilohkosis}
\babeltags{eng = english}
\end{koodilohkosis}

\noindent
Oletuskieli voidaan vaihtaa kesken dokumentin komennolla
\komento{selectlanguage}. Komennon argumentiksi annetaan kielen nimi,
esimerkiksi seuraavalla tavalla:

\komentoi{selectlanguage}
\begin{koodilohkosis}
\selectlanguage{english}
\end{koodilohkosis}

\noindent
Eri kielille on mahdollista asettaa eri kirjainperheet
\komento{babelfont}\-/ komennolla. Tämä komento korvaa
\paketti{fontspec}\-/ paketin fontinvalintakomennot (luku
\ref{luku/fontin-valinta}) ja muodostaa uuden korkeamman tason komennon,
joka sisältää myös kielen. Jos siis käyttää komentoa
\komento{babelfont}, ei saa samanaikaisesti käyttää komentoja
\komento{setmainfont}, \komento{setsansfont} eikä \komento{setmonofont}.

Oman dokumentin kirjainperheet voitaisiin määrittää
\komento{babelfont}\-/ komennolla esimerkin \ref{esim/babelfont} tavoin.
Ensin esimerkissä määritellään peruskirjainperhe (\koodi{rm}),
pääteviivaton perhe (\koodi{sf}) ja tasalevyinen perhe (\koodi{tt}). Sen
jälkeen määritellään kreikan kielessä (\koodi{greek}) käytettävä
peruskirjainperhe ja pääteviivaton perhe.

\begin{esimerkki*}
  \komentoi{babelfont}

\begin{koodilohko}
\babelfont{rm}{TeX Gyre Termes}
\babelfont{sf}[Scale=MatchLowercase]{TeX Gyre Heros}
\babelfont{tt}[Scale=MatchLowercase]{TeX Gyre Cursor}

\babelfont[greek]{rm}[Scale=MatchLowercase]{GFS Artemisia}
\babelfont[greek]{sf}[Scale=MatchLowercase]{GFS Neohellenic}
\end{koodilohko}
  \caption{\komento{babelfont}\-/komennon käyttö dokumentin
    kirjainperheiden valintaan ja kielikohtaisten kirjainperheiden
    valintaan}
  \label{esim/babelfont}
\end{esimerkki*}

Esimerkissä ennen fontin nimeä (esim. TeX Gyre Heros) oleva valinnainen
argumentti on sama kuin \paketti{fontspec}\-/paketin
fontinvalintakomennoissa. Sen avulla määritellään kyseisen
kirjainperheen asetuksia. Lisätietoa voi lukea fontteja käsittelevästä
luvusta \ref{luku/kirjaintyypit} ja \pakettictan{fontspec}\-/ paketin
ohjekirjasta.

\section{Tavutus}
\label{luku/tavutus}

Sanojen tavutus eli katkaiseminen rivin lopussa tavurajan kohdalta
kytkeytyy läheisesti kieliasetuksiin (luku \ref{luku/kieliasetukset}),
joihin on syytä perehtyä ennen tämän luvun lukemista.

Tex tavuttaa eli katkaisee sanat automaattisesti rivien lopussa, jotta
se saisi tekstikappaleet näyttämään tasapainoisilta. Tavutus perustuu
yksinkertaisiin kirjainpohjaisiin sääntöihin: millaisia kirjainten
yhdistelmiä tavuissa voi olla. Tavutussäännöt määräytyvät valitun kielen
perusteella ja joskus kieleen liittyvien muiden asetusten perusteella.

Automaattinen tavutus auttaa paljon, mutta se ei yksinään riitä. Se
tavuttaa välillä kielen kannalta väärin tai tuottaa muuten suositusten
vastaista jälkeä. Niinpä kirjoittajan täytyy auttaa välillä eli
kirjoittaa tavutusvihjeitä. Texissä itsessään on tavutukseen vaikuttavia
erikoismerkkejä ja toimintoja, mutta myös kielipaketeissa
\paketti{polyglossia} ja \paketti{babel} on omia tavutukseen vaikuttavia
erikoisuuksiaan, jotka voivat olla eri kielissä erilaisia.
Kielipakettien ominaisuuksia käsitellään luvuissa
\ref{luku/tavutus-polyglossia} ja \ref{luku/tavutus-babel}.

\subsection{Yleiset tavutusvihjeet}
\label{luku/hyphenation-komento}

Yksi tapa tavutusvihjeiden kirjoittamiseen on \komento{hyphenation}\-/
komento, jolla määritellään yksittäisten sanojen tavutuskohdat
kaikkialla dokumentissa. Seuraava esimerkki selventää komennon käyttöä:

\komentoi{hyphenation}
\begin{koodilohkosis}
\hyphenation{
  ala-indek-si alku-osa nimen-omaan
  typo-gra-fi-nen Latex
}
\end{koodilohkosis}

\noindent
Komennon argumentiksi eli aaltosulkeiden sisään kirjoitetaan sanoja,
jotka erotetaan toisistaan sanaväleillä. Sanoihin kirjoitetaan
yhdysmerkki niihin kohtiin, joista sanan katkaiseminen on sallittua. Jos
yhdysmerkkiä ei ole, sanaa ei tavuteta mistään kohdasta. Jos taas sana
on yhdyssana ja sen osien väliin kuuluu on yhdysmerkki, ei sen tavutusta
voi käsitellä tällä komennolla. Luvussa \ref{luku/tavutuksen-merkit}
kerrotaan muita tapoja.

\komento{hyphenation}\-/komennon voi sijoittaa dokumentin esittelyosaan
tai tekstiosaan, mutta sijainti vaikuttaa sen toimintaan. Jos komennon
sijoittaa dokumentin esittelyosaan ennen kuin mitään kieltä on ladattu
tai valittu, se vaikuttaa kaikkien sanojen tavutukseen kielestä
riippumatta. Jos komennon sijoittaa dokumentin tekstiosaan eli kielen
valitsemisen jälkeen, se vaikuttaa vain kyseisen kielen eli yleensä
dokumentin pääasiallisen kielen tavutukseen. Kielipaketit
\paketti{polyglossia} (luku \ref{luku/tavutus-polyglossia}) ja
\paketti{babel} (luku \ref{luku/tavutus-babel}) sisältävät komennon,
joilla voi asettaa yleisiä kielikohtaisia tavutusvihjeitä.

\subsection{Yksittäisten sanojen tavutus}
\label{luku/tavutuksen-merkit}

Tietyt sanassa mukana olevat merkit kytkevät tavalliset tavutussäännöt
pois päältä. Jos sanassa on mukana yksikin yhdysmerkki (\koodi{\=/}),
lyhyt ajatusviiva (\mbox{\koodi{--}}) tai pitkä ajatusviiva
(\mbox{\koodi{---}}), sana katkaistaan vain näiden merkkien jälkeen, eli
muut tavutuskohdat kytketään pois käytöstä.

Unicode\-/ merkistön ajastusviivat \uctunnus{u+2013 en dash} ja
\uctunnus{u+2014 em dash} ovat toimineet tavutuksessa eri tavalla
Lualatex\-/{} ja Xelatex\-/ kääntäjissä ja mahdollisesti eri tavalla
kuin edellä mainitut Texin omat merkintätavat. Yhteensopivuussyistä on
parasta kirjoittaa ajatusviivat Texin merkintätavoilla.

Myös tavutusvihje (\komento{-}) vaikuttaa sanan tavutuskohtiin: jos
sanassa on mukana yksikin tavutusvihje, tavalliset tavutuskohdat
kytkeytyvät pois käytöstä, ja sana katkaistaan vain tavutusvihjeiden
kohdalta sekä edellä mainittujen viivavälimerkkien jälkeen. Kielipaketti
\paketti{babel} (luku \ref{luku/tavutus-babel}) käyttäytyy kuitenkin
suomen kielessä eri tavoin. Se määrittelee ainakin suomen kielessä
tavutusvihjeen siten, että se sallii sanan tavutuksen muistakin kohdista
kuin tavutusvihjeen kohdalta.

\subsection{Tavutus sanan reunasta}
\label{luku/tavutus-reunasta}

Asetukset \komento{lefthyphenmin}\komentojatko{=N} ja
\komento{righthyphenmin}\komentojatko{=N} vaikuttavat tavutukseen sanan
reunoissa. Argumentti \koodi{N} on positiivinen kokonaisluku, joka
määrittelee, kuinka monta merkkiä vähintään sanan vasemmasta tai
oikeasta reunasta pidetään yhdessä. Oletusarvot ovat kielikohtaisia ja
ne on määritelty \paketti{polyglossia}\-/{} ja \paketti{babel}\-/
paketeissa. Suomen kielessä kumpikin asetus on kaksi (2) merkkiä.

Nämä asetukset alustetaan automaattisesti kielikohtaisiin oletusarvoihin
aina, kun kieliasetukset tulevat voimaan. Näin on esimerkiksi dokumentin
aloittavan \ymparistox{document}\-/ ympäristön alussa ja aina kielen
vaihtuessa. Jos kirjoittaja haluaa muuttaa asetuksia, täytyy omat
muutokset tehdä joka kerta edellä mainittujen asioiden jälkeen.
Toisaalta kätevämpää on sisällyttää omat asetukset kielikohtaisiin
alustuskomentoihin. Sillä tavoin omat asetukset tulevat voimaan
automaattisesti kielen vaihtuessa.

\komentoi{addto}
\komentoi{captionsfinnish}
\komentoi{lefthyphenmin}
\komentoi{righthyphenmin}
\begin{koodilohkosis}
\addto{\captionsfinnish}{
  \lefthyphenmin=3 \righthyphenmin=2
}
\end{koodilohkosis}

\noindent
Edellisessä esimerkissä oleva komento \komento{addto} on
\paketti{polyglossia}\-/{} ja \paketti{babel}\-/ paketin ominaisuus,
jolla lisätään omia komentoja kielikohtaisiin asetuksiin. Tässä
esimerkissä käsitellään suomen kielen asetuksia
(\komento{captionsfinnish}). Komento täytyy suorittaa lähdedokumentin
esittelyosassa.

\subsection{Polyglossia ja tavutus}
\label{luku/tavutus-polyglossia}

Kielipaketti \pakettictan{polyglossia} (myös luku
\ref{luku/polyglossia}) sisältää tavutukseen vaikuttavia asetuksia,
jotka ovat eri kielissä erilaisia. Suomen kielessä on hyödyllistä ottaa
käyttöön asetus \koodi{babelshorthands}, joka tuo mukaan muutaman
tavutukseen vaikuttavan erikoismerkin.

\komentoi{setdefaultlanguage}
\begin{koodilohkosis}
\setdefaultlanguage[babelshorthands]{finnish}
\end{koodilohkosis}

\noindent
Asetus \koodi{babelshorthands} lisää erikoismerkkivalikoimaan seuraavat
tavutukseen vaikuttavat merkit. Taulukossa \ref{tlk/polyglossia-tavutus}
on esimerkkejä tavutusvihjeiden, yhdysmerkin ja ajatusviivan sekä
\paketti{polyglossia}\-/ paketin erikoismerkkien vaikutuksesta.

\newcommand{\shorthandsyhdysmerkki}{%
\item [\textquotedbl-] Sitova yhdysmerkki (\=/), joka sitoo merkin
  seuraavaan sanaan eli estää katkaisemisen yhdysmerkin jälkeen. Tämä
  merkki sallii sanan tavuttamisen muista kohdista. Tämän sijasta voi
  käyttää Unicode\-/ merkistön sitovaa yhdysmerkkiä \uctunnus{u+2011
    non\-/ breaking hyphen}.}

\newcommand{\shorthandspystyviiva}{%
\item [\textquotedbl|] Estää typografisten ligatuurien (luku
  \ref{luku/typo-liga}) muodostumisen tähän kohtaan. Esimerkiksi fi\-/
  ligatuurin muodostumisen voi estää kirjoittamalla \koodi{f"|i}, ja
  tuloksena on kaksi erillistä merkkiä: f\textcompwordmark i.}

\newcommand{\shorthandslainausmerkki}{%
\item [\textquotedbl\textquotedbl] Tavutusvihje, joka ei lisää
  tavutuskohtaan yhdysmerkkiä. Tämä on hyödyllinen esimerkiksi
  teknisissä ilmauksissa, joissa sanaan tarvitaan mahdollinen
  katkaisukohta mutta ei haluta lisätä ylimääräistä yhdysmerkkiä
  sotkemaan ilmausta.}

\begin{maaritelma}{\koodi{#1}}
  \shorthandsyhdysmerkki

% Polyglossian tulevassa versiossa (jälkeen 30.12.2023).
% \item [\textquotedbl=] Sama kuin edellinen (\koodi{\textquotedbl-}).

  \shorthandspystyviiva

  \shorthandslainausmerkki

\item [\textquotedbl/] Vinoviiva (/), jonka jälkeen on tavutusvihje.
  Tavutuskohtaan ei lisätä yhdysmerkkiä. Tavutus sallitaan sanan muista
  kohdista.
\end{maaritelma}

\leijutlk{
  \begin{tabular}{lll}
    \toprule
    \ots{Lähde}
    & \ots{Tavutus}
    & \ots{Merkitys} \\
    \midrule

    \koodi{erikoisalalla}
    & eri\tavukohta koi\tavukohta sa\tavukohta lal\tavukohta la
    & tavutus kaikista kohdista (väärin) \\

    \koodi{erikois\komento{-}alalla}
    & erikois\tavukohta alalla
    & vain tavutusvihjeen kohdalta \\

    \koodi{matka-aika}
    & matka-\tavukohta aika
    & vain yhdysmerkin jälkeen \\

    \koodi{matka-ai\komento{-}ka}
    & matka-\tavukohta ai\tavukohta ka
    & vain yhdysmerkki ja tavutusvihje \\

    \koodi{Oulu--Rovaniemi}
    & Oulu--\tavukohta Rovaniemi
    & vain ajatusviivan jälkeen \\

    \koodi{Oulu--Rova\komento{-}niemi}
    & Oulu--\tavukohta Rova\tavukohta niemi
    & vain ajatusviiva ja tavutusvihje \\

    \koodi{matka-}
    & matka-
    & ei tavutuskohtia \\

    \koodi{-aika}
    & -\tavukohta aika
    & vain yhdysmerkin jälkeen (väärin) \\

    \koodi{\textquotedbl-aika}
    & -ai\tavukohta ka
    & ei sitovan yhdysmerkin jälkeen \\

    \bottomrule
  \end{tabular}
}{
  \caption{Kielipaketti \paketti{polyglossia} ja tavutusvihjeen,
    yhdysmerkin ja ajatusviivan sekä kielipaketin omien erikoismerkkien
    (\koodi{babelshorthands}) vaikutus tavutukseen}
  \label{tlk/polyglossia-tavutus}
}

\noindent
Yleisiä koko kieleen vaikuttavia tavutusvihjeitä voi määritellä
komennolla \komento{pghyphenation}. Se vastaa Texin
\komento{hyphenation}\-/ komentoa (luku \ref{luku/hyphenation-komento}),
mutta tavutusvihjeet määritellään kielikohtaisesti. Komennon ensimmäinen
argumentti on kielen nimi, ja toiseen argumenttiin luetellaan sanat ja
niiden sallitut tavutuskohdat yhdysmerkin avulla.

\komentoi{pghyphenation}
\begin{koodilohkosis}
\pghyphenation{finnish}{
  ala-indek-si alku-osa nimen-omaan typo-gra-fi-nen Latex
}
\end{koodilohkosis}

\noindent
Kielikohtaisesti voi määrittää myös sen, kuinka monta kirjainta sanan
reunasta jätetään tavuttamatta. \paketti{polyglossia}\-/ paketissa on
komento \komento{setlanghyphenmins}, jonka ensimmäinen argumentti on
kielen nimi. Toinen ja kolmas argumentti ovat kokonaislukuja, joilla
asetetaan sanan vasemman ja oikean reunan tavuttamaton merkkimäärä.

\komentoi{setlanghyphenmins}
\begin{koodilohkosis}
\setlanghyphenmins{finnish}{3}{2}
\end{koodilohkosis}

\noindent
Suomen kielellä vaihtoehtoisen tavutussäännöstön voi kytkeä päälle
käyttämällä kielen asettamisen yhteydessä valitsinta
\koodi{schoolhyphens}. Tämä sallii tavutuksen kaikista mahdollisista
kohdista. Lisätietoa suomen tavutuksen erityispiirteistä on luvussa
\ref{luku/suomi-tavutus}.

\komentoi{setdefaultlanguage}
\begin{koodilohkosis}
\setdefaultlanguage[schoolhyphens]{finnish}
\end{koodilohkosis}

\subsection{Babel ja tavutus}
\label{luku/tavutus-babel}

Kielipaketti \pakettictan{babel}\yipilkku\avctan{babel-finnish} (myös
luku \ref{luku/babel}) lisää automaattisesti moniin kieliin -- myös
suomen kieleen -- muutaman erikoismerkin, joilla voi vaikuttaa
tavutukseen. Erikoismerkit saa pois päältä käyttämällä paketin
lataamisen yhteydessä valitsinta \koodi{shorthands=\katk off}.
Seuraavassa on selitetty suomen kielen erikoismerkkejä.

\begin{maaritelma}{\koodi{#1}}
\item [\komento{-}] Tämä on Texin normaali tavutusvihje (luku
  \ref{luku/tavutuksen-merkit}), mutta \paketti{babel} määrittelee sen
  suomen kielessä uudelleen siten, että tavutus sallitaan sanassa myös
  muista kohdista tavutussääntöjen mukaan.

  \shorthandsyhdysmerkki

\item [\textquotedbl=] Sama kuin edellinen (\koodi{\textquotedbl-}).

  \shorthandspystyviiva

  \shorthandslainausmerkki

\end{maaritelma}

\leijutlk{
  \begin{tabular}{lll}
    \toprule
    \ots{Lähde}
    & \ots{Tavutus}
    & \ots{Merkitys} \\
    \midrule

    \koodi{erikoisalalla}
    & eri\tavukohta koi\tavukohta sa\tavukohta lal\tavukohta la
    & tavutus kaikista kohdista (väärin) \\

    \koodi{erikois\komento{-}alalla}
    & eri\tavukohta kois\tavukohta alal\tavukohta la
    & tavutusvihjeen kohdalta ja muualta \\

    \koodi{matka-aika}
    & matka-\tavukohta aika
    & vain yhdysmerkin jälkeen \\

    \koodi{matka-ai\komento{-}ka}
    & matka-\tavukohta ai\tavukohta ka
    & vain yhdysmerkki ja tavutusvihje \\

    \koodi{Oulu--Rovaniemi}
    & Oulu--\tavukohta Rovaniemi
    & vain ajatusviivan jälkeen \\

    \koodi{Oulu--Rova\komento{-}niemi}
    & Oulu--\tavukohta Rova\tavukohta nie\tavukohta mi
    & tavutusvihjeen kohdalta ja muualta \\

    \koodi{matka-}
    & matka-
    & ei tavutuskohtia \\

    \koodi{-aika}
    & -\tavukohta aika
    & vain yhdysmerkin jälkeen (väärin) \\

    \koodi{\textquotedbl-aika}
    & -ai\tavukohta ka
    & ei sitovan yhdysmerkin jälkeen \\

    \bottomrule
  \end{tabular}
}{
  \caption{Kielipaketti \paketti{babel} ja suomen kielellä
    tavutusvihjeen, yhdysmerkin ja ajatusviivan sekä kielipaketin omien
    erikoismerkkien vaikutus tavutukseen}
  \label{tlk/babel-tavutus}
}

\noindent
Taulukossa \ref{tlk/babel-tavutus} on esimerkkejä suomen kielen
tavutuksesta, kun sanassa on tavutusvihjeitä, yhdysmerkkejä,
ajatusviivoja tai \paketti{babel}\-/ paketin erikoismerkkejä.

Taulukon viimeinen rivi soittaa, että kun sanan alussa on yhdysmerkki,
täytyy suomen kielessä käyttää sitovaa yhdysmerkkiä, joka estää
tavutuksen heti yhdysmerkin jälkeen. Toisaalta \paketti{babel}\-/
paketissa on suomen kielelle asetus, joka käsittelee sanan alussa olevan
yhdysmerkin automaattisesti oikein. Ominaisuus kytketään päälle
kielipaketin lataamisen jälkeen \komento{babelprovide}\-/ komennolla
seuraavasti:

\komentoi{usepackage}
\pakettii{babel}
\komentoi{babelprovide}
\begin{koodilohkosis}
\babelprovide[transforms=prehyphen.nobreak]{finnish}
\end{koodilohkosis}

\noindent
Kun edellä mainittu asetus on päällä, voi lähdetiedostoon kirjoittaa
huoletta esimerkiksi \koodi{matkasuunnitelma ja \=/aika} (ilman sitovaa
yhdysmerkkiä), ja silti sanaan \emph{\mbox{-aika}} sisältyvä yhdysmerkki
pysyy kiinni sanassa eikä jää koskaan yksin rivin loppuun.

Yleisiä kaikkiin kieliin tai tiettyyn kieleen vaikuttavia
tavutusvihjeitä voi asettaa komennolla \komento{babelhyphenation}.
Komento vastaa Texin \komento{hyphenation}\-/ komentoa (luku
\ref{luku/hyphenation-komento}), mutta sille voi antaa valinnaisen
argumentin, joka rajaa vaikutuksen vain tiettyyn kieleen. Seuraavassa
esimerkissä asetetaan tavutusvihjeitä vain suomen kielen sanoille:

\komentoi{babelhyphenation}
\begin{koodilohkosis}
\babelhyphenation[finnish]{
  ala-indek-si alku-osa nimen-omaan typo-gra-fi-nen Latex
}
\end{koodilohkosis}

\noindent
Vaihtoehtoisen suomen kielen tavutussäännöstön voi kytkeä päälle
komennolla \komento{babelprovide}, seuraavan esimerkin mukaisesti. Tämä
vaihtoehto sallii sanan katkaisemisen kaikista mahdollisista suomen
kielen tavutuskohdista. Lisätietoa suomen tavutuksen erityispiirteistä
on luvussa \ref{luku/suomi-tavutus}.

\komentoi{babelprovide}
\begin{koodilohkosis}
\babelprovide[hyphenrules=schoolfinnish]{finnish}
\end{koodilohkosis}

\subsection{Viivojen erikoispaketti extdash}
\label{luku/tavutus-extdash}

Jos Texin ja kielipakettien välimerkit ja tavutusvihjeet eivät riitä,
mahdollisuuksia saa lisää \pakettictan{extdash}\-/ paketin avulla. Se
tuo mukanaan uusia komentoja viivavälimerkeille ja samalla lisää
vaihtoehtoja tavutuksen hallintaan. Komennot ovat sellaisia kuin
\komento{Hyphdash} ja \komento{Endash}, mutta niille on saatavilla myös
lyhyemmät vastineet, jos paketin lataa käyttämällä
\koodi{short\-cuts}\-/ valitsinta.

\komentoi{usepackage}
\pakettii{extdash}
\begin{koodilohkosis}
\usepackage[shortcuts]{extdash}
\end{koodilohkosis}

\noindent
Paketti sisältää kaksi lisävaihtoehtoa kolmelle viivavälimerkille eli
yhdysmerkille, lyhyelle ajatusviivalle ja pitkälle ajatusviivalle. Kun
Texin viivavälimerkit (luku \ref{luku/tavutuksen-merkit}) aina estävät
tavutuksen muualta kuin välimerkin jälkeen, \paketti{extdash}\-/ paketin
perusvaihtoehdot sallivat tavutuksen muualtakin. Lisäksi kaikille
kolmelle viivavälimerkille on sitova versio, joka estää tavutuksen
välimerkin jälkeen mutta sallii muualta.

Taulukossa \ref{tlk/extdash} ovat \paketti{extdash}\-/paketin tärkeimmät
komennot ja niiden merkitykset. Taulukossa \ref{tlk/extdash-vertailu}
vertaillaan \paketti{extdash}\-/paketin komentoja Texin vastaaviin.

\leijutlk{
  \begin{tabular}{ll}
    \toprule
    \ots{Komento} & \ots{Merkitys} \\
    \midrule
    \komento{-/} & tavutuksen salliva yhdysmerkki \\
    \komento{=/} & sitova, tavutuksen salliva yhdysmerkki \\
    \komento{--} & tavutuksen salliva lyhyt ajatusviiva \\
    \komento{==} & sitova, tavutuksen salliva lyhyt ajatusviiva \\
    \komento{---} & tavutuksen salliva pitkä ajatusviiva \\
    \komento{===} & sitova, tavutuksen salliva pitkä ajatusviiva \\
    \bottomrule
  \end{tabular}
}{
  \caption{\paketti{extdash}-paketin komentoja, kun paketti on ladattu
    käyttämällä asetusta \koodi{short\-cuts}}
  \label{tlk/extdash}
}

\leijutlk{
  \begin{tabular}{llll}
    \toprule
    \ots{Lähde} & \ots{Tavutus}
    & \ots{Lähde} & \ots{Tavutus} \\
    \cmidrule(r){1-2}
    \cmidrule(l){3-4}
    \koodi{matka-aika}
                & matka-\tavukohta aika
    & \koodi{Oulu--Rovaniemi}
                  & Oulu--\tavukohta Rovaniemi \\
    \koodi{matka\komento{-/}aika}
                & mat\tavukohta ka-\tavukohta ai\tavukohta ka
    & \koodi{Oulu\komento{--}Rovaniemi}
                  & Ou\tavukohta lu--\tavukohta Ro\tavukohta va\tavukohta
                    nie\tavukohta mi \\
    \koodi{matka\komento{=/}aika}
                & mat\tavukohta ka-ai\tavukohta ka
    & \koodi{Oulu\komento{==}Rovaniemi}
                  & Ou\tavukohta lu--Ro\tavukohta va\tavukohta
                    nie\tavukohta mi \\
    \koodi{matka-}
                & matka-
    & \koodi{-aika}
                  & -\tavukohta aika (väärin) \\
    \koodi{matka\komento{-/}}
                & mat\tavukohta ka-
    & \koodi{\komento{=/}aika}
                  & -ai\tavukohta ka \\
    \bottomrule
  \end{tabular}
}{
  \caption{Texin ja \paketti{extdash}-paketin viivamerkkien vertailua}
  \label{tlk/extdash-vertailu}
}

On tärkeää huomioida, että \paketti{extdash}\-/paketin komennot ovat
todellakin normaaleja komentoja. Se tarkoittaa, että komennon jälkeiset
sanavälit syödään pois. Tämä asia saattaa unohtua seuraavanlaisessa
tilanteessa, jossa käytetään tavutuksen sallivaa yhdysmerkkikomentoa:

\komentoi{-/}
\begin{koodilohkosis}
matka\-/ ja aika\-/arvio
\end{koodilohkosis}

\begin{tulossis}
  matka\-/ ja aika\-/arvio
\end{tulossis}

\noindent
Ensimmäisen yhdysmerkkikomennon jälkeinen sanaväli hävisi, ja syntyi
virheellinen sana \emph{matka-ja}. Sanavälin saa säilymään, kun
kirjoittaa komennon perään aaltosuljeparin tai kenoviivan sekä
sanavälin.

\komentoi{-/}
\begin{koodilohkosis}
matka\-/{} ja aika\-/arvio \\
matka\-/\ ja aika\-/arvio
\end{koodilohkosis}

\begin{tulossis}
  matka\-/{} ja aika\-/arvio \\
  matka\-/\ ja aika\-/arvio
\end{tulossis}

\subsection{Muita tavutusasetuksia}
\label{luku/tavutus-muut}

Varsinainen tavutuksen peruskomento on \komento{discretionary}, joka
mahdollistaa omanlaisten tavutuskohtien määrittelyn. Komento
kirjoitetaan sanassa juuri siihen kohtaan, johon tavutuskohta halutaan,
ja komennon argumenttien rakenne on seuraavanlainen:

\komentoi{discretionary}
\begin{koodilohkosis}
\discretionary{loppu}{alku}{katkaisematon}
\end{koodilohkosis}

\noindent
Komennon ensimmäinen argumentti \koodi{loppu} ilmaisee
katkaisutilanteessa rivin loppuun ladottavat merkit. Toinen argumentti
\koodi{alku} ilmaisee katkaisutilanteessa seuraavan rivin alkuun
ladottavat merkit, ja kolmas argumentti tarkoittaa katkaisemattomaan
sanaan ladottavia merkkejä. Normaali tavutuskohta sanaan \emph{tavu}
määriteltäisiin seuraavan esimerkin tavoin:

\komentoi{discretionary}
\begin{koodilohkosis}
ta\discretionary{-}{}{}vu
\end{koodilohkosis}

\noindent
Edellisessä esimerkissä komennon ensimmäinen argumentti on yhdysmerkki,
koska rivin loppuun tietenkin halutaan yhdysmerkki silloin, kun sana
katkaistaan tästä kohdasta. Toinen argumentti on tyhjä, koska seuraavan
rivin alkuun ei kirjoiteta suomen kielessä mitään ylimääräistä. Myös
kolmas argumentti on tyhjä, koska katkaisemattomaan sanaan ei haluta
mitään merkkiä tavujen väliin.

\komento{discretionary}\-/komennolla voi luoda myös tavutuskohtia,
joihin ei ilmesty yhdysmerkkiä eikä mitään muutakaan merkkiä
katkaisutilanteessa. Tämä vastaa kielipakettien tavutusvihjettä
\koodi{\textquotedbl\textquotedbl}. \komento{discretionary}\-/
komennolla sama toteutetaan jättämällä kaikki argumentit tyhjäksi:

\komentoi{discretionary}
\begin{koodilohkosis}
\discretionary{}{}{}
\end{koodilohkosis}

\noindent
Joissakin kielissä sanan katkaiseminen yhdysmerkin kohdalta vaatii, että
rivin loppuun kirjoitetaan yksi yhdysmerkki ja seuraavan rivin alkuun
toinen. Näin ilmaistaan, että sanassa on pysyvä yhdysmerkki eikä vain
väliaikaisesti katkaisun merkkinä. Tällaisia kieliä varten saattaa
kielipaketissa olla omat yhdysmerkkitoiminnot tai oikeanlainen
tavutuslogiikka voi jo sisältyä kieliasetuksiin, mutta tällaisen
tavutuskohdan saa myös seuraavalla tavalla:

\komentoi{discretionary}
\begin{koodilohkosis}
\discretionary{-}{-}{-}
\end{koodilohkosis}

\noindent
Tavutuksen suunnittelussa ja tutkimisessa voi auttaa
\pakettictan{showhyphenation}\-/ paketti. Kun paketti on ladattuna,
kaikkien sanojen mahdollisiin tavutuskohtiin ladotaan pieni merkki.
Paketti hyödyntää Lualatex\-/ kääntäjän ominaisuuksia, eikä se siis
toimi muiden kääntäjien kanssa.

\subsection{Suomen kielen tavutus}
\label{luku/suomi-tavutus}

Suomen kielelle on olemassa Latexissa kaksi erilaista tavutussäännöstöä.
Oletuksena toimivat säännöt kuvaavat tavallisimmat tavujen rakenteet
mutta pyrkivät huomioimaan myös typografisia suosituksia. Sanaa ei
esimerkiksi katkaista vokaalien välistä, koska se ei ole suositeltavaa.
Myös jotkin yleiset vierasperäisten sanojen konsonanttiyhdistelmät
pidetään yhdessä, vaikka ne suomen tavurakenteen mukaan kuuluisivat
joskus eri tavuihin.

Toinen, vaihtoehtoinen tavutussäännöstö kuvaa pelkästään suomen tavujen
rakenteen, eli se tavuttaa sanat kaikista mahdollisista kohdista eikä
huomioi typografisia suosituksia mitenkään.%
\footnote{Lisätietoa:
  \url{https://github.com/hyphenation/basic-finnish}} Asetus sanan
reunasta tavuttamisesta on kuitenkin voimassa (luku
\ref{luku/tavutus-reunasta}). Tämä vaihtoehtoinen tavutustapa täytyy
kytkeä erikseen päälle kielipaketin asetuksista. Sen nimi on Latexissa
\emph{\englanti{school}}, koska se on ''peruskoulutavuttamista'' eikä
typografian mukaista tavuttamista. Katso kielipakettikohtaiset ohjeet
luvuista \ref{luku/tavutus-polyglossia} ja \ref{luku/tavutus-babel}.

Texin kirjainyhdistelmiin perustuvat tavutussäännöt eivät yksinään ole
tarpeeksi älykkäitä suomen kieleen, ja esimerkiksi yhdyssanojen
rajakohdat tuottavat usein ongelmia. Sana \emph{alku\-osa} katkaistaan
Latexissa kohdista \emph{al-kuo-sa}. Se on kyllä oikein tavurakenteen
kannalta mutta käytännössä ongelmallinen. Tässä ei ole kyse \emph{uo}\-/
diftongista eli samaan tavuun kuuluvista vokaaleista, vaan yhdyssanan
rajalla on myös tavutuskohta (\emph{al-ku-o\=/sa}). Lisäksi sanaa ei
saisi katkaista siten, että siitä jää yksittäinen kirjain eri riville
(ei: \emph{o\=/sa}).

Parasta olisi katkaista suomen kielen yhdyssanat vain yhdysosien välistä
(\emph{alku-osa}). Muualtakin voi katkaista (\emph{al-ku-osa}), kunhan
sanasta eikä sen yhdysosasta ei jää yksittäinen kirjain eri riville.
Mielellään ei katkaista myöskään kahden vokaalin välistä, jos ne
kuuluvat samaan sanaan (ei: \emph{kau-emmin}). Joskus halutaan välttää
myös niin sanottuja orpotavuja eli sitä, että tekstikappaleen
viimeiselle riville jää vain yksi tavu.

Käytännössä siis suomenkielinen teksti ja hyvä typografia vaativat
välillä tavutusvihjeiden kirjoittamista. Yhdyssanojen osien väliin
tarvitaan tavutusvihje silloin, kun jälkimmäinen osa alkaa vokaalilla
tai useammalla kuin yhdellä konsonantilla. Joitakin tällaisia tapauksia
Tex osaa tavuttaa oikein ilman tavutusvihjeitäkin.

\komentoi{-}
\begin{koodilohkosis}
alku\-osa pusku\-traktori
\end{koodilohkosis}

\noindent
Sitova yhdysmerkki on suomen kielessä tarpeellinen, kun sana alkaa
yhdysmerkillä eli ennen yhdysmerkkiä on sanaväli. Esimerkiksi
ilmauksessa \emph{matkasuunnitelma ja \mbox{-aika}} ei riviä saa
katkaista sanassa \emph{\mbox{-aika}} olevan yhdysmerkin jälkeen. Täytyy
siis käyttää sitovaa yhdysmerkkiä eli merkkiä, joka sitoo yhdysmerkin
kiinni seuraavaan sanaan. Sitovan yhdysmerkin käyttöä neuvotaan luvuissa
\ref{luku/tavutus-polyglossia}, \ref{luku/tavutus-babel} ja
\ref{luku/tavutus-extdash}.

Yhdysmerkin tai ajatusviivan sisältävät pitkät yhdyssanat voivat vaatia
tavutuskohtien lisäämistä, koska yhdysmerkki ja ajatusviivat estävät
tavutuksen muualta kuin näiden merkkien jälkeen. Ilman tavutuskohtien
lisäämistä Texillä ei ehkä ole riittävästi vaihtoehtoja tekstikappaleen
rivittämiseen. Voi syntyä liian suuria sanavälejä, tai joistakin
riveistä tulee ylipitkiä, eli ne yltävät marginaalin puolelle.

Tavutuskohtia voi lisätä sopiviin kohtiin tavutusvihjeillä.
Vaihtoehtoisesti voi käyttää \paketti{extdash}\-/ paketin
välimerkkikomentoja (taulukko \ref{tlk/extdash}), jotka sallivat
tavutuksen kaikista kielen tavutussääntöjen mukaisista kohdista.
Seuraavassa esimerkissä on tavutuksen hallintaa sanalle
\emph{Molo\-tov--Ribben\-trop-sopi\-mus}:

\komentoi{-}
\komentoi{--}
\komentoi{-/}
\begin{koodilohkosis}
Molo\-tov--Ribben\-trop-sopi\-mus % tavutusvihjeet
Molotov\--Ribben\-trop\-/sopimus  % tavutuksen sallivat välimerkit
\end{koodilohkosis}

\noindent
\komento{discretionary}\-/komentoa (luku \ref{luku/tavutus-muut}) voi
hyödyntää suomen kielen sanoissa silloin, kun niissä on heittomerkki
erottamassa kahta tavurajan molemmin puolin olevaa samaa vokaalia,
esimerkiksi sanoissa \emph{vaa'an} ja \emph{liu'uttaa}. Normaalisti Tex
ei katkaise heittomerkin kohdalta lainkaan, eikä se olisi suomen
kielessä suositeltavaakaan, koska vokaalien välistä ei mielellään
katkaista sanaa. Jos tavutuksen kuitenkin haluaa myös heittomerkin
kohdalle, täytyy huolehtia, että tavutustilanteessa heittomerkki poistuu
ja sen paikalle tulee yhdysmerkki rivin loppuun. Sanan \emph{vaa'an}
voisi siis kirjoittaa seuraavalla tavalla:

\komentoi{discretionary}
\begin{koodilohkosis}
vaa\discretionary{-}{}{'}an
\end{koodilohkosis}

\noindent
Komennon kolmas argumentti on heittomerkki, koska se pitää latoa
tavurajalle silloin, kun sanaa ei katkaista tästä kohdasta. Mikäli
tällaisia tarvitsee paljon, on järkevää määritellä sitä varten
yksinkertaisempi komento, jota sitten käytetään sanoissa heittomerkin
sijasta.

\komentoi{discretionary}
\begin{koodilohkosis}
\newcommand{\hm}{\discretionary{-}{}{'}}
vaa\hm an, liu\hm uttaa
\end{koodilohkosis}

\noindent
Suomen kielessä käytetään heittomerkkiä myös taivutuspäätteen, liitteen
tai johtimen edellä silloin, kun sanavartalon kirjoitusasu päättyy
konsonanttiin mutta ääntöasu vokaaliin, esimerkiksi \emph{show'ssa}.
Näissä tilanteissa ei ole kyse tavurajasta vaan morfeemirajasta eli
merkityksellisten osien rajakohdasta. Tavurajakin voi sattua samaan
paikkaan, mutta tavutettaessa heittomerkki säilyy: \emph{show'-hun}.%
\footnote{Asiaa ei yleensä mainita kielenhuolto\-/oppaissa. Tieto
  perustuu Kielikello\-/lehden 2/2006 artikkeliin:
  \kulmaurl{https://www.kielikello.fi/-/lainausmerkit-}. Viittauspäivä
  6.7.2020.} Mikäli tällainen tavutuskohta halutaan mukaan, käytetään
sanassa tavallista tavutusvihjettä: \koodi{show'}\komento{-}\koodi{hun}.
Mieluummin ei kuitenkaan katkaista sanoja heittomerkin kohdalta.

\subsection{Tavutus pois päältä}

Tavutuksen voi kytkeä kokonaan pois päältä \paketti{polyglossia}\-/
kielipaketin komennolla \komento{disablehyphenation}. Tavutuksen saa
takaisin päälle komennolla \komento{enablehyphenation}.

Toinen vaihtoehto on käyttää fontin asetusta \koodi{Hyphen\-Char=\katk
  None} kirjainperheen määrittelyn yhteydessä. Oletuksena tavutus on
pois päältä tasalevyisestä fontista, eli esimerkiksi \komento{texttt}\-/
komennon argumenttina olevaa tekstiä ei tavuteta. Fontteja ja niiden
asetuksia käsitellään luvussa \ref{luku/kirjaintyypit}.

Käytännössä tavutus menee pois päältä myös silloin, kun tekstikappaleet
tasaa vain vasempaan reunaan eli tekee oikealle liehureunan komennolla
\komento{raggedright}. Tekstikappaleiden tasaamista ja palstan muotoa
käsitellään luvussa \ref{luku/kappaleen-tasaus}.
