% Tekijä:   Teemu Likonen <tlikonen@iki.fi>
% Lisenssi: Creative Commons Nimeä-JaaSamoin 4.0 Kansainvälinen (CC BY-SA 4.0)
% https://creativecommons.org/licenses/by-sa/4.0/legalcode.fi

\chapter{Erikoisdokumentit}

Kaikki dokumentit ja painotuotteet eivät ole samanlaista tekstivirtaa,
joka koostuu jäsennellystä rakenteesta, väliotsikoista ynnä muusta
sellaisesta. Tässä luvussa käsitellään dokumentteja, jotka vaativat
osittain toisenlaista rakennetta ja tekniikkaa kuin luvussa
\ref{luku/rakenne} on käsitelty.

\section{Esitysgrafiikka}
\label{luku/esitysgrafiikka}

Esitelmien ja muiden puhe\-/ esitysten tukena käytetään usein
esitysgrafiikkaohjelmia kuten \englanti{Microsoft Powerpointia} tai
\englanti{Libreoffice Impressiä}. Myös Latex sopii esitysgrafiikan
tekemiseen, ja se voi olla osaavissa käsissä jopa kaikkein nopein ja
typografisesti tyylikkäin työkalu esitysdiojen tuottamiseen.

Latexin tunnetuin ja monipuolisin esitysgrafiikkajärjestelmä on
\luokkactan{beamer}, joka on oma dokumenttiluokkansa. Sen ohjekirjassa
on reilu kaksisataa sivua, joten tässä oppaassa voidaan esitellä vain
perusasioita, joilla pääsee alkuun.

\begin{esimerkki*}
  \komentoi{author}
  \komentoi{date}
  \komentoi{documentclass}
  \komentoi{institute}
  \komentoi{item}
  \komentoi{maketitle}
  \komentoi{pause}
  \komentoi{section}
  \komentoi{setdefaultlanguage}
  \komentoi{title}
  \komentoi{usepackage}
  \luokkai{beamer}
  \pakettii{hyperref}
  \pakettii{polyglossia}
  \ymparistoi{enumerate}
  \ymparistoi{frame}

\begin{koodilohko}
\documentclass[aspectratio=169,t]{beamer}
\usepackage{polyglossia} \setdefaultlanguage{finnish}
\usepackage{hyperref}

\title{Esitysgrafiikkaa Latexilla}
\author{Teppo Tekijä}
\institute{Ladontatieteiden tiedekunta}
\date{13.9.2021}

\begin{document}

\section{Otsikkodia}

\begin{frame}
  \maketitle
\end{frame}

\section{Ensimmäinen väliotsikko}

\begin{frame}{Pääotsikko}{Alaotsikko}

  \pause Seuraavassa luetellaan jotakin: \pause

  \begin{enumerate}
  \item ensimmäinen \pause
  \item toinen \pause
  \item kolmas
  \end{enumerate}

\end{frame}

\section{Toinen väliotsikko}

\begin{frame}[c]

  Sisältö on keskitetty pystysuunnassa.

\end{frame}

\end{document}
\end{koodilohko}

  \caption{\luokka{beamer}\-/ dokumenttiluokan avulla tehdyn
    diaesityksen runko}
  \label{esim/beamer-runko}
\end{esimerkki*}

\subsection{Diaesityksen rakentaminen}

Esimerkissä \ref{esim/beamer-runko} on kokonainen Latexin lähdetiedosto
ja diaesityksen runko. Dokumenttiluokan \luokka{beamer} lataamisen
yhteydessä (rivi~1) määritellään valitsimella \koodi{aspectratio}, mitkä
ovat diojen mittasuhteet. Tässä esimerkissä arvoksi annetaan
\koodi{169}, joka tarkoittaa leveyden ja korkeuden suhdetta 16:9.
Oletusarvo on \koodi{43} eli kuvasuhde 4:3. Muita mahdollisia arvoja
ovat esimerkiksi \koodi{1610} (16:10) ja \koodi{54} (5:4).

Esimerkin ensimmäisellä rivillä oleva toinen valitsin \koodi{t}
(\englanti{top}) tarkoittaa, että diojen sisältö sijoitetaan dian
yläreunaan. Oletusasetus on \koodi{c} (\englanti{center}), joka
keskittää dian sisällön pystysuunnassa. Tämän asetuksen voi muuttaa
kuhunkin diaan käyttämällä \ymparisto{frame}\-/ ympäristön valinnaista
argumenttia, kuten esimerkin rivillä 34 on tehty.

Riveillä 5--8 määritellään dokumentin perustiedot (luku
\ref{luku/dokumentin-perustiedot}), mutta perus Latexiin verrattuna
mukana on uusi komento \komento{institute}, jolla voi määritellä tekijän
edustaman laitoksen. Nämä perustiedot ladotaan näkyviin vasta
\komento{maketitle}\-/ komennolla, joka on esimerkin rivillä~15.

Jokainen yksittäinen sivu eli dia täytyy kirjoittaa \ymparisto{frame}\-/
ympäristön sisään. Ympäristö siis aloittaa aina puhtaan sivun.
Ympäristön alussa voi olla aaltosulkeissa sivun pääotsikko ja toisissa
aaltosulkeissa alaotsikko. Esimerkin rivillä 20 on tehty juuri näin.
Toisen tai molemmat otsikot voi jättää poiskin.

Sivun eli \ymparisto{frame}\-/ ympäristön sisällä voi olla tavallista
tekstiä ja käyttää suunnilleen normaaleja Latexin komentoja. Esimerkissä
\ref{esim/beamer-runko} on käytetty muutaman kerran \komento{pause}\-/
komentoa, joka tekee esitykseen tauon. Käytännössä se jakaa sivun
sisällön erillisiksi pdf\-/ sivuiksi, minkä avulla sisällön voi
paljastaa yleisölle vaihe vaiheelta. Esimerkin riveillä 24--28 käytetään
luetelmaa eli \ymparisto{enumerate}\-/ ympäristöä. Se toimii suunnilleen
samoin kuin perus Latexin vastaava ympäristö (luku
\ref{luku/luetelma-perus}), mutta \luokka{beamer}\-/ versiossa on hieman
enemmän ominaisuuksia.

Tästä esimerkistä puuttuu hyödyllinen \ymparisto{block}\-/ ympäristö,
jolla saa dian sisälle pienen väliotsikon ja siihen kuuluvan
tekstikokonaisuuden. Otsikon teksti annetaan ympäristön argumentissa:

\ymparistoi{block}
\begin{koodilohkosis}
\begin{block}{Väliotsikko}
  ...
\end{block}
\end{koodilohkosis}

\noindent
Otsikkokomennot \komento{chapter}, \komento{section},
\komento{subsection} ym. täytyy kirjoittaa \ymparisto{frame}\-/
ympäristöjen ulkopuolelle. \luokka{beamer}\-/ luokassa otsikkokomennot
eivät lado mitään itse dokumenttiin, vaan ne ainoastaan muodostavat
pdf\-/ tiedoston sisällysluettelon. Tosin sisällysluettelon kaltaista
rakennetietoa on mahdollista saada näkyviin itse diojen reunoillekin.
Niiden tarkoituksena on helpottaa esimerkiksi pitkän diaesityksen
seuraamista ja jäsentämistä. Näistä ominaisuuksista voi lukea lisää
\luokka{beamer}\-/ luokan ohjekirjasta.

Dioihin voi lisätä kuvatiedostoja tai vektorigrafiikkaa normaalin
Latexin tavoin eli luvun \ref{luku/grafiikka} ohjeilla. Palstoja ei
kuitenkaan kannata toteuttaa luvun \ref{luku/palstat} keinoilla vaan
\luokka{beamer}\-/ luokan oman \ymparisto{columns}\-/ ympäristön avulla.

\subsection{Ulkoasuteemat}

\luokka{beamer}\-/ dokumenttiluokka sisältää valmiita ulkoasuteemoja eli
ulkoasuun vaikuttavia asetusten kokonaisuuksia. Käyttämällä valmista
teemaa saa helposti käyttöönsä jonkun henkilön suunnitteleman tyylikkään
kokonaisuuden.

Yleinen teemojen valintakomento on \komento{usetheme}. Sillä valitaan
teema, joka voi määritellä vähän kaikkea diaesityksen ulkoasuun
liittyvää: fontit, värit, sisällön asettelua ja diojen reunoille
ladottavaa lisätietoa. Alemmantasoisilla teemakomennoilla valitaan vain
jonkin pienemmän osa-alueen teema. Esimerkiksi komennoilla
\komento{usecolortheme} ja \komento{usefonttheme} valitaan vain väri-
tai fonttiteema.

Fontit tai kirjainperheet sinänsä määritellään ja otetaan käyttöön
samalla tavalla kuin Latexissa muutenkin (luku
\ref{luku/kirjaintyypit}), mutta \luokka{beamer}\-/ luokan teema voi
määritellä, mitä kirjainperhettä tai \=/leikkausta käytetään missäkin
tilanteessa, esimerkiksi diojen otsikoissa tai leipätekstissä. Näiden
asetusten muuttaminen voi vaatia \luokka{beamer}\-/ luokan omia
komentoja, joista annetaan tietoa luvussa \ref{luku/beamer-asetuksia}.

Alemmantasoisia teemakomentoja ovat myös \komento{useinnertheme} ja
\komento{useoutertheme}. Niillä vaikutetaan erilaisiin diojen
sisältöelementtien (\englanti{inner}) ulkoasuun ja diojen reunoille
(\englanti{outer}) ladottavaan lisätietoon.

\leijutlk{
  \providecommand{\rivi}{}
  \renewcommand{\rivi}[2]{\midrule \komento{#1} & #2 \\}
  \providecommand{\teema}{}
  \renewcommand{\teema}[2][,]{\mbox{\koodi{#2}#1}}

  \begin{tabularx}{\linewidth}{lL}
    \toprule
    \ots{Komento} & \ots{Valmiita teemoja} \\
    \rivi{usetheme}{
      \teema{default}
      \teema{boxes}
      \teema{Bergen}
      \teema{Boadilla}
      \teema{Madrid}
      \teema{AnnArbor}
      \teema{CambridgeUS}
      \teema{EastLansing}
      \teema{Pittsburgh}
      \teema{Rochester}
      \teema{Antibes}
      \teema{JuanLesPins}
      \teema{Montpellier}
      \teema{Berkeley}
      \teema{PaloAlto}
      \teema{Goettingen}
      \teema{Marburg}
      \teema{Hannover}
      \teema{Berlin}
      \teema{Ilmenau}
      \teema{Dresden}
      \teema{Darmstadt}
      \teema{Frankfurt}
      \teema{Singapore}
      \teema{Szeged}
      \teema{Copenhagen}
      \teema{Luebeck}
      \teema{Malmoe}
      \teema[]{Warsaw}
    }

    \rivi{usecolortheme}{
      \teema{default}
      \teema{structure}
      \teema{sidebartab}
      \teema{albatross}
      \teema{beetle}
      \teema{crane}
      \teema{dove}
      \teema{fly}
      \teema{monarca}
      \teema{seagull}
      \teema{wolverine}
      \teema{beaver}
      \teema{spruce}
      \teema{lily}
      \teema{orchid}
      \teema{rose}
      \teema{whale}
      \teema{seahorse}
      \teema{dolphin}
      \teema[]{seahorse}
    }

    \rivi{usefonttheme}{
      \teema{default}
      \teema{serif}
      \teema{structurebold}
      \teema{structureitalicserif}
      \teema[]{structuresmallcapsserif}
    }

    \rivi{useinnertheme}{
      \teema{default}
      \teema{circles}
      \teema{rectangles}
      \teema{rounded}
      \teema[]{inmargin}
    }

    \rivi{useoutertheme}{
      \teema{default}
      \teema{infolines}
      \teema{miniframes}
      \teema{smoothbars}
      \teema{sidebar}
      \teema{split}
      \teema{shadow}
      \teema{tree}
      \teema[]{smoothtree}
    }
    \bottomrule
  \end{tabularx}
}{
  \caption{\luokka{beamer}\-/ luokan teemanvalintakomennot ja valmiita
    teemoja}
  \label{tlk/usetheme}
}

Taulukkoon \ref{tlk/usetheme} on koottu \luokka{beamer}\-/ luokan
teemakomennot ja valmiita teemoja. Taulukon ensimmäisessä sarakkeessa
ovat teeman valintakomennot ja toisessa sarakkeessa teemojen nimiä,
joita voi antaa komennolle argumentiksi. Ensimmäisenä mainittu
\koodi{default}\-/ teema on käytössä oletuksena. Teemat otetaan käyttöön
kirjoittamalla lähdedokumentin esittelyosaan teemakomentoja, esimerkiksi
seuraavalla tavalla:

\komentoi{usetheme}
\komentoi{usecolortheme}
\komentoi{usefonttheme}
\begin{koodilohkosis}
\usetheme{Bergen}
\usecolortheme{albatross}
\usefonttheme{serif}
\end{koodilohkosis}

\noindent
Teemanvalintakomennoille voi antaa hakasulkeissa valinnaisen argumentin,
jonka avulla määritetään kyseisen teeman asetuksia, jos teema sellaisia
tukee. Asetukset ovat teemakohtaisia, ja niistä voi lukea lisää
dokumenttiluokan ohjekirjasta. Seuraavassa on yleisesimerkki komentojen
rakenteesta:

\komentoi{usetheme}
\komentoi{useoutertheme}
\begin{koodilohkosis}
\usetheme[valitsin]{teeman nimi}
\useoutertheme[valitsin=arvo]{teeman nimi}
\end{koodilohkosis}

\noindent
Fonttiteema \koodi{serif} vaihtaa koko diaesityksen kirjainperheeksi
antiikvan. Sen myötä oletuksena olevaa groteskia (\englanti{sans serif})
ei siis käytetä enää missään. \koodi{serif}\-/ teema sisältää kuitenkin
muutaman valitsimen, joilla tähän voi tehdä pieniä hyödyllisiä
poikkeuksia. Valitsimia on koottu taulukkoon
\ref{tlk/beamer-serif-teema}, ja niitä voi antaa useampia kerralla.

Seuraava esimerkki asettaa groteskin suuriin elementteihin eli
otsikoihin ja pieniin elementteihin eli dian reunojen mahdollisiin
lisätietoihin. Muualla käytetään antiikvaa.

\komentoi{usefonttheme}
\begin{koodilohkosis}
\usefonttheme[stillsansseriflarge, stillsansserifsmall]{serif}
\end{koodilohkosis}

\leijutlk{
  \providecommand{\rivi}{}
  \renewcommand{\rivi}[2]{\koodi{#1} & #2 \\}

  \begin{tabularx}{\linewidth}{lL}
    \toprule
    \ots{Valitsin} & \ots{Merkitys} \\
    \midrule
    \rivi{stillsansseriflarge}{suuret elementit eli otsikot käyttävät
    groteskifonttia}
    \rivi{stillsansserifsmall}{pienet elementit kuten dian reuna\-/
    alueiden lisätiedot käyttävät groteskia}
    \rivi{stillsansseriftext}{normaali teksti groteskilla}
    \rivi{stillsansserifmath}{matematiikkatilassa groteski}
    \bottomrule
  \end{tabularx}
}{
  \caption{\luokka{beamer}\-/ luokan \koodi{serif}\-/ fonttiteeman
    asetusvalitsimia}
  \label{tlk/beamer-serif-teema}
}

\subsection{Muita asetuksia}
\label{luku/beamer-asetuksia}

Jos haluaa vaikuttaa \luokka{beamer}\-/ dokumenttien otsikoiden tai
muiden rakenteellisten osien fontteihin ja väreihin, täytyy käyttää
dokumenttiluokan omia komentoja. Niiden käyttöä neuvotaan tässä
alaluvussa. Sen sijaan diojen sisällä väliaikaiset kirjaintyypin tai
\=/leikkauksen muutokset tehdään samalla tavalla kuin Latexissa
muutenkin. Fonttien tekniikkaa käsitellään luvussa
\ref{luku/kirjaintyypit} ja tekstin korostamisen typografiaa luvussa
\ref{luku/korostus}.

\luokka{beamer}\-/ luokan erityisten tekstielementtien kuten otsikoiden
fonttiin voi vaikuttaa komennolla \komento{setbeamerfont}. Sen
ensimmäinen argumentti on tekstielementin nimi ja toinen argumentti
sisältää fonttiasetukset. Esimerkiksi seuraava komento muuttaa
aloitusdian (\komento{maketitle}) otsikon kirjainperheen ja
\=/leikkauksen:

\komentoi{setbeamerfont}
\begin{koodilohkosis}
\setbeamerfont{title}{family=\rmfamily, series=\bfseries,
  shape=\itshape, size=\Huge}
\end{koodilohkosis}

\leijutlk{
  \providecommand{\rivi}{}
  \renewcommand{\rivi}[2]{\koodi{#1} & #2 \\}

  \begin{tabular}{ll}
    \toprule
    \ots{Tekstielementti} & \ots{Merkitys} \\
    \midrule
    \rivi{title}{aloitusdian otsikko (\komento{maketitle})}
    \rivi{frametitle}{diojen otsikot}
    \rivi{framesubtitle}{diojen alaotsikot}
    \rivi{blocktitle}{\ymparisto{block}\-/ ympäristön otsikot}
    \rivi{normal text}{normaali teksti diojen sisällä}
    \rivi{footnote}{alaviitteet}
    \rivi{item}{luetelmien luetelmamerkit}
    \bottomrule
  \end{tabular}
}{
  \caption{\luokka{beamer}\-/ luokan tekstielementtejä. Elementtien
    nimiä käytetään esimerkiksi komentojen \komento{setbeamerfont} ja
    \komento{setbeamercolor} kanssa}
  \label{tlk/beamer-tekstielementtejä}
}

\noindent
Aloitusdian otsikko on tekstielementti nimeltä \koodi{title}, ja siksi
se oli edellisen komennon ensimmäisenä argumenttina. Muita oleellisia
tekstielementtejä on koottu taulukkoon
\ref{tlk/beamer-tekstielementtejä}.

Tekstielementtien väreihin vaikutetaan komennolla
\komento{setbeamercolor}, joka toimii lähes samalla tavalla kuin edellä
esitelty komento \komento{setbeamerfont}. Värin asettamisessa käytetään
valitsimia \koodi{fg} ja \koodi{bg}, joista ensin mainittu vaihtaa
varsinaisen värin (\englanti{foreground}) ja jälkimmäinen taustavärin
(\englanti{background}). Seuraavassa esimerkissä vaihdetaan diojen
otsikon teksti valkeaksi ja tausta siniseksi:

\komentoi{setbeamercolor}
\begin{koodilohkosis}
\setbeamercolor{frametitle}{fg=white, bg=blue}
\end{koodilohkosis}

\noindent
Värien nimien täytyy olla ennalta määriteltyjä. Perusvärit on määritelty
valmiiksi, mutta lisää värejä voi määritellä ohjeilla, joita kerrotaan
luvussa \ref{luku/korostus-värit}.

Oletuksena luetelmaympäristöt \ymparisto{itemize} ja
\ymparisto{enumerate} latovat luetelmamerkit eri värillä kuin normaalin
tekstin. Väri riippuu käytetystä väriteemasta (\komento{usecolortheme}).
Jos haluaa, että luetelmamerkit ovat samalla värillä kuin normaali
teksti, kannattaa käyttää seuraavan esimerkin komentoa:

\komentoi{setbeamercolor}
\begin{koodilohkosis}
\setbeamercolor{item}{parent={normal text}}
\end{koodilohkosis}

\noindent
Edellisessä esimerkissä viitataan \koodi{item}\-/ nimiseen
tekstielementtiin, joka tarkoittaa luetelmamerkkejä. Valitsin
\koodi{parent} tarkoittaa, että väri peritään toiselta
tekstielementiltä, tässä tapauksessa normaalilta tekstiltä
(\koodi{\englanti{normal text}}). Toki tässäkin voi käyttää valitsimia
\koodi{fg} ja~\koodi{bg}.

Normaalin tekstin taustaväri tarkoittaa koko dian tekstialueen
taustaväriä. Seuraavassa on käytännön esimerkki, joka muuttaa diojen
otsikon taustan kirkkaan vihreäksi (70\,\%) ja tekstialueen taustan
vaalean vihreäksi (30\,\%).

\komentoi{setbeamercolor}
\begin{koodilohkosis}
\setbeamercolor{frametitle} {fg=black, bg=green!90}
\setbeamercolor{normal text}{fg=black, bg=green!30}
\end{koodilohkosis}

\noindent
Numeroimattomien luetelmien eli \ymparisto{itemize}\-/ ympäristön
luetelmamerkki on \luokka{beamer}\-/ dokumenttiluokassa oletuksena
kolmionmuotoinen, mutta asetusta voi muuttaa komennolla
\komento{setbeamertemplate}. Tällä komennolla tehdään sekalaisia diojen
asetuksia, joista annetaan tässä yhteydessä vain pari esimerkkiä.
Luetelmamerkki vaihdetaan seuraavan esimerkin komennoilla.

\komentoi{setbeamertemplate}
\begin{koodilohkosis}
\setbeamertemplate{itemize item}[circle]          % perustaso
\setbeamertemplate{itemize subitem}[square]       % toinen taso
\setbeamertemplate{itemize subsubitem}[triangle]  % kolmas taso
\end{koodilohkosis}

\noindent
Vielä monipuolisemmin voi luetelmamerkkeihin vaikuttaa, kun vaihtaa
hakasulkeissa olevan argumentin tilalle aaltosulkeet. Tällaiseen
argumenttiin voi kirjoittaa suunnilleen mitä hyvänsä Latex\-/ komentoja,
joilla luetelmamerkki tuotetaan. Seuraavassa esimerkissä tätä voimakasta
komentomuotoa käytetään maltillisesti pelkästään ajatusviivan
tuottamiseen.

\komentoi{setbeamertemplate}
\begin{koodilohkosis}
\setbeamertemplate{itemize item}{--}
\end{koodilohkosis}

\noindent
\komento{setbeamertemplate}\-/ komennolla voi asettaa myös diojen
oikeaan alareunaan ladottaviin navigointisymboleihin. Tässä yhteydessä
ei käsitellä niitä sen syvällisemmin, mutta navigointisymbolien
poistaminen onnistuu helposti seuraavalla komennolla:

\komentoi{setbeamertemplate}
\begin{koodilohkosis}
\setbeamertemplate{navigation symbols}{}
\end{koodilohkosis}

\section{Kirjeet}
\label{luku/kirjeet}

Latexin \luokka{letter}\-/ dokumenttiluokka on tarkoitettu kirjeiden
latomiseen. Tyylillisesti se soveltuu ehkä paremmin virallisiin
kirjeisiin kuin henkilökohtaisiin. Esimerkiksi joukkojakelukirjeiden
tekemiseen se voi olla käytännöllinen, koska tietokoneohjelman avulla
voi helposti tuottaa Latex\-/ koodia eli lähes samanlaisia kirjeitä eri
vastaanottajille.

\begin{esimerkki*}
  \komentoi{address}
  \komentoi{cc}
  \komentoi{closing}
  \komentoi{documentclass}
  \komentoi{encl}
  \komentoi{makelabels}
  \komentoi{opening}
  \komentoi{ps}
  \komentoi{signature}
  \komentoi{usepackage}
  \luokkai{letter}
  \ymparistoi{letter}

\begin{koodilohko}
\documentclass{letter}
\usepackage[a4paper]{geometry}
\usepackage{polyglossia}
\setdefaultlanguage{finnish}

\address{Liisa Lähettäjä \\ Katuosoite 1 \\ 00000 Kaupunki}
\signature{Liisa Lähettäjä}

\makelabels

\begin{document}

\begin{letter}{Virpi Vastaanottaja \\ Tiennimi 3 \\ 99999 Kunta}

  \opening{Hei!}

  Tässä kirjeessä ei ole kovin mielenkiintoista sisältöä, mutta tähän
  kohtaan se kirjoitettaisiin.

  \closing{Terveisin}

  \ps{Jk. Tässä on kirjeen jälkikirjoitus.}

  \cc{Mauno Muuhenkilö}
  \encl{lippu, lappu, paperi}

\end{letter}

\end{document}
\end{koodilohko}

  \caption{Latexin \luokka{letter}\-/ dokumenttiluokka on tarkoitettu
    kirjeiden latomiseen}
  \label{esim/letter}
\end{esimerkki*}

Esimerkissä \ref{esim/letter} on yhden kirjeen runko ja oleelliset
komennot. Dokumentin esittelyosassa olevalla \komento{address}\-/
komennolla määritellään lähettäjän nimi ja osoitetiedot. Ne ladotaan
jokaisen kirjeen alkuun. Myös komennolla \komento{signature} ilmaistaan
lähettäjän nimi eli allekirjoitus, joka ladotaan kirjeen loppuun.
Komento \komento{makelabels} aiheuttaa sen, että kaikkien kirjeiden
jälkeen ladotaan sivu (tai useampia), jossa ovat kaikkien
vastaanottajien osoitetiedot. Tämä on tarkoitettu osoitetarrojen
tulostamiseen.

Varsinainen kirjeen sisältö toteutetaan \ymparisto{letter}\-/ ympäristön
avulla. Ympäristölle annetaan yksi argumentti, jossa on kyseisen kirjeen
vastaanottajan nimi ja osoitetiedot. Itse kirje alkaa
\komento{opening}\-/ komennolla, jolla ilmaistaan tervehdys tai muut
kirjeen aloitussanat.

Kirjeen lopussa \komento{closing}\-/ komennolla on sopivaa ilmaista
lopputoivotus tai muu vastaava. Lähettäjän nimi ladotaan sen jälkeen
automaattisesti, jos lähettäjä on ilmaistu aiemmin
\komento{signature}\-/ komennolla. Mahdollisen jälkikirjoituksen voi
ilmaista komennolla \komento{ps}, kirjeen jakelutietoja komennolla
\komento{cc} ja liitteet komennolla \komento{encl}.

Yksi Latex\-/ lähdedokumentti voi sisältää useita \ymparisto{letter}\-/
ympäristöjä, ja jokainen niistä muodostaa erillisen, uudelta sivulta
alkavan kirjeen. Kaikkiin kirjeisiin ladotaan sama lähettäjä, ellei
lähettäjätietoja välillä vaihda \komento{address}\-/\ ja
\komento{signature}\-/ komennoilla.
