% Tekijä:   Teemu Likonen <tlikonen@iki.fi>
% Lisenssi: Creative Commons Nimeä-JaaSamoin 4.0 Kansainvälinen (CC BY-SA 4.0)
% https://creativecommons.org/licenses/by-sa/4.0/legalcode.fi

\chapter{Esipuhe}

Tex ja Latex kehitettiin alun perin parantamaan tietokoneavusteista
tekstin latomista. Ohjelmat syntyivät 1970\-/ luvun lopulla ja 1980\-/
luvun alkupuolella eli ajalla, jolloin tietokoneilla ei vielä kovin
hyvin osattu tuottaa laadukasta typografiaa ja painotuotteita. Texin
alkuperäinen luoja on Donald Knuth, ja Latexin kehitti Leslie Lamport.

Texin ja Latexin myötä tietokonetypografia parani: tekstieditorin ja
merkintäkielen avulla kirjoittaja pystyi varsin helposti tuottamaan
hyvää jälkeä niin kuin typografian ja ladonnan ammattilaiset aikoinaan.
Varsinkin akateemisten tekstien tuottamisessa Latex sai vankan
jalansijan. Kirjoittaja keskittyy lähinnä sisältöön, ja
tietokoneohjelmat hoitavat taittamisen eli ulkoasun suunnittelun (Latex)
ja lopullisen dokumentin latomisen (Tex). Typografinen ammattitaito on
ohjelmoitu sisään näihin tietokoneohjelmiin.

Sitten tulivat tekstinkäsittelyohjelmat, taitto\-/ ohjelmat ja uusi
käsite \textsc{wy\-si\-wyg} eli \englantik{what you see is what you
  get}. Suoraan tietokoneen ruudulta eli graafisesta tietokoneohjelmasta
näki lopullisen painotuotteen sellaisenaan. Mikä sen helpompaa? Jokainen
kirjoittaja oli nyt samassa hetkessä myös latoja, joka heittelee
kirjakkeita tietokoneen ruudulle peräkkäin: sanoiksi, riveiksi,
palstoiksi ja lopulta valmiiksi dokumentiksi. Tekstien parissa
työskentely helpottui, mutta typografiaa tämä kehitys ei parantanut,
koska yhä useammat kirjoittajat saivat vastuulleen myös ulkoasun
suunnittelun ja lopullisten dokumenttien valmistuksen. Vuosisatojen
aikana kehittynyt typografinen osaaminen ei enää sisältynyt
tietokoneohjelmaan.

Tässä tilanteessa olemme edelleenkin: lähes jokainen kirjoittaja vastaa
itse niin sisällöstä kuin typografiastakin eli lopullisesta ulkoasusta.
Jokainen voi olla tekstinsä julkaisija. Visuaalisten
tekstinkäsittely\-/\ ja taitto\-/ ohjelmien yleistyminen ei kuitenkaan
lopettanut Latexin tarinaa. Latexin kehitys ei jäänyt 1980\-/ luvulle,
vaan sen ympärillä tapahtuu jatkuvasti edelleenkin. Esimerkiksi
nykyaikainen fonttitekniikka ja tietokoneiden laajat merkistöt ovat myös
Latexin käyttäjän arkipäivää.

Jos siis typografia, laadukkaat julkaisut ja niihin liittyvä tekniikka
kiinnostavat, on Latexilla varmasti paljon annettavaa nykyajallekin.
Toivottavasti tästä oppaasta on apua tutustumismatkassa -- ja sen
jälkeenkin. Tervetuloa mukaan!
