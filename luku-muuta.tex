% Tekijä:   Teemu Likonen <tlikonen@iki.fi>
% Lisenssi: Creative Commons Nimeä-JaaSamoin 4.0 Kansainvälinen (CC BY-SA 4.0)
% https://creativecommons.org/licenses/by-sa/4.0/legalcode.fi

\chapter{Muuta tekniikkaa}

Tähän päälukuun on koottu sekalaisia ohjeita, jotka eivät oikein sovi
mihinkään aiempaan päälukuun. Kyse on sellaisesta Latexin tekniikasta,
jota voi hyödyntää monenlaisissa dokumenteissa ja tilanteissa.

\section{Päiväykset ja kellonajat}

Latexissa on muutama komento, joilla voi latoa dokumenttiin
automaattisesti kääntämishetken päivämäärän. Sellaista voidaan tarvita
esimerkiksi dokumentin kansi- tai nimiösivulle ilmaisemaan
julkaisuajankohtaa. Yleisimmin käytetty komento lienee \komento{today},
joka latoo nykyisen kielivalinnan mukaisen pitkän päivämäärän:

\komentoi{today}
\begin{koodilohkosis}
\today
\end{koodilohkosis}

\begin{tulossis}
  12. kesäkuuta 2023
\end{tulossis}

\noindent
Edellä on esimerkki suomen kielen kokonaisesta päiväysmerkinnästä.
Muilla kielillä komento tuottaa tietenkin jotain muuta. Päivämäärään
sisältyvät lukuarvot voi latoa kahden komennon yhdistelmillä: ensin
annetaan komento \komento{the} ja sen jälkeen päivää, kuukautta tai
vuotta ilmaiseva komento, kuten seuraavasta esimerkistä ilmenee.

\komentoi{the}
\komentoi{day}
\komentoi{month}
\komentoi{year}
\begin{koodilohkosis}
\the\day.\the\month.\the\year
\end{koodilohkosis}

\begin{tulossis}
  12.6.2023
\end{tulossis}

\noindent
Monipuolisempia ajanilmauksia voi latoa paketin \pakettictan{datetime2}
avulla. Se sisältää paljon komentoja ajanilmausten tuottamiseen eri
muodoissa ja eri kielillä. \paketti{datetime2}\-/ paketti pitäisi ladata
kieliasetusten (luku \ref{luku/kieliasetukset}) ja kielipaketin jälkeen,
ja lataamisen yhteydessä voi asettaa, minkä kielten mukaisesti
ajanilmaukset halutaan latoa. Seuraavassa esimerkissä ladataan suomen
(\koodi{finnish}) ja brittienglannin (\koodi{en-GB}) ajanilmaukset sekä
asetetaan, että kellonajoissa ei ole sekunteja mukana.

\pakettii{datetime2}
\begin{koodilohkosis}
% Ensin kielipaketti polyglossia tai babel. Sitten:
\usepackage[finnish, en-GB, showseconds=false]{datetime2}
\end{koodilohkosis}

\noindent
Ajanilmausten kielen tai muun tyylin voi muuttaa komennolla
\komento{DTMsetstyle}, kuten seuraava esimerkki osoittaa. Dokumentin
kääntämishetken päiväys ladotaan komennolla \komento{DTMtoday}, joka
vastaa perus Latexin \komento{today}\-/ komentoa. Nykyhetken kellonaika
saadaan komennolla \komento{DTMcurrenttime}.

\komentoi{DTMsetstyle}
\komentoi{DTMtoday}
\komentoi{DTMcurrenttime}
\begin{koodilohkosis}
\DTMsetstyle{finnish}
Tänään on \DTMtoday\ ja kello on \DTMcurrenttime. \\
\DTMsetstyle{en-GB}
Today is \DTMtoday\ and the time is \DTMcurrenttime.
\end{koodilohkosis}

\begin{tulossis}
  Tänään on 12. kesäkuuta 2023 ja kello on 10.46. \\
  Today is 12th June 2023 and the time is 10:46am.
\end{tulossis}

\noindent
Pitkän päivämäärän sijasta on mahdollista latoa myös pelkistä luvuista
koostuva päiväys. Tämä vaatii, että ajanilmausten tyyliä muutetaan
komennolla \komento{DTMsetstyle}. Esimerkiksi suomen kielelle on
olemassa tyyli \koodi{finnish-numeric} eli numeeristen päiväysten tyyli.

\komentoi{DTMsetstyle}
\komentoi{DTMtoday}
\begin{koodilohkosis}
\DTMsetstyle{finnish-numeric}
\DTMtoday
\end{koodilohkosis}

\begin{tulossis}
  12.6.2023
\end{tulossis}

\noindent
\paketti{datetime2}\-/ paketissa on myös komentoja minkä tahansa
päivämäärän tai kellonajan latomiseen, ei pelkästään nykyhetken.
Peruskomento päiväyksille on \komento{DTMdisplaydate}, jota käytetään
seuraavasti:

\komentoi{DTMdisplaydate}
\begin{koodilohkosis}
\DTMdisplaydate{vuosi}{kuukausi}{päivä}{viikonpäivä}
\end{koodilohkosis}

\noindent
Komennon kaikki argumentit ovat lukuja. Viimeinen argumentti on
viikonpäivän numero: \koodi{0}~on maanantai, \koodi{1}~on tiistai jne.
Kaikki ajanilmaustyylit tai \=/kielet eivät huomioi viikonpäivää
mitenkään. Esimerkiksi suomen kielellä sitä ei huomioida. Argumentiksi
voi asettaa \koodi{-1}, jolloin viikonpäivä jätetään aina huomioimatta.

\komentoi{DTMsetstyle}
\komentoi{DTMdisplaydate}
\begin{koodilohkosis}
\DTMsetstyle{finnish} Suomi julistautui itsenäiseksi
\DTMdisplaydate{1917}{12}{6}{-1}. \\
\DTMsetstyle{en-GB} Finland declared its independence on
\DTMdisplaydate{1917}{12}{6}{-1}.
\end{koodilohkosis}

\begin{tulossis}
  Suomi julistautui itsenäiseksi 6. joulukuuta 1917. \\
  Finland declared its independence on 6th December 1917.
\end{tulossis}

\noindent
Yleensä lienee kätevämpää kirjoittaa kiinteät päivämäärät ja muut
ajanilmaukset ihan normaalina tekstinä eikä minkään komennon avulla.
Komento \komento{DTMdisplaydate} voi kuitenkin olla hyödyllinen silloin,
kun saman Latex\-/ koodin avulla tuotetaan ajanilmauksia useiden
erikielisten dokumenttien osaksi. Tai ehkä kirjoittaja ei tunne
kohdekielen ajanilmausten merkintätapoja ja luottaa, että
\paketti{datetime2}\-/ paketin avulla ne saa oikeanlaiseksi~(?).

Mitä hyvänsä kellonaikoja saa ladottua komennolla
\komento{DTMdisplaytime}, joka huomioi voimassa olevat kieli- tai
tyyliasetukset. Komennolle annetaan kolme argumenttia: tunnit, minuutit
ja sekunnit. Sekunteja ei kuitenkaan näytetä, jos \paketti{datetime2}\-/
paketin lataamisen yhteydessä on käytetty asetusta
\koodi{showseconds=\katk false}.

\komentoi{DTMsetstyle}
\komentoi{DTMdisplaytime}
\begin{koodilohkosis}
\DTMsetstyle{finnish} \DTMdisplaytime{11}{43}{27}
\DTMsetstyle{en-GB}   \DTMdisplaytime{11}{43}{27}
\end{koodilohkosis}

\begin{tulossis}
   11.43 11:43am
\end{tulossis}

\noindent
Paketti \paketti{datetime2} sisältää paljon muitakin ominaisuuksia ja
komentoja ajanilmausten käsittelyyn ja latomiseen. Paketin hyödyt
korostuvat erityisesti silloin, kun tarvitaan automaattisia
ajanilmauksia eri kielillä. Tarkempaa tietoa on luettavissa paketin
ohjekirjasta, mutta seuraavassa on vielä yksi esimerkki tilanteesta,
jossa puhutaan jostakin kuukaudesta, ilman että kirjoitusvaiheessa
tiedetään, mistä kuukaudesta on kyse.

\komentoi{DTMsavenow}
\komentoi{DTMfinnishMonthname}
\komentoi{DTMfetchmonth}
\komentoi{DTMfetchyear}
\begin{koodilohkosis}
\DTMsavenow{nyt}
\DTMfinnishMonthname{\DTMfetchmonth{nyt}}ssa vuonna \DTMfetchyear{nyt}
eli tämän tekstin julkaisun aikaan -- --.
\end{koodilohkosis}

\begin{tulossis}
  \DTMsavenow{nyt}
  \DTMfinnishMonthname{\DTMfetchmonth{nyt}}ssa vuonna \DTMfetchyear{nyt}
  eli tämän tekstin julkaisun aikaan -- --.
\end{tulossis}

\section{Vapaasti sijoitettava sisältö KESKEN}

...
