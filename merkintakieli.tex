\chapter{Merkintäkieli}

\section{Erikoismerkit}

Muutamalla merkillä on Latex\-/dokumentissa erikoismerkitys, ja siksi
merkkejä ei voi käyttää normaalilla tavalla. Merkit ovat seuraavat:

\begin{koodilohkosis}
  % ^ _ # & $ { } ~ \
\end{koodilohkosis}

Useimmat merkit voi suojata erikoismerkitykseltä kirjoittamalla niiden
eteen kenoviivan (\textbackslash). Kenoviiva on oikeastaan
Latex\-/komennon aloittava merkki (luku \ref{luku:komennot}), mutta
seuraavien erikoismerkkien kanssa syntyy komento, joka käytännössä latoo
merkin sellaisenaan.

\begin{koodilohkosis}
  \% \^ \_ \# \& \$ \{ \}
\end{koodilohkosis}

Erikoismerkeistä tilden (\textasciitilde) eikä kenoviivan
(\textbackslash) suojaaminen ei onnistu pelkällä kenoviivalla. Kenoviiva
ja tilde muodostavat yhdessä komennon, joka latoo seuraavan kirjaimen
päälle tilden. Esimerkiksi \koodi{\textbackslash\textasciitilde{a}}
tuottaa ladottuna merkin \~a (\textsc{u+00e3 \textenglish{latin small
    letter a with tilde}}). Pelkkä tildemerkki saadaan kirjoitetaan
komennolla \koodi{\textbackslash\textasciitilde\{\}} tai \koodi{\keno
  textasciitilde}. Kaksi kenoviivaa (\koodi{\keno\keno}) puolestaan on
komento, joka tekee rivinvaihdon. Pelkkä kenoviiva täytyy kirjoittaa
komennolla \koodi{\keno textbackslash}. Lisätietoa komentojen
toiminnasta on luvussa \ref{luku:komennot}.

% Pitää kertoa, missä on lisätietoa näistä merkeistä.

\subsection{Sanavälit}

Välilyönti, sarkainmerkki ja yksi rivinvaihto ovat kaikki vain
tavallisia sanavälejä Latex\-/dokumentissa, ja näillä kolmella on sama
merkitys. Esimerkiksi rivin lopussa oleva rivinvaihto tarkoittaa samaa
kuin sanojen välissä oleva välilyönti. Kirjoittaja voi itse vapaasti
päättää, kuinka monta sanaa kirjoittaa yhdelle riville tekstieditorissa.
Latex ei välitä.

Toinen tarpeellinen tieto on, että välilyöntejä ja sarkainmerkkejä voi
kirjoittaa useita peräkkäin, mutta peräkkäiset välit silti tarkoittavat
vain yhtä väliä.

\begin{koodilohkosis}
  Nämä      kaikki
       ovat        vain
  sanoja  peräkkäin,  ja      kaikki                    kuuluvat
      samaan kappaleeseen.     
\end{koodilohkosis}

Tyhjä rivi on kuitenkin eri asia. Tyhjä rivi on sellainen, joka ei
sisällä mitään muuta kuin rivinvaihdon tai joka sisältää vain
välilyöntejä tai sarkainmerkkejä ja lopuksi rivinvaihdon. Latexissa
tyhjä rivi on kappaleen vaihtumisen merkki, eli sen jälkeen alkaa uusi
tekstikappale. Tyhjiä rivejä voi olla useita peräkkäin, mutta ne
tarkoittavat samaa kuin yksi tyhjä rivi.

Ladotuissa teksteissä uuden tekstikappaleen alku ilmaistaan usein
sisennetyllä rivillä, ja niin on tässäkin oppaassa. Sisennyksiä eikä
muitakaan muotoiluja ei tehdä tekstieditorissa välien avulla. Siihen on
omat keinonsa (luku \ref{luku:kappale}).

\subsection{Kommentit (\%)}

Latex\-/dokumentissa prosentin merkki (\koodi{\%}) on kommenttimerkki,
jonka jälkeisen rivin\-osan kääntäjä jättää huomioimatta. Merkki on
tarkoitettu kirjoittajan omien kommenttien ja muistiinpanojen
kirjoittamiseen.

\begin{koodilohkosis}
  % Nyt ei tosin ole
  % mitään kommentoitavaa.
\end{koodilohkosis}

Kommenttimerkki vaikuttaa kääntäjään kuitenkin siten, että se syö kaikki
välilyönnit ja sarkainmerkit, jotka tulevat kyseisen kommentin jälkeen.
Toisin sanoen kommenttimerkin avulla voi yhdistää eri riveillä olevan
tekstin. Seuraava esimerkki tuottaa ladottuna ehjän sanan ''Latex'':

\begin{koodilohkosis}
  La% Tässä on kommentti
    t% ja myöskin tässä.
      ex
\end{koodilohkosis}

\section{Komennot}
\label{luku:komennot}
\section{Ympäristöt}
\section{Mitat}
\section{Laskurit}
