\chapter{Merkintäkieli}

\section{Sanavälit}

Välilyönti, sarkainmerkki ja yksi rivinvaihto ovat kaikki vain
tavallisia sanavälejä Latex\-/dokumentissa, ja näillä kolmella on sama
merkitys. Esimerkiksi rivin lopussa oleva rivinvaihto tarkoittaa samaa
kuin sanojen välissä oleva välilyönti. Kirjoittaja voi vapaasti päättää,
kuinka monta sanaa kirjoittaa yhdelle riville tekstieditorissa. Latex ei
välitä. Välilyöntejä ja sarkainmerkkejä voi kirjoittaa useita peräkkäin,
mutta ne ovat sama asia kuin yksi väli.

\begin{koodilohkosis}
  Nämä      kaikki
       ovat            vain
  sanoja  peräkkäin,  ja      kaikki                kuuluvat
      samaan kappaleeseen.     
\end{koodilohkosis}

Tyhjä rivi on kuitenkin eri asia: se tarkoittaa kappaleen vaihtumista.
Rivi on tyhjä silloin, kun se ei sisällä mitään muuta kuin rivinvaihdon
tai kun se sisältää vain välilyöntejä tai sarkainmerkkejä ja lopuksi
rivinvaihdon. Tyhjiä rivejä voi olla useita peräkkäin, mutta ne
tarkoittavat samaa kuin yksi tyhjä rivi.

Ladotuissa teksteissä uuden tekstikappaleen alkaminen ilmaistaan usein
sisennetyllä rivillä, ja niin on tässäkin kappaleessa ja koko oppaassa.
Sisennyksiä eikä muitakaan muotoiluja ei tehdä tekstieditorissa välien
avulla. Siihen on omat keinonsa (luku \ref{luku:kappale}).

\section{Merkistö}

Latex\-/dokumenttiin voi kirjoittaa tekstiä Unicode\-/merkistöllä ja sen
\textsc{utf}\=/8\/-koodauksella, jos kääntäjänä on Unicoden osaava
la\-to\-ja\-oh\-jel\-ma kuten Xelatex tai Lualatex. Pääasiassa siis
merkit kirjoitetaan sellaisenaan lähdedokumenttiin. On kuitenkin muutama
poikkeus eli tiettyyn erityistarkoitukseen varattuja merkkejä, ja niitä
käsitellään tässä alaluvussa.

\subsection{Varatut erikoismerkit}

Muutamalla merkillä on perus Latexissa erikoismerkitys, eikä niitä ei
voi käyttää normaalilla tavalla. Merkit ovat seuraavat:

\begin{koodilohkosis}
  % $ ^ _ # & { } ~ \
\end{koodilohkosis}

Useimmat näistä merkeistä voi suojata erikoismerkitykseltä
kirjoittamalla niiden eteen kenoviivan (\keno). Tildeä
(\textasciitilde), sirkumfleksia (\textasciicircum) eikä kenoviivaa
itseään ei voi suojata pelkän kenoviivan avulla, koska kenoviivan kanssa
ne muodostavat eräitä muita komentoja. Taulukossa
\ref{tlk:merkkien_suojaus} on koottuna, kuinka edellä mainitut
erikoismerkit suojataan eli saadaan ladottua dokumenttiin sellaisenaan.

\leijutlk{
  \begin{tabular}{cll}
    \toprule
    \ots{Merkki} & \multicolumn{2}{l}{\ots{Kirjoittaminen}} \\
    \midrule
    \koodi{\%} & \koodi{\keno\%} \\
    \koodi{\$} & \koodi{\keno\$} & \koodi{\keno textdollar} \\
    \koodi{\^{}} & \koodi{\keno\^{}\{\}} & \koodi{\keno textasciicircum} \\
    \koodi{\_} & \koodi{\keno\_} & \koodi{\keno textunderscore} \\
    \koodi{\#} & \koodi{\keno\#} \\
    \koodi{\&} & \koodi{\keno\&} \\
    \koodi{\{} & \koodi{\keno\{} & \koodi{\keno textbraceleft} \\
    \koodi{\}} & \koodi{\keno\}} & \koodi{\keno textbraceright} \\
    \koodi{\~{}} & \koodi{\keno\~{}\{\}} & \koodi{\keno textasciitilde} \\
    \koodi{\keno} && \koodi{\keno textbackslash} \\
    \bottomrule
  \end{tabular}
}{
  \caption{Varattujen erikoismerkkien kirjoittaminen}
  \label{tlk:merkkien_suojaus}
}

Jotkin makropaketit määrittelevät muitakin erikoismerkkejä. Esimerkiksi
kieli\-asetuksiin (luku \ref{luku:kieliasetukset}) liittyvät
Polyglossia\-/{} ja Babel\-/paketit voivat määritellä pari
lainausmerkillä (\koodi{\textquotedbl}) alkavaa, tavutuksen hallintaan
liittyvää komentoa.

\subsection{Kommentit}

Latex\-/dokumentissa prosentin merkki (\koodi{\%}) on kommenttimerkki,
jonka jälkeisen rivin\-osan kääntäjä jättää huomioimatta. Merkki on
tarkoitettu kirjoittajan omien kommenttien ja muistiinpanojen
kirjoittamiseen.

\begin{koodilohkosis}
  % Nyt ei tosin ole
  % mitään kommentoitavaa.
\end{koodilohkosis}

Kommenttimerkki vaikuttaa kääntäjään myös siten, että se syö kaikki
välilyönnit ja sarkainmerkit, jotka tulevat kyseisen kommentin jälkeen.
Toisin sanoen kommenttimerkin avulla voi yhdistää eri riveillä olevan
tekstin. Seuraava esimerkki tuottaa ladottuna ehjän sanan ''Latex'':

\begin{koodilohkosis}
  La% Tässä on kommentti
    t% ja myöskin tässä.
      ex
\end{koodilohkosis}

\subsection{Lainausmerkit ja heittomerkki}

Kuulunee typografian perus\-asioihin, että käytetään oikeita
typografisia lainausmerkkejä. Näppäilemällä vaihtonäppäin ja 2 (shift+2)
saadaan yleensä lainausmerkin yleisversio (\textquotedbl), niin sanottu
\textsc{ascii}\-/lainausmerkki, joka ei ole tietääkseni minkään kielen
varsinainen lainausmerkki.

Eri kielissä lainausmerkkikäytännöt ovat erilaiset. Suomen kielessä
käytetään ''tällaisia'' lainausmerkkejä tai joskus >>tällaisia>>
kulmalainausmerkkejä. Jos lainauksen sisään tarvitaan lainaus, täytyy
sisempi lainaus kirjoittaa 'tällaisten' puolilainausmerkkien avulla.
Yksittäin käytettynä se on nimeltään heittomerkki ('). Englannin
kielessä lainauksen alussa ja lopussa on erilainen merkki, ja ``tässä''
siitä esimerkki. Samoin on puolilainausmerkin kohdalla: `näin'.

Latexissa voi käyttää Unicode\-/merkistöä ja yleensä lähdedokumenttiin
voi kirjoittaa suoraan ne merkit, jotka halutaan ladottavaksi, mutta
edellä mainituille merkeille on myös omat komentonsa. Näppäimistöltä
kirjoitettava yleisheittomerkki (\koodi{'}) tuottaa automaattisesti
oikean kaarevan heittomerkin ('). Kun kirjoittaa kaksi heittomerkkiä
peräkkäin (\koodi{''}), on lopputuloksena yksi kaareva lainausmerkki
(''). Kahdella suurempi kuin \=/merkillä (\koodi{>>}) saadaan
kulmalainausmerkki (>>).

\begin{koodilohkosis}
  ''Lainaus, jonka 'sisällä' on lainaus.''
  >>Lainaus, jonka 'sisällä' on lainaus.>>
\end{koodilohkosis}

Yllä mainitut riittävät suomen kieleen, mutta englantia ja muita kieliä
varten tarvitaan myös toisinpäin oleva merkki (``), joka tehdään
kahdella gra\-vis\-ak\-sen\-til\-la (\koodi{``}). Vastaava
puolilainausmerkki (`) tehdään yhdellä aksentilla (\koodi{`}). Ainakin
ranskan kielessä käytetään erilaisia kulmalainausmerkkejä lainauksen
alussa ja lopussa. Lainauksen aloittava merkki (<<) tehdään kahdella
pienempi kuin \=/merkillä (\koodi{<<}). Muunlaisiin lainausmerkkeihin
tarvittaneen Unicode\-/merkkejä. Tosin kielikohtaiset asetukset (luku
\ref{luku:kieliasetukset}) voivat tuoda mukanaan omia keinojaan myös
kielikohtaisten lainausmerkkien tuottamiseen.

Joskus ladottuun tekstiin todella halutaan yleislainausmerkki
(\textquotedbl) tai yleisheittomerkki (\textquotesingle). Ne saadaan
komennoilla \koodi{\keno textquotedbl} ja \koodi{\keno textquotesingle}.
Oletuksena myös tasalevyinen fontti (luku \ref{luku:kirjaintyypit})
kytkee pois erikoiset lainausmerkkikomennot ja tuottaa suoraan niitä
merkkejä, joita lähdedokumenttiin on kirjoitettu.

\subsection{Yhdysmerkki, ajatusviiva ja miinusmerkki}

Yhdyssanojen osien välissä käytettävä yhdysmerkki on Latexissa
tavallinen näppäimistöltä saatava yleis\-yhdysmerkki (\=/). Se vaikuttaa
sanojen tavutukseen, josta on tarkempaa tietoa luvussa
\ref{luku:tavutus}.

Ajatusviivaa tarvitaan esimerkiksi äärikohtien (7--12,
Oulu--Rova\-niemi), luetelmien, vuorosanojen ja virkkeen irrallisen
lisäysten merkitseminen. Suomen kielessä käytetään yleensä vain lyhyttä
ajatusviivaa \mbox{(--)}, joka tehdään Latexissa kahdella peräkkäisellä
yhdysmerkillä (\koodi{--}). Pitkä ajatusviiva \mbox{(---)} tehdään
kolmella yhdysmerkillä (\koodi{---}). Myös Unicoden ajatusviivamerkkejä
voi käyttää: \unicode{u+2013 en dash} ja \unicode{u+2014 em dash}.
Ajatusviivatkin vaikuttavat sanan tavutukseen (luku \ref{luku:tavutus}).

Miinusmerkille ei Latexissa ole erityistä merkintätapaa muuten kuin
matematiikkatilassa (luku \ref{luku:matematiikka}). Ajatusviivaa saa
käyttää myös miinusmerkkinä, mutta vielä parempi olisi käyttää
varsinaista Unicoden miinusmerkkiä \unicode{u+2212 minus sign}, koska se
on fonteissa suunniteltu typografisesti yhteensopivaksi muiden
matemaattisten merkkien kanssa. Tämän oppaan fontissakin on pieni ero
merkkien välillä: miinusmerkki on samalla korkeudella kuin plusmerkin
vaakaviiva; ajatusviiva on ylempänä (kuva
\ref{kuva:ajatusviiva_miinusmerkki}). Joissakin fonteissa eroa on myös
pituudessa.

\leijukuva{
  \addfontfeatures{Scale=11, LetterSpace=-5}
  +−= \hfill +–=
}{
  \caption{Vasemmalla plusmerkki, miinusmerkki ja yhtäsuuruusmerkki;
    oikealla plusmerkki, ajatusviiva ja yhtäsuuruusmerkki}
  \label{kuva:ajatusviiva_miinusmerkki}
}

\subsection{Sitova välilyönti}

Sitova välilyönti on samanlainen tyhjä merkki kuin tavallinenkin
välilyönti, mutta rivinvaihtoa ei sallita sen kohdalta. Sitovaa
välilyöntiä kannattaa käyttää esimerkiksi mittaluvun ja mittayksikön
lyhenteen välissä (3~km). Latexissa sitova välilyönti saadaan joko
tildemerkillä (\koodi{\textasciitilde}) tai nimenomaan siihen
tarkoitetulla merkillä \unicode{u+00a0 no-break space}.

Nämä kaksi eri merkkiä, tilde (\koodi{\textasciitilde}) ja
\unicode{u+00a0}, toimivat hieman eri tavoin. Molemmat tietenkin estävät
rivinvaihdon, mutta \koodi{\textasciitilde}\=/merkki sallii myös välin
venymisen samalla tavalla kuin tavallinenkin sanaväli sallii. Rivillä
sanavälit venyvät silloin, kun tekstipalsta tasataan molemmista
reunoista. Merkki \unicode{u+00a0} on vakiolevyinen eikä siis veny
muiden sanavälien tapaan.

\subsection{Kolme pistettä eli ellipsi}

Ajatuksen katkeamista ja muuta sellaista ilmaisevalle kolmelle pisteelle
eli ellipsille (…) on oma merkkinsä, ja fontissa se saattaa näyttää
hieman erilaiselta kuin kolme peräkkäistä pistemerkkiä. Tyypillisesti
ellipsimerkissä pisteet ovat hieman harvemmassa ja erottuvat toisistaan
paremmin kuin kolmena erillisenä merkkinä ladotut pisteet. Ellipsi
tehdään Latexissa komennolla \koodi{\keno ldots} tai suoraan
Unicode\-/merkillä \unicode{u+2026 horizontal ellipsis}.

\subsection{Ligatuurit}
\subsection{Tarkkeet}
\label{luku:tarkkeet}
\section{Rivin ja sivun vaihto}
\section{Komennot}
\label{luku:komennot}
\section{Ympäristöt}
\label{luku:ymparistot}
\section{Mitat}
\section{Laskurit}
