\chapter{Merkintäkieli}

% Latex on merkintäkieli, eli se sisältää omat tapansa dokumentin
% rakenteen ja sisällön kuvaamiseen. Tässä luvussa käsitellään
% merkintäkielen perus\-asioita.

\section{Merkistö}

Latex\-/dokumenttiin voi kirjoittaa tekstiä Unicode\-/merkistöllä ja sen
\textsc{utf}\=/8\/-koodauksella, jos kääntäjänä on Unicoden osaava
ohjelma kuten Xelatex tai Lualatex. Pääasiassa siis merkit kirjoitetaan
sellaisenaan lähdedokumenttiin, mutta on kuitenkin kaikenlaisia
poikkeuksia, ja niitä käsitellään tässä alaluvussa.

\subsection{Varatut erikoismerkit}

Muutamalla merkillä on perus Latexissa erikoismerkitys, eikä niitä voi
käyttää normaalilla tavalla. Merkit ovat seuraavat:

\begin{koodilohkosis}
  % $ ^ _ # & { } ~ \
\end{koodilohkosis}

Useimmat näistä merkeistä voi suojata erikoismerkitykseltä
kirjoittamalla niiden eteen kenoviivan (\keno). Tildeä
(\textasciitilde), sirkumfleksia (\textasciicircum) eikä kenoviivaa
itseään ei voi suojata pelkän kenoviivan avulla, koska kenoviivan kanssa
ne muodostavat eräitä muita komentoja. Taulukossa
\ref{tlk:merkkien_suojaus} on koottuna, kuinka edellä mainitut
erikoismerkit suojataan eli saadaan ladottua dokumenttiin sellaisenaan.

\leijutlk{
  \begin{tabular}{cll}
    \toprule
    \ots{Merkki} & \multicolumn{2}{l}{\ots{Kirjoittaminen}} \\
    \midrule
    \koodi{\%} & \koodi{\keno\%} \\
    \koodi{\$} & \koodi{\keno\$} & \koodi{\keno textdollar} \\
    \koodi{\^{}} & \koodi{\keno\^{}\{\}} & \koodi{\keno textasciicircum} \\
    \koodi{\_} & \koodi{\keno\_} & \koodi{\keno textunderscore} \\
    \koodi{\#} & \koodi{\keno\#} \\
    \koodi{\&} & \koodi{\keno\&} \\
    \koodi{\{} & \koodi{\keno\{} & \koodi{\keno textbraceleft} \\
    \koodi{\}} & \koodi{\keno\}} & \koodi{\keno textbraceright} \\
    \koodi{\~{}} & \koodi{\keno\~{}\{\}} & \koodi{\keno textasciitilde} \\
    \koodi{\keno} && \koodi{\keno textbackslash} \\
    \bottomrule
  \end{tabular}
}{
  \caption{Varattujen erikoismerkkien kirjoittaminen}
  \label{tlk:merkkien_suojaus}
}

Jotkin makropaketit määrittelevät muitakin erikoismerkkejä. Esimerkiksi
kieli\-asetuksiin (luku \ref{luku:kieliasetukset}) liittyvät
Polyglossia\-/{} ja Babel\-/paketit voivat määritellä pari
lainausmerkillä (\koodi{\textquotedbl}) alkavaa, tavutuksen hallintaan
liittyvää komentoa.

\subsection{Sanaväli}
\label{luku:sanavali}

Välilyönti, sarkainmerkki ja yksi rivinvaihto ovat kaikki tavallisia
sanavälejä Latex\-/dokumentissa, ja näillä kolmella on sama merkitys.
Esimerkiksi rivin lopussa oleva rivinvaihto tarkoittaa samaa kuin
sanojen välissä oleva välilyönti. Välilyöntejä ja sarkainmerkkejä voi
kirjoittaa useita peräkkäin, mutta ne ovat sama asia kuin yksi väli.

\begin{koodilohkosis}
  Nämä      kaikki
       ovat            vain
  sanoja  peräkkäin  ja               kuuluvat
      samaan kappaleeseen.     
\end{koodilohkosis}

Sanavälien leveys ei ole vakio. Silloin kun tekstipalsta tasataan
molemmista reunoista -- kuten tämänkin oppaan leipätekstissä \==,
rivillä olevia sanavälejä venytetään sopivasti, jotta tekstipalstan
molemmat reunat saadaan tasaiseksi. Tietynlevyisiä välejä saa tehtyä
komennolla \koodi{\keno hspace} (luku \ref{luku:mitat}).

\subsection{Kappaleen vaihtuminen}

Tyhjä rivi tarkoittaa kappaleen vaihtumista. Rivi on tyhjä silloin, kun
se ei sisällä mitään muuta kuin rivinvaihdon tai kun se sisältää vain
välilyöntejä tai sarkainmerkkejä ja lopuksi rivinvaihdon. Tyhjiä rivejä
voi olla useita peräkkäin, mutta ne tarkoittavat samaa kuin yksi tyhjä
rivi.

\begin{koodilohkosis}
  Nämä rivit kuuluvat
  samaan kappaleeseen.

  Tässä on toinen tekstikappale.
  Nyt ei oteta kantaa siihen, miten
  rivit ja kappaleet muotoillaan.
\end{koodilohkosis}

Ladotuissa teksteissä uuden tekstikappaleen alkaminen ilmaistaan usein
sisennetyllä rivillä, mutta sisennyksiä eikä muitakaan muotoiluja ei
tehdä tekstieditorissa välien avulla. Kappaleiden muotoiluun on omat
keinonsa, ja niistä kerrotaan luvussa \ref{luku:kappale}.

\subsection{Sitova välilyönti}

Sitova välilyönti on samanlainen tyhjä merkki kuin tavallinenkin
välilyönti, mutta rivinvaihtoa ei sallita sen kohdalta. Sitovalla
välilyönnillä kannattaa estää esimerkiksi pienistä osista koostuvan
ilmauksen hajoaminen eri riveille (\emph{osa~5}). Latexissa sitova
välilyönti saadaan joko tildemerkillä (\koodi{\textasciitilde}) tai
nimenomaan siihen tarkoitetulla merkillä, jonka Unicode\-/tunnus on
\unicode{u+00a0 no-break space}.

Nämä kaksi eri merkkiä, tilde ja \unicode{u+00a0}, toimivat hieman eri
tavoin. Molemmat estävät rivinvaihdon, mutta
\koodi{\textasciitilde}\=/merkki sallii välin venymisen samalla tavalla
kuin tavallinenkin sanaväli sallii (luku \ref{luku:sanavali}). Merkki
\unicode{u+00a0} on vakiolevyinen eikä siis veny muiden sanavälien
tapaan. Rivillä sanavälejä täytyy venyttää silloin, kun tekstipalsta
tasataan molemmista reunoista.

\subsection{Ohuke}

Ohuke on tavallista sanaväliä kapeampi väli, ja se tehdään
komennolla~\koodi{\keno,} (kenoviiva ja pilkku). Ohuke on tasalevyinen
ja sitova, eli se ei veny muiden sanavälien tapaan, ja se estää
rivinvaihdon. Siksi ohuke sopii esimerkiksi pitkien lukujen ja
puhelinnumeroiden ryhmittelyyn paremmin kun sanaväli.

\begin{koodilohkosis}
  12\,750\,000
  J.\,R.\,R. Tolkien
\end{koodilohkosis}

Myös henkilön etunimen alkukirjainten välissä voi käyttää ohuketta, jos
tavallinen sanaväli vie kirjaimet turhan kauas toisistaan. Sukunimi
erotetaan kuitenkin aina sanavälillä. Joskus myös päiväyksissä käytetään
ohuketta järjestysluvun pisteiden jälkeen. Taulukossa \ref{tlk:ohuke}
vertaillaan sanaväliä, ohuketta ja yhteen kirjoittamista.

\leijutlk{
  \begin{tabular}{lrll}
    \toprule
    & \ots{Luku} & \ots{Päiväys} & \ots{Nimi} \\
    \midrule
    \otsrivi{Sanaväli} & 12 750 000 & \st{9. 5. 2020} & J. R. R. Tolkien \\
    \otsrivi{Ohuke} & 12\,750\,000 & 9.\,5.\,2020 & J.\,R.\,R. Tolkien \\
    \otsrivi{Yhteen} & 12750000 & 9.5.2020 & \st{J.R.R. Tolkien} \\
    \bottomrule
  \end{tabular}
}{
  \caption{Sanavälin, ohukkeen ja yhteen kirjoittamisen vertailu. Suomen
    kielen vastaiset kirjoitus\-asut on viivattu yli}
  \label{tlk:ohuke}
}

\subsection{Lainausmerkit ja heittomerkki}

Näppäilemällä \textsc{shift} eli vaihtonäppäin ja 2 saadaan yleensä
lainausmerkin yleisversio, niin sanottu \textsc{ascii}\-/lainausmerkki
(\textquotedbl), mutta se ei taida olla minkään kielen varsinainen
lainausmerkki. On siis syytä käyttää oikeita lainausmerkkejä, ja se käy
Latexissa varsin helposti.

Eri kielissä lainausmerkkikäytännöt ovat erilaiset. Suomen kielessä
käytetään ''tällaisia'' lainausmerkkejä tai joskus >>tällaisia>>
kulmalainausmerkkejä. Jos lainauksen sisään tarvitaan lainaus, täytyy
sisempi lainaus kirjoittaa 'tällaisten' puolilainausmerkkien avulla.
Yksittäin käytettynä se on nimeltään heittomerkki. Englannin kielessä
lainauksen alussa ja lopussa on erilainen merkki, ja ``tässä'' siitä
esimerkki. Samoin on puolilainausmerkin kohdalla: `näin'.

Latexissa voi käyttää Unicode\-/merkistöä ja lähdedokumenttiin voi
kirjoittaa suoraan ne lainausmerkit, jotka halutaan ladottavaksi, mutta
edellä mainituille merkeille on myös omat komentonsa. Näppäimistöltä
kirjoitettava yleisheittomerkki (\koodi{'}) tuottaa ladottuna
automaattisesti oikean kaarevan heittomerkin ('). Kun kirjoittaa kaksi
heittomerkkiä peräkkäin (\koodi{''}), on lopputuloksena yksi kaareva
lainausmerkki (''). Kahdella suurempi kuin \=/merkillä (\koodi{>>})
saadaan kulmalainausmerkki~(>>).

\begin{koodilohkosis}
  ''Lainaus, jonka 'sisällä' on lainaus.''
  >>Lainaus, jonka 'sisällä' on lainaus.>>
\end{koodilohkosis}

Yllä mainitut riittävät suomen kieleen, mutta englantia ja muita kieliä
varten tarvitaan myös toisinpäin oleva merkki (``), joka tehdään
kahdella gra\-vis\-ak\-sen\-til\-la (\koodi{``}). Vastaava
puolilainausmerkki (`) tehdään yhdellä aksentilla (\koodi{`}). Joissakin
kielissä käytetään erilaisia kulmalainausmerkkejä lainauksen alussa ja
lopussa. Vasemmalle osoittava merkki (<<) tehdään kahdella pienempi kuin
\=/merkillä (<\mbox{}<). Muunlaisiin lainausmerkkeihin tarvittaneen
Unicode\-/merkkejä. Tosin kielikohtaiset asetukset (luku
\ref{luku:kieliasetukset}) voivat tuoda mukanaan omia keinojaan myös
kielikohtaisten lainausmerkkien tuottamiseen.

Joskus todella halutaan latoa yleislainausmerkki (\textquotedbl) tai
yleisheittomerkki (\textquotesingle). Ne saadaan komennoilla
\koodi{\keno textquotedbl} ja \koodi{\keno textquotesingle}. Oletuksena
myös tasalevyinen fontti (luku \ref{luku:kirjaintyypit}) kytkee pois
erikoiset lainausmerkkikomennot ja tuottaa suoraan niitä merkkejä, joita
lähdedokumenttiin on kirjoitettu. Muita tapoja lainausmerkkitoiminnon
estämiseen on esimerkissä \ref{esim:pois-tex-liga}.

\subsection{Yhdysmerkki, ajatusviiva ja miinusmerkki}

Yhdyssanan osien välissä käytettävä yhdysmerkki on Latexissa tavallinen
näppäimistöltä saatava yleis\-yh\-dys\-merk\-ki (\=/). Merkillä on
sivuvaikutus tavutukseen, mistä kerrotaan tarkemmin luvussa
\ref{luku:tavutus}.

Ajatusviivaa tarvitaan esimerkiksi äärikohtien (7--12,
Oulu--Rova\-niemi), luetelmien, vuorosanojen ja virkkeen irrallisen
lisäysten merkitseminen. Suomen kielessä käytetään yleensä vain lyhyttä
ajatusviivaa \mbox{(--)}, joka tehdään Latexissa kahdella peräkkäisellä
yhdysmerkillä \mbox{(\koodi{--})}. Pitkä ajatusviiva \mbox{(---)}
tehdään kolmella yhdysmerkillä \mbox{(\koodi{---})}. Myös Unicoden
ajatusviivamerkkejä voi käyttää: \unicode{u+2013 en dash} ja
\unicode{u+2014 em dash}. Ajatusviivatkin vaikuttavat sanan tavutukseen
(luku \ref{luku:tavutus}).

\begin{koodilohkosis}
  7--12-vuotiaat
  Oulu--Rovaniemi-yhteys
\end{koodilohkosis}

Silloin kun todella täytyy latoa kaksi tai kolme peräkkäistä
yhdysmerkkiä, voi käyttää tasalevyistä fonttia (luku
\ref{luku:kirjaintyypit}), joka oletuksena kytkee pois Latexin
ajatusviivatoiminnon. Muita tapoja on esimerkissä
\ref{esim:pois-tex-liga}.

\begin{esimerkki}
\begin{koodilohko}
  \addfontfeatures{Ligatures=TeXReset} % Fontista Tex-ligatuurit pois.
  -\mbox{}-  % Estetään ajatusviiva.
  >\mbox{}>  % Estetään kulmalainausmerkit.
  \texttt{…} % Tasalevyinen fontti estää Tex-ligatuurit.
\end{koodilohko}
\caption{Keinoja Tex-ligatuurien estämiseen. Komento \koodi{\keno
    addfontfeatures} on Fontspec\-/makropaketin ominaisuus}
\label{esim:pois-tex-liga}
\end{esimerkki}

Miinusmerkille ei Latexissa ole erityistä merkintätapaa muuten kuin
matematiikkatilassa (luku \ref{luku:matematiikka}). Ajatusviivaa saa
käyttää myös miinusmerkkinä, mutta vielä parempi olisi käyttää
varsinaista Unicoden miinusmerkkiä \unicode{u+2212 minus sign}, koska se
on fonteissa suunniteltu typografisesti yhteensopivaksi muiden
matemaattisten merkkien kanssa. Tämän oppaan fontissakin on pieni ero
merkkien välillä: miinusmerkki on samalla korkeudella kuin plusmerkin
vaakaviiva; ajatusviiva on ylempänä (kuva
\ref{kuva:ajatusviiva_miinusmerkki}). Joissakin fonteissa eroa on myös
pituudessa.

\leijukuva{
  \addfontfeatures{Scale=9, LetterSpace=-5}
  +−= \hfill +–=
}{
  \caption{Vasemmalla plusmerkki, miinusmerkki ja yhtäsuuruusmerkki;
    oikealla plusmerkki, ajatusviiva ja yhtäsuuruusmerkki}
  \label{kuva:ajatusviiva_miinusmerkki}
}

\subsection{Kolme pistettä eli ellipsi}

Ajatuksen katkeamista ja muuta sellaista ilmaisevalle kolmelle pisteelle
eli ellipsille (…) on oma merkkinsä, ja fontissa se saattaa näyttää
hieman erilaiselta kuin kolme peräkkäistä pistemerkkiä. Tyypillisesti
ellipsimerkissä pisteet ovat hieman harvemmassa ja erottuvat toisistaan
paremmin kuin kolmena erillisenä merkkinä ladotut pisteet. Ellipsi
tehdään Latexissa komennolla \koodi{\keno ldots} tai Unicode\-/merkillä
\unicode{u+2026 horizontal ellipsis}.

\subsection{Ligatuurit}
\subsection{Tarkkeet}
\label{luku:tarkkeet}

\subsection{Kommentit}

Latex\-/dokumentissa prosentin merkki (\koodi{\%}) on kommenttimerkki,
jonka jälkeisen rivin\-osan kääntäjä jättää huomioimatta. Merkki on
tarkoitettu kirjoittajan omien kommenttien ja muistiinpanojen
kirjoittamiseen.

\begin{koodilohkosis}
  % Nyt ei tosin ole
  % mitään kommentoitavaa.
\end{koodilohkosis}

Kommenttimerkki vaikuttaa kääntäjään myös siten, että se syö kaikki
välilyönnit ja sarkainmerkit, jotka tulevat kyseisen kommentin jälkeen.
Tämän vuoksi kommenttimerkin avulla voi yhdistää eri riveillä olevan
tekstin. Seuraava esimerkki tuottaa ladottuna ehjän sanan \emph{Latex}.

\begin{koodilohkosis}
  La% Nämä rivit
    t% yhdistyvät.
      ex
\end{koodilohkosis}

\section{Rivin ja sivun vaihto}
\section{Komennot}
\label{luku:komennot}
\section{Ympäristöt}
\label{luku:ymparistot}
\section{Mitat}
\label{luku:mitat}
\section{Laskurit}
