\chapter{Merkintäkieli}
\section{Erikoismerkit}

Muutamalla merkillä on Latex\-/dokumentissa erikoismerkitys, ja siksi
merkkejä ei voi käyttää normaalilla tavalla. Merkit ovat seuraavat:

\begin{koodilohkosis}
  % ^ _ # & $ { } ~ \
\end{koodilohkosis}

Useimmat merkit voi suojata erityismerkitykseltä kirjoittamalla niiden
eteen kenoviivan:

\begin{koodilohkosis}
  \% \^ \_ \# \& \$ \{ \}
\end{koodilohkosis}

Erikoismerkeistä tilden (\textasciitilde) eikä kenoviivan
(\textbackslash) suojaaminen ei onnistu pelkällä kenoviivalla. Kenoviiva
ja tilde muodostavat yhdessä komennon, joka tekee sen, että seuraavan
kirjaimen päälle tulee tilde. Esimerkiksi
\koodi{\textbackslash\textasciitilde{a}} tuottaa ladottuna merkin ã.
Pelkkä tildemerkki saadaan kirjoitetaan komennolla
\koodi{\textbackslash\textasciitilde\{\}} tai \koodi{\keno
  textasciitilde}. Kaksi kenoviivaa (\koodi{\keno\keno}) puolestaan on
komento, joka tekee rivinvaihdon. Pelkkä kenoviiva täytyy kirjoittaa
komennolla \koodi{\keno textbackslash}.

% Pitää kertoa, missä on lisätietoa näistä merkeistä.

\subsection{Sanavälit}

Välilyönti, sarkainmerkki ja yksi rivinvaihto ovat kaikki vain
tavallisia sanavälejä Latex\-/dokumentissa, ja näillä kolmella on sama
merkitys. Esimerkiksi rivin lopussa oleva rivinvaihto tarkoittaa samaa
kuin sanojen välissä oleva välilyönti. Kirjoittaja voi itse vapaasti
päättää, kuinka monta sanaa kirjoittaa yhdelle riville tekstieditorissa.
Latex ei välitä.

Toinen tarpeellinen tieto on, että välilyöntejä ja sarkainmerkkejä voi
kirjoittaa useita peräkkäin, mutta peräkkäiset välit silti tarkoittavat
vain yhtä väliä.

\begin{koodilohkosis}
  Nämä      kaikki
       ovat        vain
  sanoja  peräkkäin,  ja      kaikki                    kuuluvat
      samaan kappaleeseen.     
\end{koodilohkosis}

Tyhjä rivi on kuitenkin eri asia. Tyhjä rivi on sellainen, joka ei
sisällä mitään muuta kuin rivinvaihdon tai joka sisältää vain
välilyöntejä tai sarkainmerkkejä ja lopuksi rivinvaihdon. Latexissa
tyhjä rivi on kappaleen vaihtumisen merkki, eli sen jälkeen alkaa uusi
tekstikappale.

Ladotuissa teksteissä uuden tekstikappaleen alku ilmaistaan usein
sisennetyllä rivillä, ja niin on tässäkin oppaassa. Sisennyksiä eikä
muitakaan muotoiluja ei kuitenkaan tehdä tekstieditorissa välien avulla.
Siihen on omat keinonsa.

\subsection{Kommentit (\%)}

Latex\-/dokumentissa \koodi{\%}\=/merkki on kommenttimerkki, jota
kääntäjä ei huomioi ja jonka jälkeisen rivin\-osan kääntäjä jättää
huomioimatta. Merkki on tarkoitettu kirjoittajan omien kommenttien ja
muistiinpanojen kirjoittamiseen.

Kommenttimerkki myös syö yhden rivinvaihdon sekä kaikki välilyönnit ja
sarkainmerkit, jotka tulevat kyseisen kommentin jälkeen. Seuraava
esimerkki tuottaa ladottuna ehjän sanan ''Latex'':

\begin{koodilohkosis}
  La% Tässä on kommentti
   t%  ja myöskin tässä.
    ex
\end{koodilohkosis}

Toista rivinvaihtoa kommentti ei kuitenkaan syö, eli tyhjä rivi
kommentin jälkeen tarkoittaa normaalia kappaleen vaihtumista.

\section{Komennot}
\section{Ympäristöt}
\section{Mitat}
\section{Laskurit}
