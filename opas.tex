\documentclass[a4paper,10pt,notitlepage,oneside]{book}
\usepackage[vscale=.75, hscale=.68, footskip=1.5cm]{geometry}
\usepackage{fontspec}
\usepackage{polyglossia}
\usepackage{ragged2e}
\usepackage{footmisc}
\usepackage{titling}
\usepackage{titlesec}
\usepackage{color}
\usepackage{floatrow}
\usepackage{caption}
\usepackage{newfloat}
\usepackage{fancyvrb}
%\usepackage{placeins}
\usepackage{booktabs}
\usepackage{fancyhdr}
%\usepackage{noindentafter}
\usepackage[unicode,hyperfootnotes=false]{hyperref}
\usepackage[shortcuts]{extdash}

\hypersetup{ hidelinks, bookmarksopen, bookmarksnumbered,
  pdfauthor={Teemu Likonen} }

\setdefaultlanguage{finnish}
\setotherlanguage{english}

\setmainfont{Linux Libertine O}
[Scale=1.4, Numbers=Lowercase]

\setsansfont{Linux Biolinum O}
[Scale=MatchLowercase, Numbers=Lowercase]

\setmonofont{Linux Libertine Mono O}
[Scale=MatchLowercase, LetterSpace=-2]

\linespread{1.58}

% \newcommand{\gemenanum}{\addfontfeatures{Numbers=Lowercase}}
% \newcommand{\versaalinum}{\addfontfeatures{Numbers=Uppercase}}

\clubpenalty=10000
\widowpenalty=10000
\setlength{\emergencystretch}{1em}

\setlength{\parindent}{1.2em}
\setlength{\parskip}{0em}
\setlength{\footnotemargin}{\parindent}
\setlength{\floatsep}{3ex}
\setlength{\textfloatsep}{4ex}

\DeclareFloatingEnvironment[ name={Esimerkki} ]{esim}

\DefineVerbatimEnvironment{koodilohko}{Verbatim}{
  fontsize=\footnotesize, gobble=1, frame=lines, numbers=left,
  numbersep=.3em, xleftmargin=0em, xrightmargin=0em, baselinestretch=1.3 }

\floatsetup{ font={small}, justification=raggedright,
  margins=raggedright, captionskip=1ex, capposition=bottom }

\DeclareCaptionFont{rivivali}{\linespread{1.3}\selectfont}

\captionsetup{ font={small, sf, rivivali}, labelfont={bf}, textfont={},
  textformat=period, margin=.5em, justification=raggedright,
  singlelinecheck=off }

\captionsetup[esim]{ skip=-.5ex, margin=.5em }

\newcommand{\leiju}[3]{
  \begin{#1}
    \floatbox{#1}{#2}{#3}
  \end{#1}
}

\fancyhf{}
% \fancyhf[hel]{\small\nouppercase{\leftmark}}
% \fancyhf[hor]{\small\nouppercase{\rightmark}}
\fancyhf[fc]{\thepage}
\renewcommand{\headrulewidth}{0pt}

\fancypagestyle{plain}{%
  \fancyhf{}
  \fancyhf[fc]{\thepage}
  %\renewcommand{\headrulewidth}{0pt}
}

\definecolor{tavu}{rgb}{1,0,0}
\definecolor{sitova}{rgb}{1,0,0}

\newcommand{\keno}{\textbackslash}
\newcommand{\koodi}[1]{\mbox{\texttt{#1}}}
\newcommand{\tavuvihje}{\keno-}
\newcommand{\tavukohta}{\textcolor{tavu}{\rule{1pt}{1.5ex}}}
\newcommand{\sitovaym}{\textcolor{sitova}{-}}

\newcommand{\ots}[1]{{\sffamily\bfseries #1}}

\newcommand{\otsikkotyyli}{%
  \raggedright
  \linespread{1.1}
  \sffamily
  \bfseries
}

\titleformat{\section}{\otsikkotyyli\Large}
{\thesection}{.8em}{}[]
\titlespacing*{\section}{0pt}{*4}{*2}

\titleformat{\subsection}{\otsikkotyyli\normalsize}
{\thesubsection}{.8em}{}[]
\titlespacing*{\section}{0pt}{*3}{*1}

\begin{document}
\pagestyle{fancy}

\hyphenation{
  Ext-dash
  Lua-la-tex
  ni-men-omaan
  Open-type
  Poly-glos-sia
  True-type
  Xe-la-tex
}

\chapter*{Esipuhe}
\addcontentsline{toc}{1}{Esipuhe}
\phantomsection

Suurin innoittajani tämän kirjan kirjoittamiselle on ollut Latex itse.

\chapter{Ensiaskeleet}
\section{Käsitteitä ja nimityksiä}

Se, mitä sanalla Latex yleensä tarkoitetaan, on monimutkainen
kokonaisuus erilaisia pieniä palikoita kuten tietokone\-ohjelmia,
makropaketteja ja asetustiedostoja. Yritän tässä luvussa selventää
joitakin peruskäsitteitä.

Tex on tietokoneohjelma (\koodi{tex}), joka osaa lukea tietynmuotoisia
tekstitiedostoja ja latoa niiden perusteella tekstidokumentin, joka on
tarkoitettu ihmisten luettavaksi. Tex on myös tekstin ladontaan
tarkoitettu ohjelmointikieli. Se noudattaa sille annettuja ohjeita ja
latoo kirjaimia ja muuta tavaraa peräkkäin sivulle. Ihmisten
näkökulmasta Tex on hyvin tekninen ja matalatasoinen järjestelmä, eikä
sellaisten kanssa yleensä haluta olla missään tekemisissä. Siirtykäämme
siis eteenpäin.

Latex toimii korkeammalla abstraktiotasolla kuin Tex. Se on kokoelma
Tex\-/ohjelmointikielen makroja, jotka piilottavat monimutkaiset
tekniset yksityiskohdat ja toteuttavat varsin helppokäyttöisen
merkkauskielen, joilla tekstidokumenttien rakenne ja ulko\-asua
kuvataan. Ihmiset siis kirjoittavat yleensä Latex\-/dokumentteja, ja
Latex eli makrot puolestaan huolehtivat Texin käskyttämisestä. Latex on
myös tietokone\-ohjelma (\koodi{latex}), jolla lähdetiedoston voi
kääntää julkaistavaksi dokumentiksi.

Ajan myötä mukaan on tullut sekalainen joukko muitakin ohjelmia, joista
kannattaa tässä yhteydessä mainita kaksi: Xelatex ja Lualatex. Ne ovat
erilaisia Latexin toteutuksia ja tietokone\-ohjelmia (\koodi{xelatex},
\koodi{lualatex}), joilla lähdedokumentti käännetään. Nyky\-aikana
käytetään näistä yleensä jompaakumpaa, ja esimerkiksi tämän kirjan
Latex\-/lähdetiedosto on käännetty PDF\-/tiedostoksi
\koodi{xelatex}\-/ohjelmalla.

Jotta kaikki olisi mahdollisimman sekavaa, sana Latex toimii myös
yleisenä nimityksenä tälle kaikelle. Se esiintyy ilmaisuissa kuten
''Toteutin dokumentin Latexilla'' tai ''Tämä artikkeli on tehty
Latexilla''. Ilmaukset sitten tarkoittavat jotakin seuraavanlaista:
Henkilöllä on asennettuna tietokoneelle Latex\-/jakelu (kuten Tex Live).
Hän on kirjoittanut teksti\-editorilla (kuten GNU Emacsilla)
tekstitiedoston, jossa on dokumentin sisältö ja Latex\-/makroille
tarkoitettuja komentoja. Sitten hän on kääntänyt eli ladotuttanut
tekstitiedostonsa PDF\-/tiedostoksi Latex-la\-don\-ta\-oh\-jel\-man
jollakin toteutuksella (kuten \koodi{xelatex}illa).

Meille \marginpar{\Large\LaTeX} taitaa riittää vain Latexista puhuminen,
mutta siitäkin on mainittava vielä yksi huomio. Latexin harrastajat
tykkäävät käyttää dokumenttiensa leipätekstissä logoja kuten \TeX{} ja
\LaTeX{}. Usein teksteissä näkyy myös logojen pohjalta mukailtuja
kirjoitus\-asuja TeX ja LaTeX. Mielestäni logojen eikä muodikkaiden
kIRjoiTus\-AsuJen paikka ei ole leipätekstissä, koska ne erottuvat
tekstipalstasta liiaksi ja tekevät siitä rauhattoman näköisen. Tässä
kirjassa viittaan kaikkiin nimiin kielenhuollon normien mukaisesti eli
käytän tavallisia erisnimiä kuten Tex ja Latex. Koodi ja komennot ovat
siinä muodossa kuin ne tietokoneelle annetaan, esimerkiksi
\koodi{xelatex}.

\section{Käyttöönotto}
\subsection{Jakelukokoelma}
\subsection{Apuohjelmia}

% texdoc, latexmk
%
% ~/texmf/tex/latex/local

\section{Ensimmäinen dokumentti}

Ehkä olisi parasta päästä vain nopeasti kokeilemaan omia
Latex\-/dokumentteja eikä lukea pitkiä jaaritteluja muinaisten
kreikkalaisten Latex\-/filosofioista. Tallenna esimerkin
\ref{esim:ensimmainen} sisältö teksti\-editorin avulla tiedostoon
vaikkapa nimellä ''teksti.tex''. Käännä eli lado se PDF\-/tiedostoksi
komennolla ''\koodi{xelatex} \koodi{teksti.tex}'' tai ''\koodi{latexmk}
\koodi{-xelatex} \koodi{teksti.tex}''.

Latin Modern Roman \=/fontin tilalle voi toki kokeilla muitakin
(rivi~6). Fontin oletuskoko on 10 pistettä, mutta tässä esimerkissä se
venytetään 1,3\-/kertaiseksi eli 13 pisteeseen. Riviväliin liittyvä
kerroin asetetaan rivillä 7.

\begin{esim}
\begin{koodilohko}
  \documentclass[a4paper]{article}
  \usepackage{fontspec}
  \usepackage{polyglossia}

  \setdefaultlanguage{finnish}
  \setmainfont{Latin Modern Roman}[Scale=1.3]
  \linespread{1.4}

  \begin{document}

  Minun Latex-dokumenttini!

  \end{document}
\end{koodilohko}
\caption{Ensimmäinen Latex-dokumentti}
\label{esim:ensimmainen}
\end{esim}

\chapter{Kirjoittaminen}
\section{Kappale}
\section{Erikoismerkit}
\section{Tavutus}

\chapter{Asetukset ja virittely}
\section{Dokumenttiluokat}
\section{Kieli}
\section{Sivu}
\section{Kirjaintyypit}

\chapter{Ympäristöjä}
\section{Luetelmat}
\section{Taulukot}
\section{Leijuvat elementit}

\chapter{Matematiikka}
\chapter{Grafiikka}

\end{document}
