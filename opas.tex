% Tekijä:   Teemu Likonen <tlikonen@iki.fi>
% Lisenssi: Creative Commons Nimeä-JaaSamoin 4.0 Kansainvälinen (CC BY-SA 4.0)
%           https://creativecommons.org/licenses/by-sa/4.0/legalcode.fi

\documentclass{book}
\usepackage[a5paper, twoside,
hscale=.72, vscale=.77,
hratio=17:28, vratio=20:32,
footskip=12mm, footnotesep=13bp,
marginparwidth=50bp, marginparsep=10bp]{geometry}
\usepackage[quiet]{fontspec}
% \usepackage[mode=text, input-decimal-markers={,}, output-decimal-marker={,},
% per-mode=symbol]{siunitx}
\usepackage{polyglossia}
\usepackage[finnish,showseconds=false]{datetime2}
\usepackage{ragged2e}
\usepackage[hang,bottom,norule]{footmisc}
\usepackage[clearempty]{titlesec}
%\usepackage{titletoc}
%\usepackage{graphicx}
\usepackage{color}
\usepackage{floatrow}
\usepackage{caption}
\usepackage{newfloat}
\usepackage{fancyvrb}
\usepackage{booktabs}
\usepackage{noindentafter}
\usepackage{soul}
% \usepackage{imakeidx} \makeindex[intoc]
% \usepackage{ifthen}
\usepackage{chngcntr}
\usepackage{realscripts}
\usepackage{csquotes}
\usepackage[style=authoryear, dashed=false, maxbibnames=99]{biblatex}
\usepackage{tikz}
\usepackage{totcount}
\usepackage{nowidow} \setnowidow
%\usepackage{showhyphens}
\usepackage[unicode,hyperfootnotes=false]{hyperref}
\usepackage[shortcuts]{extdash}

\hypersetup{ hidelinks, bookmarksopen, bookmarksnumbered,
  pdfauthor={Teemu Likonen}, pdftitle={Käytännöllistä Latexia} }

\setdefaultlanguage{finnish}
\setotherlanguage{english}
\newcommand{\englishfont}{}
\newcommand{\englanti}[1]{\textenglish{#1}}

\defaultfontfeatures{Ligatures={TeX, Common}, Numbers=Lowercase}

% Perusfontin vaatimuksia: Numbers={Lowercase, SlashedZero}, Fractions,
% pienversaalit, aitoja ylä- ja alaindeksejä.
\setmainfont {libertinusserif}
[UprightFont=*-regular,
ItalicFont=*-italic,
BoldFont=*-semibold,
BoldItalicFont=*-semibolditalic,
Extension=.otf]

\setsansfont {libertinussans}
[Scale=MatchLowercase,
UprightFont=*-regular,
ItalicFont=*-italic,
BoldFont=*-bold,
BoldItalicFont=*-bolditalic,
Extension=.otf]

% Fonttien asetuksia käsittelevässä luvussa viitataan tähän fonttiin ja
% välistyksen tiivistämiseen (FakeStretch).
\setmonofont {libertinusmono}
[Scale=MatchLowercase, Ligatures={TeXReset, NoCommon}, FakeStretch=.8,
Numbers=Uppercase,
HyphenChar=-,
UprightFont=*-regular,
ItalicFont=*-italic,
BoldFont=*-bold,
BoldItalicFont=*-bolditalic,
Extension=.otf]

% Esimerkkien tulosteet tällä fontilla:
\newcommand{\erikoisfontti}{\rmfamily}

\newfontface{\viivastofontti}{libertinusserif-regular.otf}

\renewcommand{\footnotesize}{\fontsize{8bp}{9bp}\selectfont}
\renewcommand{\small}{\fontsize{9bp}{10bp}\selectfont}
\renewcommand{\normalsize}{\fontsize{10.5bp}{13bp}\selectfont}
\renewcommand{\large}{\fontsize{13bp}{15bp}\selectfont}
\renewcommand{\Large}{\fontsize{16bp}{18bp}\selectfont}
\renewcommand{\LARGE}{\fontsize{20bp}{22bp}\selectfont}
\renewcommand{\huge}{\fontsize{24bp}{26bp}\selectfont}
\linespread{1}
\normalsize

\newcommand{\gemenanum}{\addfontfeatures{Numbers=Lowercase}}
\newcommand{\versaalinum}{\addfontfeatures{Numbers=Uppercase}}
\newcommand{\murtoluku}[2]{{\addfontfeatures{Fractions=On}#1/#2}}

\urlstyle{sf}
\newcommand{\kulmaurl}[1]
{\href{#1}{\guilsinglleft\nolinkurl{#1}\guilsinglright}}
\newcommand{\kulmasp}[1]
{\href{mailto:#1}{\guilsinglleft\nolinkurl{#1}\guilsinglright}}

\setlength{\emergencystretch}{1em}

\setlength{\parindent}{1em}
\setlength{\bibhang}{\parindent}
\setlength{\bibitemsep}{.5ex plus .1ex minus .1ex}
\newlength{\sisennys}\setlength{\sisennys}{1.8em}
\setlength{\parskip}{0em}
\setlength{\footnotemargin}{.8em}
\setlength{\floatsep}{.8em plus .3em minus .1em}
\setlength{\textfloatsep}{1.3em plus .3em minus .1em}

\addbibresource{kirjallisuutta.bib}
\nocite{*}
\renewcommand{\bibfont}{\RaggedRight}

\DeclareDelimFormat[bib]{nametitledelim}{\addcolon\space}
\DeclareDelimFormat[bib]{multinamedelim}{\space--\space}
\DeclareDelimFormat[bib]{finalnamedelim}{\space--\space}
\DeclareFieldFormat{url}{Saatavissa: \kulmaurl{#1}}
\DeclareNameAlias{sortname}{family-given}
\DeclareNameAlias{default}{family-given}
\DefineBibliographyStrings{finnish}{andothers = {ym.}}

\DeclareFloatingEnvironment[ name={Esimerkki} ]{esimerkki}

\renewcommand{\theFancyVerbLine}
{\sffamily\versaalinum\fontsize{6bp}{7bp}\selectfont\arabic{FancyVerbLine}}

\DefineVerbatimEnvironment{koodilohko}{Verbatim}{ fontsize=\small,
  gobble=2, frame=single, framesep=.4em, numbers=left, numbersep=.3em,
  xleftmargin=0em, xrightmargin=0mm, baselinestretch=1 }

\DefineVerbatimEnvironment{koodilohkosis}{Verbatim}{
  fontsize=\small, gobble=2, frame=none, numbers=none,
  numbersep=0em, xleftmargin=\sisennys, xrightmargin=0mm,
  baselinestretch=1, samepage=true }

\newenvironment{tulossis}{%
    \begin{list}{}{
        \setlength{\leftmargin}{\sisennys}
        \erikoisfontti\small
      }\item[⇒]}{%
    \end{list}}

\floatsetup{ font={small}, justification=raggedright,
  margins=raggedright, captionskip=2ex, capposition=bottom }

\captionsetup{ font={small, sf}, labelfont={bf}, textfont={},
  textformat=period, margin=.5em, justification=RaggedRight,
  singlelinecheck=off }

\captionsetup[esimerkki]{ skip=-0.5ex, margin=.5em }

\newcommand{\leijutlk}[2]{%
  \begin{table*}
    \floatbox{table}{\versaalinum #1}{#2}
  \end{table*}}

\newcommand{\leijukuva}[2]{%
  \begin{figure*}
    \floatbox{figure}{#1}{#2}
  \end{figure*}}

\newenvironment{nluetelma}{%
  \begin{list}{\theenumi.}{
      \usecounter{enumi}
      \setlength{\leftmargin}{1.3em}
      \setlength{\labelsep}{.3em}
      \setlength{\itemsep}{.2ex plus .2ex}
      \setlength{\parsep}{0em}
      \setlength{\topsep}{.2ex plus .2ex}
      \RaggedRight
    }}{%
  \end{list}}

\definecolor{tavu}{rgb}{1,0,0}

\newcommand{\keno}{\textbackslash}
\newcommand{\koodi}[1]{\texttt{#1}}
\newcommand{\marginaali}[1]{\marginpar{\footnotesize #1}}
\newcommand{\koodimargin}[1]{\marginaali{\koodi{#1}}}
\newcommand{\tavukohta}{\textcolor{tavu}{\raisebox{-.2ex}{\rule{.6bp}{2ex}}}}
\newcommand{\uctunnus}[1]{\textsc{\englanti{#1}}}
\newcommand{\paketti}[1]{\textsf{#1}}
\newcommand{\avctan}[1]{\footnote{\url{https://www.ctan.org/pkg/#1}}}

\newcommand{\ots}[1]{{\sffamily\bfseries #1}}
\newcommand{\otsrivi}[1]{{\sffamily #1}}
\newcommand{\katk}{\discretionary{}{}{}}

\NoIndentAfterEnv{koodilohkosis}
\NoIndentAfterEnv{tulossis}
\NoIndentAfterEnv{nluetelma}

\addto{\captionsfinnish}{
  \renewcommand{\contentsname}{Sisällys}
}

\setcounter{tocdepth}{2}
\setcounter{secnumdepth}{2}

\counterwithout{footnote}{chapter}

\newcommand{\otsikkotyyli}{ \raggedright \sffamily \bfseries }

\titleformat{\chapter}[display]{\Large\bfseries}
{\chaptertitlename\hspace{.3em}\thechapter}{1.5ex}{\otsikkotyyli\huge}[]
\titlespacing*{\chapter}{0em}{*13}{*8}

\titleformat{\section}{\otsikkotyyli\large}
{\thesection}{.8em}{}[]
\titlespacing*{\section}{0pt}{*4}{*2}

\titleformat{\subsection}{\otsikkotyyli\normalsize}
{\thesubsection}{.8em}{}[]
\titlespacing*{\subsection}{0bp}{*2}{*1}

\titleformat{\subsubsection}{\normalsize\bfseries\itshape}
{\thesubsubsection}{.8em}{}[]
\titlespacing*{\subsubsection}{\parindent}{*2}{*1}

\regtotcounter{chapter}

\begin{document}
\pagestyle{empty}

\hyphenation{
  abst-ra-hoin-ti
  abst-rak-te-ja
  abst-rak-tio-ta-soil-la
  abst-rak-tio-ta-sol-la
  abst-rak-tio-ta-sol-taan
  abst-rak-tio-ta-son
  ala-in-dek-se-jä
  ala-in-dek-seil-le
  ala-in-dek-si-ko-men-not
  ala-in-dek-sien
  ala-in-dek-sin
  ala-in-dek-sit
  ala-in-dek-siä
  alku-osaa
  an-tiik-va
  an-tiik-vaa
  an-tiik-vaan
  an-tiik-van
  ar-ti-cle
  au-thor
  ba-bel
  beam-er
  bib-latex
  dis-cre-tionary
  doc-u-ment
  doc-u-ment-class
  ed-i-tor
  esit-te-ly-osaan
  esit-te-ly-osak-si
  esit-te-ly-osas-sa
  ext-dash
  font-size
  font-spec
  font-ti-ase-tuk-siin
  font-ti-ase-tuk-sil-le
  font-ti-ase-tus
  font-ti-ase-tus-ta
  ge-om-e-try
  gro-tes-ki
  gro-tes-kin
  hspace
  hy-phen-ation
  jul-kai-su-oh-jel-miin
  jul-kai-su-oh-jel-mis-sa
  kie-li-ase-tuk-set
  kie-li-ase-tuk-sia
  kie-li-ase-tuk-sis-ta
  kie-li-ase-tus-ten
  konk-reet-ti-ses-ti
  konk-reet-ti-sia
  käyt-töön-ot-toon
  käyt-töön-oton
  käyt-töön-otos-sa
  käyt-töön-otto
  käyt-töön-ottoa
  la-don-ta-oh-jel-man
  La-tex-mk
  La-tex-mk-rc
  large
  line-spread
  Lua-latex
  Match-Low-er-case
  mem-oir
  mit-ta-yk-si-köi-den
  mit-ta-yk-si-köi-tä
  mit-ta-yk-si-köl-le
  mit-ta-yk-si-kön
  new-com-mand
  new-en-vi-ron-ment
  ni-men-omaan
  note
  ole-tus-ar-vo
  ole-tus-ar-voa
  ole-tus-ase-tuk-set
  ole-tus-ase-tuk-sia
  ole-tus-ase-tuk-sil-la
  page
  pages
  par-in-dent
  para-graph
  pe-rus-osas-ta
  pe-rus-osat
  pe-rus-osien
  pe-rus-osiin
  pien-aak-ko-set
  pien-aak-kos-ten
  poly-glos-sia
  pub-lish-er
  pää-asial-li-nen
  pää-asial-li-sek-si
  pää-asial-li-sen
  pää-asias-sa
  re-new-com-mand
  re-new-en-vi-ron-ment
  rm-fam-i-ly
  sa-man-ai-kai-ses-ti
  Scale
  sec-tion
  sf-fam-i-ly
  slides
  small
  style
  sub-sec-tion
  suur-aak-kos-ten
  table
  teks-ti-edi-to-ria
  teks-ti-edi-to-ril-la
  teks-ti-edi-to-rin
  teks-ti-edi-to-ris-sa
  teks-ti-edi-to-ris-sa-kin
  teks-ti-edi-to-ris-ta
  teks-ti-osa
  teks-ti-osaan
  teks-ti-osan
  teks-ti-osas-sa
  the-bib-li-og-ra-phy
  tie-to-kone-oh-jel-ma
  tie-to-kone-oh-jel-mat
  tie-to-kone-oh-jel-mia
  title
  tt-fam-i-ly
  ty-po-gra-fi-sen
  ty-po-gra-fi-ses-ta
  ty-po-gra-fi-ses-ti
  ty-po-gra-fi-set
  ty-po-gra-fi-sia
  ty-po-gra-fi-sis-ta
  ty-po-gra-fia
  ty-po-gra-fiaan
  ty-po-gra-fian
  ty-po-gra-fias-sa
  ty-po-gra-fis-ten
  ulko-asu
  ulko-asua
  ulko-asun
  ulko-asus-ta
  ulko-asuun
  Uni-code
  Uni-coden
  use-pack-age
  vaih-to-eh-don
  vaih-to-eh-to
  vaih-to-eh-to-ja
  vaih-to-eh-toi-ses-ti
  vir-he-il-moi-tuk-sen
  vir-he-il-moi-tus
  vir-he-il-moi-tus-ta
  vol-ume
  vspace
  väli-ai-kai-ses-ti
  väli-ai-kais-tie-dos-to-jen
  väli-ai-kais-tie-dos-toa
  väli-ai-kais-tie-dos-toon
  väli-ai-kais-tie-dos-tot
  väli-ot-si-kon
  väli-ot-si-kot
  Xe-latex
  ylä-in-dek-si
  ylä-in-dek-sit
}

\newgeometry{top=1cm, bottom=1.8cm, hmargin=1.3cm}
\pdfbookmark[0]{Kansi}{luku:kansi}
\DTMsetstyle{finnish}

\vspace*{.2\textheight}

{

  \setlength{\parindent}{0pt}

  \fontsize{14bp}{14bp}\rmfamily Teemu Likonen

  \fontsize{48bp}{48bp}\sffamily\bfseries%
  \hspace{-3bp}%
  Käytännöllistä

  \fontsize{60bp}{60bp}\selectfont%
  \hspace{-5bp}%
  \LaTeX{}ia

}

\vfill

{

  \raggedleft
  Latex-ladontajärjestelmän opas \\
  \DTMtoday

}

\clearpage
\restoregeometry
\pdfbookmark[0]{Tekijänoikeus}{luku:tekijänoikeus}
\DTMsetstyle{finnish-numeric}

\null\vfill

{
  \setlength{\parindent}{0em}
  \setlength{\parskip}{1.2ex plus .1ex}

  \section*{Käytännöllistä Latexia}

  Tekijä: Teemu Likonen \kulmasp{tlikonen@iki.fi}

  Päiväys: \DTMtoday{} kello \DTMcurrenttime{} (Ensijulkaisu: ei ole
  julkaistu vielä)

  Lisenssi: \emph{Creative Commons Nimeä-Jaa\-Samoin 4.0 Kansainvälinen}
  (\textsc{cc by-sa} 4.0). Lisenssi antaa sinulle luvan kopioida ja
  levittää tätä teosta tai sen osia missä tahansa välineessä ja
  muodossa. Sisältöä saa muokata, ja sen pohjalta saa luoda uusia
  teoksia mihin tahansa tarkoitukseen, myös kaupallisesti. Ehdot ovat
  seuraavat:

  \begin{list}{\textbullet}{
      \setlength{\leftmargin}{1em}
      \setlength{\topsep}{0ex}
      \setlength{\partopsep}{0ex}
      \setlength{\itemsep}{0ex}
    }
  \item Sinun on mainittava tekijä(t) asianmukaisesti, tarjottava linkki
    lisenssin koko tekstiin (ks. alla) sekä mainittava, mikäli olet
    tehnyt muutoksia.
  \item Jos muokkaat teosta tai luot sen pohjalta uuden teoksen, sinun on
    jaettava muutoksiasi samalla lisenssillä kuin alkuperäistä teosta.
  \item Et saa asettaa sellaisia oikeudellisia ehtoja tai teknisiä
    estoja, jotka estävät muita tekemästä asioita, jotka tämä lisenssi
    sallii.
  \end{list}

  Lisenssin koko teksti: \\
  \url{https://creativecommons.org/licenses/by-sa/4.0/legalcode.fi}

  Poikkeus lisenssiin: Jos teet tästä teoksesta oman sähköisen
  dokumentin (esim. \textsc{pdf}), dokumenttiin mahdollisesti
  upotettavat fontit eivät kuulu tämän lisenssin piiriin, eli niillä saa
  olla jokin toinen lisenssi. Ilmaise sähköisen dokumentin käyttäjille
  selvästi, jos upotettujen fonttien lisenssi rajoittaa käyttöä eri
  tavalla kuin tämä lisenssi ja että tämä lisenssi on silti voimassa
  teoksen muun sisällön osalta. (Jos haluat, voit poistaa tämän
  poikkeuksen, jolloin myös fonttien lisenssin on oltava yhteensopiva
  tämän lisenssin kanssa.)

}

\cleardoublepage
\pagestyle{plain}
\pdfbookmark[0]{Sisällys}{luku:sisallys}

\tableofcontents

\cleardoublepage

\chapter*{Esipuhe}
\addcontentsline{toc}{chapter}{Esipuhe}
\phantomsection

% Miksi Latex?

Tekstidokumenttien toteuttaminen Latexilla ei ole läheskään samanlaista
kuin työskentely tietotekniikassa yleensä. Yleensähän käynnistetään
jokin sovellus\-ohjelma, joka sisältää suunnilleen kaikki tarvittavat
toiminnot. Samalla sovelluksella työ viedään alusta loppuun.

Sen sijaan Latexin kanssa työskentely on lähempänä ohjelmoijan
työskentelyä. Lähdedokumentti (''koodi'') on muodoltaan täysin erilainen
kuin lopullinen tuotos. Kirjoittaminen vaatii omanlaisensa kielen
osaamista. Lopulliseen toteutukseen tarvitaan yleensä eri tekijöiden
tuottamia makropaketteja (vrt. ohjelmakirjastot), ja niiden ohjekirjoja
täytyy joskus vilkaista. Lähdedokumentit käännetään erillisellä
kääntäjäohjelmalla lopulliseksi tuotokseksi, eikä käännettyä tuotosta
voi enää palauttaa alkuperäiseksi lähdedokumentiksi.

Latexin käyttö ei silti ole mitään ohjelmointia. Merkintäkieli on eri
asia kuin ohjelmointikieli. Työskentelyn luonteessa on kuitenkin useita
yhtäläisyyksiä, ja

\chapter{Valmistautuminen}

\section{Käsitteet ja nimet}

Latex ja sen ympärille rakentuneet ohjelmistot ovat käsitteellisesti
aika monimutkainen kokonaisuus, johon kuuluu eri\-/ikäisiä ja
abst\-rak\-tio\-ta\-sol\-taan erilaisia osia. Mukaan kuuluu tietenkin
konkreettia tie\-to\-kone\-oh\-jel\-mia, jotka tekevät konkreettisia
asoita. Mukaan kuuluu kuitenkin myös ihmisten luomia abst\-rak\-te\-ja
käsitteitä kuten Latex\-/formaatti, Tex\-/ohjelmointikieli tai muu
yleinen idea, jonka tieto\-kone\-ohjelmat pyrkivät konk\-reet\-ti\-sesti
toteuttamaan.

Internetissä näkyy silloin tällöin käsitekeskusteluja, jossa
ihmetellään, mihin mikäkin palikka kuuluu käsitteellisesti. Missä
suhteessa jotkin uudemmat osat ovat vanhempiin? Oikeiden termien
osaamisesta voi olla hyötyä, kun pyytää verkossa suurelta yleisöltä
apua. Viestintä vaatii, että puhutaan suunnilleen samaa kieltä. Yritän
seuraavaksi selventää peruskäsitteitä.

\subsection{Tex ja Latex}

Tex on tietokone\-ohjelma (\koodi{tex}), joka osaa lukea tietynmuotoisia
tekstitiedostoja ja latoa niiden perusteella tekstidokumentin, joka on
tarkoitettu ihmisten luettavaksi. Tex on myös tekstin ladontaan
tarkoitettu ohjelmointikieli. Se noudattaa sille annettuja ohjeita ja
latoo kirjaimia ja muuta tavaraa peräkkäin sivulle. Ihmisten
näkökulmasta Tex on hyvin tekninen ja matalatasoinen järjestelmä, eikä
sellaisten kanssa yleensä haluta olla missään tekemisissä. Siirtykäämme
siis eteenpäin.

Latex toimii korkeammalla abstraktiotasolla kuin Tex. Se on kokoelma
Tex\-/ohjelmointikielen makroja, jotka piilottavat monimutkaiset
tekniset yksityiskohdat ja toteuttavat varsin helppokäyttöisen
merkintäkielen, jolla dokumenttien rakenne ja ulko\-asua kuvataan.
Ihmiset siis kirjoittavat yleensä Latex\-/muotoisia dokumentteja, ja
Latex puolestaan huolehtii matalamman tason Texin käskyttämisestä. Latex
on myös tietokone\-ohjelma (\koodi{latex}), jolla lähdetiedoston voi
kääntää julkaistavaksi dokumentiksi.

\subsection{Xelatex ja Lualatex}

Ajan myötä mukaan on tullut sekalainen joukko muitakin ohjelmia, joista
kannattaa tässä yhteydessä mainita kaksi: Xelatex ja Lualatex. Ne ovat
erilaisia Latexin toteutuksia ja tietokone\-ohjelmia (\koodi{xelatex,
  lualatex}), joilla lähdedokumentti käännetään.\footnote{Englannin
  kielellä näitä on tapana kutsua yleisnimellä
  \emph{\textenglish{engine}} 'kone, moottori'.} Ne osaavat lukea
Unicode\-/merkistöllä kirjoitettuja lähdedokumentteja ja käyttää
nyky\-aikaisia True Type- ja Open Type \=/fontteja, mitä alkuperäinen
Latex ja Tex eivät osaa. Nyky\-aikana käytetään yleensä jompaakumpaa,
Xelatexia tai Lualatexia, ja esimerkiksi tämän oppaan lähdetiedosto on
käännetty PDF\-/tiedostoksi Xelatexilla.

Xelatexilla ja Lualatexilla ei ole käyttäjän kannalta suurtakaan eroa.
Jälkimmäinen sisältää Lua\-/nimisen ohjelmointikielen, ja sillä on
merkitystä joillekuille makropakettien tekijöille. Jotkin makropaketit
eivät toimi Lualatexissa ja jotkin eivät toimi Xelatexissa. Xelatex
taitaa olla hieman paremmin tuettu, koska se on vanhempi eli ehtinyt
vakiinnuttaa asemansa paremmin.

Kääntäjäohjelmien toiminnassa on pieniä eroja. Vaikka Latex\-/dokumentit
yleensä kääntyvätkin molemmilla, saattaa joskus valmiissa PDF:ssä näkyä
pieniä eroja, kun kääntäjää vaihtaa. Toisaalta on ihan hyödyllistä
kokeilla kääntää omat dokumentit kummallakin ohjelmalla, koska se voi
paljastaa huonoja, epäyhteensopivia käytäntöjä omassa Latex\-/koodissa.
En kuitenkaan suosittele vaihtamaan kääntäjää viime hetkellä ennen
tärkeän tekstin julkaisua.

\subsection{Latex yläkäsitteenä}

Jotta kaikki olisi mahdollisimman sekavaa, sana Latex toimii myös
yleisenä nimityksenä tälle kaikelle. Se esiintyy ilmaisuissa kuten
''Toteutin dokumentin Latexilla'' tai ''Tämä artikkeli on tehty
Latexilla''. Ilmaukset sitten tarkoittavat suunnilleen seuraavanlaista:
Henkilöllä on asennettuna tietokoneelle Latex\-/jakelu (kuten Tex Live).
Hän on kirjoittanut teksti\-editorilla (kuten GNU Emacsilla)
tekstitiedoston, jossa on dokumentin sisältö ja Latex\-/makroille
tarkoitettuja komentoja mutta myös joitakin Tex\-/komentoja. Sitten hän
on kääntänyt eli ladotuttanut tekstitiedostonsa PDF\-/tiedostoksi
Latex-la\-don\-ta\-oh\-jel\-man jollakin toteutuksella (kuten
Xelatexilla).

Meille taitaa riittää vain Latexista puhuminen, mutta siitäkin on
mainittava vielä yksi asia. Latexin harrastajat tykkäävät käyttää
dokumenttiensa leipätekstissä logoja kuten \TeX{} ja \LaTeX{}. Usein
teksteissä näkyy myös logojen pohjalta mukailtuja kirjoitus\-asuja TeX
ja LaTeX. Mielestäni logojen eikä muodikkaiden kIRjoiTus\-AsuJen paikka
ei ole leipätekstissä. Tässä oppaassa viittaan kaikkiin nimiin
kielenhuollon normien mukaisesti eli käytän tavallisia erisnimiä kuten
Tex ja Latex. Koodi ja komennot ovat siinä muodossa kuin ne
tietokoneelle annetaan, esimerkiksi \koodi{xelatex}.

\section{Asentaminen}
\label{luku:asentaminen}

Latex pitää tietysti asentaa tietokoneelle, jotta sitä voisi käyttää.
Miten edellisessä luvussa kuvattu sekava kokonaisuus saadaan ehjänä
omalle tietokoneelle? Onneksi muut ovat jo ratkaisseet sen ongelman aika
pitkälle.

Tavallisin tapa Latexin käyttöön\-ottoon on jonkin Latexin jakelupaketin
asentaminen. Jakelupaketti sisältää perus\-osien lisäksi paljon
makropaketteja ja niiden ohjekirjoja. Kaikkea ei koskaan tarvitse, mutta
kun yllättävä tarve tulee tai lukee vinkkejä verkkokeskusteluista, on
mukavaa huomata, että makropaketti olikin itsellä jo valmiina. Siksi
suosittelen kokonaisen jakelupaketin asentamista.

GNU/Linuxissa ja muissa Unix\-/tyyppisissä käyttöjärjestelmissä
käytetään yleensä Tex Live \=/nimistä jakelua. Se on todennäköisesti
saatavilla käyttöjärjestelmäjakelun pakettivarastoista. Esimerkiksi
Debianiin ja sen kaltaisiin järjestelmiin on asennuspaketti
\koodi{texlive-full}, joka asentaa kaiken helposti ja kerralla.

Windows\-/käyttöjärjestelmälle on saatavilla Tex Liven lisäksi Miktex ja
Protext. Mac OS \=/käyttöjärjestelmän kanssa käytettäneen yleensä
Mactex\-/nimistä jakelua.

\section{Ensimmäinen dokumentti}

Olemme varmasti jo puhuneet tarpeeksi, ja olisi hyvä päästä tekemään
jotain käytännöllistä. Tallenna esimerkin \ref{esim:ensimmainen} sisältö
teksti\-editorin avulla tiedostoon vaikkapa nimellä \koodi{teksti.tex}.
Käännä eli lado se PDF\-/tiedostoksi komennolla \koodi{xelatex
  teksti.tex} tai komennolla \koodi{lualatex teksti.tex}. Tuloksena
pitäisi olla tiedosto \koodi{teksti.pdf}, jota voi ihailla jollakin
PDF\-/tiedostojen katseluun tarkoitetulla ohjelmalla. Näin tämä
Latex\-/homma toimii.

Tutkitaan esimerkkiä \ref{esim:ensimmainen} tarkemmin. Ensimmäisellä
rivillä määritellään dokumenttiluokka \textenglish{\koodi{article}},
joka on tietynlainen sivupohja tai asetusten kokoelma, jonka perustalle
aletaan rakentaa omaa sivua. Luokka \textenglish{\koodi{article}} on
tyypillinen valinta lyhyehköille teksteille. Lisätietoa
dokumenttiluokista on luvussa \ref{luku:dokumenttiluokat}.

Toisella ja kolmannella rivillä käytetään komentoa \koodi{\keno
  usepackage} ja niiden avulla otetaan käyttöön fontti\-asetuksia
hoitava Fontspec\-/paketti ja kieli\-asetuksista vastaava
Polyglossia\-/paketti. Kumpaakin tarvitaan melkein joka kerta
dokumenteissa, ja niihin palataan tarkemmin luvuissa
\ref{luku:kirjaintyypit} ja \ref{luku:kieliasetukset}. Sivun asetuksia
käsitellään luvussa \ref{luku:sivuasetukset}.

Seuraavilla riveillä asetetaan kieleksi suomi (\koodi{finnish}) ja
määritetään oletuksena käytettävä fontti (kirjainperhe). Latin Modern
Roman \=/fontin tilalle voi toki kokeilla muitakin. Fontin oletuskoko on
10 pistettä, mutta tässä esimerkissä se venytetään 1,3\-/kertaiseksi eli
13 pisteeseen. Riviväliin liittyvä kerroin asetetaan rivillä 7.

\begin{esimerkki}
\begin{koodilohko}
  \documentclass{article}
  \usepackage{fontspec}
  \usepackage{polyglossia}

  \setdefaultlanguage{finnish}
  \setmainfont{Latin Modern Roman}[Scale=1.3]
  \linespread{1.4}

  \begin{document}

  Minun Latex-dokumenttini!

  \end{document}
\end{koodilohko}
\caption{Ensimmäinen Latex-dokumentti}
\label{esim:ensimmainen}
\end{esimerkki}

Dokumentin alku\-osaa esimerkin riville 8 saakka kutsutaan johdannoksi
tai esittelyksi (engl. \textenglish{preamble}). Tässä osassa ladataan
tarvittavat makropaketit ja määritetään dokumentin asetuksia ja
taustatietoja. Riviltä 9 alkaa varsinainen teksti\-osa eli sivulle
ladottava sisältö. Se osa kirjoitetaan \koodi{document}\-/ympäristön
sisään eli riveillä 9 ja 13 olevien ympäristön aloitus\-/{} ja
lopetuskomentojen väliin.

Tällaisen merkintäkielen ja rakenteen avulla dokumentit kirjoitetaan.
Osa merkintäkielen komennoista tulee Latexin perus\-osasta ja osa tulee
erikseen ladattavista makropaketeista (Fontspec, Polyglossia jne.).
Komentoja voi luoda itsekin.

\section{Apuohjelmia}

\subsection{Tekstieditori}

Hanki kunnollinen teksti\-editori. Teknisesti ainoa vaatimus on se, että
editori osaa tallentaa UTF\=/8\-/muotoista tekstidataa, tai vähän
kikkailemalla pelkkä ASCII\=/merkistökin riittäisi. Käytännössä
editorissa olisi hyvä olla muitakin ominaisuuksia, ja niin sanotut
ohjelmoijien tai tehokäyttäjien teksti\-editorit ovat paras valinta.

TÄNNE VIELÄ TEKSTIN SYNTAKTISESTA VÄRITTÄMISESTÄ JA EDITORIN HIENOISTA
OMINAISUUKSISTA.

\subsection{Texdoc}

Latexin kirjoittajan täytyy silloin tällöin lukea ohjekirjoja. Vaikka
Latexin perus\-osat joskus oppisikin ulkoa, ei voi koskaan muistaa
kaikkien hyödyllisten makropakettien kaikkia ominaisuuksia. Myös uusia
makropaketteja ja ihmisten suosituksia tulee vastaan esimerkiksi
verkkokeskusteluissa.

Tex Live \=/jakelun (luku \ref{luku:asentaminen}) mukana tulee mainio
komentotulkissa toimiva komento \koodi{texdoc}, jolla voi hakea ja avata
omaan järjestelmään asennettuja Latex\-/aiheisia ohjeita. Jos vaikka
haluaa tutustua esimerkissä \ref{esim:ensimmainen} mainittuun
Fontspec\-/pakettiin syvällisemmin, tarvitsee vain komentaa
\koodi{texdoc fontspec}, ja paketin PDF\-/muotoinen ohjekirja avautuu.

\subsection{Latexmk}

Hyödyllinen ohjelma on myös Latexmk, joka helpottaa dokumenttien
kääntämistä. Nimittäin varsin usein Latex\-/dokumentit täytyy kääntää
useita kertoja ennen kuin PDF\-/tiedosto on valmis. Se johtuu siitä,
että dokumentit sisältävät usein ristiviitteitä eli viittauksia
dokumentin toisiin osiin. Latex ei saa viitteitä kohdalleen yhdellä
kääntämisellä, vaan ensin se kirjoittaa viittausten kohteet muistiin
väli\-aikais\-tiedostoon ja seuraavilla kääntökerroilla käyttää
väli\-aikais\-tiedostoa apunaan.

Kääntäjä huomauttaa tietokoneen käyttäjää, kun uusintakäännös on
tarpeen, mutta Latexmk\-/ohjelma käynnistää uusintakäännöksen ihan itse,
aina kun se on tarpeellista.

Alla on kääntämiseen esimerkkikomentoja. Ensimmäinen kääntää dokumentin
Xelatexilla ja jälkimmäinen Lualatexilla.

\begin{koodilohkosis}
  latexmk -xelatex  teksti.tex
  latexmk -lualatex teksti.tex
\end{koodilohkosis}

Seuraavista esimerkeistä ensimmäinen komento poistaa kääntämisen aikana
luodut väli\-aikaistiedostot (\koodi{log, aux, out} ym.), ja
jälkimmäinen rivin komento poistaa kaikki luodut tiedostot eli
väli\-aikais\-tiedostojen lisäksi myös valmiin PDF\-/tiedoston.

\begin{koodilohkosis}
  latexmk -c teksti.tex
  latexmk -C teksti.tex
\end{koodilohkosis}

Edellisissä esimerkeissä käsitellään lähdetiedostoa nimeltä
\koodi{teksti.tex}, mutta jos lähdetiedostoa ei anna komennolle
lainkaan, käännetään kaikki nykyisessä hakemistossa olevat
\koodi{tex}\-/päätteiset tiedostot.

Latexmk-ohjelmalle voi tehdä asetustiedoston, johon voi kirjoittaa omaan
käyttöön sopivat asetukset. Asetustiedosto sijoitetaan käyttäjän
kotihakemistoon nimellä \koodi{.latexmkrc}. Alla on esimerkki, mitä se
voisi ehkä sisältää.

\begin{koodilohkosis}
  $pdf_mode = 5; # 5=xelatex, 4=lualatex
  $xelatex = 'xelatex -interaction=nonstopmode %O %S';
  $lualatex = 'lualatex -interaction=nonstopmode %O %S';
  $clean_ext = 'snm nav xdv';
\end{koodilohkosis}

Ensimmäisen rivin asetus määrittää, mitä kääntäjää käytetään oletuksena.
Toisella ja kolmannella rivillä määritellään, millä tavoin Xelatex ja
Lualatex suoritetaan. Tässä esimerkissä oletus\-asetuksiin on lisätty
\koodi{non\-stop\-mode}, joka estää kaiken vuorovaikutteisen toiminnan.
Asetus on tarpeen ainakin silloin, kun kääntäjä käynnistetään toisesta
ohjelmasta kuten teks\-ti\-edi\-to\-ris\-ta eikä vuorovaikutus kääntäjän
kanssa ole mahdollista.

Neljännellä rivillä luetellaan kääntämisen aikana syntyvien
väli\-aikais\-tiedostojen päätteitä. Yleiset väli\-aikais\-tiedostot
(\koodi{log, aux, out} ym.) on \koodi{latexmk}\-/ohjelmalla jo tiedossa,
mutta tällä asetuksella mukaan voi lisätä muitakin.

% Tekijä:   Teemu Likonen <tlikonen@iki.fi>
% Lisenssi: Creative Commons Nimeä-JaaSamoin 4.0 Kansainvälinen (CC BY-SA 4.0)
%           https://creativecommons.org/licenses/by-sa/4.0/legalcode.fi

\chapter{Merkintäkieli}

Latex on merkintäkieli, mikä tarkoittaa, että se sisältää omat tapansa
dokumentin rakenteen ja sisällön kuvaamiseen. Kaikkea ei kirjoiteta
lähdedokumenttiin sellaisenaan, vaan täytyy käyttää tiettyjä kielen
sääntöjen mukaisia merkintätapoja tai komentoja. Tässä luvussa
käsitellään merkintäkielen perus\-asioita.

\section{Merkistö}

Latex\-/dokumenttiin voi kirjoittaa tekstiä Unicode\-/merkistöllä ja sen
\textsc{utf}\=/8\/-koodauksella, jos kääntäjänä on Unicoden osaava
ohjelma kuten Lualatex tai Xelatex. Pääasiassa siis merkit kirjoitetaan
sellaisenaan lähdedokumenttiin, mutta on kuitenkin kaikenlaisia
poikkeuksia, ja niitä käsitellään tässä alaluvussa.

\subsection{Varatut erikoismerkit}

Muutamalla merkillä on perus Latexissa erikoismerkitys, eikä niitä voi
käyttää normaalilla tavalla. Merkit ovat seuraavat:

\begin{koodilohkosis}
  % $ ^ _ # & { } ~ \
\end{koodilohkosis}

Useimmat näistä merkeistä voi suojata erikoismerkitykseltään
kirjoittamalla niiden eteen kenoviivan (\koodi{\keno}). Tildeä
(\textasciitilde), sirkumfleksia (\textasciicircum) eikä kenoviivaa
itseään ei voi suojata pelkän kenoviivan avulla, koska kenoviivan kanssa
ne muodostavat eräitä muita komentoja. Taulukossa
\ref{tlk:merkkien_suojaus} on koottuna, kuinka edellä mainitut
erikoismerkit suojataan eli saadaan ladottua dokumenttiin sellaisenaan.

\leijutlk{
  \begin{tabular}{cll}
    \toprule
    \ots{Merkki} & \multicolumn{2}{l}{\ots{Kirjoittaminen}} \\
    \midrule
    \koodi{\%} & \koodi{\keno\%} \\
    \koodi{\$} & \koodi{\keno\$} & \koodi{\keno textdollar} \\
    \koodi{\^{}} & \koodi{\keno\^{}\{\}} & \koodi{\keno textasciicircum} \\
    \koodi{\_} & \koodi{\keno\_} & \koodi{\keno textunderscore} \\
    \koodi{\#} & \koodi{\keno\#} \\
    \koodi{\&} & \koodi{\keno\&} \\
    \koodi{\{} & \koodi{\keno\{} & \koodi{\keno textbraceleft} \\
    \koodi{\}} & \koodi{\keno\}} & \koodi{\keno textbraceright} \\
    \koodi{\~{}} & \koodi{\keno\~{}\{\}} & \koodi{\keno textasciitilde} \\
    \koodi{\keno} && \koodi{\keno textbackslash} \\
    \bottomrule
  \end{tabular}
}{
  \caption{Varattujen erikoismerkkien kirjoittaminen}
  \label{tlk:merkkien_suojaus}
}

Jotkin makropaketit määrittelevät muitakin erikoismerkkejä. Esimerkiksi
kieli\-asetuksiin (luku \ref{luku:kieliasetukset}) liittyvät
\paketti{polyglossia}\-/{} ja \paketti{babel}\-/paketit voivat
määritellä pari lainausmerkillä (\koodi{\textquotedbl}) alkavaa,
tavutuksen hallintaan liittyvää komentoa tai erikoismerkkiä.

\subsection{Sanaväli}
\label{luku:sanavali}

Välilyönti, sarkainmerkki ja yksi rivinvaihto ovat kaikki tavallisia
sanavälejä Latex\-/dokumentissa, ja näillä kolmella on sama merkitys.
Esimerkiksi rivin lopussa oleva rivinvaihto tarkoittaa samaa kuin
sanojen välissä oleva välilyönti. Välilyöntejä ja sarkainmerkkejä voi
kirjoittaa useita peräkkäin, mutta ne ovat sama asia kuin yksi väli.

\pagebreak[3]

\begin{koodilohkosis}
  Nämä      kaikki
       ovat            vain
  sanoja  peräkkäin  ja               kuuluvat
      samaan kappaleeseen.
\end{koodilohkosis}

\begin{tulossis}
  Nämä kaikki ovat vain sanoja peräkkäin ja kuuluvat samaan
  kappaleeseen.
\end{tulossis}

Sanavälien leveys ei ole vakio. Silloin kun tekstipalsta tasataan
molemmista reunoista -- kuten tämänkin oppaan leipätekstissä \==,
rivillä olevia sanavälejä venytetään sopivasti, jotta tekstipalstan
molemmat reunat saadaan tasaiseksi. Tietynlevyisiä vaakasuuntaisia
välejä saa tehtyä komennolla \koodi{\keno hspace} (luku
\ref{luku:mitat}).

\subsection{Kappaleen vaihtuminen}

Tyhjä rivi tarkoittaa kappaleen vaihtumista. Rivi on tyhjä silloin, kun
se ei sisällä mitään muuta kuin rivinvaihdon tai kun se sisältää vain
välilyöntejä tai sarkainmerkkejä ja lopuksi rivinvaihdon. Tyhjiä rivejä
voi olla useita peräkkäin, mutta ne tarkoittavat samaa kuin yksi tyhjä
rivi. Uuden tekstikappaleen voi aloittaa myös komennolla \koodi{\keno
  par}.

\begin{koodilohkosis}
  Nämä rivit kuuluvat
  samaan kappaleeseen.

  Tässä on toinen tekstikappale.
  Nyt ei oteta kantaa siihen, miten
  rivit ja kappaleet muotoillaan.
\end{koodilohkosis}

Ladotuissa teksteissä uuden tekstikappaleen alkaminen ilmaistaan usein
sisennetyllä rivillä, mutta sisennyksiä eikä muitakaan muotoiluja ei
tehdä tekstieditorissa välien avulla. Kappaleiden muotoiluun on omat
keinonsa, ja niistä käsitellään luvussa \ref{luku:kappale}.

\subsection{Kommentit}

Latex\-/dokumentissa prosentin merkki (\koodi{\%}) on kommenttimerkki,
jonka jälkeisen rivin\-osan kääntäjä jättää huomioimatta. Merkki on
tarkoitettu kirjoittajan omien kommenttien ja muistiinpanojen
kirjoittamiseen.

\begin{koodilohkosis}
  % Nyt ei tosin ole
  % mitään kommentoitavaa.
\end{koodilohkosis}

Kommenttimerkki vaikuttaa kääntäjään myös siten, että se syö kaikki
välilyönnit ja sarkainmerkit, jotka tulevat kyseisen kommentin jälkeen.
Tämän vuoksi kommenttimerkin avulla voi yhdistää eri riveillä olevan
tekstin. Seuraava esimerkki tuottaa ladottuna ehjän sanan \emph{Latex}:

\begin{koodilohkosis}
  La% Nämä rivit
    t% yhdistyvät.
      ex
\end{koodilohkosis}

\begin{tulossis}
  La% Nämä rivit
    t% yhdistyvät.
      ex
\end{tulossis}

\subsection{Aaltosulkeet}
\label{luku:aaltosulkeet}

Aaltosulkeet \mbox{\koodi{\{\}}} muodostavat eräänlaisen näkymättömän
ympäristön, jonka sisällä voi olla väliaikaisesti voimassa erilaiset
asetukset kuin ulkopuolella. Aaltosulkeiden sisällä suoritetut komennot,
uusien komentojen määrittelyt (luku \ref{luku:komennot}) tai asetetut
mittojen arvot (luku \ref{luku:mitat}) ovat voimassa vain kyseisen
ympäristön sisäpuolella. Seuraavassa esimerkissä aaltosulkeilla rajataan
kursivointikomennon \koodi{\keno it\-shape} vai\-ku\-tus\-aluetta.

\pagebreak[3]

\begin{koodilohkosis}
  tavallinen {\itshape kursiivi} tavallinen
\end{koodilohkosis}

\begin{tulossis}
  tavallinen {\itshape kursiivi} tavallinen
\end{tulossis}

\subsection{Sitova välilyönti}

Sitova välilyönti on samanlainen tyhjä merkki kuin tavallinenkin
välilyönti, mutta rivinvaihtoa ei sallita sen kohdalta. Sitovalla
välilyönnillä kannattaa estää esimerkiksi pienistä osista koostuvan
ilmauksen hajoaminen eri riveille (esimerkki: \emph{osa~5}). Latexissa
sitova välilyönti saadaan joko tildemerkillä (\koodi{\textasciitilde})
tai nimenomaan siihen tarkoitetulla merkillä, jonka Unicode\-/tunnus on
\uctunnus{u+00a0 no-break space}.

Nämä kaksi eri merkkiä, tilde ja \uctunnus{u+00a0}, toimivat hieman eri
tavoin. Molemmat estävät rivinvaihdon, mutta tildemerkki sallii välin
venymisen samalla tavalla kuin tavallinenkin sanaväli sallii (luku
\ref{luku:sanavali}). Sen sijaan merkki \uctunnus{u+00a0} on
vakiolevyinen eikä siis veny muiden sanavälien tapaan. Unicoden sitovaa
välilyöntiä \uctunnus{u+00a0} täytyy käyttää ainakin vuorosanaviivan
(\==) ja sitä seuraavan sanan välissä, koska se väli ei saa venyä.

\subsection{Ohuke}
\label{luku:ohuke}

Ohuke on tavallista sanaväliä kapeampi väli, ja se tehdään
komennolla~\koodi{\keno,} (kenoviiva ja pilkku). Ohukkeen leveys
Latexissa on \murtoluku{1}{6} typografisen neliön leveydestä eli
em-mitasta (luku \ref{luku:mitat}). Ohuke on tasalevyinen ja sitova, eli
se ei veny muiden sanavälien tapaan, ja se estää rivinvaihdon. Siksi
ohuke sopii esimerkiksi pitkien lukujen ja puhelinnumeroiden
ryhmittelyyn paremmin kun sanaväli.

\begin{koodilohkosis}
  12\,750\,000
  J.\,R.\,R. Tolkien
\end{koodilohkosis}

Myös henkilön etunimen alkukirjainten välissä voi käyttää ohuketta, jos
tavallinen sanaväli vie kirjaimet turhan kauas toisistaan. Sukunimi
erotetaan kuitenkin aina sanavälillä. Joskus myös päiväyksissä käytetään
ohuketta järjestysluvun pisteiden jälkeen. Taulukossa \ref{tlk:ohuke}
vertaillaan sanaväliä, ohuketta ja yhteen kirjoittamista.

\leijutlk{
  \begin{tabular}{lrll}
    \toprule
    & \ots{Luku} & \ots{Päiväys} & \ots{Nimi} \\
    \midrule
    \otsrivi{Sanaväli} & 12 750 000 & \st{9. 5. 2020} & J. R. R. Tolkien \\
    \otsrivi{Ohuke} & 12\,750\,000 & 9.\,5.\,2020 & J.\,R.\,R. Tolkien \\
    \otsrivi{Yhteen} & 12750000 & 9.5.2020 & \st{J.R.R. Tolkien} \\
    \bottomrule
  \end{tabular}
}{
  \caption{Sanavälin, ohukkeen ja yhteen kirjoittamisen vertailu. Suomen
    kielen vastaiset kirjoitus\-asut on viivattu yli}
  \label{tlk:ohuke}
}

\subsection{Lainausmerkit ja heittomerkki}
\label{luku:lainausmerkit}

Näppäilemällä \textsc{shift} eli vaihtonäppäin ja 2 saadaan yleensä
lainausmerkin yleisversio, niin sanottu \textsc{ascii}\-/lainausmerkki
(\textquotedbl), mutta se ei taida olla minkään kielen varsinainen
lainausmerkki. On siis syytä käyttää oikeita lainausmerkkejä, ja se käy
Latexissa varsin helposti.

Eri kielissä lainausmerkkikäytännöt ovat erilaiset. Suomen kielessä
käytetään ''tällaisia'' lainausmerkkejä ja joskus >>tällaisia>>
kulmalainausmerkkejä. Jos lainauksen sisään tarvitaan lainaus, täytyy
sisempi lainaus kirjoittaa 'tällaisten' puolilainausmerkkien avulla.
Yksittäin käytettynä se on nimeltään heittomerkki. Englannin kielessä
lainauksen alussa ja lopussa on erilainen merkki, ja ``tässä'' siitä
esimerkki. Samoin on puolilainausmerkin kohdalla: `näin'.

Latexissa voi käyttää Unicode\-/merkistöä ja lähdedokumenttiin voi
kirjoittaa suoraan ne lainausmerkit, jotka halutaan ladottavaksi, mutta
edellä mainituille merkeille on myös omat komentonsa. Näppäimistöltä
kirjoitettava yleisheittomerkki (\koodi{'}) tuottaa ladottuna
automaattisesti oikean kaarevan heittomerkin ('). Kun kirjoittaa kaksi
heittomerkkiä peräkkäin (\koodi{''}), on lopputuloksena yksi kaareva
lainausmerkki (''). Kahdella suurempi kuin \=/merkillä (\koodi{>>})
saadaan kulmalainausmerkki~(>>).

\pagebreak[3]

\begin{koodilohkosis}
  ''Lainaus, jonka 'sisällä' on lainaus.'' \\
  >>Lainaus, jonka 'sisällä' on lainaus.>>
\end{koodilohkosis}

\begin{tulossis}
  ''Lainaus, jonka 'sisällä' on lainaus.'' \\
  >>Lainaus, jonka 'sisällä' on lainaus.>>
\end{tulossis}

Yllä mainitut riittävät suomen kieleen, mutta englantia ja muita kieliä
varten tarvitaan myös toisinpäin oleva merkki (``), joka tehdään
kahdella gra\-vis\-ak\-sen\-til\-la (\koodi{``}). Vastaava
puolilainausmerkki (`) tehdään yhdellä aksentilla (\koodi{`}). Joissakin
kielissä käytetään erilaisia kulmalainausmerkkejä lainauksen alussa ja
lopussa. Vasemmalle osoittava merkki (<<) tehdään kahdella pienempi kuin
\=/merkillä (\koodi{<<}).

Lainausmerkkien merkintätapoja ja komentoja on koottu taulukkoon
\ref{tlk:erikoismerkit-lainaus}
(s.~\pageref{tlk:erikoismerkit-lainaus}). Toisaalta kielikohtaiset
asetukset (luku \ref{luku:kieliasetukset}) voivat tuoda mukanaan myös
kielikohtaisia keinoja lainausmerkkien kirjoittamiseen. Lisäksi
makropaketti \paketti{cs\-quotes}\avctan{csquotes} sisältää
lainausmerkkeihin liittyviä komentoja ja kielikohtaista logiikkaa.

Joskus todella halutaan latoa yleislainausmerkki (\textquotedbl) tai
yleisheittomerkki (\textquotesingle). Ne saadaan komennoilla
\koodi{\keno text\-quote\-dbl} ja \koodi{\keno text\-quote\-single}.
Yksittäinen gra\-vis\-ak\-sent\-ti (\`{}) tehdään komennolla
\koodi{\keno `\{\}}.

Latexin omat lainausmerkkien ja ajatusviivojen merkintätavat saa
poistettua kokonaan käytöstä fontin asetuksella
\koodi{Liga\-tures=\katk TeX\-Reset}. Asetus kytketään päälle esimerkiksi
komennolla \koodi{\keno add\-font\-fea\-tures}, joka sisältyy
\paketti{fontspec}\-/pakettiin. Mainittu asetus on oletuksena päällä
tasalevyisessä fontissa, jonka saa yksittäisiin sanoihin komennolla
\koodi{\keno text\-tt}. Fontteja käsitellään tarkemmin luvussa
\ref{luku:kirjaintyypit}.

\pagebreak[3]

\begin{koodilohkosis}
  {\addfontfeatures{Ligatures=TeXReset} `` '' >> '} \\
  \texttt{`` '' >> '}
\end{koodilohkosis}

\begin{tulossis}
  {\addfontfeatures{Ligatures=TeXReset} `` '' >> '} \\
  \texttt{`` '' >> '}
\end{tulossis}

\subsection{Yhdysmerkki, ajatusviiva ja miinusmerkki}

Yhdyssanan osien välissä käytettävä yhdysmerkki on Latexissa tavallinen
näppäimistöltä saatava yleis\-yh\-dys\-merk\-ki (\=/). Merkillä on
vaikutusta myös sanan tavutukseen (luku \ref{luku:tavutus}).

Ajatusviivaa tarvitaan esimerkiksi äärikohtien (27--29,
Oulu--Rova\-niemi), luetelmien, vuorosanojen ja virkkeen irrallisen
lisäysten merkitseminen. Suomen kielessä käytetään yleensä vain lyhyttä
ajatusviivaa \mbox{(--)}, joka tehdään Latexissa kahdella peräkkäisellä
yhdysmerkillä \mbox{(\koodi{--})}. Pitkä ajatusviiva \mbox{(---)}
tehdään kolmella yhdysmerkillä \mbox{(\koodi{---})}. Ajatusviivat
vaikuttavat sanan tavutukseen samoin kuin yhdysmerkki.

\pagebreak[3]

\begin{koodilohkosis}
  Oulu--Rovaniemi-yhteys
\end{koodilohkosis}

\begin{tulossis}
  Oulu--Rovaniemi-yhteys
\end{tulossis}

Myös Unicoden ajatusviivamerkit \uctunnus{u+2013 en dash} ja
\uctunnus{u+2014 em dash} toimivat, mutta tavutuksen kannalta ne
käyttäytyvät eri tavoin Lua\-latex- ja Xelatex\-/kääntäjillä.
Yhteensopivuussyistä on parasta tehdä ajatusviivat Latexin omilla
merkintätavoilla eikä Unicode\-/merkeillä.

Silloin kun todella täytyy latoa kaksi tai kolme peräkkäistä
yhdysmerkkiä, voi käyttää tasalevyistä fonttia (\koodi{\keno
  text\-tt\{\ldots\}}), joka oletuksena kytkee pois Latexin
ajatusviivatoiminnon. Saman asetuksen saa kyllä mihin tahansa fonttiin,
ja väliaikaisesti asetus kytketään seuraavasti:

\pagebreak[3]

\begin{koodilohkosis}
  {\addfontfeatures{Ligatures=TeXReset} -- ---}
\end{koodilohkosis}

\begin{tulossis}
  {\addfontfeatures{Ligatures=TeXReset} -- ---}
\end{tulossis}

Miinusmerkille ei Latexissa ole erityistä merkintätapaa muuten kuin
matematiikkatilassa (luku \ref{luku:matematiikka}). Lyhyttä ajatusviivaa
saa käyttää myös miinusmerkkinä, mutta vielä parempi olisi käyttää
varsinaista Unicoden miinusmerkkiä \uctunnus{u+2212 minus sign}, koska
se on fonteissa suunniteltu typografisesti yhteensopivaksi muiden
matemaattisten merkkien kanssa.

\subsection{Kolme pistettä eli ellipsi}

Ajatuksen katkeamista ja muuta sellaista ilmaisevalle kolmelle pisteelle
eli ellipsille (\ldots) on oma merkkinsä, ja fontissa se saattaa näyttää
hieman erilaiselta kuin kolme peräkkäistä pistemerkkiä. Tyypillisesti
ellipsimerkissä pisteet ovat hieman harvemmassa ja erottuvat toisistaan
paremmin kuin kolmena erillisenä merkkinä ladotut pisteet. Ellipsi
tehdään Latexissa komennoilla \koodi{\keno dots}, \koodi{\keno ldots},
\koodi{\keno text\-el\-lip\-sis} tai Unicode\-/merkillä \uctunnus{u+2026
  horizontal ellipsis}.

\subsection{Ylä- ja alaindeksi}

Yläindeksit (a\textsuperscript{2}) tehdään komennolla \koodi{\keno
  text\-super\-script} ja alaindeksit (a\textsubscript{2}) komennolla
\koodi{\keno text\-sub\-script}. Oletusasetuksilla Latex toteuttaa
indeksit mekaanisesti pienentämällä fonttia ja sijoittamalla pienennetyn
tekstin peruslinjan tai gemenalinjan tuntumaan. Lopputulos ei ole
typografisesti välttämättä kovin hyvä, koska fontin pienentäminen
ohentaa samalla merkkien viivoja ja ohuimmat hiusviivat voivat lähes
kadota.

\englanti{Open Type} \=/fontit sisältävät usein tuen oikeille ylä- ja
alaindekseille, jotka fontin suunnittelija on toteuttanut. Niitä
kannattaa käyttää, koska suunnittelija tuntee oman fonttinsa ja saa
todennäköisesti parempaa jälkeä kuin Latex mekaanisesti. \englanti{Open
  Type} \=/fonttien indeksit on kätevintä ottaa käyttöön
\paketti{real\-scripts}\-/paketin avulla.\avctan{realscripts}

Paketti \paketti{real\-scripts} määrittelee uudelleen Latexin ylä- ja
alaindeksikomennot, niin että ne ensisijaisesti pyrkivät käyttämään
\englanti{Open Type} \=/fontin ominaisuutta. Jos käytössä oleva fontti
ei sisällä haluttujen merkkien ylä- tai alaindeksiä,
\paketti{real\-scripts}\-/paketin komennot käyttävät automaattisesti
Latexin mekaanista keinoa. Paketti määrittelee pari muutakin hyödyllistä
komentoa, muun muassa tähdelliset versiot edellä mainituista:
\koodi{\keno text\-su\-per\-script*} ja \koodi{\keno
  text\-sub\-script*}. Nämä komennot toteuttavat aina mekaanisen ylä-
tai alaindeksin eli toimivat kuten Latexin alkuperäiset komennot.
Seuraavassa esimerkissä vertaillaan oikeita ja mekaanisia (tähdelliset)
ylä- ja alaindeksejä.

\pagebreak[3]

\begin{koodilohkosis}
  x\textsuperscript {ab36}
  x\textsuperscript*{ab36} \\
  H\textsubscript {2}SO\textsubscript {4}
  H\textsubscript*{2}SO\textsubscript*{4}
\end{koodilohkosis}

\begin{tulossis}
  x\textsuperscript {ab36}
  x\textsuperscript*{ab36} \\*
  H\textsubscript {2}SO\textsubscript {4}
  H\textsubscript*{2}SO\textsubscript*{4}
\end{tulossis}

Kuten esimerkki paljastaa, lopputuloksessa on eroa. Ensin mainitut ovat
fontin suunnittelijan versioita ja jälkimmäiset Latexin mekaanisesti
tekemiä. Mekaaninen ylä- ja ala\-in\-dek\-si\-toi\-min\-to ohentaa
merkkejä turhan paljon, eikä se ymmärrä poistaa gemenanumeroita
(3\,6\,2\,4) käytöstä vaan latoo ne sellaisenaan suunnilleen oikeaan
paikkaan.

Ylä- ja alaindeksejä käytettäessä on siis syytä ladata
\paketti{real\-scripts}\-/paketti ja käyttää indeksit hallitsevaa
\englanti{Open Type} \=/fonttia. Fonttien ominaisuuksia voi tutkia
käyttöjärjestelmän komentotulkissa komennolla \koodi{otf\-info}.
Toisaalta fonttiin sisältyviä ylä- ja alaindeksejä voi myös kirjoittaa
Unicode\-/merkistön avulla sellaisenaan. Lopputulos on sama.

\subsection{Tarkkeet ja erikoismerkit}
\label{luku:tarkkeet}

Latexissa on useita komentoja tarkkeellisten kirjainten
(\'a\,\v{s}\,\c{c}\,\~o) kirjoittamiseen sekä muille merkeille, joita ei
ehkä ihan helposti saa suoraan näppäimistöltä. Komentoja on koottu
taulukoihin \ref{tlk:tarkkeet}, \ref{tlk:erikoismerkit-lainaus} ja
\ref{tlk:erikoismerkit-muut}. Taulukon tarkekomennoissa on käytetty
a\=/kirjainta esimerkkinä, mutta tarke voi liittyä muihinkin kirjaimiin.
\marginaali{\TeX \\ \LaTeX} Merkit voi kirjoittaa
Latex\-/lähdedokumenttiin myös sellaisenaan, eli näiden komentojen
käyttö ei ole välttämätöntä. Tarke- ja erikoismerkkikomentojen lisäksi
on komennot ladontajärjestelmän logojen kirjoittamiseen: \koodi{\keno
  TeX} ja \koodi{\keno LaTeX}.

\providecommand{\rivi}{}
\renewcommand{\rivi}[3]{{\erikoisfontti #2} & \koodi{\keno #1} & #3 \\}

\leijutlk{
  \begin{tabular}[t]{cll}
    \toprule
    \multicolumn{2}{l}{\ots{Merkki}} & \ots{Selitys} \\
    \midrule
    \rivi{`a}{\`a}{gravis}
    \rivi{'a}{\'a}{akuutti}
    \rivi{\^{}a}{\^a}{sirkumfleksi}
    \rivi{\~{}a}{\~a}{tilde}
    \rivi{\textquotedbl a}{\"a}{treema}
    \rivi{H a}{\H a}{kaksoisakuutti}
    \rivi{r a}{\r a}{yläympyrä}
    \rivi{v a}{\v a}{hattu}
    %\rivi{t\{ae\}}{\t{ae}}{sidontakaari}
    \rivi{u a}{\u a}{lyhyysmerkki}
    \rivi{=a}{\=a}{pituusmerkki}
    \rivi{b a}{\b a}{alaviiva}
    \rivi{c a}{\c a}{sedilji}
    \rivi{.a}{\.a}{yläpiste}
    \rivi{d a}{\d a}{alapiste}
    \rivi{k a}{\k a}{ogonek}
    \\
    \rivi{L}{\L}{poikkiviiva-L}
    \rivi{l}{\l}{poikkiviiva-l}
    \bottomrule
  \end{tabular}\hspace{1.3em}%
  \begin{tabular}[t]{cll}
    \toprule
    \multicolumn{2}{l}{\ots{Merkki}} & \ots{Selitys} \\
    \midrule
    \rivi{O}{\O}{poikkiviiva-O}
    \rivi{o}{\o}{poikkiviiva-o}
    \rivi{DJ}{\DJ}{poikkiviiva-D}
    \rivi{dj}{\dj}{poikkiviiva-d}
    \rivi{DH}{\DH}{versaali-eth}
    \rivi{dh}{\dh}{gemena-eth}
    \rivi{NG}{\NG}{versaali-äng}
    \rivi{ng}{\ng}{gemena-äng}
    \rivi{SS}{\SS}{versaali kaksois-s}
    \rivi{ss}{\ss}{gemena kaksois-s}
    \rivi{TH}{\TH}{versaali thorn}
    \rivi{th}{\th}{gemena thorn}
    \rivi{i}{\i}{pisteetön i}
    \rivi{j}{\j}{pisteetön j}
    \rivi{AE}{\AE}{AE-ligatuuri}
    \rivi{ae}{\ae}{ae-ligatuuri}
    \rivi{OE}{\OE}{OE-ligatuuri}
    \rivi{oe}{\oe}{oe-ligatuuri}
    \bottomrule
  \end{tabular}
}{
  \caption{Komentoja tarkkeellisten ja muiden kirjainten
    kirjoittamiseen}
  \label{tlk:tarkkeet}
}

\providecommand{\rivi}{}
\renewcommand{\rivi}[3]{{\erikoisfontti #1} & \koodi{\keno #2} & #3 \\}

\leijutlk{
  \begin{tabular}{cll}
    \toprule
    \multicolumn{2}{l}{\ots{Merkki ja komennot}} & \ots{Selitys} \\
    \midrule
    \rivi{\textquotedblleft}{textquotedblleft `{}`}{vasen lainausmerkki}
    \rivi{\textquotedblright}{textquotedblright '{}'}{oikea lainausmerkki}
    \rivi{\textquotedbl}{textquotedbl}{yleislainausmerkki (\textsc{ascii})}
    \rivi{\textquoteleft}{textquoteleft \keno lq `}{vasen puolilainausmerkki}
    \rivi{\textquoteright}{textquoteright \keno rq '}{oikea
    puolilainausmerkki, heittomerkki}
    \rivi{\textquotesingle}{textquotesingle}{yleispuolilainausmerkki ja
    heittomerkki (\textsc{ascii})}
    \rivi{\guillemotleft}{guillemotleft <{}<}{vasen kulmalainausmerkki}
    \rivi{\guillemotright}{guillemotright >{}>}{oikea kulmalainausmerkki}
    \rivi{\guilsinglleft}{guilsinglleft}{vasen kulmapuolilainausmerkki}
    \rivi{\guilsinglright}{guilsinglright}{oikea
    kulmapuolilainausmerkki}
    \rivi{\quotedblbase}{quotedblbase}{rivinalinen lainausmerkki}
    \rivi{\quotesinglbase}{quotesinglbase}{rivinalinen puolilainausmerkki}
    \bottomrule
  \end{tabular}
}{
  \caption{Komentoja lainausmerkkien kirjoittamiseen}
  \label{tlk:erikoismerkit-lainaus}
}

\newcommand{\textbigcirclekorvike}{%
  \begin{tikzpicture}
    \draw (0,0) circle [radius=.65ex];
  \end{tikzpicture}}

\leijutlk{
  \begin{tabular}{cll}
    \toprule
    \multicolumn{2}{l}{\ots{Merkki ja komennot}} & \ots{Selitys} \\
    \midrule
    \rivi{\textendash}{textendash --}{lyhyt ajatusviiva}
    \rivi{\textemdash}{textemdash ---}{pitkä ajatusviiva}
    \rivi{\textexclamdown}{textexclamdown !`}{yläsalainen huutomerkki}
    \rivi{\textquestiondown}{textquestiondown ?`}{ylösalainen kysymysmerkki}
    \rivi{\textgreater}{textgreater}{suurempi kuin -merkki}
    \rivi{\textless}{textless}{pienempi kuin -merkki}
    \rivi{\ldots}{textellipsis \keno ldots \keno dots}{kolme pistettä, ellipsi}
    \rivi{\texteuro}{texteuro}{euron merkki}
    \rivi{\pounds}{textsterling \keno pounds}{punnan merkki}
    \rivi{\textdollar}{textdollar \keno \$}{dollarin merkki}
    \rivi{\S}{textsection \keno S}{pykälän merkki}
    \rivi{\P}{textparagraph \keno P}{kappaleen merkki}
    \rivi{\copyright}{textcopyright \keno copyright}{tekijänoikeusmerkki}
    \rivi{\textregistered}{textregistered}{rekisteröity tavaramerkki}
    \rivi{\texttrademark}{texttrademark}{tavaramerkki}
    \rivi{\dag}{textdagger \keno dag}{risti}
    \rivi{\ddag}{textdaggerdbl \keno ddag}{kaksoisristi}
    \rivi{\textasciicircum}{textasciicircum \keno \^{}\{\}}{sirkumfleksi}
    \rivi{\textasciitilde}{textasciitilde \keno \~{}\{\}}{tilde}
    \rivi{\textasteriskcentered}{textasteriskcentered}{rivinkeskinen
    asteriski, tähti}
    \rivi{\textbackslash}{textbackslash}{kenoviiva}
    \rivi{\textbar}{textbar}{pystyviiva}
    \rivi{\textbardbl}{textbardbl}{kaksoispystyviiva}
    \rivi{\textbraceleft}{textbraceleft \keno \{}{vasen aaltosulje}
    \rivi{\textbraceright}{textbraceright \keno \}}{oikea aaltosulje}
    \rivi{\textbullet}{textbullet}{luetelmaympyrä}
    \rivi{\textbigcirclekorvike}{textbigcircle}{suuri ympyrä}
    % \rivi{\textcircled{a}}{textcircled\{\ldots\}}{ympyröity merkki}
    \rivi{\textleftarrow}{textleftarrow}{nuoli vasemmalle}
    \rivi{\textrightarrow}{textrightarrow}{nuoli oikealle}
    \rivi{\textordfeminine}{textordfeminine}{feminiininen järjestysluvun
    merkki}
    \rivi{\textordmasculine}{textordmasculine}{maskuliininen järjestysluvun
    merkki}
    \rivi{\textperiodcentered}{textperiodcentered}{rivinkeskinen piste}
    \rivi{\textunderscore}{textunderscore \keno\_}{alaviiva}
    \rivi{\textvisiblespace}{textvisiblespace}{näkyvä välilyönti}
    \bottomrule
  \end{tabular}
}{
  \caption{Komentoja erikoismerkkien kirjoittamiseen}
  \label{tlk:erikoismerkit-muut}
}

\section{Komennot}
\label{luku:komennot}

Latexin komennot alkavat kenoviivalla (\koodi{\textbackslash}), jonka
jälkeen tulee komennon nimi. Nimi koostuu yleensä pienistä tai isoista
kirjaimista, mutta komento voi olla myös merkkejä.

Komennot voivat ottaa vastaan argumentteja eli lisätietoa, jota komento
käsittelee ja tarvitsee toimintaansa. Jotkin argumentit voivat olla
pakollisia ja jotkin valinnaisia. Pakolliset kirjoitetaan
aaltosulkeisiin \koodi{\{\ldots\}} ja valinnaiset hakasulkeisiin
\koodi{[\ldots]}.

\begin{koodilohkosis}
  \komento
  \komento{argu}{mentteja}
  \komento[valinnainen]{argu}{mentteja}
\end{koodilohkosis}

Jos pakolliseen argumenttiin haluaa sisällyttää aaltosulkeen, täytyy sen
eteen kirjoittaa kenoviiva (\koodi{\keno komento\{\keno\}\}}), tai voi
myös käyttää taulukossa \ref{tlk:merkkien_suojaus}
(s.~\pageref{tlk:merkkien_suojaus}) mainittuja komentoja aaltosulkeiden
tuottamiseen. Sama pätee aaltosulkeisiin muutenkin.

Hakasulkeet sen sijaan ladotaan tekstiin normaalisti, eikä niiden kanssa
käytetä kenoviivaa. Poikkeustilanne on komennon valinnaisen argumentin
sisällä, koska valinnainen argumentti jo sinänsä kirjoitetaan
hakasulkeiden sisään. Hakasulkeita ei voi suojata kenoviivalla, koska
\koodi{\keno[} ja \koodi{\keno]} ovat jo muuhun tarkoitettuja komentoja:
niillä luodaan matematiikkatilassa (luku \ref{luku:matematiikka}) oleva
tekstilohko. Valinnaisen argumentin sisään saa hakasulkeen, kun sen
kirjoittaa aaltosulkeiden sisään: \koodi{\keno komento[\{]\}]}.

Komennon yhteydessä sanavälejä käsitellään hieman poikkeuksellisesti.
Esimerkiksi komennon nimen perässä olevat sanavälit syödään pois, jos
komennolle ei anneta yhtään argumenttia. Seuraavassa esimerkissä sana
\emph{Latex} ladotaan ehjänä, jos vain \koodi{\keno ko\-men\-to}
itsessään ei kirjoita mitään eikä vaikuta tekstin latomiseen.

\pagebreak[3]

\begin{koodilohkosis}
  La\komento   tex
\end{koodilohkosis}

\begin{tulossis}
  Latex
\end{tulossis}

Jos haluaa komennon jälkeen yhden sanavälin, täytyy kirjoittaa komennon
nimen jälkeen aaltosulkeet (\koodi{\keno komento\{\}}) tai kenoviiva ja
sanaväli (\koodi{\keno komento\keno~}). Komennon nimen ja argumenttien
välissä voi olla sanavälejä, ja ne kaikki syödään pois. Komennon ja sen
argumentit voi siis kirjoittaa vaikka seuraavalla tavalla:

\begin{koodilohkosis}
  \komento  [valinnainen]
     {argu}   {mentteja}
\end{koodilohkosis}

\subsection{Omat komennot ja abstrahointi}

Omien komentojen tärkein tarkoitus on merkintätapojen abstrahointi eli
teknisen toteutuksen ja yksityiskohtien piilottaminen. Abstrahointi voi
helpottaa työskentelyä huomattavastikin ja tehdä lähdedokumentista
helppolukuisen. Otetaan esimerkki tämän oppaan toteutuksesta.

Tämä opas käsittelee välillä Unicode\-/merkkejä ja niiden tunnuksia
kuten \uctunnus{u+0061 latin small letter~a}. Merkkien tunnukset
ladotaan erilaisella kirjainleikkauksella, pienversaalilla eli
kapiteelilla. Jos fontti vain tukee pienversaalia, sitä saadaan helposti
komennolla \koodi{\keno text\-sc}. Silti tätä komentoa ei ehkä kannata
käyttää suoraan dokumentissa. Jos nimittäin kirjoittaa dokumentin
täyteen \koodi{\keno text\-sc}\-/komentoja, voi myöhemmin olla työlästä
muuttaa Unicode\-/tunnusten ulko\-asu ja toteutus toisenlaiseksi.

Jossain vaiheessa kirjoittaja oivaltaa, että Unicode\-/tunnukset ovat
englannin kieltä ja että sanoja pitäisi tavuttaa englannin sääntöjen
mukaan eikä suomen. Pitäisi siis liittää \koodi{\keno
  text\-sc}\-/komentojen ympärille vielä \koodi{\keno
  text\-en\-glish}\-/komento.%
\footnote{Komento \koodi{\keno text\-english} on
  \paketti{polyglossia}\-/paketin ominaisuus.} Lähdedokumentin
kirjoittaminen on ikävää ja ulko\-asu sekava, jos Unicode\-/tunnusten
kohdalla on aina \koodi{\keno text\-english\{\keno text\-sc\{\ldots\}\}}.

Parasta olisi jo ennen kirjoittamista miettiä, mitä rakenteita teksti
sisältää. Erilaisia rakenteita tai ilmauksia varten kannattaa luoda
komento, jotta sen toteutusta voi myöhemmin muuttaa helposti.
Esimerkiksi Unicoden merkkitunnuksille voisi luoda oman komennon
\koodi{\keno uctunnus}, jota käytetään aina tunnusten ilmaisemiseen
Latex\-/dokumentissa. Komento määritellään vain yhdessä paikassa, ja
määritelmää on helppoa muuttaa. Tekstikin pysyy selkeänä, kun siellä on
vain korkean tason komentoja, jotka ilmaisevat tarkoituksen eivätkä
teknistä toteutusta.

\subsection{Komentojen määrittely}
\label{luku:komennot-määr}

Komentojen määrittelyyn on kolme erilaista komentoa, ja niille kaikille
annetaan samanlaiset argumentit. Komennot ovat seuraavat:

\begin{koodilohkosis}
  \newcommand     {\nimi}[n][oletus]{määritelmä}
  \renewcommand   {\nimi}[n][oletus]{määritelmä}
  \providecommand {\nimi}[n][oletus]{määritelmä}
\end{koodilohkosis}

Ensimmäinen pakollinen argumentti on komennon nimi (\koodi{\keno
  ni\-mi}), ja se voi koostua vain kirjaimista. Komento \koodi{\keno
  newcommand} määrittelee uuden komennon. Mikäli komento on jo olemassa,
annetaan virheilmoitus. Toinen komento \koodi{\keno renewcommand}
määrittelee olemassa olevan komennon uudelleen. Se antaa
virheilmoituksen, jos komentoa ei ollut olemassa. Kolmas komento
\koodi{\keno pro\-vide\-command} puolestaan määrittelee uuden komennon
vain siinä tapauksessa, että sellaista ei ollut ennen olemassa. Se ei
anna virheilmoitusta.

Komentojen toinen pakollinen argumentti (\koodi{mää\-ri\-tel\-mä})
sisältää komennon määritelmän eli mitä tahansa tekstiä tai komentoja.
Suoritusvaiheessa komento ikään kuin vaihdetaan sen määritelmäksi.

\pagebreak[3]

\begin{koodilohkosis}
  \newcommand{\komento}{Minua komennettiin!}
  \komento
\end{koodilohkosis}

\begin{tulossis}
  Minua komennettiin!
\end{tulossis}

Komentojen ensimmäinen valinnainen argumentti (\koodi{n}) on luku, joka
kertoo, kuinka monta argumenttia määriteltävä komento käsittelee.
Määritelmässä voi käyttää parametreja \koodi{\#1}, \koodi{\#2},
\koodi{\#3} jne., ja ne korvautuvat komennon suoritusvaiheessa
ensimmäisellä, toisella, kolmannella jne. argumentilla.

\pagebreak[3]

\begin{koodilohkosis}
  \newcommand{\komento}[2]{Sanoit #1 ja #2!}
  \komento{hip}{hei}
\end{koodilohkosis}

\begin{tulossis}
  Sanoit hip ja hei!
\end{tulossis}

Toinen valinnainen argumentti (\koodi{ole\-tus}) -- jos se on mukana --
kertoo, että määriteltävän komennon ensimmäinen argumentti on
valinnainen ja että tämä on sen ole\-tus\-arvo. Ole\-tus\-arvoa
käytetään silloin, kun valinnaista argumenttia ei ole annettu.

\pagebreak[3]

\begin{koodilohkosis}
  \newcommand{\komento}[3][tyyppi]{Hei #1, sanoit #2 ja #3!}
  \komento{hip}{hei} \\
  \komento[Leslie]{hip}{hei}
\end{koodilohkosis}

\begin{tulossis}
  Hei tyyppi, sanoit hip ja hei! \\
  Hei Leslie, sanoit hip ja hei!
\end{tulossis}

Joskus yhden komennon määritelmä sisältää komennon \koodi{\keno
  renewcommand}, joka sitten määrittelee uudelleen jonkin toisen
komennon. Silloin parametrit \koodi{\#1}, \koodi{\#2} jne. on
tarkoitettu ensimmäisen eli uloimman kerroksen käsiteltäväksi. Sisempi
kerros käyttää parametreja \koodi{\#\#1}, \koodi{\#\#2} jne.

Kaikista kolmesta komentojen määrittelykomennosta on olemassa
tähdellinen versio eli sellainen, jonka komennon nimen lopussa on tähti
(\koodi{*}). Latexin komennoissa on tapana, että tähdellinen versio --
jos sellainen on olemassa -- tarjoaa samaan asiaan jonkin toisenlaisen
näkökulman.

\begin{koodilohkosis}
  \newcommand*     {\nimi}[n][oletus]{määritelmä}
  \renewcommand*   {\nimi}[n][oletus]{määritelmä}
  \providecommand* {\nimi}[n][oletus]{määritelmä}
\end{koodilohkosis}

Komentojen määrittelyssä tähdelliset versiot toimivat muuten samalla
tavalla, mutta ne antavat virheilmoituksen, jos komennolle annetut
argumentit sisältävät enemmän kuin yhden tekstikappaleen. Niinpä
seuraava esimerkki tuottaa käännettäessä virheen:

\begin{koodilohkosis}
  \newcommand*{\komento}[1]{Teksti: #1}

  \komento{
    Ensimmäinen tekstikappale.

    Toinen tekstikappale.
  }
\end{koodilohkosis}

\koodi{\keno newcommand*}\-/komennolla määritelty \koodi{\keno
  ko\-men\-to} ei siis suostu ottamaan vastaan argumentteja, jotka
sisältävät kappaleen vaihtumisen eli enemmän kuin yhden tekstikappaleen.
Tähdelliset versiot tuovat tällaisen suo\-jaus\-omi\-nai\-suu\-den.

\subsection{Muita vinkkejä}
\label{luku:komennot-lisä}

Komennon viimeistä argumenttia ei välttämättä tarvitse kirjoittaa
aaltosulkeisiin, jos argumentiksi halutaan vain yksi merkki. Tällaisessa
tilanteessa komento poimii argumentiksi seuraavan merkin, joka ei ole
sanaväli.

\pagebreak[3]

\begin{koodilohkosis}
  \newcommand{\x}[1]{Argumentti: <#1>}
  \x abc \\
  \x.abc
\end{koodilohkosis}

\begin{tulossis}
  Argumentti: <a>bc \\
  Argumentti: <.>abc
\end{tulossis}

Mikäli argumenttina on kenoviivalla alkava komento, sitäkään ei tarvitse
kirjoittaa aaltosulkeisiin. Latex\-/koodin lukemisen kannalta tällainen
ei välttämättä ole hyvä käytäntö, koska joskus voi hämärtyä, onko kyse
kahdesta peräkkäisestä komennosta vai onko toinen komento vain
argumenttina toiselle.

\pagebreak[3]

\begin{koodilohkosis}
  \newcommand{\x}[1]{Argumentti: <#1>}
  \newcommand{\yyy}{abc}
  \x\yyy
\end{koodilohkosis}

\begin{tulossis}
  Argumentti: <abc>
\end{tulossis}

Tätä merkintätapaa esiintyy jokin verran komentojen määrittelyssä, niin
että jätetään \koodi{\keno newcommand}\-/komennon ensimmäisenä
argumenttina oleva komennon nimi ilman aaltosulkeita.

\begin{koodilohkosis}
  \newcommand\yyy{abc}
\end{koodilohkosis}

Komennon määrittelyssä on välillä hyötyä \koodi{\keno
  ig\-nore\-spaces}\-/komennosta, joka jättää sanavälit huomioimatta
komennon jälkeen. Ilman tätä komentoa tulisi seuraavassa esimerkissä
sanojen väliin yksi välilyönti.

\pagebreak[3]

\begin{koodilohkosis}
  \newcommand{\komento}[1]{#1\ignorespaces}
  \komento{yhdys}       sana
\end{koodilohkosis}

\begin{tulossis}
  yhdyssana
\end{tulossis}

Aaltosulkeilla (luku \ref{luku:aaltosulkeet}) voi rajata
komentomäärittelyn vai\-ku\-tus\-aluet\-ta. Seuraavan esimerkin alussa
asetetaan \koodi{\keno ko\-men\-to} tiettyyn alkuperäismääritelmään.
Aaltosulkeiden sisällä se määritellään väliaikaisesti uudestaan.
Aaltosulkeilla rajatun ympäristön jälkeen komennon uusi määritelmä
lakkaa ja komento palautuu alkuperäiseksi.

\pagebreak[3]

\begin{koodilohkosis}
  \newcommand{\komento}{alkuperäinen}
  \komento
  {%
    \renewcommand{\komento}{muutettu}
    \komento
  }
  \komento
\end{koodilohkosis}

\begin{tulossis}
  alkuperäinen muutettu alkuperäinen
\end{tulossis}

Monimutkaisiin komentoihin voidaan tarvita ehtorakenteita. Ne saa
toteutettua \paketti{ifthen}\=/paketin\avctan{ifthen} tarjoaman
\koodi{\keno if\-then\-else}\-/komennon avulla. Se on
ohjelmointikielistä tuttu ehtorakenne: jos annettu ehtolauseke on tosi,
käsitellään then\-/haara; muussa tapauksessa käsitellään else-haara.

\section{Ympäristöt}
%\label{luku:ymparistot}

Ympäristöt ovat rakenteita, joilla on aloittava \koodi{\keno
  begin}\-/komento ja lopettava \koodi{\keno end}\-/komento sekä nimi.
Ympäristöjen ajatuksena on, että jokin ominaisuus tai jotkin toiminnot
ovat voimassa vain ympäristön sisällä ja asiat palautuvat ennalleen
ympäristön jälkeen. Tässä mielessä ne toimivat samalla tavalla kuin
aaltosulkeet (luku \ref{luku:aaltosulkeet}). Jos esimerkiksi
fonttiasetusta (luku \ref{luku:kirjaintyypit}) muuttaa ympäristön
sisäpuolella, asetus palautuu ympäristön jälkeen samaksi kuin se oli
ennen ympäristön alkua. Samoin ympäristön sisällä määritellyt komennot
ovat voimassa vain kyseisessä ympäristössä.

\begin{koodilohkosis}
  \begin{nimi}
    % ympäristön
    % vaikutusalue
  \end{nimi}
\end{koodilohkosis}

Yleisin ympäristö on nimeltään \koodi{document}, jonka sisään koko
dokumentin sisältö kirjoitetaan. Muita ympäristöjä käytetään
tavallisesta leipätekstistä poikkeavien rakenteiden ilmaisemiseen,
esimerkiksi luetelmiin ja taulukoihin (luvut \ref{luku:luetelmat} ja
\ref{luku:taulukot}). Ympäristöjä voi tehdä itsekin mihin hyvänsä
tarkoitukseen. Niitä määritellään seuraavilla komennoilla:

\begin{koodilohkosis}
  \newenvironment   {nimi}[n][oletus]{aloitus}{lopetus}
  \renewenvironment {nimi}[n][oletus]{aloitus}{lopetus}
\end{koodilohkosis}

Komento \koodi{\keno newenvironment} määrittelee uuden ympäristön. Se
antaa virheilmoituksen, jos samanniminen ympäristö on jo olemassa.
Komento \koodi{\keno renewenvironment} puolestaan määrittelee uudelleen
ympäristön, joka on jo olemassa. Se antaa virheilmoituksen, jos
ympäristöä ei ollutkaan olemassa.

Argumentit ovat lähes samanlaiset kuin komentojen määrittelyssä (luku
\ref{luku:komennot-määr}). Ympäristöjen määrittelykomennoilla on kolme
pakollista argumenttia: ensimmäinen on ympäristön nimi, toinen on
ympäristön aloitusmääritelmä (\koodi{aloi\-tus}) ja kolmas on
lopetusmääritelmä (\koodi{lope\-tus}).

\pagebreak[3]

\begin{koodilohkosis}
  \newenvironment{ymp}{Tästä se alkaa.}{Tähän se päättyy.}

  \begin{ymp}
    Ympäristön sisältöä.
  \end{ymp}
\end{koodilohkosis}

\begin{tulossis}
  Tästä se alkaa. Ympäristön sisältöä. Tähän se päättyy.
\end{tulossis}

Tavallisesti ympäristön aloitusmääritelmään kirjoitetaan jonkin toisen
ympäristön aloituskomento sekä mahdollisesti suoritetaan joitakin
asetuskomentoja. Vastaavasti lopetusmääritelmässä lopetetaan ympäristö
eli palataan normaaliin tilaan. Tarkoituksena on abstrahoida jokin
monimutkaisempi kokonaisuus eli tehdä uusi helppokäyttöinen ympäristö, joka
piilottaa teknisen toteutuksen.

\begin{koodilohkosis}
  \newenvironment{ymp}
  {\begin{mahtavuus}
      \omia\hienoja\asetuksia}
    {\end{mahtavuus}}
\end{koodilohkosis}

Omille ympäristölle voi määritellä argumentteja samalla tavalla kuin
komennoillekin. Ensimmäinen valinnainen argumentti (\koodi{n}) on luku
joka kertoo, kuinka monta argumenttia määriteltävä ympäristö käsittelee.
Ympäristön aloitusmääritelmässä voi argumentteihin viitata parametreilla
\koodi{\#1}, \koodi{\#2}, \koodi{\#3} jne.

Jos toinen valinnainen argumentti (\koodi{oletus}) on mukana, se
ilmaisee, että määriteltävän ympäristön ensimmäinen argumentti on
valinnainen ja että tämä on sen ole\-tus\-arvo. Oletusta käytetään
silloin, kun valinnaista argumenttia ei ole annettu. Argumentit annetaan
ympäristön aloittavan \koodi{\keno begin}-ko\-men\-non yhteydessä.

\begin{koodilohkosis}
  \begin{ymp}[valinnainen]{argu}{mentteja}
    % ympäristön
    % vaikutusalue
  \end{ymp}
\end{koodilohkosis}

Ympäristön määrittelykomennoille on myös tähdelliset versiot
\koodi{\keno newenvironment*} ja \koodi{\keno renewenvironment*}. Ne
toimivat samoin kuin edellä kuvatut tavallisetkin komentoversiot, mutta
ne eivät salli, että määritellylle ympäristölle annetut argumentit
sisältävät enemmän kuin yhden tekstikappaleen. Toiminta on siis sama
kuin komentojenkin määrittelyn tähdellisissä versioissa (luku
\ref{luku:komennot-määr}).

\begin{koodilohkosis}
  \newenvironment*   {nimi}[n][oletus]{aloitus}{lopetus}
  \renewenvironment* {nimi}[n][oletus]{aloitus}{lopetus}
\end{koodilohkosis}

Joskus ympäristöjen määrittelyyn on hyödyllistä sisällyttää komento
\koodi{\keno ig\-nore\-spaces}, joka jättää huomioimatta tämän komennon
jälkeiset sanavälit. Toinen hyödyllinen on \koodi{\keno
  ig\-nore\-spaces\-af\-ter\-end}, joka jättää huomioimatta ympäristön
lopettavan \koodi{\keno end}\-/komennon jälkeiset sanavälit.

\pagebreak[3]

\begin{koodilohkosis}
  \newenvironment{ymp}
    {Yhdys\ignorespaces}
    {esi\ignorespacesafterend}

  \begin{ymp}
    sana% Kommentti poistaa seuraavan sanavälin.
  \end{ymp}   merkki.
\end{koodilohkosis}

\begin{tulossis}
  Yhdyssanaesimerkki.
\end{tulossis}

\section{Tavutus}
\label{luku:tavutus}

Tex tavuttaa eli katkaisee sanoja automaattisesti rivien lopussa, jotta
se saa tekstikappaleet näyttämään tasapainoisilta. Lähtökohtaisesti
tavutus määräytyy kielikohtaisten tavutussääntöjen ja \=/asetusten
perusteella, mutta kirjoittaja voi tehdä poikkeuksia kirjoittamalla
tavutusvihjeitä. Käytännössä tavutusvihjeitä tarvitaan välillä.

Kielen valintaa ja yleisiä kieliasetuksia käsitellään tarkemmin luvussa
\ref{luku:kieliasetukset}, mutta mainitaan tässäkin, että suomen kieli
ja sen mukainen tavutus saadaan päälle, kun dokumentin esittelyosaan
kirjoittaa seuraavat komennot:

\begin{koodilohkosis}
  \usepackage{polyglossia}
  \setdefaultlanguage{finnish}
\end{koodilohkosis}

Tavutusta käsitellään tässä yhteydessä lähinnä
\paketti{polyglossia}\-/kielipaketin näkökulmasta. Toinen kielipaketti
\paketti{babel} toimii joissakin tavutukseen liittyvissä asioissa eri
tavalla, ja pakettien eroja mainitaan tekstin lomassa.

\subsection{Yleiset tavutussäännöt}

Texin automaattinen tavutus ei perustu varsinaiseen sanojen eikä
taivutusmuotojen tunnistamiseen vaan yksinkertaisiin kirjainpohjaisiin
sääntöihin. Säännöt pyrkivät kuvaamaan kielen tavujen rakenteen ja
huomioimaan myös typografiaan liittyviä suosituksia.

Automaattinen tavutus auttaa paljon, mutta se ei yksinään riitä, koska
se tekee välillä virheitä tai tuottaa muuten suositusten vastaista
jälkeä. Kirjoittajan täytyy kirjoittaa välillä eli tavutusvihjeitä. Yksi
tapa tavutusvihjeiden kirjoittamiseen on \koodi{\keno
  hyphenation}\-/komento, jolla määritellään yksittäisten sanojen
tavutuskohdat kaikkialla dokumentissa. Seuraava esimerkki selventää
komennon käyttöä:

\begin{koodilohkosis}
  \hyphenation{
    ala-indek-si alku-osa
    nimen-omaan
    typo-gra-fi-nen
  }
\end{koodilohkosis}

Komennon perään aaltosulkeiden sisään kirjoitetaan sanoja, jotka
erotetaan toisistaan sanaväleillä. Sanoihin kirjoitetaan yhdysmerkki
niihin kohtiin, joista sanan katkaiseminen on sallittua. Jos sanassa
itsessään on yhdysmerkki, sen tavutusta ei voi käsitellä tällä
komennolla. Luvussa \ref{luku:tavutuksen_merkit} kerrotaan muita tapoja.

\koodi{\keno hyphenation}\-/komennon voi sijoittaa dokumentin
esittelyosaan tai tekstiosaan, mutta sijainti vaikuttaa sen toimintaan.
Jos komennon sijoittaa dokumentin esittelyosaan ennen kuin mitään kieltä
on ladattu tai valittu, se vaikuttaa kaikkien sanojen tavutukseen
kielestä riippumatta. Jos komennon sijoittaa dokumentin tekstiosaan eli
kielen valitsemisen jälkeen, se vaikuttaa vain kyseisen kielen eli
yleensä dokumentin pääasiallisen kielen tavutukseen.

\subsection{Yksittäisten sanojen tavutus}
\label{luku:tavutuksen_merkit}

Tietyt sanassa mukana olevat merkit kytkevät muut tavutussäännöt pois
päältä ja muuttavat sanan tavutuksen yksilölliseksi. Jos sanassa on
mukana yksikin tavutusvihje (\koodi{\keno-}), yhdysmerkki (\koodi{-}),
lyhyt ajatusviiva (\mbox{\koodi{--}}) tai pitkä ajatusviiva
(\mbox{\koodi{---}}), sana katkaistaan vain näiden kohdalta
(\koodi{\keno-}) tai jälkeen.%
\footnote{\paketti{polyglossia}\-/kielipaketti ja Tex toimivat tekstissä
  kuvatulla tavalla. Sen sijaan \paketti{babel}\-/paketti määrittelee
  ainakin suomen kielelle tavutusvihjeen (\koodi{\keno-}) siten, että se
  sallii sanan tavutuksen muistakin kohdista kuin tavutusvihjeen
  kohdalta. Tavutusvihjeen molemmin puolin sanan osat tavutetaan
  yleisten sääntöjen mukaisesti, ellei tavutuksen estäviä merkkejä kuten
  yhdysmerkkejä tai ajatusviivoja ole. \paketti{babel}\-/paketissa on
  myös tavutusvihje \koodi{\textquotedbl\textbar}, joka estää tavutuksen
  muualta.}

Myös Unicoden ajatusviivamerkit \uctunnus{u+2013 en dash} ja
\uctunnus{u+2014 em dash} toimivat, mutta ne käyttäytyvät tavutuksen
kannalta eri tavalla Lualatex\-/{} ja Xelatex\-/kääntäjillä.
Yhteensopivuussyistä kannattanee välttää Unicoden ajatusviivoja
Latex\-/lähdetiedostossa.

\leijutlk{
  \begin{tabular}{lll}
    \toprule
    \ots{Lähde}
    & \ots{Tavutus}
    & \ots{Selitys} \\
    \midrule
    \koodi{matkustaa}
    & mat\tavukohta kus\tavukohta taa
    & tavutus kaikista kohdista \\
    \koodi{matkus\keno-taa}
    & matkus\tavukohta taa
    & vain tavutusvihjeen kohdalta \\
    \koodi{matka-aika}
    & matka-\tavukohta aika
    & vain yhdysmerkin jälkeen \\
    \koodi{matka-ai\keno-ka}
    & matka-\tavukohta ai\tavukohta ka
    & vain yhdysmerkki ja tavutusvihje \\
    \koodi{Oulu--Rovaniemi}
    & Oulu--\tavukohta Rovaniemi
    & vain ajatusviivan jälkeen \\
    \koodi{Oulu--Rova\keno-niemi}
    & Oulu--\tavukohta Rova\tavukohta niemi
    & vain ajatusviiva ja tavutusvihje \\
    \koodi{matka-}
    & matka-
    & ei tavutuskohtia \\
    \koodi{-aika}
    & -\tavukohta aika
    & vain yhdysmerkin jälkeen \\
    \bottomrule
  \end{tabular}
}{
  \caption{Tavutusvihjeen, yhdysmerkin ja ajatusviivan vaikutus
    tavutukseen}
  \label{tlk:tex-tavutus}
}

Taulukossa \ref{tlk:tex-tavutus} on esimerkkejä tavutusvihjeiden,
yhdysmerkin ja ajatusviivan vaikutuksesta. Ensimmäisessä sarakkeessa on
esimerkkisana siinä muodossa kuin se kirjoitetaan lähdetiedostoon.
Toisessa sarakkeessa on ladottu sana, johon on pystyviivalla merkittynä
mahdolliset tavutuskohdat.

Taulukon viimeinen rivi paljastaa suomen kielen kannalta ongelmallisen
tilanteen. Esimerkiksi ilmauksessa \emph{matkasuunnitelma ja
  \mbox{-aika}} ei riviä saa katkaista sanassa \emph{\mbox{-aika}}
olevan yhdysmerkin jälkeen, koska rivin loppuun jäisi yksinäinen
yhdysmerkki. Latexin peruskeinoilla sana täytyy laittaa näkymättömään
laatikkoon, joka pitää merkit yhdessä: \koodi{\keno mbox\{-aika\}}.
Toinen vaihtoehto on sitovan yhdysmerkin käyttö, jota käsitellään
seuraavassa alaluvussa.

\subsection{Tavutuksen sallivia ja sitovia merkkejä}

Unicode\-/merkistön sitova yhdysmerkki \uctunnus{u+2011 non\-/breaking
  hyphen} näkyy tavallisena yhdysmerkkinä, mutta se estää sanan
katkaisemisen yhdysmerkin vierestä. \uctunnus{u+2011}\-/merkkiä voi
käyttää Xelatex\-/kääntäjän kanssa, mutta Lualatex\-/kääntäjä kadottaa
koko merkin. Yhteensopivuussyistä ei kannata käyttää Unicoden sitovaa
yhdysmerkkiä, sillä muitakin vaihtoehtoja on.

Makropaketti \paketti{extdash}\avctan{extdash} tuo uusia komentoja ja
mahdollisuuksia tavutuksen hallintaan. Komennot ovat sellaisia kuin
\koodi{\keno Hyph\-dash} ja \koodi{\keno En\-dash}, mutta niille on
saatavilla myös lyhemmät muodot, jos paketin lataa käyttämällä
\koodi{short\-cuts}\-/valitsinta.

\begin{koodilohkosis}
  \usepackage[shortcuts]{extdash}
\end{koodilohkosis}

Paketti sisältää kaksi lisä\-vaihto\-ehtoa kolmelle viivavälimerkille
eli yhdysmerkille, lyhyelle ajatusviivalle ja pitkälle ajatusviivalle.
Kun Texin viivavälimerkit (luku \ref{luku:tavutuksen_merkit}) aina
estävät tavutuksen muualta kuin välimerkin jälkeen,
\paketti{extdash}\-/paketin perus\-vaihto\-ehdot sallivat tavutuksen
muualtakin. Lisäksi kaikille kolmelle viivavälimerkille on sitova
versio, joka estää tavutuksen välimerkin jälkeen (mutta sallii muualta).

Taulukossa \ref{tlk:extdash} ovat \paketti{extdash}\-/paketin tärkeimmät
komennot ja niiden merkitykset. Taulukossa \ref{tlk:extdash-vertailu}
vertaillaan \paketti{extdash}\-/paketin komentoja Texin vastaaviin.

\leijutlk{
  \begin{tabular}{ll}
    \toprule
    \ots{Komento} & \ots{Merkitys} \\
    \midrule
    \koodi{\keno -/} & tavutuksen salliva yhdysmerkki \\
    \koodi{\keno =/} & sitova, tavutuksen salliva yhdysmerkki \\
    \koodi{\keno --} & tavutuksen salliva lyhyt ajatusviiva \\
    \koodi{\keno ==} & sitova, tavutuksen salliva lyhyt ajatusviiva \\
    \koodi{\keno ---} & tavutuksen salliva pitkä ajatusviiva \\
    \koodi{\keno ===} & sitova, tavutuksen salliva pitkä ajatusviiva \\
    \bottomrule
  \end{tabular}
}{
  \caption{\paketti{extdash}-paketin komentoja}
  \label{tlk:extdash}
}

\leijutlk{
  \begin{tabular}{llll}
    \toprule
    \ots{Lähde} & \ots{Tavutus}
    & \ots{Lähde} & \ots{Tavutus} \\
    \midrule
    \koodi{matka-aika}
                & matka-\tavukohta aika
                & \koodi{Oulu--Rovaniemi}
                & Oulu--\tavukohta Rovaniemi \\
    \koodi{matka\keno -/aika}
                & mat\tavukohta ka-\tavukohta ai\tavukohta ka
                & \koodi{Oulu\keno --Rovaniemi}
                & Ou\tavukohta lu--\tavukohta Ro\tavukohta va\tavukohta
                  nie\tavukohta mi \\
    \koodi{matka\keno =/aika}
                & mat\tavukohta ka-ai\tavukohta ka
                & \koodi{Oulu\keno ==Rovaniemi}
                & Ou\tavukohta lu--Ro\tavukohta va\tavukohta
                  nie\tavukohta mi \\
    \koodi{matka-}
                & matka-
                & \koodi{-aika}
                & -\tavukohta aika \\
    \koodi{matka\keno -/}
                & mat\tavukohta ka-
                & \koodi{\keno =/aika}
                & -ai\tavukohta ka \\
    \bottomrule
  \end{tabular}
}{
  \caption{Texin ja \paketti{extdash}-paketin komentojen vertailua}
  \label{tlk:extdash-vertailu}
}

On tärkeää huomioida, että \paketti{extdash}\-/paketin komennot ovat
todellakin normaaleja komentoja. Se tarkoittaa, että komennon jälkeiset
sanavälit syödään pois. Tämä asia saattaa unohtua seuraavanlaisessa
tilanteessa, jossa käytetään tavutuksen sallivaa yhdysmerkkikomentoa
\koodi{\keno-/}:

\pagebreak[3]

\begin{koodilohkosis}
  matka\-/ ja aika\-/arvio
\end{koodilohkosis}

\begin{tulossis}
  matka\-/ ja aika\-/arvio
\end{tulossis}

Ensimmäisen yhdysmerkkikomennon jälkeinen sanaväli hävisi, ja syntyi
virheellinen sana \emph{matka-ja}. Sanavälin saa säilymään, kun
kirjoittaa komennon perään aaltosuljeparin tai kenoviivan ja sanavälin.

\pagebreak[3]

\begin{koodilohkosis}
  matka\-/{} ja aika\-/arvio \\
  matka\-/\ ja aika\-/arvio
\end{koodilohkosis}

\begin{tulossis}
  matka\-/{} ja aika\-/arvio \\
  matka\-/\ ja aika\-/arvio
\end{tulossis}

\subsection{Tavutus sanan reunasta}

Asetukset \koodi{\keno left\-hy\-phen\-min=N} ja \koodi{\keno
  right\-hy\-phen\-min=N} vaikuttavat tavutukseen sanan reunoissa.
Argumentti \koodi{N} on positiivinen kokonaisluku, ja se määrittelee,
kuinka monta merkkiä vähintään sanan vasemmasta tai oikeasta reunasta
pidetään yhdessä. Oletus\-arvot ovat kielikohtaisia ja määritellään
\paketti{polyglossia}\-/{} ja \paketti{babel}\-/paketeissa. Suomen
kielessä kumpikin asetus on kaksi (2) merkkiä.

Nämä asetukset alustetaan kielikohtaisiin oletus\-arvoihin aina, kun
kieliasetukset tulevat voimaan. Näin on esimerkiksi dokumentin
aloittavan \koodi{document}\-/ympäristön alussa tai kielen vaihtuessa.
Tästä seuraa, että jos asetuksia ha\-luaa muuttaa, täytyy ne asettaa
\koodi{document}\-/ympäristön alussa tai aina kielen vaihtamisen
jälkeen.

Omat kieli\-asetukset saa mukaan automaattisesti, kun lisää muutokset
kielikohtaisiin alustuskomentoihin alla olevan esimerkin mukaisesti.
Tällä tavoin omat asetukset tulevat voimaan samalla kuin muutkin
kieliasetukset, esimerkiksi kielen vaihtuessa.

\begin{koodilohkosis}
  \addto{\captionsfinnish}{
    \lefthyphenmin=3
    \righthyphenmin=3
  }
\end{koodilohkosis}

Esimerkissä oleva komento \koodi{\keno add\-to} on
\paketti{polyglossia}\-/{} ja \paketti{babel}\-/paketin ominaisuus, ja
tässä käsitellään suomen kielen asetuksia (\koodi{\keno
  cap\-tions\-finnish}).

\subsection{Muita tavutusasetuksia ja -vinkkejä}
\label{luku:tavutus_muut}

Aiemmin mainittu \paketti{extdash}\-/makropaketti sisältää tavutuksen
hallintaan liittyviä välimerkkejä, mutta sellaisia on myös
kielipaketeissa \paketti{polyglossia} ja \paketti{babel}. Kielipakettien
englanninkielisissä ohjekirjoissa niitä kutsutaan nimellä
\emph{\englanti{short\-hands}}.

\paketti{babel}\-/paketti tuo suomen kieleen sitovan, tavutuksen
sallivan yhdysmerkin \koodi{\textquotedbl-}, joka toimii samoin kuin
\paketti{extdash}\-/paketin komento \koodi{\keno=/}. Lisäksi on
tavutusvihje (\koodi{\textquotedbl\textquotedbl}), joka ei tee
yhdysmerkkiä tavutuskohtaan. Sitä voi käyttää joidenkin teknisten
ilmausten tavutusvihjeenä, jotta ilmaus säilyy täsmälleen samanlaisena
myös rivien katkaisemisessa. \paketti{babel}\-/paketin toiminnot
vaihtelevat eri kielten välillä.

\paketti{polyglossia}\-/pakettiin saa samat lisätoiminnot ja eräitä
muitakin, kun paketin lataa käyttämällä valitsinta
\koodi{babel\-short\-hands}. Edellä mainittujen ''Baabelin lyhenteiden''
lisäksi tulee yhdysmerkki \koodi{\textquotedbl\textasciitilde}, joka on
sitova ja samalla kaiken tavutuksen estävä yhdysmerkki. Tälle merkille
ei ole vastinetta \paketti{extdash}\-/paketissa. Pari muutakin
lainausmerkillä alkavaa, tavutukseen liittyvää merkkiä on mukana.

Varsinainen%
\koodimargin{\keno discretionary} tavutuksen peruskomento on
\koodi{\keno discretionary}, joka mahdollistaa omanlaisten
tavutuskohtien määrittelyn. Komento kirjoitetaan sanassa juuri siihen
kohtaan, johon tavutuskohta halutaan. Komennolle annetaan kolme
argumenttia järjestyksessä seuraavasti: 1) katkaisutilanteessa rivin
loppuun ladottavat merkit, 2) katkaisutilanteessa seuraavan rivin alkuun
ladottavat merkit ja 3) katkaisemattomaan sanaan ladottavat merkit.
Normaali tavutuskohta sanaan \emph{tavu} määriteltäisiin alla olevan
esimerkin tavoin:

\begin{koodilohkosis}
  ta\discretionary{-}{}{}vu
\end{koodilohkosis}

Edellisessä esimerkissä komennon ensimmäinen argumentti on yhdysmerkki,
koska rivin loppuun tietenkin halutaan yhdysmerkki silloin, kun sana
katkaistaan tästä kohdasta. Toinen argumentti on tyhjä, koska seuraavan
rivin alkuun ei kirjoiteta suomen kielessä mitään ylimääräistä. Myös
kolmas argumentti on tyhjä, koska katkaisemattomaan sanaan ei haluta
mitään merkkiä tavujen väliin.

\koodi{\keno discretionary}\-/komennolla voi luoda myös tavutuskohtia,
joihin ei ilmesty yhdysmerkkiä eikä mitään muutakaan merkkiä
katkaisutilanteessa. Se saadaan aikaan jättämällä kaikki argumentit
tyhjäksi. Tällainen sopii tavutusvihjeeksi teknisiin ilmauksiin, joihin
ei selvyyden vuoksi haluta mitään ylimääräisiä merkkejä edes sanan
katkaisutilanteessa.

\begin{koodilohkosis}
  \discretionary{}{}{}
\end{koodilohkosis}

Joissakin kielissä sanan katkaiseminen yhdysmerkin kohdalta vaatii, että
rivin loppuun kirjoitetaan yksi yhdysmerkki ja seuraavan rivin alkuun
toinen. Näin ilmaistaan, että sanassa on pysyvä yhdysmerkki eikä vain
väliaikaisesti katkaisun merkkinä. Tällaisia kieliä varten saattaa
kielipaketissa (tai muualla) olla omat yhdysmerkkitoiminnot, mutta
edellä kuvatun tavutuskohdan saa myös seuraavalla tavalla:

\begin{koodilohkosis}
  \discretionary{-}{-}{-}
\end{koodilohkosis}

Tavutuksen suunnittelussa ja tutkimisessa voi auttaa
\paketti{show\-hy\-phens}\-/paketti,\avctan{showhyphens} jonka
lataamalla kaikki tavutuskohdat tulevat näkyviin. Dokumentin kaikkien
sanojen tavutuskohtiin piirretään ohut punainen pystyviiva. Tämä paketti
hyödyntää Lualatex\-/kääntäjän ominaisuuksia, eikä se siis toimi muiden
kääntäjien kanssa.

\subsection{Suomen kielen tavutus}

Texin kirjainyhdistelmiin perustuvat säännöt eivät yksinään riitä suomen
kieleen, ja esimerkiksi yhdyssanojen rajakohdat tuottavat usein
ongelmia. Sana \emph{alku\-osa} katkaistaan sääntöjen mukaan kohdista
\emph{al-kuo-sa}. Se on kyllä oikein tavurakenteen kannalta mutta
käytännössä ongelmallinen. Tässä ei ole kyse \emph{uo}\-/diftongista,
vaan yhdyssanan rajalla on myös tavutuskohta (\emph{al-ku-o-sa}).
Lisäksi sanaa ei saa katkaista siten, että siitä jää yksittäinen kirjain
eri riville (\emph{o-sa}).

Parasta olisi katkaista suomen kielen yhdyssanat vain yhdys\-osien
välistä (\emph{alku-osa}). Muualtakin voi katkaista (\emph{al-ku-osa}),
kunhan sanasta eikä sen yhdys\-osasta ei jää yksittäinen kirjain eri
riville. Mielellään ei katkaista myöskään kahden vokaalin välistä, jos
ne kuuluvat samaan sanaan (\emph{kau-emmin}). Joskus halutaan välttää
myös niin sanottuja orpotavuja eli sitä, että tekstikappaleen
viimeiselle riville jää vain yksi tavu.

Käytännössä siis suomenkielinen teksti ja hyvä typografia vaativat
välillä tavutusvihjeiden kirjoittamista. Yhdyssanojen osien väliin
tarvitaan tavutusvihje silloin, kun jälkimmäinen osa alkaa vokaalilla
tai useammalla kuin yhdellä konsonantilla. Joitakin tällaisia tapauksia
Tex osaa tavuttaa oikein ilman tavutusvihjeitäkin, mutta etukäteen sitä
ei tiedä, ellei ole kokemusta.

\begin{koodilohkosis}
  alku\-osa pusku\-traktori
\end{koodilohkosis}

Yhdysmerkin tai ajatusviivan sisältävät pitkät yhdyssanat voivat vaatia
tavutuskohtien lisäämistä, koska Texin yhdysmerkki ja ajatusviivat
estävät tavutuksen muualta kuin näiden merkkien jälkeen. Ilman
tavutuskohtien lisäämistä Texillä ei ehkä ole riittävästi vaihtoehtoja
tekstikappaleen rivittämiseen. Voi syntyä liian suuria sanavälejä, tai
joistakin riveistä tulee ylipitkiä, eli ne yltävät marginaalin puolelle.

Tavutuskohtia voi lisätä sopiviin kohtiin omilla tavutusvihjeillä
(\koodi{\keno-}). Vaihtoehtoisesti kaikki yleisten tavutussääntöjen
mukaiset tavutuskohdat saa käyttöön \paketti{extdash}\-/paketin
komennoilla (taulukko \ref{tlk:extdash}). Seuraavassa esimerkissä on
tavutuksen hallintaa sanalle \emph{Molo\-tov--Ribben\-trop-sopi\-mus}:

\begin{koodilohkosis}
  Molo\-tov--Ribben\-trop-sopi\-mus % tavutusvihjeet
  Molotov\--Ribben\-trop\-/sopimus  % tavutuksen sallivat välimerkit
\end{koodilohkosis}

\koodi{\keno discretionary}\-/komentoa (luku \ref{luku:tavutus_muut})
voi hyödyntää suomen kielen sanoissa silloin, kun niissä on heittomerkki
erottamassa kahta tavurajan molemmin puolin olevaa samaa vokaalia,
esimerkiksi sanoissa \emph{vaa'an} ja \emph{liu'uttaa}. Normaalisti Tex
ei katkaise heittomerkin kohdalta lainkaan, eikä se olisi suomen
kielessä suositeltavaakaan, koska vokaalien välistä ei mielellään
katkaista sanaa. Jos tavutuksen kuitenkin haluaa myös heittomerkin
kohdalle, täytyy huolehtia, että tavutustilanteessa heittomerkki poistuu
ja sen paikalle tulee yhdysmerkki rivin loppuun. Sanan \emph{vaa'an}
voisi siis kirjoittaa seuraavalla tavalla:

\begin{koodilohkosis}
  vaa\discretionary{-}{}{'}an
\end{koodilohkosis}

Komennon kolmas argumentti on heittomerkki, koska se pitää latoa
tavurajalle silloin, kun sanaa ei katkaista tästä kohdasta. Mikäli
tällaisia tarvitsee paljon, on järkevää määritellä sitä varten
yksinkertaisempi komento, jota sitten käytetään sanoissa heittomerkin
sijasta.

\begin{koodilohkosis}
  \newcommand{\hm}{\discretionary{-}{}{'}}
  vaa\hm an
\end{koodilohkosis}

Suomen kielessä käytetään heittomerkkiä myös taivutuspäätteen, liitteen
tai johtimen edellä silloin, kun sanavartalon kir\-joi\-tus\-asu päättyy
konsonanttiin mutta ääntö\-asu vokaaliin, esimerkiksi \emph{show'ssa}.
Näissä tilanteissa ei ole kyse tavurajasta vaan morfeemirajasta eli
merkityksellisten osien rajakohdasta. Tavurajakin voi sattua samaan
paikkaan, mutta tavutettaessa heittomerkki säilyy: \emph{show'-hun}.%
\footnote{Asiaa ei yleensä mainita kielenhuolto\-/oppaissa. Tieto
  perustuu Kielikello\-/lehden 2/2006 artikkeliin:
  \kulmaurl{https://www.kielikello.fi/-/lainausmerkit-}. Viittauspäivä
  6.7.2020.} Mikäli tällainen tavutuskohta halutaan mukaan, käytetään
sanassa tavallista tavutusvihjettä: \koodi{show'\keno-hun}. Mieluummin
ei kuitenkaan katkaista sanoja heittomerkin kohdalta.

\subsection{Tavutus pois päältä}

Tavutuksen voi kytkeä kokonaan pois päältä
\paketti{polyglossia}\-/kielipaketin komennolla%
\koodimargin{\keno disable\-hyphenation} \koodi{\keno
  dis\-able\-hy\-phen\-ation}. Tavutuksen saa takaisin päälle komennolla
\koodi{\keno en\-able\-hy\-phen\-ation}. Toinen vaihtoehto on käyttää
fontin asetusta \koodi{Hyphen\-Char=\katk None} samalla, kun ottaa
käyttöön tietyn kirjainperheen.

\begin{koodilohkosis}
  \setmainfont{Fontin nimi}[HyphenChar=None]
\end{koodilohkosis}

Edellä oleva komento on peräisin \paketti{fontspec}\-/fonttipaketista
(luku \ref{luku:kirjaintyypit}). Oletuksena tavutus on pois päältä
tasalevyisen fontin kanssa, eli esimerkiksi \koodi{\keno
  text\-tt\{…\}}\-/komennon argumenttina olevaa tekstiä ei tavu\-teta.

Käytännössä tavutus menee pois päältä myös silloin, kun tekstikappaleet
tasaa vain vasempaan reunaan eli tekee oikealle liehureunan komennolla
\koodi{\keno ragged\-right}. Tekstikappaleiden latomista käsitellään
luvussa \ref{luku:kappale}.

\section{Mitat}
\label{luku:mitat}

\subsection{Mittayksiköt}

Koska typografia on pitkälti teksti- ja muiden elementtien sijoittelua,
tarvitaan sitä varten mittavälineitä. Niinpä Latexissakin on
pituusmittoja (engl. \emph{length}) ja useita pituuden mittayksiköitä.
Taulukkoon \ref{tlk:mittayksikot} on koottu mittayksiköiden lyhenteet ja
merkitykset. Teknisesti on samantekevää, mitä yksiköitä käyttää, sillä
ne ovat vain välineitä pituuden ilmaisemiseen. Sisäisesti Tex käyttää
sp-yksikköä, joka ilmaisee samalla mittojen tarkkuuden: pienin jakamaton
mitta on 1\,sp.

\leijutlk{
  \begin{tabular}{ll}
    \toprule
    \ots{Lyh.} & \ots{Merkitys} \\
    \midrule
    bp & piste uudessa pica-järjestelmässä, 1/72 tuumaa, 0,3528 mm \\
    pt & piste vanhassa pica-järjestelmässä, 1/72,27 tuumaa, 0,3515 mm \\
    pc & pica eli 12 pt-pistettä \\
    sp & 1/65536 pt-pistettä, Texin sisäisesti käyttämä yksikkö \\
    dd & piste Didot-järjestelmässä, 0,376 mm \\
    cc & cicero eli 12 dd-pistettä \\
    mm & millimetri \\
    cm & senttimetri \\
    in & tuuma, 25,4 mm \\
    ex & nykyisen fontin x-korkeus, perus- ja gemenalinjan etäisyys \\
    em & typografisen neliön sivun pituus, sama kuin fontin koko \\
    \bottomrule
  \end{tabular}
}{
  \caption{Latexin mittayksiköiden lyhenteet ja merkitykset}
  \label{tlk:mittayksikot}
}

Usein tietyt yksiköt ovat vakiintuneet tiettyihin tilanteisiin.
Esimerkiksi fonttien kokoja ja rivikorkeuksia on tapana mitata
pistemittojen avulla. Nykyään käytetään lähinnä bp\-/yksikön mukaista
pistettä, joka tuli käyttöön Post Script \=/standardin myötä vuonna 1984
ja jota käytetään julkaisuohjelmissa. Sivun mittoja kuten leveyttä,
korkeutta ja marginaaleja ilmaistaan tavallisesti metrijärjestelmän
avulla eli senttimetreissä tai millimetreissä.

Latexin mittojen mittaluvuissa desi\-maa\-li\-erot\-ti\-me\-na on piste,
ja mittayksikön lyhenne kirjoitetaan kiinni mittalukuun. Seuraavassa
esimerkissä tehdään vaakasuuntaisia ja pystysuuntaisia välejä
komennoilla \koodi{\keno hspace} ja \koodi{\keno vspace}:

\pagebreak[3]

\begin{koodilohkosis}
  Sanat\hspace{1.2cm}hassusti
  \vspace{2mm}

  \hspace{1.75em}erillään.
\end{koodilohkosis}

\begin{tulossis}
  Sanat\hspace{1.2cm}hassusti  \nopagebreak
  \vspace{2mm}

  \hspace{1.75em}erillään.
\end{tulossis}

\subsection{Mittakomennot ja typografinen viivasto}

Mittoja tallennetaan eräänlaisiin muuttujiin, jotka näyttävät päällepäin
komennoilta eli niiden alussa on kenoviiva (\koodi{\keno}) ja sitten
kirjaimista koostuva nimi. Esimerkiksi mitta \koodi{\keno text\-width}
on teks\-ti\-alueen leveys nykyisellä sivulla. Komentomaisesta
ulko\-asus\-taan huolimatta mittoja ei voi suorittaa komentoina; ne
sopivat vain komennon argumentiksi, silloin kun tarvitaan mitta.

Uusia mittoja luodaan komennolla \koodi{\keno new\-length} ja olemassa
olevia mittoja asetetaan esimerkiksi komennoilla \koodi{\keno
  set\-length} ja \koodi{\keno add\-to\-length}. Omien mittoja on
tarpeen luoda silloin, kun halutaan määritellä tietynsuuruinen mitta,
jota käytetään Latex\-/koodissa useita kertoja.

\begin{koodilohkosis}
  \newlength{\omamitta}         % Luodaan mitta.
  \setlength{\omamitta}{2.3em}  % Asetetaan mitta.
  \addtolength{\omamitta}{1em}  % Lisätään mittaan.
  \addtolength{\omamitta}{-1em} % Vähennetään mitasta.
\end{koodilohkosis}

Näin luotuja mittoja voi käyttää mittayksiköiden tavoin, eli niille voi
asettaa eteen kertoimen. Esimerkiksi seuraava \koodi{\keno
  hspace}\-/komento luo vaakasuuntaisen välin, jonka pituus on 0,7
kertaa \koodi{\keno oma\-mitta}:

\begin{koodilohkosis}
  \hspace{0.7\omamitta}
\end{koodilohkosis}

Koska vaakasuuntaisten välien tekeminen on typografiassa varsin
tavallista, on niitä varten olemassa omia komentoja. Typografisen neliön
levyisen (1~em) välin voi tehdä komennolla \koodi{\keno quad}. Sen
puolikkaan (\murtoluku{1}{2}\,em) saa komennolla \koodi{\keno
  en\-space}. Ohuke eli \murtoluku{1}{6}\,em-väli tehdään
\koodi{\keno,}\-/komennolla, josta on tarkempaa tietoa luvussa
\ref{luku:ohuke}.

\leijukuva{
  \begin{tikzpicture}
    [viiva/.style={line width=.7bp}, xscale=1.1, yscale=1.9,
    baseline=0pt]
    \node at (.04,.25) {\fontsize{120bp}{120bp}\viivastofontti p};

    \draw [viiva, color=red] (-1,.98) rectangle (1,-.5);
    \draw [viiva, color=red] (1,.98) -- (2.25,.98);
    \draw [viiva, color=red] (-1,.02) -- (2.25,.02);
    \draw [viiva, color=red] (1,-.5) -- (2.25,-.5);

    \draw [viiva, <->, color=black] (1.2,.02) -- (1.2,.98);
    \node at (1.9,.5) {korkeus};

    \draw [viiva, <->, color=black] (1.2,.02) -- (1.2,-.5);
    \node at (1.9,-.25) {syvyys};

    \draw [viiva, <->, color=black] (-1,1.1) -- (1,1.1);
    \node at (0,1.25) {leveys};

    \node at (3.13,1.64) {ylälinja};
    \node at (3.13,1.44) {versaalilinja};
    \node at (3.13,.98) {gemenalinja};
    \node at (3.13,.02) {peruslinja};
    \node at (3.13,-.5) {alalinja};

    \node at (6.52,.74) {\fontsize{120bp}{120bp}\viivastofontti H};

    \draw [viiva, color=red] (5.05,1.47) rectangle (7.95,.02);
    \draw [viiva, color=red] (5.05,.02) -- (4,.02);
    \draw [viiva, color=red] (5.05,1.47) -- (4,1.47);
    \draw [viiva, color=red] (7.95,1.62) -- (4,1.62);
    \draw [viiva, color=red] (7.95,.98) -- (4,.98);

    \draw [viiva, <->, color=black] (4.86,.02) -- (4.86,1.47);
    \node at (4.1,.5) {korkeus};

    \draw [viiva, <->, color=black] (5.05,-.1) -- (7.95,-.1);
    \node at (6.5,-.3) {leveys};
\end{tikzpicture}
}{
  \caption{Kirjainten mitat ja typografinen viivasto}
  \label{kuva:kirjainmitat}
}

Latex sisältää myös komennot mittojen poimimiseen kirjaimista tai muista
merkeistä. Mittoja on kolme: leveys (\englanti{width}), korkeus
(\englanti{height}) ja syvyys (\englanti{depth}). Leveys on merkin
viemä tila leveyssuunnassa. Korkeus tarkoittaa merkin korkeutta
peruslinjan yläpuolella, ja syvyys on merkin korkeus peruslinjan
alapuolella. Kuva \ref{kuva:kirjainmitat} havainnollistaa näitä kolmea
mittaa sekä typografista viivastoa.

Gemenalinja sijaitsee gemenakirjainten eli pienaakkosten korkeudella.
Peruslinjan ja gemenalinjan etäisyys on fontin x\=/korkeus (1\,ex).
Optisen vaikutelman vuoksi gemenalinja ei tarkoita ehdotonta
gemenakirjainten ylintä kohtaa. Esimerkiksi p\=/kirjaimen yläpääte ja
silmukan yläkaari voivat yltää hieman gemenalinjan yläpuolelle ja
silmukan alakaari peruslinjan alapuolelle. Fonttien suunnittelijat
tekevät tällaisia optisia korjauksia, jotta kirjainten hahmot näyttävät
yhtä korkeilta ja tasapainoisilta yhdessä.

Versaalilinja sijaitsee versaalikirjainten eli suuraakkosten
korkeudella. Jotkin kirjaimet voivat yltää hieman versaalilinjan
yläpuolellekin. Esimerkiksi joidenkin fonttien Hk\=/kirjaimia
vertailemalla näkee, että k\=/kirjaimen yläpääte yltää H\=/kirjaimen eli
versaalilinjan yläpuolelle, ylälinjan tuntumaan. Myös tarkemerkit (luku
\ref{luku:tarkkeet}) voivat sijaita versaalilinjan yläpuolella.

\pagebreak[3]

Merkkien leveyden, korkeuden ja syvyyden voi mitata alla olevan
esimerkin komennoilla. Esimerkin ensimmäisellä rivillä luodaan uudet
mitat \koodi{\keno leveys}, \koodi{\keno korkeus} ja \koodi{\keno
  syvyys}, joihin sitten tallennetaan merkkien mitat.

\begin{koodilohkosis}
  \newlength{\leveys} \newlength{\korkeus} \newlength{\syvyys}
  \settowidth{\leveys}{abc} % merkkien ”abc” leveys
  \settoheight{\korkeus}{H} % merkin ”H” korkeus
  \settodepth{\syvyys}{p}   % merkin ”p” syvyys
\end{koodilohkosis}

\pagebreak[3]

Mitat saa näkyviin kirjoittamalla mitan eteen komennon \koodi{\keno
  the}. Näin mitan pituus ladotaan dokumenttiin. Yksikkönä on
typografinen piste (pt).

\begin{koodilohkosis}
  Leveys: \the\leveys, korkeus: \the\korkeus, syvyys: \the\syvyys.
\end{koodilohkosis}

\begin{tulossis}
  Leveys: 14.62865pt, korkeus: 6.81897pt, syvyys: 2.4662pt.
\end{tulossis}

Aiemmin mainituista optisista korjauksista johtuu, että gemenakirjaimet
eivät ole välttämättä täsmälleen samankorkuisia. Gemena\=/x asettuu
perus- ja gemenalinjojen väliin, mutta gemena\=/o voi yltää aavistuksen
verran linjojen ylä- ja alapuolelle. Ihmisen silmään ne näyttävät yhtä
korkeilta.

\subsection{Venyvät mitat}

Edellä on puhuttu vain kiinteistä mitoista, mutta Latex tuntee myös
venyvät mitat. Niiden ajatuksena on, että Latexille voi antaa luvan
kutistaa tai venyttää mittaa tietyissä rajoissa. Venyviä mittoja
käytettään usein pystysuuntaisissa väleissä, esimerkiksi väliotsikon
edellä ja jälkeen tai tekstikappaleiden välissä. Latex pystyy latomaan
sivut yleensä paremman näköiseksi, kun sille antaa venyvien mittojen
avulla hieman säätövaraa. Alla on esimerkki venyvän kappalevälin
(\koodi{\keno par\-skip}) määrittämisestä. Samalla asetetaan kappaleen
ensimmäisen rivin sisennys (\koodi{\keno par\-in\-dent}) nollaan
(\koodi{0em}).

\begin{koodilohkosis}
  \setlength{\parskip}{2ex plus 0.2ex minus 0.1ex}
  \setlength{\parindent}{0em}
\end{koodilohkosis}

Venyvissä mitoissa mainitaan ensin normaali pituus ja sitten sanoilla
\koodi{plus} ja \koodi{minus} ilmaistaan, kuinka paljon mitta voi venyä
tai kutistua. Molempia ei välttämättä tarvitse antaa.

Venyvät mitat voivat sisältää ''äärettömän'' mittayksikön \koodi{fill},
joka antaa luvan venyttää mittaa niin, että kaikki käytettävissä oleva
tila täyttyy. Äärettömän mittayksikön kanssa mittaluku ilmaisee
suhdeluvun muihin äärettömiin mittoihin. Seuraava esimerkki
havainnollistaa asiaa:

\pagebreak[3]

\begin{koodilohkosis}
  x\hspace{0mm plus 1fill}x\hspace{0mm plus 2fill}x
\end{koodilohkosis}

\begin{tulossis}
  x\hspace{0mm plus 1fill}x\hspace{0mm plus 2fill}x
\end{tulossis}

Edellisen esimerkin mittojen luonnollinen pituus on nolla millimetriä
(\koodi{0mm}), mutta \koodi{plus} ja \koodi{fill} \=/mitan vuoksi ne
voivat venyä ja täyttää koko käytettävissä olevan tilan. Ensin
mainitulla on suhdeluku 1 (\koodi{1fill}) ja jälkimmäisellä suhdeluku 2
(\koodi{2fill}), joten jälkimmäinen väli on kaksinkertainen ensimmäiseen
verrattuna. Saman äärettömästi venyvän mitan ja suhdeluvut voi ilmaista
myös \koodi{\keno stretch}\-/mitan tai \koodi{\keno hfill}\-/komentojen
avulla seuraavasti:

\pagebreak[3]

\begin{koodilohkosis}
  x\hspace{\stretch{1}}x\hspace{\stretch{2}}x \\
  x\hfill x\hfill\hfill x
\end{koodilohkosis}

\begin{tulossis}
  x\hspace{\stretch{1}}x\hspace{\stretch{2}}x \\
  x\hfill x\hfill\hfill x
\end{tulossis}

Vastaavasti pystysuuntaisen äärettömästi venyvän välin voi tehdä
komennolla \koodi{\keno vfill}, joka tekee saman asian kuin \koodi{\keno
  vspace\{0mm plus 1fill\}}.

Äärettömästi venyvälle, koko tilan täyttävälle mittayksikölle on itse
asiassa kolme eri versiota: \koodi{fil}, \koodi{fill} ja \koodi{filll}.
Niiden erona on tärkeysjärjestys. Ensin mainittu \koodi{fil} on vähiten
tärkeä, ja viimeinen eli \koodi{filll} on tärkein. Jos samassa
yhteydessä käytetään eri tär\-keys\-as\-tei\-sia yksiköitä,
korkeammantasoiset mitätöivät alemmantasoiset.

\pagebreak[3]

\begin{koodilohkosis}
  x\hspace{0mm plus 1filll}x\hspace{0mm plus 1fill}x
\end{koodilohkosis}

\begin{tulossis}
  x\hspace{0mm plus 1filll}x\hspace{0mm plus 1fill}x
\end{tulossis}

Edellisessä esimerkissä ensimmäinen venyvä mitta \koodi{1filll} mitätöi
jälkimmäisen \koodi{1fill} kokonaan. Yleensä lienee parasta käyttää
keskimmäistä (\koodi{fill}), mutta eri tär\-keys\-as\-teil\-le voi olla
käyttöä esimerkiksi makropaketin koodissa: \koodi{fil} sallii, että
käyttäjä tai muu koodi syrjäyttää asetuksen; \koodi{filll} on ehdoton
sääntö, joka syrjäyttää muut.

\section{Laskurit}

Laskureiden avulla Latex pitää kirjaa esimerkiksi sivunumeroista,
lukujen ja alalukujen sekä kuvien ja muiden elementtien numeroinnista.
Esimerkiksi nyt olemme pääluvun~\arabic{chapter}
alaluvussa~\arabic{section}. Edellisen virkkeen luvut tulevat Latexin
laskureista automaattisesti.

\leijutlk{
  \ttfamily
  \begin{tabular}{llll}
    \toprule
    part & paragraph & figure & enumi \\
    chapter & subparagraph & table & enumii \\
    section & page & footnote & enumiii \\
    subsection & equation & mpfootnote & enumiv \\
    subsubsection \\
    \bottomrule
  \end{tabular}
}{
  \caption{Latexin laskurit}
  \label{tlk:latexin_laskurit}
}

Taulukossa \ref{tlk:latexin_laskurit} ovat Latexin perus laskurit. Monet
niistä ovat otsikoiden numerointia varten (\englanti{\koodi{part},
  \koodi{chapter}, \koodi{section}, \koodi{paragraph}}). Sivunumeroita
varten on \koodi{page}\-/laskuri ja matemaattisten kaavojen
numerointiin \koodi{equation}\-/laskuri. Leijuvista kuvista ja
taulukoista (luku \ref{luku:leijuosat}) pidetään kirjaa laskureissa
\koodi{figure} ja \koodi{table}, ja alaviitteiden (luku
\ref{luku:alaviitteet}) numerointi on laskureissa \koodi{footnote} ja
\koodi{mpfootnote}. Numeroidut luetelmat (luku \ref{luku:luetelmat})
käsitellään \koodi{enum}\-/alkuisten laskureiden avulla siten, että
perustasolla käytetään laskuria \koodi{enumi}, ja jos sen luetelmakohta
sisältää toisen numeroidun luetelman, käytetään siinä laskuria
\koodi{enumii}, sen sisällä laskuria \koodi{enumiii} jne.

Edellä mainituista laskureista ei tarvitse yleensä itse huolehtia, sillä
ne ovat vain tekniikkaa, joka toimii korkeamman tason toimintojen
taustalla. Joskus on kuitenkin käyttöä myös omille laskureille. Alla
ovat laskureiden käsittelyn peruskomennot. Niissä käytetään laskuria
nimeltä \koodi{oma}. Laskurien nimet koostuvat pelkistä kirjaimista,
eikä niiden alussa ole kenoviivaa niin kuin komentojen ja mittojen
alussa.

\begin{koodilohkosis}
  \newcounter{oma}       % Luodaan uusi laskuri ”oma”.
  \setcounter{oma}{3}    % Asetetaan laskurin arvoksi 3.
  \addtocounter{oma}{1}  % Lisätään laskurin arvoon 1.
  \addtocounter{oma}{-1} % Vähennetään laskurin arvosta 1.
\end{koodilohkosis}

Laskuri on sisäisesti kokonaisluku, mutta sen arvon voi tulostaa eri
muodoissa: arabialaisena tai roomalaisena lukuna, kirjaimena tai
symbolien sarjana. Taulukossa \ref{tlk:laskurien_tulostus} ovat komennot
laskurin arvon tulostamiseen. Komennon argumentiksi annetaan laskurin
nimi. Laskurin voi tulostaa kirjainmuodossa, jos sen arvo on 1--26;
symbolit toimivat vain lu\-ku\-alueel\-la 1--9.

\leijutlk{
  \begin{tabular}{ll}
    \toprule
    \ots{Komento} & \ots{Merkitys} \\
    \midrule
    \koodi{\keno arabic\{\ldots\}} & arabialainen luku: 1, 2, 3\ldots \\
    \koodi{\keno roman\{\ldots\}} & roomalainen luku: i, ii, iii\ldots \\
    \koodi{\keno Roman\{\ldots\}} & roomalainen luku: I, II, III\ldots \\
    \koodi{\keno alph\{\ldots\}} & kirjain: a, b, c\ldots \\
    \koodi{\keno Alph\{\ldots\}} & kirjain: A, B, C\ldots \\
    \koodi{\keno fnsymbol\{\ldots\}} & symboli:
                                  \textasteriskcentered
                                  \textdagger
                                  \textdaggerdbl\S\P\ldots \\
    \bottomrule
  \end{tabular}
}{
  \caption{Komennot laskurien arvon tulostamiseen}
  \label{tlk:laskurien_tulostus}
}

Taulukon \ref{tlk:laskurien_tulostus} komentojen lisäksi laskuriin
liittyvän arvon voi tulostaa komennolla, joka alkaa kirjaimilla
\koodi{\keno the} ja jatkuu laskurin nimellä. Esimerkiksi komento
\koodi{\keno the\-page} tulostaa sivunumeron eli tekee käytännössä saman
asian kuin komento \koodi{\keno arabic\{page\}}. Aina nämä eivät
kuitenkaan ole sama asia, ja \koodi{\keno the}\-/alkuinen komento voi
olla määritelty toisellakin tavalla. Esimerkiksi tämän oppaan leijuvat
taulukot numeroidaan laskurilla \koodi{table}, mutta komento
\koodi{\keno the\-table} tulostaa ensin pääluvun ja sen perään pisteen
ja taulukon numeron. Viimeisin taulukko oli numeroltaan \thetable{} eli
pääluvun \arabic{chapter} taulukko \arabic{table}.

Jos laskurin arvoa tarvitaan Latexin teknisessä tilanteessa eikä
tarkoituksena ole latoa sitä näkyviin itse dokumenttiin, täytyy käyttää
komentoa \koodi{\keno val\-ue}. Seuraava esimerkki luo laskurin nimeltä
\koodi{mitta}, jonka arvoa (3) käytetään \koodi{\keno hspace}\-/komennon
argumenttina mitan ilmaisemiseen (3\,em).

\pagebreak[3]

\begin{koodilohkosis}
  \newcounter{mitta}
  \setcounter{mitta}{3}
  x\hspace{\value{mitta}em}x
\end{koodilohkosis}

\begin{tulossis}
  x\hspace{3em}x
\end{tulossis}

\subsection{Hierarkkiset laskurit}

Laskurit voivat olla hierarkkisia ja riippuvaisia toisista laskureista:
kun yhden laskurin arvo kasvaa, nollautuu jokin toinen laskuri
automaattisesti. Tällaista toimintoa käytetään dokumentin lukujen eli
otsikoiden numeroinnissa. Kirjan pääluvun~1 alaluvut ovat esimerkiksi
1.1, 1.2, 1.3 jne. Kun alkaa seuraava pääluku~2, nollautuu alaluvun
laskuri, jotta saadaan oikein numeroidut alaluvut 2.1, 2.2, 2.3 jne.
Latex huolehtii lukujen numeroinnista automaattisesti, mutta omille
laskureille täytyy määritellä riippuvuussuhteet itse. Seuraavassa
esimerkissä luodaan oma laskuri, joka nollautuu automaattisesti aina,
kun sivu vaihtuu eli laskurin \koodi{page} arvo kasvaa.

\begin{koodilohkosis}
  \newcounter{oma}[page]
\end{koodilohkosis}

Ylemmän tason laskureita ei kannata kasvattaa komennolla \koodi{\keno
  add\-to\-counter}, koska se vain muuttaa laskurin arvoa mutta ei
huolehdi alemman tason laskurien nollaamisesta. Hierarkkisten laskurien
kasvattamiseen on seuraavat kaksi komentoa:

\begin{koodilohkosis}
  \stepcounter{laskuri}
  \refstepcounter{laskuri}
\end{koodilohkosis}

Edellä mainitut komennot kasvattavat argumenttina annetun laskurin arvoa
yhdellä ja nollaavat siitä riippuvaiset alemman tason laskurit.
Jälkimmäinen komento \koodi{\keno ref\-step\-counter} asettaa lisäksi
ristiviitteen numeron, joten tätä komentoa seuraava \koodi{\keno
  label}\-/komento viittaa tähän laskuriin ja sen \koodi{\keno
  the\-las\-ku\-ri}\-/komennon tulosteeseen. Ristiviitteitä käsitellään
tarkemmin luvussa \ref{luku:ristiviitteet}.

Makropaketti \paketti{chngcntr}\avctan{chngcntr} helpottaa
hierarkkisten laskurien muuttamista jälkikäteen. Paketin komentojen
avulla voi jälkikäteen asettaa jonkin laskurin riippuvaiseksi toisesta
laskurista tai poistaa riippuvuuden.

\begin{koodilohkosis}
  \newcounter{oma}           % Luodaan laskuri ”oma”.
  \counterwithin {oma}{page} % Asetetaan riippuvuus page-laskurista.
  \counterwithout{oma}{page} % Poistetaan riippuvuus.
\end{koodilohkosis}

Edellä mainitut \paketti{chngcntr}\-/paketin komennot määrittelevät
uudelleen myös \koodi{\keno the\-oma}\-/komennon, jolla
\koodi{oma}\-/laskurin arvon voi tulostaa. Se tulostaa mukaan myös
ylemmäntasoisen laskurin arvon. Jos tätä \koodi{\keno the}\-/alkuista
komentoa ei halua määritellä uudelleen, täytyy käyttää tähdellisiä
komennon versioita: \koodi{\keno counter\-with\-in*} ja \koodi{\keno
  counter\-with\-out*}. Laskuriin liittyvän \koodi{\keno the}\-/komennon
voi aina itsekin määritellä haluamansa laiseksi komennolla \koodi{\keno
  renewcommand} (luku \ref{luku:komennot-määr}), esimerkiksi seuraavalla
tavalla:

\begin{koodilohkosis}
  \renewcommand{\theoma}{\arabic{page}/\alph{oma}}
\end{koodilohkosis}

\subsection{Kokonaislaskurit}

Latex ei tiedä etukäteen, mihin arvoon laskurit lopulta yltävät. Se vain
latoo sivuja peräkkäin eikä tiedä mitään tulevasta. Sen vuoksi
esimerkiksi dokumentin sivumäärää ei voi ihan yksinkertaisella
komennolla latoa dokumentin alkusivuille.

Vastaus tämäntyyppisiin ongelmiin löytyy ristiviitteistä (luku
\ref{luku:ristiviitteet}). Ne toimivat sisäisesti niin, että dokumentin
ensimmäisellä kääntökerralla kirjoitetaan väliaikaistiedostoon muistiin
tarpeellisia asioita ja seuraavalla kääntökerralla hyödynnetään
väliaikaistiedostoa. Dokumentin sivumäärän ja viimeisen sivunumeron
tallentamiseen on olemassa makropaketti
\paketti{tot\-pages},\avctan{totpages} joka lisää viimeiselle sivulle
automaattisesti ristiviitteen nimeltä \koodi{Tot\-Pages}. Sivumäärän ja
viimeisen sivun numeron voi tulostaa komennoilla \koodi{\keno ref} ja
\koodi{\keno page\-ref}. Esimerkki \ref{esim:totpages} havainnollistaa
niiden käyttöä.

\begin{esimerkki*}
\begin{koodilohko}
  \documentclass{article}
  \usepackage{totpages}

  \begin{document}
  Sivumäärä: \ref{TotPages}. Viimeinen sivu: \pageref{TotPages}.

  \addtocounter{page}{10} % Sivunumerot 11, 12, 13...
  \newpage Jotain... \newpage ...sisältöä.
  \end{document}
\end{koodilohko}
\caption{Dokumentin sivumäärän ja viimeisen sivun tulostaminen}
\label{esim:totpages}
\end{esimerkki*}

Muunlaisten kokonaislaskurien toteuttamiseen voi käyttää
\paketti{tot\-count}\-/pakettia.\avctan{totcount} Se tarjoaa komennon,
jolla rekisteröidään olemassa oleva laskuri kokonaislaskuriksi. Lisäksi
on komento, jolla tulostetaan tai palautetaan laskurin lopullinen arvo.

\begin{koodilohkosis}
  \newcounter{oma}      % Luodaan laskuri ”oma”.
  \regtotcounter{oma}   % Rekisteröidään ”oma” kokonaislaskuriksi.
  \addtocounter{oma}{1} % Kasvatetaan laskurin arvoa.
  \total{oma}           % Tulostetaan laskurin lopullinen arvo.
  \totvalue{oma}        % Palautetaan laskurin lopullinen arvo.
\end{koodilohkosis}

Edellä olevan esimerkin viimeistä komentoa \koodi{\keno tot\-value} ei
käytetä laskurin arvon latomiseen itse dokumenttiin vaan sitä käytetään
teknisissä yhteyksissä kuten toisen Latex\-/komennon argumenttina.

Myös Latexin valmiita laskureita voi rekisteröidä kokonaislaskureiksi.
Tässä oppaassa on yhteensä \total{chapter} numeroitua päälukua, ja
mainittu lukumäärä saatiin laskettua seuraavasti:

\begin{koodilohkosis}
  \regtotcounter{chapter} % Sijoitetaan dokumentin esittelyosaan.
  \total{chapter}         % Tulostaa päälukujen (chapter) lukumäärän.
\end{koodilohkosis}

\chapter{Dokumentin asetukset}
\section{Dokumenttiluokat}
\label{luku:dokumenttiluokat}
\section{Fontit}
\label{luku:kirjaintyypit}

Fontit ja niiden asettaminen on Latexissa melko monimutkainen
kokonaisuus, koska fonteilla on paljon ominaisuuksia ja niihin
vaikutetaan monilla eri asetuksilla ja abstraktiotasoilla. Aika monta
asiaa pitää ymmärtää, jotta voi tehokkaasti työskennellä fonttien
kanssa.

Fontti jo itsessään on moniselitteinen käsite, joka vaatii
typo\-gra\-fias\-sa usein täsmentäviä ilmauksia. Sana \emph{fontti} voi
tarkoittaa kokonaista kirjainperhettä, eli muutaman yhteensopivan
kirjainleikkauksen muodostamaa kokonaisuutta. Samaan kirjainperheeseen
kuuluu yleensä neljä eri leikkausta: tavallinen, kursiivi, lihavoitu ja
lihavoitu kursiivi. Joihinkin perheisiin kuuluu leikkauksia paljon
enemmänkin, kuten useita eri vahvuuksia. Joissakin puheissa sana
\emph{fontti} tarkoittaa vain yhtä kirjainleikkausta, ja silloin koko
perheeseen viitataan sanalla fonttiperhe.

Tässä oppaassa käytän \emph{fontti}\-/sanaa yleisnimityksenä Latexin
kirjaimiin liittyville asetuksille. Se tarkoittaa kirjainperhettä tai
siihen kuuluvaa yksittäistä leikkausta. Silloin kun merkitystä pitää
täsmentää, käytän suomenkielisiä nimiä kirjainperhe ja kirjainleikkaus.
Sen sijaan sanan \emph{kirjasin} jätän kokonaan pois, koska se
tarkoittaa vanhassa metalliladonnassa ja mekaanisissa kirjoituskoneissa
metallisen kirjakkeen päähän muotoiltua kirjaimen kohokuviota, joka
painaa mustejäljen paperille.

Kuten Latexissa yleensäkin myös fonttien kanssa kannattaa käyttää
korkean abstraktiotason komentoja, jotka piilottavat yksityiskohdat ja
teknisen toteutuksen. Latexin fonttitoiminnot on suunniteltu juuri
siihen: matalan tason fontti\-asetukset määritellään mieluiten vain
kerran dokumentin esittely\-osassa, ja dokumentin teksti\-osassa
käytetään pelkästään korkean tason komentoja.

\subsection{Fontin valinta}

Latexin fonttien perus\-toiminnot rakentuvat kolmen erityyppisen
kirjainperheen varaan: antiikva eli pääteviivallinen
(\textenglish{serif, roman}), groteski eli pääteviivaton
(\textenglish{sans serif}) ja tasalevyinen kirjoituskoneen kaltainen
perhe (\textenglish{typewriter, monospace}). Kuvassa
\ref{kuva:kirjainperhetyypit} on tässä oppaassa käytetyt kolme eri
kirjainperhetyyppiä. Leipätekstissä käytetään pääteviivallista,
otsikoissa pääteviivatonta ja koodi\-esimerkeissä tasalevyistä.

\leijukuva{
  {\rmfamily\addfontfeatures{Scale=7}Amf}
  \hfill
  {\sffamily\addfontfeatures{Scale=7}Amf}
  \hfill
  {\ttfamily\addfontfeatures{Scale=7}Amf}
}{
  \caption{Vasemmalla pääteviivallinen, keskellä pääteviivaton ja
    oikealla tasalevyinen pääteviivallinen kirjainperhe}
  \label{kuva:kirjainperhetyypit}
}

Kirjoituskoneen kaltainen tasalevyinen kirjainperhe on tässä tapauksessa
tyypiltään pääteviivallinen, mutta se voisi olla muutakin. Tasalevyisyys
on sen kirjainperheen tärkein määrittävä tekijä Latexin asetusten
näkökulmasta.

Kirjainperheet otetaan käyttöön Fontspec\-/paketin komennoilla seuraavan
esimerkin mukaisesti.

\begin{koodilohkosis}
  \usepackage{fontspec}
  \setmainfont{Linux Libertine O}[Scale=1]
  \setsansfont{Linux Biolinum O}[Scale=MatchLowercase]
  \setmonofont{Linux Libertine Mono O}[Scale=MatchLowercase]
\end{koodilohkosis}

Samalla voi määritellä lukuisia kirjainperheeseen sisältyviä asetuksia
kuten ligatuureja ja optisia kokoja. Tässä esimerkissä käytetään vain
\koodi{Scale}\-/valitsinta, jolla fontin voi skaalata haluttuun kokoon.
\koodi{Scale}\-/kerroin on desimaaliluku, ja oletus\-arvo on 1.

Esimerkissä peruskirjainperheen (\koodi{\keno setmainfont}) skaalaus on
1, eli sille ei tehdä mitään, ja koko valitsimen voisi jättää pois. Sen
sijaan kahdella muulla kirjainperheellä (\koodi{\keno setsansfont, \keno
  setmonofont}) käytetään ker\-roin\-ase\-tus\-ta
\koodi{MatchLowercase}, joka skaalaa fontin siten, että gemenakirjaimet
eli pienet kirjaimet ovat yhtä korkeita kuin peruskirjainperheessä.

Jos edellä kuvatut kolme kirjainperhettä eivät riitä, on
Fontspec\-/paketissa komennot lisäperheiden ja \=/leikkausten
määrittämiseen. Uusi perhe määritellään seuraavasti:

\begin{koodilohkosis}
  \newfontfamily{\hienoperhe}{TeX Gyre Schola}[...]
\end{koodilohkosis}

Komento \koodi{\keno newfontfamily} toimii samalla tavalla kuin aiemmin
esitellyt \koodi{\keno setmainfont} ym. komennot, mutta lisäksi
ensimmäinen parametri määrittää komennon, jolla kirjainperhe otetaan
käyttöön. Edellisessä esimerkissä luodaan komento \koodi{\keno
  hieno\-perhe}, joka kytkee päälle TeX Gyre Scho\-la \=/nimisen
kirjainperheen.
          
Jos ei tarvita kokonaista perhettä vaan yksi leikkaus riittää, käytetään
komentoa \koodi{\keno newfontface}:

\begin{koodilohkosis}
  \newfontface{\hienoleikkaus}{TeX Gyre Schola Bold}[...]
\end{koodilohkosis}

Edellä määriteltävä komento \koodi{\keno hieno\-leikkaus} ottaa käyttöön
lihavoidun (bold) kirjainleikkauksen perheestä TeX Gyre Scho\-la. Tässä
vaih\-to\-eh\-dos\-sa on se etu, että tietokoneen muistiin ladataan vain
yksi leikkaus, ei koko perhettä.

\subsection{Fontin koko ja rivikorkeus}

Fonttien koot on totuttu valitsemaan typo\-grafisen pistemitan avulla.
Esimerkiksi 10--12 pistettä on tyypillinen leipätekstin oletuskoko
teks\-tin\-kä\-sit\-tely\-ohjel\-mis\-sa. Piste on mitta\-yksikkö, jonka
pituus on 1/72 tuumaa eli 0,3528 millimetriä. Kirjainleikkauksen
pistekoko mitataan kirjaimiston ylimmän ja alimman kohdan välillä, ja
siihen luetaan mukaan ylä- ja alapuolella oleva pieni tyhjä tila, jonka
fontin suunnittelija on määritellyt.

Myös Latexissa koot voi määritellä pistemittojen (lyhenne pt) avulla,
mutta halutessaan ne voi unohtaa lähes kokonaan ja käyttää niin sanottua
suhteellista tapaa koko\-asetuksiin. Käsittelen suhteellisia ja
absoluuttisia koko\-ase\-tuk\-sia luvuissa
\ref{luku:fontti_suhteellinen} ja \ref{luku:fontti_absoluuttinen}.

Matalalla tasolla fonttien kokoon vaikuttaa Latexissa eräs yllättävä
asia. Nimittäin dokumenttiluokalle (luku \ref{luku:dokumenttiluokat})
voi antaa valitsimen, jolla koko asetetaan. Vaihto\-ehtoja on Latexin
normaaleissa dokumenttiluokissa vain kolme: \koodi{10pt} (oletus),
\koodi{11pt}, ja \koodi{12pt}. Dokumenttiluokan koko\-asetus vaikuttaa
myös sivun marginaaleihin, koska Latex pyrkii pitämään rivin
merkkimäärän lukijalle sopivana: yhdelle riville ei kannata latoa ihan
mahdottomasti merkkejä, koska luettavuus heikkenee.

Fontin koon määrittäminen dokumenttiluokan valitsimella kuuluu jo vähän
menneisyyteen, mutta voi sitä edelleen käyttää, jos se riittää ja sillä
saa halutun lopputuloksen. Yleensä kyllä jättäisin dokumenttiluokan
fontti\-asetuksen oletukseksi (\koodi{10pt}) ja käyttäisin koon
asettamiseen luvussa \ref{luku:fontti_suhteellinen} tai
\ref{luku:fontti_absoluuttinen} kerrottuja tapoja. Sivun marginaalien ja
muiden mittojen määrittämiseen on ohjeita luvussa
\ref{luku:sivuasetukset}.

Fontti\-asetuksiin kuuluu fontin koon lisäksi toinenkin mitta:
rivikorkeus (\koodi{\keno baselineskip}). Se on mitta rivin
peruslinjalta seuraavan rivin peruslinjalle. Fontin koko ja rivikorkeus
määritellään saman\-aikaisesti, koska ne ovat saman \koodi{\keno
  fontsize}\-/komennon parametreja. Esimerkki:

\begin{koodilohkosis}
  \fontsize{10pt}{12pt} \selectfont
\end{koodilohkosis}

Ensimmäinen parametri on fontin kokomitta ja toinen on rivikorkeus.
Mitta\-yksiköt voivat olla mitä tahansa Latexin mittoja, ja oletuksena
käytetään pistemittaa (pt), jos yksikköä ei ole mainittu. Komento
\koodi{\keno selectfont} on mukana, koska vasta sen myötä matalan tason
fonttikomennot tulevat voimaan. Korkean tason fonttikomennot (luku
\ref{luku:fontit_korkea}) suorittavat sen automaattisesti.

Rivikorkeus on vähintään sama kuin fontin koko, mutta yleensä se
asetetaan hieman suuremmaksi, jotta rivit eivät olisi liian lähellä
toisiaan. Esimerkissä \ref{esim:rivikorkeus} on kaksi erilaista
\koodi{\keno fontsize}\-/komentoa ja ladottu lopputulos.

\begin{esimerkki}
\begin{koodilohko}
  \fontsize{10pt}{12pt}\selectfont Tässä on pienehkö leipätekstin
  fonttikoko ja sitä hieman suurempi rivikorkeus. Rivikorkeus on
  liian pieni näin pitkille riveille.

  \fontsize{16pt}{25pt}\selectfont Tässä on melko suuri fontti ja
  reilu rivikorkeus. Rivit eivät tunnu kuuluvan enää yhteen.
\end{koodilohko}
\centering%
\parbox{.9\textwidth}{%
  \linespread{1}\addfontfeatures{Scale=1}
  \fontsize{10pt}{12pt}\selectfont Tässä on pienehkö leipätekstin
  fonttikoko ja sitä hieman suurempi rivikorkeus. Rivikorkeus on liian
  pieni näin pitkille riveille.

  \fontsize{16pt}{25pt}\selectfont Tässä on melko suuri fontti ja reilu
  rivikorkeus. Rivit eivät tunnu kuuluvan enää yhteen. } \vspace{3ex}
\caption{Fontin koon ja rivikorkeuden asettaminen ja vaikutus}
\label{esim:rivikorkeus}
\end{esimerkki}

Toinen tekstirivien peruslinjojen väliseen etäisyyteen vaikuttava asetus
on \koodi{\keno baselinestretch}. Se on desimaalilukukerroin, jolla
nykyinen rivikorkeus kerrotaan. Kerroin asetetaan helpoimmin komennolla
\koodi{\keno linespread}.\footnote{Toinen tapa: \koodi{\keno
    renewcommand\{\keno baselinestretch\}\{kerroin\}}}

\begin{koodilohkosis}
  \fontsize{10pt}{12pt} \linespread{1.3} \selectfont
\end{koodilohkosis}

Edellä oleva esimerkki asettaa fontin kooksi 10 pistettä ja
rivikorkeudeksi 12 pistettä. \koodi{\keno linespread}\-/komennolla
asetetun kertoimen vuoksi rivien peruslinjojen väliseksi etäisyydeksi
tulee lopulta 1,3 kertaa 12 pistettä eli 15,6 pistettä. Ei ole väliä,
kummassa järjestyksessä \koodi{\keno fontsize}- ja \koodi{\keno
  linespread}\-/komennot annetaan. Asetukset tulevat voimaan vasta
\koodi{\keno selectfont}\-/komennon jälkeen.

Käytännössä \koodi{\keno linespread} sopii rivikorkeuden yleistason
hienosäätöön, esimerkiksi dokumentin esittely\-osassa. Sitä ei
kannattane kovin paljon muutella, koska se vaikuttaa kaikkialla. Sen
sijaan \koodi{\keno fontsize}\-/komennolla määritetään rivikorkeus
tietylle fonttikoolle ja tiettyyn tilanteeseen.

\subsection{Korkean tason komennot}
\label{luku:fontit_korkea}

Latexissa on joukko korkean tason fontti\-komentoja, jotka on
tarkoitettu käytettäväksi sen jälkeen, kun matalan tason asetukset on
kerran määritetty. Taulukossa \ref{tlk:fonttimallikomennot} on komennot
kirjainperheen ja kirjainleikkauksen valintaan.

Komennot fontin koon valintaan ovat taulukossa
\ref{tlk:fonttikokokomennot}. Taulukko kertoo myös, mitä
fontin pistekokoa mikäkin komento tarkoittaa oletuksena. Oletus riippuu
Latexin dokumenttiluokkien (luku \ref{luku:dokumenttiluokat})
fonttikokovalitsimista \koodi{10pt, 11pt} ja \koodi{12pt}.

Kaikille korkean tason fonttikomennoille on olemassa myös samanniminen
ympäristönsä. Alla olevassa esimerkissä on kaksi fontteihin vaikuttavaa
ympäristöä sisäkkäin.

\begin{koodilohkosis}
  \begin{LARGE}
    \begin{scshape}
      Tämä teksti on isoa (LARGE) kapiteelia (scshape).
      Typografisesti varmaan aika typerää...
    \end{scshape}
  \end{LARGE}
\end{koodilohkosis}

\leijutlk{
  \begin{tabular}{llll}
    \toprule
    \multicolumn{2}{l}{\ots{Komento}}
    & \multicolumn{2}{l}{\ots{Merkitys}} \\
    \midrule
    \koodi{\keno rmfamily} & \koodi{\keno textrm\{...\}}
    & {\rmfamily perhe} & pääteviivallinen, antiikva, serif, roman \\
    \koodi{\keno sffamily} & \koodi{\keno textsf\{...\}}
    & {\sffamily perhe} & pääteviivaton, groteski, sans serif \\
    \koodi{\keno ttfamily} & \koodi{\keno texttt\{...\}}
    & {\ttfamily perhe} & tasalevyinen, kirjoituskone, typewriter \\
    \midrule
    \koodi{\keno mdseries} & \koodi{\keno textmd\{...\}}
    & {\mdseries leikkaus} & lihavoimaton, tavallinen \\
    \koodi{\keno bfseries} & \koodi{\keno textbf\{...\}}
    & {\bfseries leikkaus} & lihavoitu, bold \\
    \midrule
    \koodi{\keno upshape} & \koodi{\keno textup\{...\}}
    & {\upshape leikkaus} & pystyasento, tavallinen \\
    \koodi{\keno slshape} & \koodi{\keno textsl\{...\}}
    & {\slshape leikkaus} & kallistettu, slanted, oblique \\
    \koodi{\keno itshape} & \koodi{\keno textit\{...\}}
    & {\itshape leikkaus} & kursiivi, italic \\
    \koodi{\keno scshape} & \koodi{\keno textsc\{...\}}
    & {\scshape leikkaus} & kapiteeli, small caps \\
    \bottomrule
  \end{tabular}
}{
  \caption{Komennot kirjainperheen ja kirjainleikkauksen valintaan}
  \label{tlk:fonttimallikomennot}
}

\leijutlk{
  \begin{tabular}{lr@{}lr@{}lr@{}l}
    \toprule
    \ots{Komento}
    & \multicolumn{2}{c}{\ots{10pt}}
    & \multicolumn{2}{c}{\ots{11pt}}
    & \multicolumn{2}{c}{\ots{12pt}} \\
    \midrule
    \koodi{\keno tiny} & 5 && 6 && 6 \\
    \koodi{\keno scriptsize} & 7 && 8 && 8 \\
    \koodi{\keno footnotesize} & 8 && 9 && 10 \\
    \koodi{\keno small} & 9 && 10 && 10&,95 \\
    \koodi{\keno normalsize} & 10 && 10&,95 & 12 \\
    \koodi{\keno large} & 12 && 12 && 14&,4 \\
    \koodi{\keno Large} & 14&,4 & 14&,4 & 17&,28 \\
    \koodi{\keno LARGE} & 17&,28 & 17&,28 & 20&,74 \\
    \koodi{\keno huge} & 20&,74 & 20&,74 & 24&,88 \\
    \koodi{\keno Huge} & 24&,88 & 24&,88 & 24&,88 \\
    \bottomrule
  \end{tabular}
}{
  \caption{Fonttien oletuspistekoot dokumenttiluokkien valitsimilla
    \koodi{10pt, 11pt} ja \koodi{12pt}}
  \label{tlk:fonttikokokomennot}
}

\subsection{Koot suhteellisesti}
\label{luku:fontti_suhteellinen}

Dokumentin fonttien koot on helpointa määrittää siten, että asettaa
ensin peruskirjainperheen koon ja antaa muiden fonttien määräytyä
suhteessa siihen. Esimerkki \ref{esim:fontti_suhteellinen} selventää,
kuinka se tapahtuu. Alussa otetaan käyttöön dokumenttiluokka
\koodi{article} ja annetaan sille valitsin \koodi{10pt}, joka määrittää
fonttikooksi 10 pistettä. Se on dokumenttiluokan ole\-tus\-ase\-tus,
jota ei tarvitsisi edes kirjoittaa näkyviin. Esimerkin toisella rivillä
otetaan Fontspec\-/paketti käyttöön.

Peruskirjainperheen (rivi~4) koko skaalataan 1,4\-/kertaiseksi, eli
pistekooksi tulee 1,4 kertaa 10 pistettä eli 14 pistettä.
Normaalikokoinen peruskirjainperhe on ainoa, jonka pistekoko tiedetään.
Kaikkien muiden koot täytyisi selvittää laskemalla.

Pääteviivaton kirjainperhe (rivi~5) ja tasalevyinen perhe (rivi~6)
skaalataan samankorkuiseksi kuin perusperhe. Vertailukohtana ovat pienet
kirjaimet (\koodi{MatchLowercase}). Näiden kahden kirjainperheen
pistekokoa ei tiedetä. Se ei välttämättä ole sama kuin perusfontissa,
koska fonttien pistekoko mitataan ylimmän ja alimman kohdan välillä ja
koska fonttien mittasuhteet ovat erilaisia.

\begin{esimerkki}
\begin{koodilohko}
  \documentclass[10pt]{article} % 10pt on oletus
  \usepackage{fontspec}

  \setmainfont{Linux Libertine O}[Scale=1.4]
  \setsansfont{Linux Biolinum O}[Scale=MatchLowercase]
  \setmonofont{Linux Libertine Mono O}[Scale=MatchLowercase]
  \linespread{1.45}
\end{koodilohko}    
\caption{Fonttikokojen määrittäminen suhteessa peruskirjainperheeseen}
\label{esim:fontti_suhteellinen}
\end{esimerkki}

Viimeisellä rivillä oleva \koodi{\keno linespread}\-/komento on tärkeä.
Se asettaa rivikorkeuden kertoimeksi 1,45. Kertoimen täytyy olla
vähintään yhtä suuri kuin peruskirjainperheen skaalauskerroin (1,4),
jotta rivivälit ovat riittävän suuret. Näiden asetusten jälkeen
dokumentissa käytetään korkeamman tason komentoja fonttien valintaan,
esimerkiksi fonttikoon valintakomentoja \koodi{\keno small, \keno
  normalsize, \keno large} (taulukko \ref{tlk:fonttikokokomennot}).

Edellä kuvatussa suhteellisessa kirjainperheiden koon määrittelyssä on
sellainen ongelma tai kummallisuus, että Latex koko ajan luulee, että
peruskirjainperhe on normaalikokoisena 10 pistettä. Latexin matalan
tason fonttikomennot eivät tiedä kirjainperheen skaalauskertoimesta, ja
siksi esimerkiksi komentojen

\begin{koodilohkosis}
  \fontsize{10pt}{12pt} \selectfont
\end{koodilohkosis}

tuloksena ei todellisuudessa ole 10 pisteen fontti, vaan mukaan
lasketaan myös kirjainperheen skaalauskerroin. Tämän vuoksi \koodi{\keno
  fontsize}\-/komennon käyttö menee aika oudoksi. Parametrina annetut
mitat eivät pidä paikkaansa.

Jos korkean tason font\-ti\-koko\-komen\-to\-jen (taulukko
\ref{tlk:fonttikokokomennot}) lisäksi tarvitaan jotakin muuta kokoa,
voisi ehkä \koodi{\keno fontsize}\-/komennon sijasta käyttää
Fontspec\-/paketin tarjoamaa komentoa ja tilanteeseen sopivaa
skaalauskerrointa esimerkiksi seuraavalla tavalla:

\begin{koodilohkosis}
  \addfontfeatures{Scale=3.2} Poikkeuksellisen isoa tekstiä
\end{koodilohkosis}

Jos edellä mainitut kummallisuudet eivät häiritse eikä ole tarvetta
määritellä fontteja tarkasti tietyn pistekoon mukaiseksi, on
suhteellinen määrittelytapa todella helppo. Kaikki dokumentin fontit
määräytyvät perusfontin skaalauskertoimen kautta. Tämä tapa sopii hyvin
varsinkin dokumentin sisällön kirjoittamisvaiheeseen, jossa halutaan
vain nopeasti asettaa dokumentti suurin piirtein järkevän näköiseksi.

\subsection{Koot absoluuttisesti}
\label{luku:fontti_absoluuttinen}

Absoluuttinen fonttien koonmääritystapa tarkoittaa sitä, että koot
asetetaan tietyn kokoiseksi käyttämällä esimerkiksi pistemittoja ja että
kirjaimet myös päätyvät lopulliseen dokumenttiin juuri sen kokoisena.
Tämä tapa on myös teknisesti eheä, eli Latexin eri osat ovat samaa
mieltä siitä, minkäkokoisesta fontista on kyse. Näin ei ollut
suhteellisen tavan kanssa (luku \ref{luku:fontti_suhteellinen}).

Joskus yrityksen, oppilaitoksen tai muun tahon julkaisu\-ohjeissa
määritellään tarkasti, mitä fontteja käytetään ja mikä on leipätekstin
ja otsikoiden fonttikoko. Silloin tarvitaan tässä luvussa kuvattua tapaa
fonttien asettamiseen.

\begin{esimerkki}
\begin{koodilohko}
  \documentclass{article}
  \usepackage{fontspec}

  % Leipätekstiin samankokoiset fontit
  \setmainfont{Linux Libertine O}
  \setsansfont{Linux Biolinum O}[Scale=MatchLowercase]
  \setmonofont{Linux Libertine Mono O}[Scale=MatchLowercase]
  
  % Muualle sans ja mono ilman skaalausta
  \newfontfamily{\sffamilyabs}{Linux Biolinum O}
  \newfontfamily{\ttfamilyabs}{Linux Libertine Mono O}

  \linespread{1} % ei välttämättä tarvita

  % Kaikki tarvittavat fonttikoot ja komennot
  \renewcommand{\footnotesize}{\fontsize{10pt}{12pt}\selectfont}
  \renewcommand{\small}       {\fontsize{12pt}{14pt}\selectfont}
  \renewcommand{\normalsize}  {\fontsize{14pt}{17pt}\selectfont}
  \renewcommand{\large}       {\fontsize{17pt}{19pt}\selectfont}
  \renewcommand{\Large}       {\fontsize{20pt}{22pt}\selectfont}
  \normalsize % jotta tulee heti voimaan eikä vasta tekstiosassa
\end{koodilohko}    
\caption{Fonttikokojen määrittäminen pistekoon avulla}
\label{esim:fontti_absoluuttinen}
\end{esimerkki}

Esimerkistä \ref{esim:fontti_absoluuttinen} selviää perus\-ajatus.
Peruskirjainperhe (rivi~5) otetaan käyttöön ilman skaalausta (eli
\koodi{Scale=1}), minkä vuoksi koon voi jatkossa asettaa täsmälleen
kohdalleen \koodi{\keno fontsize}\-/komennolla. Samaa ei tehdä
pääteviivattoman eikä tasalevyisen fontin kanssa (rivit 6--7), vaan
käytetään skaalausta \koodi{MatchLowercase}, jotta tekstikappaleessa
kaikki kirjainperheet näyttävät samankokoisilta. Tässä menetetään
mahdollisuus määrittää näiden kirjainperheiden koko täsmällisesti
pistemitan avulla. Jos siihen on tarvetta esimerkiksi otsikoissa,
voidaan käyttää rivien 10--11 komentoja. Niillä luodaan uudet
kirjainperheet, jotka ovat käytännössä samoja mutta ilman skaalausta.

Uusien kirjainperheiden komentojen nimiksi on valittu \koodi{\keno
  sffamilyabs} ja \koodi{\keno ttfamilyabs} (vrt. \koodi{\keno sffamily}
ja \koodi{\keno ttfamily}), ja näillä komennoilla kirjainperheet
kytketään päälle. Jos esimerkiksi jonkin julkaisun vaatimuksiin kuuluu,
että otsikossa täytyy olla 16 pisteen lihavoitu Linux Biolinum~O
\=/kirjainleikkaus, voi esimerkissä \ref{esim:fontti_absoluuttinen}
olevien asetusten pohjalta antaa otsikolle seuraavat komennot:

\begin{koodilohkosis}
  \sffamilyabs\fontsize{16pt}{18pt}\bfseries
\end{koodilohkosis}

Esimerkin \ref{esim:fontti_absoluuttinen} riveillä 16--20 määritellään
uudelleen Latexin korkean tason komennot, joilla fonttikoot asetetaan.
Vähintäänkin täytyy määritellä komento \koodi{\keno normalsize} mutta
sen lisäksi kaikki ne, joita omassa dokumentissa tarvitaan. Tässä
esimerkissä normaali koko asetetaan 14 pisteen kokoiseksi ja muut koot
on ajateltu suhteessa siihen.

Jokaiselle fonttikoolle määritetään riveillä 16--20 myös oma
rivikorkeus, ja se on tarkoitus asettaa sopivaksi juuri kyseiselle
koolle. Rivikorkeuteen vaikuttaa myös kerroin \koodi{\keno
  baselinestretch}, joka asetetaan komennolla \koodi{\keno linespread}.
Sitä ei välttämättä tarvitse käyttää, koska kirjainperheitä ei ole
skaalattu ja koska rivikorkeus asetetaan aina \koodi{\keno
  fontsize}\-/komennolla. \koodi{\keno linespread} on kuitenkin kätevä
komento rivikorkeuden säätämiseen yleisesti kaikkialla.

Fonttikokojen määrittelyn lopuksi rivillä 21 suoritetaan komento
\koodi{\keno normalsize}, jotta se tulee heti voimaan. Dokumentin
esittely\-osassa voidaan käyttää fonttikokoon viittaavia mittoja
\koodi{em} ja \koodi{ex}, ja ne viittaavat nyt tähän kokoon.
\koodi{\keno normalsize}\-/komento suoritetaan kyllä myöhemmin
automaattisesti dokumentin teksti\-osan eli
\koodi{document}\-/ympäristön alussa.

Edellä kuvatun absoluuttisen koonmääritystavan etuna on se, että
kirjoittaja hallitsee fonttien kokoa ja rivikorkeuksia tarkasti ja että
julkaisuun saadaan juuri ne mitat, jotka halutaan tai vaaditaan. Tapa on
myös teknisesti eheä eli toimii Latexin sisäisen logiikan näkökulmasta
oikein. Haittana voi pitää sitä, että kaikki koot täytyy määritellä
erikseen (esimerkki \ref{esim:fontti_absoluuttinen}, rivit 16--20).

\section{Kieli}
\label{luku:kieliasetukset}

\subsection{Polyglossia}
\subsection{Babel}

\section{Sivu}
\label{luku:sivuasetukset}
\subsection{Marginaalit ja mitat}
\subsection{Ylä- ja alatunnisteet}


% Tekijä:   Teemu Likonen <tlikonen@iki.fi>
% Lisenssi: Creative Commons Nimeä-JaaSamoin 4.0 Kansainvälinen (CC BY-SA 4.0)
%           https://creativecommons.org/licenses/by-sa/4.0/legalcode.fi

\chapter{Dokumentin rakenne}

...

\section{Tekstikappaleet}
\label{luku:kappale}

Tekstikappale on tekstin osa, jonka pitäisi käsitellä suunnilleen yhtä
asiakokonaisuutta. Se voi olla esimerkiksi yksi aihe, näkökulma,
ajankohta tai henkilö. Tekstin seuraava kappale käsittelee jotakin
toista aihetta, näkökulmaa tms. Kappaleen vaihtuminen on lukijalle
merkki siitä, että tekstin sisällössäkin jokin muuttuu.

Latexin lähdetiedostoissa kappaleen vaihtuminen ilmaistaan
kirjoittamalla kappaleiden väliin vähintään yksi tyhjä rivi. Tätä
merkintäkielen piirrettä käsitellään myös luvussa
\ref{luku:kappaleen_vaihtuminen}. Kappale vaihtuu myös komennolla
\koodi{\keno par}, joka sopii käytettäväksi esimerkiksi komentojen
määrittelyssä (luku \ref{luku:komennot}), kun halutaan varmistaa
kappaleen vaihtuminen tietyssä kohdassa.

Ladotuissa teksteissä kuten kirjoissa ja lehdissä kappaleen vaihtuminen
ilmaistaan melkein aina siten, että uuden kappaleen ensimmäinen rivi
sisennetään hieman. Niin on tässäkin oppaassa. Toisinaan tekstikappaleet
erotetaan pystysuuntaisella välillä, ja silloin kappaleiden ensimmäistä
riviä ei sisennetä. Kappaleiden välejä, sisennyksiä, rivien tasaamista
ja muita asetuksia käsitellään seuraavissa alaluvuissa.

Monissa kappaleisiin liittyvissä asetuksissa tarvitaan Texin mittoja ja
mittayksiköitä. Mittoihin liittyvää tekniikkaa käsitellään tarkemmin
luvussa \ref{luku:mitat}, joka on syytä tuntea ennen tämän alaluvun
lukemista.

\subsection{Tasaaminen ja palstan muoto}
\label{luku:kappaleen_tasaus}

Tekstikappaleet tasataan oletuksena palstan molempiin reunoihin, ja tätä
palstan muotoa kutsutaan tasapalstaksi. Se tarkoittaa samalla sitä, että
rivillä olevia sanavälejä venytetään sopivasti, jotta jokainen rivi
näyttäisi yhtä pitkältä ja palstan molemmat reunat tasaiselta.

Käytännössä sanavälien venymiselle on määritelty yläraja, jonka yli
niitä ei venytetä. Ylärajan tarkoituksena on estää liian suuret ja rumat
sanavälit. Rajoitus on sinänsä järkevä, mutta se voi myös johtaa siihen,
että Tex ei saa tasattua kaikkia tekstikappaleita palstan oikeasta
reunasta: jotkin rivit yltävät palstan reunan yli; jotkin rivit jäävät
vajaaksi. Näin käy usein varsinkin suomen kielessä, jonka sanat ovat
usein pitkiä ja riveillä on vähänlaisesti sanavälejä. Suomen kielessä
sanavälien venymisen yläraja on usein tarpeellista asettaa oletusarvoa
suuremmaksi. Se tehdään mitan \koodi{\keno emergencystretch} avulla,
esimerkiksi seuraavasti:

\begin{koodilohkosis}
  \setlength{\emergencystretch}{1em}
\end{koodilohkosis}

Kaikenlaiset kappaleiden latomiseen liittyvät tekniset rajoitukset voi
poistaa tai asettaa hyvin suuriksi komennolla \koodi{\keno sloppy}.
\koodimargin{\keno sloppy} Komento asettaa muun muassa sanavälien
venymisen ylärajaksi 3\,em. Tämän komennon käyttö ei ole kovin
suositeltavaa, koska sillä on muitakin seurauksia ja se voi vaikuttaa
myös sellaisiin kappaleisiin, jotka muuten saataisiin ladottua nätisti.
Parempi on asettaa vain mitta \koodi{\keno emergencystretch} riittävän
suureksi. \koodimargin{\keno fussy} Sanavälien venymiseen ja kappaleiden
tasaiseen latomiseen liittyvät asetukset voi palauttaa oletusarvoihin
komennolla \koodi{\keno fussy}.

Hyvin tavallista on tasata teksti pelkästään vasempaan reunaan, jolloin
rivien pituudet vaihtelevat ja oikealla on niin sanottu liehureuna.
Oikea liehureuna sopii pitkiin teksteihin yhtä hyvin kuin tasapalstakin,
mutta se on parempi valinta erityisesti silloin, kun palsta on kapea.
Nimittäin kapealla palstalla venyviä sanavälejä on käytettävissä hyvin
vähän ja oikean reunan tasaaminen vaatii sanavälien venyttämistä joskus
kohtuuttoman paljon. Tekstiin jää rumia aukkoja.

\providecommand{\rivi}{}
\renewcommand{\rivi}[3]{\koodi{\keno #1} & \koodi{#2} & #3 \\}

\leijutlk{
  \begin{tabular}{lll}
    \toprule
    \ots{Komento} & \ots{Ympäristö} & \ots{Merkitys} \\
    \midrule
    \rivi{raggedright}{flushleft}{vasen tasaus, oikea liehu}
    \rivi{raggedleft}{flushright}{oikea tasaus, vasen liehu}
    \rivi{centering}{center}{keskitetty}
    \midrule
    \rivi{RaggedRight}{FlushLeft}
    {vasen tasaus, oikea liehu, tavutus (\paketti{ragged2e})}
    \rivi{RaggedLeft}{FlushRight}
    {oikea tasaus, vasen liehu, tavutus (\paketti{ragged2e})}
    \rivi{Centering}{Center}{keskitetty, tavutus (\paketti{ragged2e})}
    \rivi{justifying}{justify}{tasapalsta, tavutus (\paketti{ragged2e})}
    \bottomrule
  \end{tabular}
}{
  \caption{Tekstikappaleen tasaamiseen ja palstan muotoon vaikuttavat
    komennot ja ympäristöt. Osa sisältyy \paketti{ragged2e}\-/pakettiin}
  \label{tlk:kappaleen_tasauskomennot}
}

Kappaleiden tasaamiseen ja palstan muotoon vaikuttavia komentoja ja
ympäristöjä on koottu taulukkoon \ref{tlk:kappaleen_tasauskomennot}.
Taulukossa on mainittu ensin Latexin omat komennot ja sitten
\paketti{ragged2e}\-/paketin\avctan{ragged2e} vastaavat. Latexin omat
komennot estävät sanojen tavuttamisen, kun taas \paketti{ragged2e}\-/
paketin komennot sallivat tavutuksen normaalisti.

\subsection{Pystysuuntaiset välit}

Kappaleiden väliin ladottava pystysuuntainen tyhjä tila asetetaan mitan
\koodi{\keno par\-skip} \koodimargin{\keno par\-skip} avulla. Se on
oletuksena nolla, mutta pientä venymistä kuitenkin sallitaan, eli
joissakin tilanteissa kappaleiden väliin voidaan latoa pieni tyhjä tila.
Jos tyhjää tilaa ei haluta missään tilanteessa, asetetaan mitta vain
nollaksi:

\begin{koodilohkosis}
  \setlength{\parskip}{0ex}
\end{koodilohkosis}

Seuraava esimerkkikomento asettaa kappaleväliksi 1,3\,ex. Lisäksi se
sallii kappalevälin venyä 0,2\,ex:n verran tai kutistua 0,1\,ex:n
verran.

\begin{koodilohkosis}
  \setlength{\parskip}{1.3ex plus .2ex minus .1ex}
\end{koodilohkosis}

Silloin kun kappaleet ladotaan erilleen toisistaan, on yleensä hyvä
sallia kappalevälin venyä tai kutistua hieman, koska venyvät
pystysuuntaiset välit antavat Texille paremmat mahdollisuudet latoa
hyvännäköisiä sivuja. Venyvien välien avulla esimerkiksi sivun
tekstialueen ylä- ja alareunat saadaan aina samalle kohdalle. Toisaalta
myös liian suuret ja toisistaan liiaksi poikkeavat kappalevälit voivat
olla rumannäköisiä.

Tavallista kappaleväliä suurempien pystysuuntaisten välien tekemiseen on
olemassa kolme valmista komentoa: \koodimargin{\keno bigskip}
\koodimargin{\keno medskip} \koodimargin{\keno smallskip} suuremmasta
pienempään ne ovat \koodi{\keno bigskip}, \koodi{\keno medskip} ja
\koodi{\keno smallskip}. Ne sopivat käytettäväksi yksittäisiin
tilainteisiin, joissa normaali kappaleväli on liian vähän. Myös
komentojen tai ympäristöjen määrittelyssä niistä voi olla apua (luvut
\ref{luku:komennot} ja \ref{luku:ymparistot}). Jos sivunvaihto osuu
näiden komentojen kohdalle, mitään väliä ei ladota sivun loppuun eikä
seuraavan alkuun.

Edellä mainittujen komentojen latoman välin suuruuteen voi vaikuttaa
mittojen \koodi{\keno big\-skip\-amount}, \koodi{\keno
  med\-skip\-amount} ja \koodi{\keno small\-skip\-amount} avulla.
Seuraavassa on esimerkkikomennot mittojen määrittelemiseen ja samalla
niiden oletusarvot:

\begin{koodilohkosis}
  \setlength{\bigskipamount} {12pt plus 4pt minus 4pt}
  \setlength{\medskipamount}  {6pt plus 2pt minus 2pt}
  \setlength{\smallskipamount}{3pt plus 1pt minus 1pt}
\end{koodilohkosis}

Komentojen \koodi{\keno bigskip}, \koodi{\keno medskip} ja \koodi{\keno
  smallskip} sijasta voi käyttää myös komentoja \koodi{\keno bigbreak},
\koodi{\keno medbreak} tai \koodi{\keno smallbreak}. \koodimargin{\keno
  bigbreak} \koodimargin{\keno medbreak} \koodimargin{\keno smallbreak}
Nämä toimivat lähes samalla tavalla, mutta niihin sisältyy myös
sivunvaihtovihje. Toisin sanoen ne vaikuttavat ladonta\-/algoritmiin
siten, että komennon kohdalla todennäköisyys sivun vaihtumiselle kasvaa
suhteessa muihin kohtiin. Sivu voi edelleen vaihtua muustakin kohdasta,
jos algoritmi löytää omasta mielestään vielä paremman paikan.

Pystysuuntaisten välien yleiskomento on \koodi{\keno vspace},
\koodimargin{\keno vspace} \koodimargin{\keno vspace*} jolle annetaan
argumentiksi välin suuruus ja mahdolliset venymisen rajat. Tämäkin
komento jättää välin latomatta, jos se sattuu sivunvaihdon kohdalle. Sen
sijaan tähdellinen versio \koodi{\keno vspace*} latoo välin joka
tapauksessa, vaikka se olisi sivun lopussa tai alussa.

\begin{koodilohkosis}
  Tekstikappale.
  \vspace{5ex plus 1ex minus .5ex}

  Toinen tekstikappale.
\end{koodilohkosis}

Komento \koodi{\keno addvspace} \koodimargin{\keno addvspace} toimii
lähes samoin kuin \koodi{\keno vspace}, mutta se huomioi mahdolliset
peräkkäiset \koodi{\keno addvspace}\-/komennot ja varmistaa, että vain
suurin väli toteutuu. Jos siis useita \koodi{\keno
  addvspace}\-/komentoja sattuu peräkkäin, niiden määrittämiä välejä ei
ladota peräkkäin vaan ainoastaan suurin niistä ladotaan. Seuraava
esimerkki latoo kappaleiden väliin 2\,ex:n suuruisen pystysuuntaisen
välin:

\begin{koodilohkosis}
  Tekstikappale.

  \addvspace{1ex} \addvspace{2ex} \addvspace{.5ex}
  Toinen tekstikappale.
\end{koodilohkosis}

Jos edellisessä esimerkissä olisi käytetty \koodi{\keno
  vspace}\-/komentoa, pystysuuntaisen välin suuruus olisi 3,5\,ex, joka
on välien yhteenlaskettu suuruus.

\koodi{\keno addvspace}\-/komento soveltuu hyvin komentojen ja
ympäristöjen määrittelyyn (luvut \ref{luku:komennot} ja
\ref{luku:ymparistot}). Esimerkiksi itse määritellyn ympäristön alussa
ja lopussa voi \koodi{\keno addvspace}\-/komennolla varmistaa
tietynsuuruisen välin, mutta jos sama tai muu vastaava ympäristö on
dokumentissa kahdesti peräkkäin, huomioidaan pystysuuntainen väli vain
kerran eli suurimman välin mukaan. Jotkin Latexin valmiit ympäristöt
tekevät juuri näin eli käyttävät \koodi{\keno addvspace}\-/komentoa
välien asettamiseen.

Sivun alueella äärettömästi venyvän pystysuuntaisen välin saa komennolla
\koodi{\keno vfill}. \koodimargin{\keno vfill} Mitan luonnollinen arvo
on nolla, mutta se voi venyä niin, että se täyttää kaiken tyhjän tilan
sivulla. \koodi{\keno vfill}\-/komento tarkoittaa käytännössä samaa kuin
\koodi{\keno vspace\{0mm plus 1fill\}} \=/komento. Texin venyviä mittoja
ja välejä käsitellään tarkemmin luvussa \ref{luku:venyvat_mitat}.

\subsection{Ensimmäisen rivin sisennys}

Kirjojen ja lehtien typografiassa kappaleen ensimmäinen rivi on tapana
sisentää merkiksi siitä, että alkaa uusi kappale. Sisennys on lukijalle
merkki siitä, että tekstin sisällössä siirrytään seuraavaan asiaan.
Oletusasetuksilla Latex latoo sisennyksen automaattisesti kappaleen
alkuun mutta ei kuitenkaan otsikoiden jälkeen.

Ensimmäisen rivin sisennyksen suuruus asetetaan mitan \koodi{\keno
  parindent} avulla, esimerkiksi alla olevan esimerkin mukaisesti.
Sopiva mittayksikkö tähän tarkoitukseen on \emph{em}, koska se viittaa
suoraan nykyisen fontin kokoon.

\begin{koodilohkosis}
  \setlength{\parindent}{1em}
\end{koodilohkosis}

Edellä mainittu mitta pitäisi asettaa nollaan silloin, kun kappaleiden
välissä on tyhjää tilaa. Pystysuuntainen välihän jo sinänsä ilmaisee,
että kappale vaihtuu.

\begin{koodilohkosis}
  \setlength{\parskip}{1.3ex plus .2ex minus .1ex}
  \setlength{\parindent}{0em}  % Ei sisennystä.
\end{koodilohkosis}

\koodi{\keno parindent}\-/ mitan levyisen välin voi tehdä mihin tahansa
komennolla \koodi{\keno indent}. \koodimargin{\keno indent} Tätä
komentoa ei tavallisesti tarvita, koska kappaleet alkavat
automaattisesti sen suuruisella sisennyksellä. Tarpeellisempi komento on
sen vastakohta \koodi{\keno no\-in\-dent}, \koodimargin{\keno noindent}
joka voidaan kirjoittaa kappaleen alkuun estämään kyseisen kappaleen
ensimmäisen rivin sisentäminen.

\begin{koodilohkosis}
  \noindent
  Tämän tekstikappaleen ensimmäistä riviä ei sisennetä.
\end{koodilohkosis}

Suomenkielisissä julkaisuissa on tavallista, että leipätekstin
kappaleessa ei ole sisennystä, jos sitä ennen on pystysuuntainen väli.
Tällainen tilanne on aina otsikoiden jälkeen mutta myös kokonaan
sisennetyn tekstikappaleen jälkeen tai kuvan, taulukon, luetelman tai
muun vastaavan tekstin\-osan jälkeen. Esimerkiksi tässä oppaassa
Latex\-/ koodi\-esi\-merk\-kien jälkeen ei tekstikappaleissa ole
sisennystä. Tähän käytäntöön on joskus poikkeuksia suomenkielisessäkin
typografiassa, mutta eri kielten välillä käytäntö voi vaihdella
enemmänkin.

Latex estää sisennyksen automaattisesti otsikoiden jälkeen mutta latoo
sisennyksen kuitenkin kaikkien muiden elementtien ja pystysuuntaisen
välin jälkeen. Jos sisennys halutaan estää, pitäisi kyseiset kappaleet
aloittaa aina \koodi{\keno noindent}\-/komennolla. Toinen vaihtoehto on
käyttää \paketti{no\-indent\-after}\-/ pakettia\avctan{noindentafter} ja
määritellä sen tarjoaman komennon avulla, minkä ympäristöjen jälkeen ei
haluta sisennystä. Seuraava esimerkki poistaa sisennyksen aina
\koodi{list}- ja \koodi{tabular}\-/ ympäristöjen jälkeen. Lisätietoja
mainituista ympäristöistä on luvuissa \ref{luku:luetelmat} ja
\ref{luku:taulukot}.

\begin{koodilohkosis}
  \NoIndentAfterEnv{list}
  \NoIndentAfterEnv{tabular}
\end{koodilohkosis}

\subsection{Riippuva sisennys}

Riippuva sisennys tarkoittaa tekstikappaleen muotoa, jossa sisennetään
kappaleen muita rivejä mutta ei ensimmäistä. Riippuvaa sisennystä
käytetään esimerkiksi kirjallisuus\-/{} ja lähdeluetteloissa, joissa on
tarpeellista saada henkilön nimi tai muu lähdemerkinnän hakusana
erottumaan selvästi vasemmassa reunassa. Tämän oppaan lopussa sivulla
\pageref{luku:kirjallisuutta} on esimerkki lähdemerkinnöistä.

Myös virallisten asiakirjojen muotoilussa käytetään riippuvaa
sisennystä. Niissä kappaleen ensimmäinen rivi voi sisältää otsikon, joka
on tasattu vasempaan marginaaliin. Otsikon perässä on sarkainhyppy
tekstikappaleen sisennyksen tasalle, ja kappaleen muut rivit on
sisennettynä samalla tasolle.

\begin{esimerkki*}
\begin{koodilohko}
  \hangpara{2cm}{1}Tässä tekstikappaleessa on riippuva sisennys. Kappale
  alkaa yhdellä sisentämättömällä rivillä, ja kappaleen seuraavat rivit
  on sisennetty 2\,cm. Ei ole kovin vaikeaa.
\end{koodilohko}
\begin{tulos}
  \hangpara{2cm}{1}Tässä tekstikappaleessa on riippuva sisennys. Kappale
  alkaa yhdellä sisentämättömällä rivillä, ja kappaleen seuraavat rivit
  on sisennetty 2\,cm. Ei ole kovin vaikeaa.
\end{tulos}
\caption{Riippuva sisennys \paketti{hanging}\-/paketin ja sen
  \koodi{\keno hang\-para}\-/komennon avulla}
\label{esim:riippuva_sis_hangpara}
\end{esimerkki*}

Helpoin tapa riippuvien sisennysten toteuttamiseen lienee
\paketti{hang\-ing}\-/ paketin\avctan{hanging} käyttö. Paketti tuo uuden
komennon \koodi{\keno hang\-para}, \koodimargin{\keno hangpara} jonka
käyttöä esimerkki \ref{esim:riippuva_sis_hangpara} selventää. Komennon
ensimmäinen argumentti on sisennyksen mitta ja toinen argumentti
(positiivinen kokonaisluku) määrittää, kuinka monta riviä kappaleen
alusta jätetään sisentämättä. Jos taas toinen argumentti on negatiivinen
luku, luvun it\-seis\-arvo määrittää, kuinka monta riviä kappaleen
alusta sisennetään.

Komennon \koodi{\keno hang\-para} vaihtoehtona on ympäristö
\koodi{hang\-paras}, \koodimargin{hang\-paras} jonka sisällä kaikki
kappaleet sisennetään riippuvalla tyylillä samojen asetusten mukaisesti.
Ympäristön aloituskomennolle annetaan samat argumentit kuin \koodi{\keno
  hang\-para}\-/ komennollekin.

\begin{koodilohkosis}
  \begin{hangparas}{2cm}{1}
    ...
  \end{hangparas}
\end{koodilohkosis}

\begin{esimerkki*}
\begin{koodilohko}
  \hangpara{2cm}{1}\makebox[2cm][l]{Otsikko}Tässä tekstikappaleessa on
  riippuva sisennys. Kappale alkaa yhdellä sisentämättömällä rivillä,
  joka sisältää näkymättömässä 2\,cm leveässä laatikossa olevan otsikon.
  Kappaleen muut rivit on sisennetty 2\,cm.
\end{koodilohko}
\begin{tulos}
  \hangpara{2cm}{1}\makebox[2cm][l]{Otsikko}Tässä tekstikappaleessa on
  riippuva sisennys. Kappale alkaa yhdellä sisentämättömällä rivillä,
  joka sisältää näkymättömässä 2\,cm leveässä laatikossa olevan otsikon.
  Kappaleen muut rivit on sisennetty 2\,cm.
\end{tulos}
\caption{Asiakirjan tyylisten tekstikappaleiden toteutus}
\label{esim:riippuva_sis_asiakirja}
\end{esimerkki*}

Asiakirjan tyylisen otsikon saa toteutettua \koodi{\keno make\-box}\-/
komennon avulla esimerkin \ref{esim:riippuva_sis_asiakirja} tavoin.
Komento latoo näkymättömän laatikon, jonka leveys määritellään
sisennyksen levyiseksi ja jonka sisään kirjoitetaan otsikko. Jos
asiakirjatyylisiä tekstikappaleita tarvitaan useita, kannattaa
määritellä sarkainleveyttä ja sisennystä varten oma mitta ja
tekstikappaleen kirjoittamista varten oma komento. Seuraavassa on siitä
esimerkki:

\begin{koodilohkosis}
  \newlength{\sarkain}
  \setlength{\sarkain}{2.3cm}
  \newcommand{\kappale}[1][]{\par\hangpara{2\sarkain}{1}%
    \makebox[2\sarkain][l]{\ignorespaces #1}\ignorespaces}
\end{koodilohkosis}

Tämän jälkeen voi komennolla \koodi{\keno kappale} aloittaa asiakirjan
sisennetyn tekstikappaleen. Komennolle voi antaa hakasulkeissa
valinnaisen argumentin, joka on kappaleen otsikko. \koodi{\keno
  make\-box}\-/ komentoa ja muita laatikoita käsitellään tarkemmin
luvussa \ref{luku:laatikot}.

\begin{esimerkki*}
\begin{koodilohko}
  \begin{list}{}{
      \setlength{\leftmargin}{2cm}
      \setlength{\itemindent}{-2cm}
    }
  \item Tässä tekstikappaleessa on riippuva sisennys. Kappale alkaa
    yhdellä sisentämättömällä rivillä, ja kappaleen muut rivit on
    sisennetty 2\,cm.
  \end{list}
\end{koodilohko}
\caption{Riippuvan sisennyksen toteuttaminen \koodi{list}\-/ympäristön
  avulla}
\label{esim:riippuva_sis_list}
\end{esimerkki*}

Riippuvan sisennyksen voi toteuttaa myös \koodi{list}\-/ympäristön
avulla. Se on tarkoitettu luetelmien tekemiseen, mutta sopivilla
asetuksilla yksi ''luetelman'' kohta on riippuvasti sisennetty kappale.
Tarkemmin \koodi{list}\-/ympäristöä käsitellään luetelmien yhteydessä
luvussa \ref{luku:luetelmat}, mutta oheisessa esimerkissä
\ref{esim:riippuva_sis_list} on sopivat asetukset riippuvan sisennyksen
toteuttamiseen. Kappale alkaa \koodi{\keno item}\-/komennolla, koska
kyseessä on ikään kuin luetelman kohta.

\subsection{Vasen ja oikea sisennys sekä lohkolainaukset}

Dokumentteihin tarvitaan välillä kokonaisia sisennettyjä
tekstikappaleita, koska leipätekstin ohessa halutaan näyttää
muuntyyppistä sisältöä. Kyse voi olla teksti- tai
kuva\-esi\-mer\-keistä, esimerkiksi muualta lainatusta tekstistä. Tässä
oppaassa käytetään paljon sisennettyjä tekstikappaleita Latex\-/koodien
esimerkkeihin.

Kokonaan sisennettyjä tekstikappaleita kutsutaan lohkolainauksiksi,
koska ne ovat lainauksia, jotka käsittävät kokonaisen tekstilohkon.
Lainausmerkkejä ei tarvitse käyttää, koska lainaus ilmaistaan
typografisin keinoin. Sisennyksen lisäksi varsin yleistä on käyttää
hieman pienempää kirjainleikkausta ja riviväliä kuin leipätekstissä.
Joskus vasemman reunan sisennyksen lisäksi sisennetään myös oikeasta
reunasta.

Latexissa on tavallisille lohkolainauksille kolme erilaista ympäristöä:
\koodi{quotation}, \koodi{quote} ja \koodi{verse}. Kaksi ensin mainittua
on tarkoitettu normaalilla tavalla juoksevalle tekstille, kun taas
kolmas on tarkoitettu runon säkeiden ja säkeistöjen latomiseen.

\koodi{quotation}\-/ ympäristö \koodimargin{quotation} sisentää
tekstikappaleiden ensimmäisen rivin 1,5\,em:n verran, eikä kappaleiden
välissä ole pystysuuntaista tilaa. \koodi{quote}\-/ ympäristö
\koodimargin{quote} ei sisennä kappaleiden ensimmäistä riviä, ja se
puolestaan erottaa kappaleet toisistaan pystysuuntaisen tilan avulla.
\koodi{verse}\-/ympäristöä \koodimargin{verse} käytetään siten, että
lähdedokumentissa runon säkeet lopetetaan rivinvaihtokomentoon
(\koodi{\keno\keno}), lukuun ottamatta säkeistön viimeistä säettä.
Säkeistöt erotetaan toisistaan tyhjällä rivillä, kuten Latexissa
tekstikappaleet muutenkin. Lopputuloksena on useimpiin runoihin sopiva
ladontatapa, jossa säkeistöjen vasen reuna on samalla tasolla, oikealla
on liehureuna ja säkeistöjen välissä on pystysuuntaista tilaa.

Jos Latexin valmiit lohko\-lai\-naus\-ympä\-ris\-töt eivät tuota
haluttua lopputulosta, voi sisennetyt tekstikappaleet toteuttaa myös
luetelmien (luku \ref{luku:luetelmat}) tekemiseen tarkoitetun
\koodi{list}\-/ympäristön avulla. Sopivilla asetuksilla ''luetelma''
sisältää ihan tavallisen näköisiä tekstikappaleita, jotka vain on
sisennetty vasemmalta tai oikealta tai molemmista reunoista.

\begin{esimerkki*}
\begin{koodilohko}
  \newenvironment{lohkolainaus}{%
    \begin{list}{}{
        \setlength{\leftmargin}{1cm}
        \setlength{\rightmargin}{1cm}
        \setlength{\itemindent}{0bp}
        \setlength{\listparindent}{\parindent}
        \setlength{\parsep}{\parskip}
        \setlength{\topsep}{1em}
        \setlength{\partopsep}{0bp}
      }
    \item\linespread{1}\small
    }{\end{list}}
\end{koodilohko}
\caption{Lohkolainausten eli tekstikappaleen vasemman ja oikean
  sisennyksen toteutus \koodi{list}\-/ ympäristön avulla. Esimerkkikoodi
  määrittelee uuden ympäristön nimeltä \koodi{lohkolainaus}}
\label{esim:vasen_oikea_sisennys}
\end{esimerkki*}

Esimerkistä \ref{esim:vasen_oikea_sisennys} selviää, kuinka
\koodi{list}\-/ ympäristöä voi käyttää sisennyksen toteuttamiseen.
Esimerkki määrittelee uuden ympäristön nimeltä \koodi{lohkolainaus},
jota voi hyödyntää myöhemmin dokumentissa.

\begin{koodilohkosis}
  \begin{lohkolainaus}
    Tämä tekstikappale on sisennetty vasemmalta ja oikealta. Lisäksi
    kirjainleikkaus on hieman pienempi (\small) kuin leipätekstissä.
  \end{lohkolainaus}
\end{koodilohkosis}

Omien ympäristöjen määrittelyä käsitellään tarkemmin luvussa
\ref{luku:ymparistot}. Esimerkin \ref{esim:vasen_oikea_sisennys} rivillä
11 oleva \koodi{\keno item}\-/ komento on pakollinen, koska se aloittaa
\koodi{list}\-/ ympäristöön kuuluvan luetelman kohdan. Sen perässä
olevat komennot \koodi{\keno linespread} ja \koodi{\keno small} sen
sijaan ovat vapaaehtoisia. Ne ovat mukana siksi, että on varsin
tavallista latoa lohkolainaukset pienemmällä rivivälillä
(rivikorkeudella) ja kirjainleikkauksella kuin leipäteksti.

\subsection{Rivinvaihtokomennot}
\label{luku:rivinvaihtokomennot}

Latex\-/lähdedokumentissa olevat rivinvaihdot tulkitaan sanaväleiksi
siinä missä välilyönnitkin, eli ne rivinvaihdot eivät päädy ladottuun
dokumenttiin (luvut \ref{luku:sanavali} ja
\ref{luku:rivinvaihtomerkit}). \koodimargin{\keno\keno} Sen sijaan
ladottuun dokumenttiin saadaan rivinvaihto käyttämällä komentoa
\koodi{\keno\keno} eli kaksi kenoviivaa. Komennon ei tarvitse sijaita
lähdedokumentissa rivin lopussa.

\begin{koodilohkosis}
  ensimmäinen \\ toinen \\
  kolmas
\end{koodilohkosis}

\begin{tulossis}
  ensimmäinen \\* toinen \\* kolmas
\end{tulossis}

Rivinvaihtokomennolle voi antaa hakasulkeissa valinnaisen argumentin,
joka ilmaisee rivien väliin ladottavan ylimääräisen pystysuuntaisen
tilan. Argumentin on siis oltava mitta (luku \ref{luku:mitat}).

\begin{koodilohkosis}
  ensimmäinen \\ toinen \\[1.3ex] kolmas
\end{koodilohkosis}

\begin{tulossis}
  ensimmäinen \\* toinen \\*[1.3ex] kolmas
\end{tulossis}

Komennosta on olemassa tähtiversio \koodi{\keno\keno *},
\koodimargin{\keno\keno *} joka edellisten ominaisuuksien lisäksi estää
sivun vaihtumisen tämän rivinvaihdon kohdalla. Myös tähtiversiolle voi
antaa valinnaiseksi argumentiksi mitan. Sen merkitys on sama kuin
komennon normaalilla versiollakin.

Rivin voi vaihtaa myös komennolla \koodi{\keno new\-line},
\koodimargin{\keno new\-line} mutta tämä komento ei hyväksy valinnaista
argumenttia eikä siitä ole tähdellistä versiota. Komennot \koodi{\keno
  new\-line} ja \koodi{\keno\keno} käyttäytyvät eri tavoin taulukoissa,
joita käsitellään luvussa \ref{luku:taulukot}.

\subsection{Lesket ja orvot}

Leski- ja orporivit tarkoittavat typografiassa rumannäköisiä yksinäisiä
rivejä. Leskirivi (engl. \englanti{widow}) on tekstikappaleen viimeinen
rivi, joka on yksinään sivun tai palstan yläreunassa. Orporivi (engl.
\englanti{orphan}) puolestaan on tekstikappaleen ensimmäinen rivi, joka
on yksin sivun tai palstan alareunassa. Molemmat voivat näyttää
ikävältä, mutta yleensä orporivejä ei pidetä kovin vakavana virheenä;
leskien välttämisessä ollaan enemmän tosissaan.

Ulkoasun lisäksi lesket ja orvot voivat olla ikäviä myös lukemisen
kannalta. Kun tekstikappale vaihtuu, lukija pitää pienen tauon ja
valmistautuu uuteen kappaleeseen. Sivun tai palstan olisi sopivaa
vaihtua samassa kohdassa eli tekstikappaleiden välissä, mutta leski- ja
orporivit aiheuttavat kaksi taukoa melkein peräkkäin: sekä kappaleiden
välissä että sivun tai palstan vaihtumisen kohdalla.

Latexissa leski- tai orporivit lienee käytännöllisintä estää
\paketti{no\-widow}\-/paketin\avctan{nowidow} avulla. Paketin lataamisen
jälkeen käytetään komentoa \koodi{\keno set\-no\-widow}
\koodimargin{\keno set\-no\-widow}, joka estää leskirivit eli pitää
tekstikappaleen lopusta vähintään kaksi riviä yhdessä sivun tai palstan
yläreunassa. Komennolle voi antaa hakasulkeissa valinnaisen argumentin
(kokonaisluvun), joka ilmaisee, kuinka monta riviä täytyy vähintään
pysyä yhdessä. Vastaavasti orporivit estetään komennolla \koodi{\keno
  set\-no\-club}, \koodimargin{\keno set\-no\-club} joka toimii samalla
tavalla. Molemmat komennot vaikuttavat koko dokumenttiin eli kaikkiin
tekstikappaleisiin.

\begin{koodilohkosis}
  \usepackage{nowidow}
  \setnowidow   % leskirivien esto
  \setnoclub    % orporivien esto
\end{koodilohkosis}

Paketti \paketti{nowidow} määrittelee myös komennot \koodi{\keno
  no\-widow} ja \koodi{\keno no\-club}, \koodimargin{\keno no\-widow}
\koodimargin{\keno no\-club} joilla voi vaikuttaa yksittäisen
tekstikappaleen leski- ja orporiveihin. Nämä komennot täytyy sijoittaa
tekstikappaleen loppuun, ja myös niille voi antaa hakasulkeissa
valinnaiseksi argumentiksi luvun, joka kertoo yhdessä pidettävien rivien
määrän.

Jos ei halua tai voi käyttää \paketti{no\-widow}\-/pakettia, voi leski-
ja orporivit estää myös Texin matalan tason toimintojen avulla.
Ladonta\-/algoritmi käyttää leski- ja orporiveissä sisäisesti
haitallisuusarvoa tai sakko\-arvoa (engl. \englanti{penalty}), ja jos
lesket ja orvot halutaan estää, määritellään niiden haitallisuusarvo
mahdollisimman korkeaksi. Käytännössä arvo 10\,000 tarkoittaa samaa kuin
ääretön, eli silloin lesken tai orvon haitallisuus on niin suuri, että
sellaisia ei sallita.

\begin{koodilohkosis}
  \widowpenalty 10000   % leskirivien esto
  \clubpenalty  10000   % orporivien esto
\end{koodilohkosis}

Leskien ja orpojen haitallisuusarvoiksi voi kokeilla hieman pienempiäkin
lukuja. Silloin leski- tai orporivit voidaan sallia joissakin
tilanteissa, jos ladonta\-/algoritmi ei löydä parempaakaan ratkaisua.

Orvoksi kutsutaan myös tavua, joka jää yksin kappaleen viimeiselle
riville. Se on häiritsevän näköinen ainakin silloin, kun tavu on
kapeampi kuin seuraavan kappaleen ensimmäisen rivin sisennys.
Orpotavujen estämiseen ei taida olla automaattisia keinoja, mutta
kappaleen sanamäärää ja sanajärjestystä muuttamalla voi tietenkin
vaikuttaa rivien latomiseen. Kappaleen viimeiseen sanaan voi kirjoittaa
myös tavutusvihjeitä (\koodi{\keno-}) mutta jättää vihje pois ennen
viimeistä tavua. Näin estetään lyhyen orpotavun muodostuminen.
Tavutusvihjeitä ja muita tavutuksen asetuksia käsitellään
luvussa~\ref{luku:tavutus}.

\subsection{Marginaalihuomautukset}
\label{luku:marginaalihuomautukset}

Klassisessa kirja\-typo\-gra\-fias\-sa -- jota tämäkin opas suunnilleen
noudattaa -- on aukeaman ulkoreunoissa melko suuret marginaalit. Niitä
voidaan käyttää eräänlaisina apupalstoina, joihin voi kirjoittaa
lyhyehköjä huomautuksia ja lisätietoa. Esimerkiksi tämän oppaan
marginaaleissa on Latexin komentoja, ympäristöjä ja asetusvalitsimia, ja
niiden tarkoituksena on helpottaa tietojen löytämistä.

Marginaalihuomautukset tehdään komennolla \koodi{\keno margin\-par},
\koodimargin{\keno margin\-par} jonka argumentiksi annetaan marginaaliin
tuleva teksti. Marginaalin teksti ladotaan vasempaan tai oikeaan
marginaaliin riippuen siitä, kumpi on aukeaman ulkoreunassa.
Yksipuolisilla sivuilla teksti ladotaan oletuksena oikeaan marginaaliin.
Mikäli sivu on kaksipalstaisessa tilassa (luku \ref{luku:palstat}),
marginaalihuomautus ladotaan lähempänä olevaan marginaaliin.

\begin{koodilohkosis}
  Tämä on leipätekstin tekstikappale,
    \marginpar{Tämä ladotaan marginaaliin.}
  joka ladotaan sivun normaalille tekstialueelle.
\end{koodilohkosis}

% Miten valinnainen argumentti toimii? \marginpar[left]{right}

Yleensä marginaalihuomautukset ladotaan erilaisella kirjainleikkauksella
kuin leipäteksti, jotta ne erottuvat toisistaan. Jos leipätekstissä
käytetään antiikvaa, voisi marginaalissa käyttää esimerkiksi pienempää
groteskia. Tämän toteutusta varten kannattaa määritellä uusi komento,
vaikkapa nimellä \koodi{\keno marginaali}, jota käyttämällä
marginaalihuomautukset saa helposti yhdenmukaiseksi. Seuraavassa on
esimerkki tällaisen komennon määrittelystä. Omia komentoja käsitellään
tarkemmin luvussa \ref{luku:komennot} ja fonttiasetuksia
luvussa~\ref{luku:kirjaintyypit}.

\begin{koodilohkosis}
  \newcommand{\marginaali}[1]{\marginpar{\sffamily\footnotesize #1}}
\end{koodilohkosis}

Sivun marginaalissa olevan huomautuspalstan leveys voidaan määrittää
sivun asetusten yhteydessä. Asetuksia hoitavan \paketti{geometry}\-/
paketin valitsimella \koodi{margin\-par\-width} asetetaan palstan leveys
ja valitsimella \koodi{margin\-par\-sep} palstan etäisyys
leipätekstistä. Valitsimella \koodi{reverse\-margin\-par} siirretään
marginaalihuomautukset päinvastaiseen marginaaliin. Näitä ja muitakin
sivun asetuksia käsitellään luvussa~\ref{luku:sivuasetukset}.

Asetuksia voi muuttaa myös kesken dokumentin. Mitta \koodi{\keno
  margin\-par\-width} määrittää marginaalihuomautuspalstan leveyden ja
mitta \koodi{\keno margin\-par\-sep} sen etäisyyden leipätekstistä.
Lisäksi on mitta \koodi{\keno margin\-par\-push}, jolla asetetaan
peräkkäisten marginaalihuomautusten vähim\-mäis\-etäi\-syys toisistaan.
Seuraavassa on esimerkki näiden mittojen asettamisesta:

\begin{koodilohkosis}
  \setlength{\marginparwidth}{50bp}
  \setlength{\marginparsep}{10bp}
  \setlength{\marginparpush}{6bp}
\end{koodilohkosis}

Kesken dokumentin komennolla \koodi{\keno reverse\-marginpar} voidaan
vaihtaa marginaalihuomautukset vastakkaiseen marginaaliin ja komennolla
\koodi{\keno normal\-margin\-par} palautetaan oletusasetukset.

\subsection{Anfangi}

% lettrine- tms. paketti

...

\section{Tekstin korostaminen}
\label{luku:korostus}

% Katso \ref{luku:fontit_muut}: tietoa harvennuksesta.

...

\section{Sivun vaihto ja sivujen tasaaminen}

...

% newpage, clearpage, minipage, nopagebreak
% \flushbottom \raggedbottom -komennot liittyvät sivun latomiseen

\section{Otsikot ja jäsennys}
\label{luku:otsikot}

% \sectionbreak ym. ovat titlesec-paketin ominaisuuksia. titlesec
% kannattaa täällä käsitellä.
%
% Varoitus: Ei erikoisia komentoja otsikoihin (mm. extdash), koska ne
% eivät toimi ainakaan pdf:n sisällysluettelossa tai ehkä muutenkaan.
% Voi käyttää otsikkokomennon valinnaista argumenttia.

...

\subsection{Kansilehti ja dokumentin perustiedot}
\label{luku:kansilehti}

...

% titlepage, \title, \author, \date, abstract

\subsection{Esittely, pääluvut ja liitteet}
\label{luku:frontmainbackmatter}
% \frontmatter, \mainmatter, \backmatter, \appendix

% Alussa käytetään komentoa \koodi{\keno frontmatter}, joka asettaa
% sivunumeroiden tyyliksi roomalaiset numerot (i, ii, iii jne.). Tässä
% osassa ovat ainakin sisällysluettelo ja tietokirjan tai tutkimuksen
% tiivistelmä. Varsinainen sisältö aloitetaan komennolla \koodi{\keno
% mainmatter}. \backmatter lopettaa päälukujen numeroinnin.

...

\section{Luetelmat}
\label{luku:luetelmat}

...

\section{Taulukot}
\label{luku:taulukot}

...

% \newline \\ \\*

\section{Leijuvat osat}
\label{luku:leijuosat}

...

\section{Ristiviitteet}
\label{luku:ristiviitteet}

...

\section{Alaviitteet}
\label{luku:alaviitteet}

% geometry: footnotesep

...

\section{Palstat}
\label{luku:palstat}

% Käsitellään Latexin omat ja multicol-paketti erikseen.

...

\section{Lähdeluettelo ja lähdeviitteet}
\label{luku:lähteet}

Samoin kuin ristiviitteissäkin (luku \ref{luku:ristiviitteet}) Latex
sisältää komennot, joilla lähdeluettelossa mainittuihin teoksiin voidaan
viitata muualta tekstistä. Ajatus on se, että lähdeluettelo laaditaan
tiettyjen komentojen avulla niin, että jokainen lähdeteos saa jonkin
yksilöllisen tunnisteen. Muualta tekstistä viitataan lähdeteoksiin
käyttämällä samoja tunnisteita, ja Latex osaa automaattisesti poimia
lähdeluettelosta esimerkiksi teoksen tekijöiden nimet ja vuosiluvun.

Lähdeviittauksiin ja lähdemerkintöihin on useita käytäntöjä, jotka
vaihtelevat eri ammatti- ja tieteen\-aloilla, oppilaitoksilla tai
julkaisijoilla. Tässä yhteydessä käsitellään melko vakiintuneita
suomalaisia käytäntöjä, jotka kuvataan \emph{Kielitoimiston
  oikein\-kir\-joi\-tus\-oppaas\-sa} \parencite{kt_oik}. Samalla
käsitellään joitakin asetuksia, joilla kukin voi muokata viittausten
ulkoasua omiin tarpeisiin sopivaksi.

Latex sisältää lähdeluetteloiden ja \=/viittausten perustoiminnot, mutta
niiden avulla ei saa yleisen suomalaisen käytännön mukaisia
lähdeviittauksia. Siksi käytämme apuna makropakettia, jolla viittausten
ja lähdeluettelon ulkoasuun voi vaikuttaa. Ensin käsiteltävä paketti
\paketti{natbib} (luku \ref{luku:natbib}) soveltuu perustarpeisiin ja
lienee sopivin valinta useimmille kirjoittajille. Laajoja tieteellisiä
teoksia kirjoittavan kannattanee opetella käyttämään monipuolista
\paketti{biblatex}\-/pakettia (luku \ref{luku:biblatex}) ja ylläpitää
yhteistä lähdeteosten tietokantaa, josta tarvittavat teokset poimitaan
kunkin dokumentin lähdeluetteloon automaattisesti.

\subsection{Peruskäyttöön (natbib)}
\label{luku:natbib}

Makropaketti \paketti{natbib}\avctan{natbib} laajentaa Latexin
lähdeviittausten perustoimintoja sen verran, että lähteisiin voidaan
viitata teoksen tekijöiden nimen ja vuosiluvun avulla. Seuraava
esimerkki havainnollistaa paketin käyttöönottoa ja asetuksia.

\begin{koodilohkosis}
  \usepackage{natbib}
  \setcitestyle{authoryear,aysep={},notesep={: }}
\end{koodilohkosis}

Edellisessä esimerkissä viittaustyyli valitaan \koodi{\keno
  set\-cite\-style}\-/komennon argumentissa valitsimella
\koodi{author\-year} (tekijä\--vuosi). Valitsimella \koodi{aysep}
määritetään, mikä välimerkki ladotaan tekijän nimen ja vuosiluvun
väliin. Tässä se jätetään tyhjäksi \koodi{\{\}}.
\koodi{note\-sep}\-/valitsimella asetetaan merkit, jotka ladotaan
vuosiluvun ja sitä seuraavan huomautuksen kuten sivunumeroiden väliin;
tässä tapauksessa määritettiin kaksoispiste ja väli \koodi{\{:~\}},
mutta pilkkukin on yleinnen käytäntö. \koodi{\keno
  set\-cite\-style}\-/komennon valitsimet erotetaan toisistaan pilkulla,
eikä erotinpilkkujen ympärillä saa olla välilyöntejä. Lopputuloksena
lähdemerkinnät näyttävät esimerkiksi seuraavanlaisilta:

% \pagebreak[3]

\begin{koodilohkosis}
  \citet*[27--29]{johdatus} % Viittaus teokseen ”johdatus”.
\end{koodilohkosis}

\begin{tulossis}
  Meikäläinen \& Teikäläinen (2020: 27--29)
\end{tulossis}

Lähdeluettelo kirjoitetaan \koodi{thebibliography}\-/ympäristön ja
\koodi{\keno bib\-item}\-/komentojen avulla esimerkin
\ref{esim:thebibliography} tavoin. Ympäristön aloittavan komennon
(rivi~1) yhteydessä on argumentti \koodi{00}, jolla ei ole tässä
yhteydessä merkitystä. Jos lähdeviittauksen tyylinä olisi
\koodi{numbers} (eikä \koodi{author\-year}), lähdeluettelon teokset
numeroitaisiin, ja silloin \koodi{thebibliography}\-/ympäristön
argumentti ilmaisee, kuinka leveän sisennyksen numeroidut teokset
tarvitsevat. Argumentiksi voi kirjoittaa mitä tahansa merkkejä, ja Latex
mittaa niiden leveyden. Kannattaa kirjoittaa leveitä numeroita kuten
nollia (\koodi{0}) niin monta kappaletta kuin on numeroita suurimmassa
lähdemerkinnän luvussa. Yksi nolla riittää, jos lähteitä on 1--9
kappaletta, kaksi jos lähteitä on kaksinumeroinen määrä eli 10--99
kappaletta jne.

\begin{esimerkki*}
\begin{koodilohko}
  \begin{thebibliography}{00}

  \bibitem[Meikäläinen ym.(2020)Meikäläinen \& Teikäläinen]{johdatus}
    Meikäläinen, Matti \& Teikäläinen, Teija (2020): Johdatus alkeiden
    perusteisiin. Toinen painos. Kustantaja Oy.

  \bibitem[Itkonen(2019)]{typografia} Itkonen, Markus (2019):
    Typografian käsikirja. Viides, tarkistettu painos. Typoteekki.
    Graafinen suunnittelu Markus Itkonen Oy.

  \end{thebibliography}
\end{koodilohko}
\caption{Lähdeluettelon kirjoittaminen
  \koodi{thebibliography}\-/ympäristön ja \koodi{\keno
    bib\-item}\-/komentojen avulla.}
\label{esim:thebibliography}
\end{esimerkki*}

Komennolla \koodi{\keno bib\-item} tehdään varsinaiset teosmerkinnät.
Samalla määritetään teoksen yksilöllinen tunniste ja mitä tietoja
lähdeviittauksissa näytetään. Yleinen muoto on seuraavanlainen:

\begin{koodilohkosis}
  \bibitem[lyhyt(vuosi)pitkä]{tunniste} Lähdeluettelon tekstit.
\end{koodilohkosis}

Valinnaisen argumentin aluksi kirjoitetaan lähdeviittauksen lyhyt
merkintä, joka tulisi näkymään lähdeviittauksissa esimerkiksi muodossa
''Meikäläinen ym.''. Heti sen perään kirjoitetaan sulkeissa teoksen
vuosiluku ja sen perään vapaavalintainen lähdeviittauksen pitkä
merkintä, joka näkyisi esimerkiksi tekstinä ''Meikäläinen \&
Teikäläinen''. Vuosiluvun sulkeiden ympärillä ei saa olla välilyöntejä.

Komennon pakollinen argumentti on kyseisen lähdeteoksen yksilöllinen
tunniste, jonka avulla kyseiseen teokseen viitataan. Komennon
argumenttien jälkeen kirjoitetaan samaan tekstikappaleeseen teksti, joka
tulee näkymään lähdeluettelossa.

Lähdeteoksiin viittaamiseen on useita eri komentoja, jotka eroavat
toisistaan siinä, mitä tietoa lähdeviittauksessa näytetään ja onko
lähdeviittaus tai sen osa sulkeissa vai ei. Taulukossa
\ref{tlk:natbib-cite} on joitakin \paketti{natbib}\-/paketin
viittauskomentoja sekä esimerkki viittauksen ulkoasusta. Kuviteltu
esimerkkiteos on peräisin esimerkistä \ref{esim:thebibliography}.

\providecommand{\rivi}{}
\renewcommand{\rivi}[2]{\koodi{\keno #1\{\ldots\}} & #2 \\}

\leijutlk{
  \begin{tabular}{ll}
    \toprule
    \ots{Komento} & \ots{Esimerkki} \\
    \midrule
    \rivi{citet}{Meikäläinen ym. (2020)}
    \rivi{citet*}{Meikäläinen \& Teikäläinen (2020)}
    \rivi{citep}{(Meikäläinen ym. 2020)}
    \rivi{citep*}{(Meikäläinen \& Teikäläinen 2020)}
    \rivi{citealt}{Meikäläinen ym. 2020}
    \rivi{citealt*}{Meikäläinen \& Teikäläinen 2020}
    \rivi{citeauthor}{Meikäläinen ym.}
    \rivi{citeauthor*}{Meikäläinen \& Teikäläinen}
    \rivi{citeyear}{2020}
    \rivi{citeyearpar}{(2020)}
    \bottomrule
  \end{tabular}
}{
  \caption{\paketti{natbib}\-/paketin lähdeviittauskomentoja}
  \label{tlk:natbib-cite}
}

Lähdeluettelon ulkoasuun voi vaikuttaa mittojen \koodi{\keno bibhang} ja
\koodi{\keno bibsep} avulla. Ensin mainittu on lähdemerkinnän
vaakasuuntaisen riippuvan sisennyksen suuruus, ja jälkimmäinen on
lähdemerkintöjen välinen pystysuuntainen tila. Mitat asetetaan
tavalliseen tapaan \koodi{\keno set\-length}\-/komennolla (luku
\ref{luku:mitat}):

\begin{koodilohkosis}
  \setlength{\parindent}{1.1em} % tekstikappaleiden 1. rivin sisennys
  \setlength{\bibhang}{\parindent}
  \setlength{\bibsep}{.5ex plus .1ex minus .1ex}
\end{koodilohkosis}

Lähdemerkintöjen fonttiin voi vaikuttaa määrittelemällä uudelleen
komennon \koodi{\keno bibfont} ja sijoittamalla halutut fontti- tai muut
komennot kyseisen komennon määritelmään.

\begin{koodilohkosis}
  \renewcommand{\bibfont}{\sffamily\small}
\end{koodilohkosis}

Oletuksena \koodi{thebibliography}\-/ympäristö latoo lähdeluettelolle
otsikon, ja otsikon teksti määräytyy kieli\-ase\-tus\-ten (luku
\ref{luku:kieliasetukset}) ja dokumenttiluokan perusteella (luku
\ref{luku:dokumenttiluokat}). Suomenkielisen lähdeluettelon otsikon voi
määrittää dokumentin esittelyosassa seuraavan esimerkin tavoin.
Esimerkissä hyödynnetään \koodi{\keno add\-to}\-/komentoa, joka sisältyy
\paketti{polyglossia}\-/{} ja \paketti{babel}\-/paketteihin.

\begin{koodilohkosis}
  \addto{\captionsfinnish}{%
    \renewcommand{\refname}{Lähteet} % article-dokumenttiluokka
    \renewcommand{\bibname}{Lähteet} % report- ja book-luokat
  }
\end{koodilohkosis}

On myös mahdollista määritellä koko komentosarja, joka suoritetaan
lähdeluettelon otsikoinnin yhteydessä. Se tehdään määrittelemällä
uudelleen komento \koodi{\keno bibsection}.

\begin{koodilohkosis}
  \renewcommand{\bibsection}{%
    \setcounter{secnumdepth}{-1}
    \section{Lähteet}
  }
\end{koodilohkosis}

Edellisessä esimerkissä komennolla \koodi{\keno set\-counter}
määritetään, mille otsikkotasolle dokumentin otsikoiden eli lukujen
numerointi yltää. Pieni arvo \mbox{(\koodi{-1})} käytännössä tarkoittaa,
että seuraaviin otsikoihin ei tule numerointia; lähdeluettelon otsikkoon
ei numerointia välttämättä haluta. Komento \koodi{\keno section} tekee
itse otsikon.

Jos ei halua, että \koodi{thebibliography}\-/ympäristö tekee otsikon
automaattisesti, voi \koodi{\keno bibsection}\-/komennon määrittää
tyhjäksi.

\begin{koodilohkosis}
  \renewcommand{\bibsection}{}
\end{koodilohkosis}

Tässä alaluvussa on käsitelty lähdeluettelon ja lähdeviitteiden
tekemistä \paketti{natbib}\-/paketin toimintojen avulla. Paketti
sisältää muitakin ominaisuuksia, joihin kannattaa tutustua paketin
ohjekirjan avulla. On muun muassa mahdollista tehdä lähdeteoksista
tietokanta Bibtex\-/järjestelmän avulla. Jos kuitenkin siihen suuntaan
haluaa edetä, ei kannata käyttää \paketti{natbib}\-/pakettia eikä
Bibtexiä vaan monipuolisempaa pakettia \paketti{biblatex}, jota
käsitellään seuraavassa alaluvussa.

\subsection{Vaativaan käyttöön (biblatex)}
\label{luku:biblatex}

Suurten lähde- ja kirjallisuusluetteloiden ylläpito voi olla aika
työlästä: pitää jatkuvasti varmistaa, että kaikki viitatut teokset ovat
luettelossa ja että luettelo on pilkulleen yhdenmukainen. Makropaketti
\paketti{biblatex}\avctan{biblatex} on vastaus sellaisiin tarpeisiin.

Ajatuksena on, että kaikki tiedonlähteet ja kirjallisuus kirjoitetaan
tietokantaan, josta \paketti{biblatex}\-/paketin komennot hakevat tiedot
automaattisesti. Kirjoittaja tai työryhmä voi ylläpitää yhtä
kirjallisuustietokantaa, joka voi olla saatavilla oman laitoksen
verkkopalvelimella tai julkisella verkkosivullakin. Dokumentin tekstissä
viitataan teoksiin yksilöllisen tunnisteen avulla, ja pelkän viittauksen
perusteella oikeat teokset ilmestyvät lähdeluetteloon automaattisesti
aakkosjärjestyksessä ja yhdenmukaisessa muodossa. Yhtään tiedonlähdettä
ei tarvitse kirjoittaa lopulliseen lähdeluetteloon käsin.

\paketti{biblatex}\-/paketin käyttö vaatii hieman opettelua -- varsinkin
jos on tarve muokata lähdeluettelon ja lähdeviittausten ulkoasua.
Muutaman tiedonlähteen ylläpito on todennäköisesti paljon helpompaa ja
nopeampaa niillä keinoilla, jotka on kuvattu luvussa \ref{luku:natbib}
(\paketti{natbib}). Sen sijaan laajoja tieteellisiä artikkeleita
kirjoittaville \paketti{biblatex} voi olla suuri apu, koska
artikkeleissa on yleensä paljon lähteitä ja useissakin artikkeleissa
viitataan yleensä samoihin lähteisiin.

\subsubsection{Teostietokanta}

Lähdeteosten tietokanta on erillinen tekstitiedosto, joka tavallisesti
nimetään \koodi{bib}\-/päätteiseksi, esimerkiksi \koodi{teokset.bib}.
Tiedosto koostuu \koodi{@}\=/merkillä ja teostyypin nimellä alkavista
tietueista, joiden yleinen muoto on seuraavanlainen:

\begin{koodilohkosis}
  @teostyyppi{tunniste,
    author = {...},
    title = "..."
  }
\end{koodilohkosis}

Teostyypin nimen jälkeen aaltosulkeiden sisään kirjoitetaan teoksen
kaikki tiedot. Ne alkavat teoksen yksilöllisellä tunnisteella, jota
käytetään lähdeviittauksissa. Tunnisteen jälkeen tulevat muut kentät.
Eri kentät kuten \koodi{author} ja \koodi{title} erotetaan toisistaan
pilkulla. Kentän nimi ja sen sisältö erotetaan toisistaan
yhtäsuuruusmerkillä (\koodi{=}), ja kentän sisältö kirjoitetaan
aaltosulkeiden tai lainausmerkkien sisään, kuten edellinen esimerkki
näyttää.

\begin{esimerkki*}
\begin{koodilohko}
  @book{itkonen_typogr,
    author = {Itkonen, Markus},
    title = {Typografian käsikirja},
    date = {2019},
    edition = {5},
    publisher = {Typoteekki. Graafinen suunnittelu Markus Itkonen Oy}
  }

  @incollection{likonen_teams,
    author = {Likonen, Teemu and Riskilä, Kaisa},
    title = {Verkkoyhteistyö Teams-ympäristössä},
    editor = {Tammi, Tuomo and Horila, Mikko},
    booktitle = {Oppimis- ja toimintaympäristöjen kehittäminen
      harjoittelukouluissa II},
    booksubtitle = {Tilat ja tekniikka pedagogisen kehittämisen tukena},
    publisher = {E-norssi. Opettajankouluttajien yhteistyöverkosto},
    date = {2020},
    pages = {85-92},
    url = {http://www.enorssi.fi/oppimisymparistojulkaisu2020/}
  }

  @article{likonen_tietokanta,
    author = {Likonen, Teemu},
    title = {Tietoa kantaan ja takaisin},
    journaltitle = {Skrolli},
    journalsubtitle = {Tietokonekulttuurin erikoislehti},
    date = {2015},
    volume = {2015},
    number = {4},
    pages = {52-55},
    url = {https://skrolli.fi/numerot/2015-4/}
  }

  @online{ctan,
    title = {Comprehensive TeX Archive Network},
    shorttitle = {CTAN},
    date = {1992/},
    url = {https://www.ctan.org/}
  }
\end{koodilohko}
\caption{Lähdeteosten tietokantatiedosto}
\label{esim:bib-tiedosto}
\end{esimerkki*}

Todellista käyttöä vastaava tietokanta tai sen osa on esimerkissä
\ref{esim:bib-tiedosto}, jossa on neljä erityyppistä teostietuetta:
\koodi{book}, \koodi{incollection}, \koodi{article} ja \koodi{online}.
Ensin mainittu%
\koodimargin{book} teostyyppi \koodi{book} sopii tavallisille kirjoille,
joissa tietyt tekijät (\koodi{author}) vastaavat suunnilleen koko
teoksen sisällöstä ja teoksella on jokin julkaisijataho
(\koodi{publisher}).

Teostyyppi%
\koodimargin{incollection} \koodi{incollection} tarkoittaa esimerkiksi
artikkelikokoelmaa, jonka yksittäiseen artikkeliin (\koodi{title}) ja
sen kirjoittajaan (\koodi{author}) on tarkoitus viitata. Voidaan mainita
myös artikkelin alku- ja lop\-pu\-si\-vut (\koodi{pages}). Kokoelmalla
on toimittaja (\koodi{editor}) ja yhteinen nimi (\koodi{book\-title}).

Tyyppi%
\koodimargin{article} \koodi{article} sopii säännöllisesti julkaistavan
aikakaus- tai muun lehden artikkeleihin. Viittauskohteena on yksittäinen
artikkeli ja sen kirjoittaja. Julkaisutiedoissa mainitaan lehden nimi
(\koodi{jour\-nal\-title}), julkaisukausi (\koodi{vol\-ume}), kauteen
kuuluvan julkaisun järjestysnumero (\koodi{num\-ber}) sekä mahdollisesti
artikkelin sivut (\koodi{pages}).

Verkkolähteiden%
\koodimargin{online} merkitsemiseen sopii \koodi{on\-line}\-/teostyyppi,
joissa on ta\-van\-omais\-ten kenttien lisäksi ainakin verkko\-/osoite
eli \koodi{url}\-/kenttä ja mahdollisesti viittauspäivä
(\koodi{url\-date}) osoittamassa, milloin viitatut tiedot olivat
saatavilla.

Teostyyppejä ja teoksiin liittyviä tietokenttiä on olemassa paljon
muitakin. Niiden merkitystä ja käyttöä neuvotaan tarkemmin
\paketti{biblatex}\-/paketin ohjeissa. Seuraavassa on kuitenkin pari
huomiota tietokannan ja kenttien kieli\-opillisista asioista.

Tietueissa joidenkin kenttien sisältö voi koostua useasta osasta kuten
saman teoksen eri tekijöistä. Eri tekijöiden nimet erotetaan
\koodi{author}- ja \koodi{editor}\-/kentissä toisistaan
\koodi{and}\-/sanalla. Oletuksena \paketti{biblatex} katsoo, että
tekijät ovat henkilöitä, ja käsittelee esimerkiksi etu- ja sukunimet
tietyllä tavalla: jos mukana on pilkku, sitä ennen on sukunimi, ja
etunimet tulevat pilkun jälkeen; jos pilkkua ei ole, etunimet ovat
ensin, ja sukunimi on lopussa.

Jos kuitenkin teoksen tekijänä on yritys tai yhteisö, täytyy sen nimi
kirjoittaa kokonaan aaltosulkeisiin, jottei sitä tulkittaisi
henkilönnimeksi. Tällaisten aaltosulkeiden sisällä voi käyttää
\koodi{and}\-/sanaa normaalisti, eikä sitä tulkita eri tekijöiden
erottimeksi. Seuraavassa on näistä esimerkit:

\begin{koodilohkosis}
  author = {Meikäläinen, Matti and Teikäläinen, Teija}
  author = {{Org. of Latex and Typography} and Meikäläinen, Matti}
\end{koodilohkosis}

Muunkinlaisia useasta osasta koostuvia kenttiä on olemassa.
Asiasanakentän (\koodi{keywords}) eri sanat erotetaan toisistaan
pilkulla, ja sivunumeroissa (\koodi{pages}) voi olla myös luku\-alueita,
jotka ilmaistaan yhdysmerkillä \mbox{(\koodi{-})}.

\begin{koodilohkosis}
  keywords = {eri, sanoja, peräkkäin}
  pages = {15-19}
\end{koodilohkosis}

Teostietokantaan voi määrittää vakiosisältöisiä muuttujia käyttämällä
\koodi{@string}\-/rakennetta. Vakioihin voi sitten viitata
teostietueiden kentistä esimerkin \ref{esim:bib-muuttujat} tavoin.
Vakiot ovat hyödyllisiä silloin, kun sama kentän sisältö toistuu useissa
teoksissa, kuten tässä esimerkissä sama tekijä (\koodi{author}) ja
aikakauslehden nimi (\koodi{jour\-nal\-title}). Vakioita voi yhdistää
saman kentän muuhun sisältöön käyttämällä \koodi{\#}\=/merkkiä, kuten
esimerkin rivillä 13 on tehty.

\begin{esimerkki*}
\begin{koodilohko}
  @string{
    oma = {Meikäläinen, Matti},
    lehti = {Hienon hieno aikakauslehti}
  }

  @article{hieno_artikkeli,
    author = oma,
    journaltitle = lehti,
    ...
  }

  @article{toinen_artikkeli,
    author = oma # { and Teikäläinen, Teija},
    journaltitle = lehti,
    ...
  }
\end{koodilohko}
\caption{Muuttujien käyttö ja \koodi{@string}\-/rakenne}
\label{esim:bib-muuttujat}
\end{esimerkki*}

\subsubsection{Käyttöönotto}

\paketti{biblatex}\-/makropaketti otetaan käyttöön esimerkin
\ref{esim:biblatex-käyttöönotto} rivien avulla. Mukana ovat myös paketit
\paketti{polyglossia} ja \paketti{cs\-quotes}. Jälkimmäinen sisältää
lainausmerkkien käyttöön liittyvää logiikkaa (luku
\ref{luku:lainausmerkit}), jota ilman \paketti{biblatex} ei saa eri
kielten erilaisia lainausmerkkejä oikein vaan käyttää pelkästään
amerikkalaisia (``~'').

\begin{esimerkki*}
\begin{koodilohko}
  % Polyglossia tai babel on ladattava ennen biblatexia.
  \usepackage{polyglossia}

  % Kielikohtaiset lainausmerkit oikein csquoten avulla.
  \usepackage{csquotes}

  \usepackage[style=authoryear]{biblatex}
\end{koodilohko}
\caption{\paketti{biblatex}\-/makropaketin käyttöönotto ja asetuksia}
\label{esim:biblatex-käyttöönotto}
\end{esimerkki*}

Paketin asetuksissa käytetään valitsinta \koodi{style} ja sen asetusta
\koodi{author\-year}, joka asettaa lähdeviittausten ja lähdeluettelon
tyyliksi tekijän ja vuosiluvun. Se on yleinen käytäntö suomenkielisissä
teksteissä. Vastaavia tyylejä ovat myös \koodi{author\-year-comp},
\koodi{author\-year-ibid} ja \koodi{author\-year-icomp}, jotka lisäksi
tiivistävät peräkkäisiä lähdeviittauksia, jos teoksen tekijä on sama.

Numerointiin tai kirjainlyhenteisiin perustuvat lähdeluettelo\-/{} ja
viittaustyylit ovat nimeltään \koodi{nu\-mer\-ic} ja
\koodi{al\-pha\-bet\-ic}. Muitakin tyylejä on olemassa, mutta tämän
oppaan esimerkeissä käsitellään tekijä--vuosi-tyyliä.

Makropaketin omien lähdeluettelo\-/{} ja viittaustyylien lisäksi
Latex\-/jakelupaketissa on todennäköisesti mukana myös ulkopuolisten
tahojen tekemiä tyylejä. Tyylikokonaisuus nimeltä
\paketti{bib\-latex-ext}\avctan{biblatex-ext} laajentaa
\paketti{biblatex}\-/paketin tavallisten tyylien ominaisuuksia.
Laajennettujen tyylien käyttäminen ei vaadi erillisen makropaketin
lataamista, vaan tyylin saa käyttöön yksinkertaisesti vain
kirjoittamalla sen nimen \paketti{biblatex}\-/paketin lataamisen
yhteydessä. Laajennetut tyylit alkavat kirjaimilla \mbox{\koodi{ext-},}
esimerkiksi \koodi{ext-author\-year} tai \koodi{ext-author\-year-comp}.

Esimerkin \ref{esim:biblatex-käyttöönotto} komentojen lisäksi täytyy
komennolla \koodi{\keno add\-bib\-re\-source} nimetä kaikki käyttöön
otettavat teostietokantatiedostot. Komentoja ja tiedostoja voi olla
useampiakin, ja tietokanta voi olla myös verkko\-/osoitteen takana oleva
tiedosto. \koodi{\keno add\-bib\-re\-source}\-/komennot täytyy
kirjoittaa Latex\-/dokumentin esittelyosaan.

\begin{koodilohkosis}
  \addbibresource{teokset.bib}
  \addbibresource{~/texmf/omat_kirjoitukset.bib}
  \addbibresource[location=remote]{http://osoite.netissä/yhteiset.bib}
\end{koodilohkosis}

Lähdeluettelo ladotaan dokumenttiin komennolla \koodi{\keno
  print\-bib\-li\-og\-ra\-phy}. Komennolle voi antaa valinnaisen
argumentin, jonka valitsimilla vaikutetaan esimerkiksi lähdeluettelon
otsikon tekstiin tai poistetaan automaattinen otsikointi kokonaan. On
myös olemassa erilaisia lähdeteosten rajaamisvalitsimia, joiden avulla
voi määrittää, mitä teoksia kyseiseen luetteloon halutaan. Näin voidaan
esimerkiksi rajata painetut lähteet yhteen luetteloon, julkaisemattomat
toiseen ja verkkolähteet kolmanteen.

\begin{koodilohkosis}
  \printbibliography
  \printbibliography[title={Lähteet}]
  \printbibliography[heading=none,  % Ei automaattista otsikkoa,
    type=online]           % ja rajataan vain online-tyyppisiin.
\end{koodilohkosis}

Lähdeluetteloon tulevat mukaan vain ne teokset, joihin on viitattu.
Mitään ei siis näy, jos ei ole lähdeviittauksia. Seuraavassa alaluvussa
käsitellään lähdeviittauskomentoja ja myös ''näkymätöntä''
viittauskomentoa, jolla teoksia saadaan mukaan luetteloon ilman näkyvää
viittausta.

\subsubsection{Lähdeviittaukset}

\providecommand{\rivi}{}
\renewcommand{\rivi}[2]{\koodi{\keno #1\{\ldots\}} & #2 \\}

\leijutlk{
  \begin{tabular}{ll}
    \toprule
    \ots{Komento} & \ots{Esimerkki} \\
    \midrule
    \rivi{cite}{Meikäläinen 2020}
    \rivi{textcite}{Meikäläinen (2020)}
    \rivi{parencite}{(Meikäläinen 2020)}
    \rivi{citeauthor}{Meikäläinen}
    \rivi{citeyear}{2020}
    \rivi{citetitle}{[teoksen nimi]}
    \rivi{footcite}{Meikäläinen 2020 [alaviitteessä]}
    \rivi{nocite}{[näkymätön viittaus]}
    \bottomrule
  \end{tabular}
}{
  \caption{\paketti{biblatex}\-/paketin lähdeviittauskomentoja}
  \label{tlk:biblatex-cite}
}

Taulukkoon \ref{tlk:biblatex-cite} on koottu tavallisimpia
\paketti{biblatex}\-/paketin viittauskomentoja. Komennoille voi antaa
valinnaisen argumentin, jolla kerrotaan täsmentävää tietoa
lähdeviittauksesta. Yleensä se on viitattavan teoksen sivunumero.
Viittaus näkyy dokumentissa esimerkiksi seuraavalla tavalla:

% \pagebreak[3]

\begin{koodilohkosis}
  \textcite[27--29]{johdatus} % Viittaus teokseen ”johdatus”.
\end{koodilohkosis}

\begin{tulossis}
  Meikäläinen ja Teikäläinen (2020, s. 27--29)
\end{tulossis}

Jos halutaan sisällyttää lähdeluetteloon teoksia, joihin ei ole
välttämättä viitattu, käytetään dokumentissa kerran ''näkymätöntä''
viittauskomentoa \koodi{\keno no\-cite}. Sille annetaan argumentiksi
tunnisteet niistä teoksista, jotka halutaan mukaan luetteloon.
Argumentti~\koodi{*} (tähti) valitsee kaikki teokset.

\begin{koodilohkosis}
  \nocite{meikäläinen, teikäläinen} % Nämä teokset mukaan.
  \nocite{*}                        % Kaikki mukaan.
\end{koodilohkosis}

\subsubsection{Lähdetiedostojen kääntäminen}

Latexin kääntäjä\-ohjelmat Lualatex tai Xelatex eivät yksinään riitä,
sillä teostietokanta ei ole tavallinen Latex\-/muotoinen tiedosto.
Tarvitaan myös Latex\-/jakelun mukana tulevaa komentoa \koodi{bib\-er},
joka käsittelee teostietokantaan liittyviä tiedostoja. Lopulta
Latex\-/kääntäjääkin täytyy kutsua kaksi kolme kertaa, jotta kaikki
ristiviitteet saadaan kuntoon. Komentojen suoritusjärjestys on
seuraavanlainen:

\begin{koodilohkosis}
  lualatex teksti.tex
  biber teksti.bcf
  lualatex teksti.tex
  lualatex teksti.tex
\end{koodilohkosis}

Edellisen esimerkin komennoissa voi tiedoston nimistä jättää päätteet
pois (\koodi{.tex}, \koodi{.bcf}). \koodi{lualatex}\-/ohjelman paikalla
voi olla myös \koodi{xelatex}. Kääntäminen on vielä helpompaa, kun
käyttää \koodi{latexmk}\-/ohjelmaa (luku \ref{luku:latexmk}), joka osaa
automaattisesti suorittaa myös \koodi{bib\-er}\-/ohjelman ja tarvittavat
uudelleen kääntämiset. Yksi komento riittää käyttäjälle:

\begin{koodilohkosis}
  latexmk -lualatex teksti.tex    % tai: -xelatex
\end{koodilohkosis}

\subsubsection{Lähdeluettelon mittoja}

Lähdeluettelon ulkoasuun voi vaikuttaa muutaman eri mitan avulla, joista
esitellään tässä yhteydessä vain osa. Lähdemerkinnän riippuvan
sisennyksen suuruus määräytyy mitan \koodi{\keno bibhang} avulla.
Yleensä lienee sopivaa asettaa se samaksi kuin tekstikappaleiden
ensimmäisen rivin sisennys \koodi{\keno parindent}.

\begin{koodilohkosis}
  \setlength{\parindent}{1.1em} % tekstikappaleiden 1. rivin sisennys
  \setlength{\bibhang}{\parindent}
\end{koodilohkosis}

Mitta \koodi{\keno bib\-item\-sep} on lähdemerkintöjen välinen
pystysuuntainen tila. Sen avulla voi harventaa lähdeluetteloa, jolloin
lähdemerkinnät erottuvat paremmin toisistaan. Mitan \koodi{\keno
  bib\-name\-sep} avulla voi tehdä suuremman pystysuuntaisen välin
lähdemerkintöjen väliin silloin, kun teoksen tekijä vaihtuu
(\koodi{author} tai \koodi{editor}). Toisin sanoen tämän mitan avulla
voi ryhmitellä saman tekijän teokset tiiviimmin yhteen ja jättää väliä
seuraavan tekijän teoksiin. Vastaavanlainen mitta on \koodi{\keno
  bib\-init\-sep}, jota käytetään silloin, kun lähdemerkinnän aloittava
kirjain vaihtuu. Tämän avulla voi ryhmitellä lähdemerkinnät aakkosittain
eli tehdä suuremman välin aina lähdemerkinnän alkukirjaimen vaihtuessa.

\begin{koodilohkosis}
  \setlength{\bibitemsep}{.5ex plus .1ex minus .1ex}
  \setlength{\bibnamesep}{1em  plus .2ex minus .1ex}
  \setlength{\bibinitsep}{2em  plus .2ex minus .1ex}
\end{koodilohkosis}

\subsubsection{Muita asetuksia}

Lähdemerkintöjen fonttiin voi vaikuttaa määrittelemällä uudelleen
komennon \koodi{\keno bibfont} ja sijoittamalla halutut fontti- tai muut
komennot kyseisen komennon määritelmään.

\begin{koodilohkosis}
  \renewcommand{\bibfont}{\sffamily\small}
\end{koodilohkosis}

Lähdemerkinnät itsessään muodostetaan automaattisesti tiettyjen
tyyli\-ase\-tus\-ten perusteella. Omiakin tyylejä voi tehdä, mutta
yleensä riittää vain yksittäisen asetusten muuttaminen. Niistä
käsitellään tässä yhteydessä muutama. Asetusten muuttamiseen tarvitaan
yleensä \paketti{biblatex}\-/paketin omia asetuskomentoja.

Lähdeluettelon nimet näkyvät oletusasetuksilla siten, että teoksen
ensimmäisen tekijän sukunimi mainitaan ensin (luettelon
aakkosjärjestyksen vuoksi) mutta saman teoksen muiden tekijöiden etunimi
mainitaan ensin. Tekijöiden nimet näkyvät siis seuraavalla tavalla:
''Meikäläinen, Matti ja Teija Teikäläinen''. Suomessa on kuitenkin
tapana kirjoittaa kaikki nimet samalla tavalla ja mainita sukunimi aina
ensin. Tämä saadaan toteutettua seuraavilla komennolla:

% \pagebreak[3]

\begin{koodilohkosis}
  \DeclareNameAlias{default} {family-given}
  \DeclareNameAlias{sortname}{family-given}
\end{koodilohkosis}

\begin{tulossis}
  Meikäläinen, Matti ja Teikäläinen, Teija (2020). --~--
\end{tulossis}

Saman teoksen eri tekijöiden nimet erotetaan oletuksena toisistaan
pilkuilla paitsi kahden viimeisen nimen välissä on ja-sana. Usein on
kuitenkin tapana käyttää \&\=/merkkiä ainakin lähdeluettelossa.
Seuraavat esimerkkikomennot asettavat lähdeluettelon kaikkien nimien
erottimeksi \&\=/merkin.

% \pagebreak[3]

\begin{koodilohkosis}
  \DeclareDelimFormat[bib]{multinamedelim}{\space\&\space}
  \DeclareDelimFormat[bib]{finalnamedelim}{\space\&\space}
\end{koodilohkosis}

\begin{tulossis}
  Meikäläinen, Matti \& Teikäläinen, Teija \& Tutkija, Tuija (2020).
  --~--
\end{tulossis}

Edellisten esimerkkikomentojen valinnainen argumentti \koodi{bib}
tarkoittaa, että vaikutetaan vain lähdeluetteloon. Argumentti voi olla
myös \mbox{\koodi{cite}}, jolloin vaikutetaan lähdeviittauksiin.
Ero\-tin\-merkki\-ase\-tuk\-sen nimi \koodi{multi\-name\-delim}
tarkoittaa muiden kuin kahden viimeisen tekijän nimen välissä olevaa
erotinta. Kahden viimeisen nimen erotin määritellään asetuksella
\koodi{final\-name\-delim}.

Useiden saman teoksen tekijöiden luettelot lyhennetään automaattisesti
esimerkiksi muotoon ''Meikäläinen et~al.'', ja lyhentämisen säännöt
määritellään tiettyjen \koodi{max}- ja \koodi{min}\-/alkuisten paketin
valitsimien avulla. Lähdeluettelossa teoksen tekijäluetteloon
vaikutetaan valitsimilla \koodi{max\-bib\-names} ja
\koodi{min\-bib\-names}, kun taas lähdeviittausten tekijäluetteloon
vaikutetaan valitsimilla \koodi{max\-cite\-names} ja
\koodi{min\-cite\-names}. Asetukset toimivat siten, että jos
enimmäismäärä (max) ylittyy, typistetään tekijäluettelo vähimmäismäärään
(min) ja lisätään ilmaus ''et al.'' tms.

Tekijäluetteloa ei kuitenkaan välttämättä lyhennetä, jos luettelosta
tulisi täsmälleen samanlainen kuin jollakin toisella teoksella. Tähän
asiaan puolestaan vaikutetaan valitsimella \koodi{unique\-list}, joka on
oletuksena päällä viittaustyylissä \koodi{author\-year}.

\begin{koodilohkosis}
  \usepackage[style=authoryear, maxbibnames=99, minbibnames=3,
    maxcitenames=3, mincitenames=1, uniquelist=true]{biblatex}
\end{koodilohkosis}

Kun halutaan näyttää lähdeluettelossa vain tekijän etunimen alkukirjain
eikä koko etunimeä, käytetään paketin valitsinta \koodi{given\-inits}.

% \pagebreak[3]

\begin{koodilohkosis}
  \usepackage[…, giveninits]{biblatex}
\end{koodilohkosis}

\begin{tulossis}
  Meikäläinen, M. (2020). -- --
\end{tulossis}

Lähdeluettelossa näytetään teoksen tekijän nimen kohdalla ajatusviiva,
jos tekijä on sama kuin luettelon edelliselläkin teoksella. Mikäli tätä
(sinänsä yleistä) käytäntöä ei haluta, täytyy käyttää paketin asetusta
\koodi{dash\-ed=\katk false}.

\begin{koodilohkosis}
  \usepackage[…, dashed=false]{biblatex}
\end{koodilohkosis}

Joskus on tapana latoa lähdeluettelossa tekijöiden nimet esimerkiksi
pienversaalilla, jotta ne erottuvat luettelosta paremmin. Tällainen
muutos vaatii, että lähdeluettelon tulostamisen yhteydessä määritellään
uudelleen henkilön nimiin liittyvät tulostuskomennot \koodi{\keno
  mk\-bib\-name\-fam\-i\-ly}, \koodi{\keno mk\-bib\-name\-giv\-en},
\koodi{\keno mk\-bib\-name\-pre\-fix} ja \koodi{\keno
  mk\-bib\-name\-suf\-fix}. Se saadaan automaattiseksi seuraavilla
komennoilla:

% \pagebreak[3]

\begin{koodilohkosis}
  \AtBeginBibliography{%
    \renewcommand{\mkbibnamefamily}[1]{\textsc{#1}}
    \renewcommand{\mkbibnamegiven} [1]{\textsc{#1}}
    \renewcommand{\mkbibnameprefix}[1]{\textsc{#1}}
    \renewcommand{\mkbibnamesuffix}[1]{\textsc{#1}}
  }
\end{koodilohkosis}

\begin{tulossis}
  \textsc{Meikäläinen}, \textsc{Matti} \& \textsc{Teikäläinen},
  \textsc{Teija} (2020). -- --
\end{tulossis}

Lähdeluettelon eri osien erottimena on yleensä piste. Joskus kuitenkin
tekijöiden nimien ja vuosiluvun jälkeen on kaksoispiste. Se saadaan
toteutettua seuraavalla komennolla:

% \pagebreak[3]

\begin{koodilohkosis}
  \DeclareDelimFormat[bib]{nametitledelim}{\addcolon\space}
\end{koodilohkosis}

\begin{tulossis}
  Meikäläinen, Matti (2020): -- --
\end{tulossis}

Oletuksena \paketti{biblatex} kursivoi \koodi{book}\-/tyyppisten teosten
nimen (\koodi{title}). Sen sijaan artikkelikokoelmissa
(\koodi{incollection}) ja aikakauslehdissä (\koodi{article})
kursivoidaan julkaistun kokoelman nimi (\koodi{book\-title}) ja
aikakauslehden nimi (\koodi{jour\-nal\-title}). Näissä teostyypeissä
viitatun artikkelin nimi (\koodi{title}) kirjoitetaan lainausmerkkeihin.
Käytäntö tuntuu järkevältä, sillä kursivoituna on aina julkaistu
kokonainen teos eikä sen osa. Käytännössähän tiedonlähde joudutaan
hakemaan teoksen nimen perusteella. Joku voi silti haluta muuttaa näiden
ulkoasua ja esimerkiksi kursivoida aina viittauksen kohteena olevan
artikkelin. Seuraavassa on esimerkkikomennot edellä mainittujen
lähdeluettelon kenttien muuttamiseen.

\begin{koodilohkosis}
  \DeclareFieldFormat[article,incollection]{title}{\textit{#1}}
  \DeclareFieldFormat[article]{journaltitle}{#1}
  \DeclareFieldFormat[incollection]{booktitle}{#1}
\end{koodilohkosis}

Edellä olevissa esimerkkikomennoissa on valinnaisena argumenttina ne
teostyypit, joihin halutaan vaikuttaa. Jos valinnaisen argumentin jättää
pois, vaikutetaan kaikkiin teostyyppeihin, ellei tarkempaa
teostyyppikohtaista määritelmää ole olemassa. Ensimmäinen pakollinen
argumentti on kentän nimi teostietokannassa, ja toinen pakollinen
argumentti on sisältö, joka ladotaan lähdemerkintään kyseisen tiedon
kohdalle. Teostietokannasta tulevan kentän sisältö on parametrissa
\koodi{\#1}.

Mikäli haluaa jonkin teoksen tiedon lainausmerkkeihin tai sulkeisiin,
kannattaa käyttää komentoa \koodi{\keno mk\-bib\-quote} tai \koodi{\keno
  mk\-bib\-parens}. Ne ymmärtävät ottaa huomioon eri kielten
lainausmerkkikäytännöt ja mahdolliset sisäkkäiset sulkeet.

\begin{koodilohkosis}
  \DeclareFieldFormat[incollection]{booktitle}{\mkbibquote{#1}}
\end{koodilohkosis}

Oletuksena teoksen vuosiluku tai muu päiväys ladotaan lähdeluetteloon
sulkeissa. Joissakin lähdeluettelokäytännöissä sulkeita ei kuitenkin
ole, joten seuraavaksi käsitellään keino sulkeiden poistamiseen.
Tavallisessa \paketti{biblatex}\-/paketin lähdeluettelotyylissä
\koodi{author\-year} ei ole omaa asetusta teoksen päiväyksen ulkoasun
muuttamiseen, mutta jos käyttää tyyliä \koodi{ext-author\-year} (tms.),
sekin puute korjaantuu, ja voi käyttää
\koodi{bib\-label\-date}\-/asetusta.

% \pagebreak[3]

\begin{koodilohkosis}
  \usepackage[style=ext-authoryear]{biblatex}

  \DeclareFieldFormat{biblabeldate}{#1}
\end{koodilohkosis}

\begin{tulossis}
  Meikäläinen, Matti 2020. -- --
\end{tulossis}

Artikkelikokoelmissa (teostyyppi \koodi{incollection}) mainitaan
oletuksena kokoelman nimi ja toimittajat seuraavassa muodossa:
''Teoksessa: \textit{Hieno artikkelikokoelma.} Toim. Kirjailija,
Kaisa''. Ensin siis mainitaan julkaisun nimi ja sen jälkeen toimittajien
nimet. Suomessa on tapana kirjoittaa nämä tiedot toisinpäin ja laittaa
toimittajarooli sulkeisiin. Tällaiset asetukset saa käyttämällä tyyliä
\koodi{ext-author\-year} (tms.), paketin asetusta
\koodi{in\-name\-before\-title=\katk true} ja seuraavia komentoja:

% \pagebreak[3]

\begin{koodilohkosis}
  \usepackage[style=ext-authoryear, innamebeforetitle=true]{biblatex}

  \DeclareFieldFormat{editortype}{\mkbibparens{#1}}
  \DeclareDelimFormat{editortypedelim}{\addspace}
\end{koodilohkosis}

\begin{tulossis}
  -- -- Teoksessa: Kirjailija, Kaisa (toim.). \textit{Hieno
    artikkelikokoelma}. -- --
\end{tulossis}

Lähdeviittauksissa vuosiluvun ja sivunumeroiden välissä käytetään
välillä pilkkua ja välillä kaksoispistettä. Sivunumeroiden yhteydessä
voi olla lyhenne ''s.'' tai se voidaan jättää pois. Seuraavilla
komennoilla vaikutetaan näihin asetuksiin:

% \pagebreak[3]

\begin{koodilohkosis}
  \DeclareFieldFormat{postnote}{#1} % Lyhenne ”s.” pois.
  \DeclareDelimFormat{postnotedelim}{\addcolon\space} % Kaksoispiste.

  \textcite[15--16]{tunniste} toteaa artikkelissaan -- --
\end{koodilohkosis}

\begin{tulossis}
  Meikäläinen (2020: 15--16) toteaa artikkelissaan -- --
\end{tulossis}

Kun saman teoksen usean tekijän luettelo lyhennetään, käytetään
oletuksena latinankielistä ilmausta pois jäävien nimien tilalla:
''Meikäläinen et al.''. Ilmauksen voi muuttaa suomenkieliseksi
seuraavalla komennolla:

% \pagebreak[3]

\begin{koodilohkosis}
  \DefineBibliographyStrings{finnish}{
    andothers = {ym.},
  }
\end{koodilohkosis}

\begin{tulossis}
  Meikäläinen ym. (2020)
\end{tulossis}

\paketti{biblatex}\-/paketti sisältää valtavan paljon asetuksia ja
mahdollisuuksia lähdeluettelon ja \=/viitteiden ulkoasun säätämiseen.
Esimerkiksi komennolla \koodi{\keno
  De\-clare\-Bib\-li\-og\-ra\-phy\-Driv\-er} voi ottaa täysin haltuun,
miten tietty teostyyppi ladotaan lähdeluetteloon. Komennolla
\koodi{\keno De\-clare\-Sort\-ing\-Tem\-plate} voi määritellä omia
aakkostustapoja. Lisätietoa saa paketin ohjekirjasta.%
\avctan{biblatex}

\section{Kuvat ja värit}
\label{luku:grafiikka}

...

\section{Diaesitykset}
\label{luku:diaesitykset}

...

\section{Kirjeet}
\label{luku:kirjeet}

...


\chapter{Matematiikka}
\label{luku:matematiikka}

% \section{Käyttöönotto}
% \subsection{Unicode-math}
% \subsection{Mathspec}
% \subsection{Vanha tapa}
% \section{Jotain sisältöäkin tarvitaan}
% \section{Ympäristöjä}
% align... (amsmath)

...

\chapter{Muuta}

...

\section{Päiväykset ja kellonajat}
% \day\month\year
% Em. laskureissa on kenoviiva
% datetime2-paketti

...

\section{Omat paketit}

...

\section{Omat dokumenttiluokat}
\label{luku:omat_dokumenttiluokat}

Oman dokumenttiluokan tekeminen voi olla hyödyllistä silloin, kun täytyy
tuottaa paljon samantyylisiä dokumentteja ja haluaa määrittää monet
asetukset valmiiksi. Jotkin oppilaitokset ovat tehneet oman
dokumenttiluokan, koska opinnäytetöille ja muille julkaisuille halutaan
yhtenäinen ulkoasu ja typografia.

\backmatter

\chapter [Kirjallisuutta] {Kirjallisuutta \enspace {\normalsize\ldots ja
    muuta hyödyllistä}}

\printbibliography[heading=none]

%\printindex

\end{document}

% Mietittävää
%
% Mahdollisesti korvataan valitsin-sanat asetus-sanoilla.
