\documentclass[notitlepage,oneside]{book}
\usepackage[a4paper, vscale=.75, hscale=.68, footskip=1.5cm]{geometry}
\usepackage{fontspec}
\usepackage{polyglossia}
\usepackage{ragged2e}
\usepackage[hang,bottom]{footmisc}
%\usepackage{titling}
\usepackage{titlesec}
\usepackage{titletoc}
\usepackage{graphicx}
\usepackage{color}
\usepackage{floatrow}
\usepackage{caption}
\usepackage{newfloat}
\usepackage{fancyvrb}
%\usepackage{placeins}
\usepackage{booktabs}
%\usepackage{fancyhdr}
\usepackage{noindentafter}
\usepackage[unicode,hyperfootnotes=false]{hyperref}
\usepackage[shortcuts]{extdash}

\hypersetup{ hidelinks, bookmarksopen, bookmarksnumbered,
  pdfauthor={Teemu Likonen} }

\setdefaultlanguage{finnish}
\setotherlanguage{english}

\setmainfont{Linux Libertine O}
[Scale=1.4, Numbers=Lowercase]

\setsansfont{Linux Biolinum O}
[Scale=MatchLowercase, Numbers=Lowercase]

\setmonofont{Linux Libertine Mono O}
[Scale=MatchLowercase, LetterSpace=-2]

\linespread{1.58}

\newcommand{\gemenanum}{\addfontfeatures{Numbers=Lowercase}}
\newcommand{\versaalinum}{\addfontfeatures{Numbers=Uppercase}}

\clubpenalty=10000
\widowpenalty=10000
\setlength{\emergencystretch}{1em}

\setlength{\parindent}{1em}
\setlength{\parskip}{0em}
\setlength{\footnotemargin}{.8em}
\setlength{\floatsep}{3ex}
\setlength{\textfloatsep}{4ex}

\DeclareFloatingEnvironment[ name={Esimerkki} ]{esimerkki}

\DefineVerbatimEnvironment{koodilohko}{Verbatim}{
  fontsize=\footnotesize, gobble=1, frame=lines, numbers=left,
  numbersep=.3em, xleftmargin=0em, xrightmargin=0em, baselinestretch=1.3 }

\DefineVerbatimEnvironment{koodilohkosis}{Verbatim}{
  fontsize=\footnotesize, gobble=2, frame=none, numbers=none,
  numbersep=0em, xleftmargin=\parindent, xrightmargin=0em,
  baselinestretch=1.3, samepage=true }

\floatsetup{ font={small}, justification=raggedright,
  margins=raggedright, captionskip=2ex, capposition=bottom }

\DeclareCaptionFont{rivivali}{\linespread{1.3}\selectfont}

\captionsetup{ font={small, sf, rivivali}, labelfont={bf}, textfont={},
  textformat=period, margin=.5em, justification=raggedright,
  singlelinecheck=off }

\captionsetup[esimerkki]{ skip=-0.5ex, margin=.5em }

\newcommand{\leiju}[3]{
  \begin{#1}
    \floatbox{#1}{#2}{#3}
  \end{#1}
}

\newcommand{\leijutlk}[2]{\leiju{table}{\versaalinum #1}{#2}}
\newcommand{\leijukuva}[2]{\leiju{figure}{#1}{#2}}

%\fancyhf{}
% \fancyhf[hl]{\small\nouppercase{\leftmark}}
% \fancyhf[hr]{\small\nouppercase{\rightmark}}
% \fancyhf[fc]{\thepage}
%\renewcommand{\headrulewidth}{1pt}

% \fancypagestyle{plain}{%
%   \fancyhf{}
%   \fancyhf[fc]{\thepage}
%   \renewcommand{\headrulewidth}{0pt}
% }

\definecolor{tavu}{rgb}{1,0,0}
\definecolor{sitova}{rgb}{1,0,0}

\newcommand{\keno}{\textbackslash}
\newcommand{\koodi}[1]{\textsf{\versaalinum #1}}
\newcommand{\tavuvihje}{\keno-}
\newcommand{\tavukohta}{\textcolor{tavu}{\rule{1pt}{1.5ex}}}
\newcommand{\sitovaym}{\textcolor{sitova}{-}}

\newcommand{\ots}[1]{{\sffamily\bfseries\versaalinum #1}}

\NoIndentAfterEnv{koodilohkosis}

\addto{\captionsfinnish}{
  \renewcommand{\contentsname}{Sisällys}
}

\setcounter{tocdepth}{3}
\setcounter{secnumdepth}{3}

\newcommand{\otsikkotyyli}{ \raggedright \linespread{1.1} \sffamily
  \bfseries }

\titleformat{\chapter}[display]{\LARGE\bfseries}
{\chaptertitlename\hspace{.3em}\thechapter}{1.5ex}{\otsikkotyyli\Huge}[]
\titlespacing*{\chapter}{0em}{*13}{*8}

\titleformat{\section}{\otsikkotyyli\Large}
{\thesection}{.8em}{}[]
\titlespacing*{\section}{0pt}{*4}{*2}

\titleformat{\subsection}{\otsikkotyyli\normalsize}
{\thesubsection}{.8em}{}[]
\titlespacing*{\subsection}{0pt}{*3}{*1}

% \titleformat{\subsubsection}{\otsikkotyyli\normalsize\itshape}
% {\thesubsubsection}{.8em}{}[]
% \titlespacing*{\subsubsection}{0pt}{*3}{*1}

\begin{document}
\pagestyle{empty}

\hyphenation{
  an-tiik-va
  base-line-skip
  docu-ment
  Ext-dash
  font-size
  Font-spec
  foot-note-size
  gro-tes-ki
  huge
  large
  line-spread
  Lua-la-tex
  Match-Lower-case
  mono-space
  new-font-face
  new-font-family
  ni-men-omaan
  normal-size
  Poly-glos-sia
  Scale
  script-size
  select-font
  set-main-font
  set-mono-font
  set-sans-font
  sf-family
  sf-family-abs
  small
  tiny
  tt-family
  tt-family-abs
  type-writer
  Xe-la-tex
}

{
  \renewcommand{\thispagestyle}[1]{}
  %\tableofcontents
}

\clearpage
\pagestyle{plain}

\chapter*{Esipuhe}
\addcontentsline{toc}{chapter}{Esipuhe}
\phantomsection

% Miksi Latex?

... Niinpä tekstidokumenttien toteuttaminen Latexilla ei ole läheskään
samanlaista kuin työskentely tietotekniikassa yleensä. Yleensähän
käynnistetään jokin sovellus\-ohjelma, joka sisältää suunnilleen kaikki
tarvittavat toiminnot. Samalla sovelluksella työ viedään alusta loppuun.

Sen sijaan Latexin kanssa työskentely on lähempänä ohjelmoijan
työskentelyä. Lähdedokumentti (''koodi'') on muodoltaan täysin erilainen
kuin lopullinen tuotos. Kirjoittaminen vaatii omanlaisensa kielen
osaamista. Lopulliseen toteutukseen tarvitaan yleensä eri tekijöiden
tuottamia makropaketteja (vrt. ohjelmakirjastot), ja niiden ohjekirjojen
lukeminen on osa normaalia työskentelyä. Lähdedokumentit käännetään
erillisellä kääntäjäohjelmalla lopulliseksi tuotokseksi, eikä käännettyä
tuotosta voi enää palauttaa alkuperäiseksi lähdedokumentiksi.

Latexin käyttö ei silti ole mitään ohjelmointia. Merkintäkieli on eri
asia kuin ohjelmointikieli. Työskentelyn luonteessa on kuitenkin useita
yhtäläisyyksiä, ja 

\chapter{Valmistautuminen}

\section{Käsitteet ja nimet}

Latex ja sen ympärille rakentuneet ohjelmistot ovat käsitteellisesti
aika monimutkainen kokonaisuus, johon kuuluu eri\-/ikäisiä ja
abst\-rak\-tio\-ta\-sol\-taan erilaisia osia. Mukaan kuuluu tietenkin
konkreettia tie\-to\-kone\-oh\-jel\-mia, jotka tekevät konkreettisia
asoita. Mukaan kuuluu kuitenkin myös ihmisten luomia abst\-rak\-te\-ja
käsitteitä kuten Latex\-/formaatti, Tex\-/ohjelmointikieli tai muu
yleinen idea, jonka tieto\-kone\-ohjelmat pyrkivät konk\-reet\-ti\-sesti
toteuttamaan.

Internetissä näkyy silloin tällöin käsitekeskusteluja, jossa
ihmetellään, mihin mikäkin palikka kuuluu käsitteellisesti. Missä
suhteessa jotkin uudemmat osat ovat vanhempiin? Oikeiden termien
osaamisesta voi olla hyötyä, kun pyytää verkossa suurelta yleisöltä
apua. Viestintä vaatii, että puhutaan suunnilleen samaa kieltä. Yritän
seuraavaksi selventää peruskäsitteitä.

\subsection{Tex ja Latex}

Tex on tietokone\-ohjelma (\koodi{tex}), joka osaa lukea tietynmuotoisia
tekstitiedostoja ja latoa niiden perusteella tekstidokumentin, joka on
tarkoitettu ihmisten luettavaksi. Tex on myös tekstin ladontaan
tarkoitettu ohjelmointikieli. Se noudattaa sille annettuja ohjeita ja
latoo kirjaimia ja muuta tavaraa peräkkäin sivulle. Ihmisten
näkökulmasta Tex on hyvin tekninen ja matalatasoinen järjestelmä, eikä
sellaisten kanssa yleensä haluta olla missään tekemisissä. Siirtykäämme
siis eteenpäin.

Latex toimii korkeammalla abstraktiotasolla kuin Tex. Se on kokoelma
Tex\-/ohjelmointikielen makroja, jotka piilottavat monimutkaiset
tekniset yksityiskohdat ja toteuttavat varsin helppokäyttöisen
merkintäkielen, jolla dokumenttien rakenne ja ulko\-asua kuvataan.
Ihmiset siis kirjoittavat yleensä Latex\-/muotoisia dokumentteja, ja
Latex puolestaan huolehtii matalamman tason Texin käskyttämisestä. Latex
on myös tietokone\-ohjelma (\koodi{latex}), jolla lähdetiedoston voi
kääntää julkaistavaksi dokumentiksi.

\subsection{Xelatex ja Lualatex}

Ajan myötä mukaan on tullut sekalainen joukko muitakin ohjelmia, joista
kannattaa tässä yhteydessä mainita kaksi: Xelatex ja Lualatex. Ne ovat
erilaisia Latexin toteutuksia ja tietokone\-ohjelmia (\koodi{xelatex,
  lualatex}), joilla lähdedokumentti käännetään.\footnote{Englannin
  kielellä näitä on tapana kutsua yleisnimellä
  \emph{\textenglish{engine}} 'kone, moottori'.} Ne osaavat lukea
Unicode\-/merkistöllä kirjoitettuja lähdedokumentteja ja käyttää
nyky\-aikaisia True Type- ja Open Type \=/fontteja, mitä alkuperäinen
Latex ja Tex eivät osaa. Nyky\-aikana käytetään yleensä jompaakumpaa,
Xelatexia tai Lualatexia, ja esimerkiksi tämän oppaan lähdetiedosto on
käännetty PDF\-/tiedostoksi Xelatexilla.

Xelatexilla ja Lualatexilla ei ole käyttäjän kannalta suurtakaan eroa.
Jälkimmäinen sisältää Lua\-/nimisen ohjelmointikielen, ja sillä on
merkitystä joillekuille makropakettien tekijöille. Jotkin makropaketit
eivät toimi Lualatexissa ja jotkin eivät toimi Xelatexissa. Xelatex
taitaa olla hieman paremmin tuettu, koska se on vanhempi eli ehtinyt
vakiinnuttaa asemansa paremmin.

Kääntäjäohjelmien toiminnassa on pieniä eroja. Vaikka Latex\-/dokumentit
yleensä kääntyvätkin molemmilla, saattaa joskus valmiissa PDF:ssä näkyä
pieniä eroja, kun kääntäjää vaihtaa. Toisaalta on ihan hyödyllistä
kokeilla kääntää omat dokumentit kummallakin ohjelmalla, koska se voi
paljastaa huonoja, epäyhteensopivia käytäntöjä omassa Latex\-/koodissa.
En kuitenkaan suosittele vaihtamaan kääntäjää viime hetkellä ennen
tärkeän tekstin julkaisua.

\subsection{Latex yläkäsitteenä}

Jotta kaikki olisi mahdollisimman sekavaa, sana Latex toimii myös
yleisenä nimityksenä tälle kaikelle. Se esiintyy ilmaisuissa kuten
''Toteutin dokumentin Latexilla'' tai ''Tämä artikkeli on tehty
Latexilla''. Ilmaukset sitten tarkoittavat suunnilleen seuraavanlaista:
Henkilöllä on asennettuna tietokoneelle Latex\-/jakelu (kuten Tex Live).
Hän on kirjoittanut teksti\-editorilla (kuten GNU Emacsilla)
tekstitiedoston, jossa on dokumentin sisältö ja Latex\-/makroille
tarkoitettuja komentoja mutta myös joitakin Tex\-/komentoja. Sitten hän
on kääntänyt eli ladotuttanut tekstitiedostonsa PDF\-/tiedostoksi
Latex-la\-don\-ta\-oh\-jel\-man jollakin toteutuksella (kuten
Xelatexilla).

Meille taitaa riittää vain Latexista puhuminen, mutta siitäkin on
mainittava vielä yksi asia. Latexin harrastajat tykkäävät käyttää
dokumenttiensa leipätekstissä logoja kuten \TeX{} ja \LaTeX{}. Usein
teksteissä näkyy myös logojen pohjalta mukailtuja kirjoitus\-asuja TeX
ja LaTeX. Mielestäni logojen eikä muodikkaiden kIRjoiTus\-AsuJen paikka
ei ole leipätekstissä. Tässä oppaassa viittaan kaikkiin nimiin
kielenhuollon normien mukaisesti eli käytän tavallisia erisnimiä kuten
Tex ja Latex. Koodi ja komennot ovat siinä muodossa kuin ne
tietokoneelle annetaan, esimerkiksi \koodi{xelatex}.

\section{Asentaminen}
\label{luku:asentaminen}

Latex pitää tietysti asentaa tietokoneelle, jotta sitä voisi käyttää.
Miten edellisessä luvussa kuvattu sekava kokonaisuus saadaan ehjänä
omalle tietokoneelle? Onneksi muut ovat jo ratkaisseet sen ongelman aika
pitkälle.

Tavallisin tapa Latexin käyttöön\-ottoon on jonkin Latexin jakelupaketin
asentaminen. Jakelupaketti sisältää perus\-osien lisäksi paljon
makropaketteja ja niiden ohjekirjoja. Kaikkea ei koskaan tarvitse, mutta
kun yllättävä tarve tulee tai lukee vinkkejä verkkokeskusteluista, on
mukavaa huomata, että makropaketti olikin itsellä jo valmiina. Siksi
suosittelen kokonaisen jakelupaketin asentamista.

GNU/Linuxissa ja muissa Unix\-/tyyppisissä käyttöjärjestelmissä
käytetään yleensä Tex Live \=/nimistä jakelua. Se on todennäköisesti
saatavilla käyttöjärjestelmäjakelun pakettivarastoista. Esimerkiksi
Debianiin ja sen kaltaisiin järjestelmiin on asennuspaketti
\koodi{texlive-full}, joka asentaa kaiken helposti ja kerralla.

Windows\-/käyttöjärjestelmälle on saatavilla Tex Liven lisäksi Miktex ja
Protext. Mac OS \=/käyttöjärjestelmän kanssa käytettäneen yleensä
Mactex\-/nimistä jakelua.

\section{Ensimmäinen dokumentti}

Olemme varmasti jo puhuneet tarpeeksi, ja olisi hyvä päästä tekemään
jotain käytännöllistä. Tallenna esimerkin \ref{esim:ensimmainen} sisältö
teksti\-editorin avulla tiedostoon vaikkapa nimellä \koodi{teksti.tex}.
Käännä eli lado se PDF\-/tiedostoksi komennolla \koodi{xelatex
  teksti.tex} tai komennolla \koodi{lualatex teksti.tex}. Tuloksena
pitäisi olla tiedosto \koodi{teksti.pdf}, jota voi ihailla jollakin
PDF\-/tiedostojen katseluun tarkoitetulla ohjelmalla. Näin tämä
Latex\-/homma toimii.

Tutkitaan esimerkkiä \ref{esim:ensimmainen} tarkemmin. Ensimmäisellä
rivillä määritellään dokumenttiluokka \textenglish{\koodi{article}},
joka on tietynlainen sivupohja tai asetusten kokoelma, jonka perustalle
aletaan rakentaa omaa sivua. Luokka \textenglish{\koodi{article}} on
tyypillinen valinta lyhyehköille teksteille. Lisätietoa
dokumenttiluokista on luvussa \ref{luku:dokumenttiluokat}.

Toisella ja kolmannella rivillä käytetään komentoa \koodi{\keno
  usepackage} ja niiden avulla otetaan käyttöön fontti\-asetuksia
hoitava Fontspec\-/paketti ja kieli\-asetuksista vastaava
Polyglossia\-/paketti. Kumpaakin tarvitaan melkein joka kerta
dokumenteissa, ja niihin palataan tarkemmin luvuissa
\ref{luku:kirjaintyypit} ja \ref{luku:kieliasetukset}. Sivun asetuksia
käsitellään luvussa \ref{luku:sivuasetukset}.

Seuraavilla riveillä asetetaan kieleksi suomi (\koodi{finnish}) ja
määritetään oletuksena käytettävä fontti (kirjainperhe). Latin Modern
Roman \=/fontin tilalle voi toki kokeilla muitakin. Fontin oletuskoko on
10 pistettä, mutta tässä esimerkissä se venytetään 1,3\-/kertaiseksi eli
13 pisteeseen. Riviväliin liittyvä kerroin asetetaan rivillä 7.

\begin{esimerkki}
\begin{koodilohko}
  \documentclass{article}
  \usepackage{fontspec}
  \usepackage{polyglossia}

  \setdefaultlanguage{finnish}
  \setmainfont{Latin Modern Roman}[Scale=1.3]
  \linespread{1.4}

  \begin{document}

  Minun Latex-dokumenttini!

  \end{document}
\end{koodilohko}
\caption{Ensimmäinen Latex-dokumentti}
\label{esim:ensimmainen}
\end{esimerkki}

Dokumentin alku\-osaa esimerkin riville 8 saakka kutsutaan johdannoksi
tai esittelyksi (engl. \textenglish{preamble}). Tässä osassa ladataan
tarvittavat makropaketit ja määritetään dokumentin asetuksia ja
taustatietoja. Riviltä 9 alkaa varsinainen teksti\-osa eli sivulle
ladottava sisältö. Se osa kirjoitetaan \koodi{document}\-/ympäristön
sisään eli riveillä 9 ja 13 olevien ympäristön aloitus\-/{} ja
lopetuskomentojen väliin.

Tällaisen merkintäkielen ja rakenteen avulla dokumentit kirjoitetaan.
Osa merkintäkielen komennoista tulee Latexin perus\-osasta ja osa tulee
erikseen ladattavista makropaketeista (Fontspec, Polyglossia jne.).
Komentoja voi luoda itsekin.

\section{Apuohjelmia}

\subsection{Tekstieditori}

Hanki kunnollinen teksti\-editori. Teknisesti ainoa vaatimus on se, että
editori osaa tallentaa UTF\=/8\-/muotoista tekstidataa, tai vähän
kikkailemalla pelkkä ASCII\=/merkistökin riittäisi. Käytännössä
editorissa olisi hyvä olla muitakin ominaisuuksia, ja niin sanotut
ohjelmoijien tai tehokäyttäjien teksti\-editorit ovat paras valinta.

TÄNNE VIELÄ TEKSTIN SYNTAKTISESTA VÄRITTÄMISESTÄ JA EDITORIN HIENOISTA
OMINAISUUKSISTA.

\subsection{Texdoc}

Latexin kirjoittajan täytyy silloin tällöin lukea ohjekirjoja. Vaikka
Latexin perus\-osat joskus oppisikin ulkoa, ei voi koskaan muistaa
kaikkien hyödyllisten makropakettien kaikkia ominaisuuksia. Myös uusia
makropaketteja ja ihmisten suosituksia tulee vastaan esimerkiksi
verkkokeskusteluissa.

Tex Live \=/jakelun (luku \ref{luku:asentaminen}) mukana tulee mainio
komentotulkissa toimiva komento \koodi{texdoc}, jolla voi hakea ja avata
omaan järjestelmään asennettuja Latex\-/aiheisia ohjeita. Jos vaikka
haluaa tutustua esimerkissä \ref{esim:ensimmainen} mainittuun
Fontspec\-/pakettiin syvällisemmin, tarvitsee vain komentaa
\koodi{texdoc fontspec}, ja paketin PDF\-/muotoinen ohjekirja avautuu.

\subsection{Latexmk}

Hyödyllinen ohjelma on myös Latexmk, joka helpottaa dokumenttien
kääntämistä. Nimittäin varsin usein Latex\-/dokumentit täytyy kääntää
useita kertoja ennen kuin PDF\-/tiedosto on valmis. Se johtuu siitä,
että dokumentit sisältävät usein ristiviitteitä eli viittauksia
dokumentin toisiin osiin. Latex ei saa viitteitä kohdalleen yhdellä
kääntämisellä, vaan ensin se kirjoittaa viittausten kohteet muistiin
väli\-aikais\-tiedostoon ja seuraavilla kääntökerroilla käyttää
väli\-aikais\-tiedostoa apunaan.

Kääntäjä huomauttaa tietokoneen käyttäjää, kun uusintakäännös on
tarpeen, mutta Latexmk\-/ohjelma käynnistää uusintakäännöksen ihan itse,
aina kun se on tarpeellista.

Alla on kääntämiseen esimerkkikomentoja. Ensimmäinen kääntää dokumentin
Xelatexilla ja jälkimmäinen Lualatexilla.

\begin{koodilohkosis}
  latexmk -xelatex  teksti.tex
  latexmk -lualatex teksti.tex
\end{koodilohkosis}

Seuraavista esimerkeistä ensimmäinen komento poistaa kääntämisen aikana
luodut väli\-aikaistiedostot (\koodi{log, aux, out} ym.), ja
jälkimmäinen rivin komento poistaa kaikki luodut tiedostot eli
väli\-aikais\-tiedostojen lisäksi myös valmiin PDF\-/tiedoston.

\begin{koodilohkosis}
  latexmk -c teksti.tex
  latexmk -C teksti.tex
\end{koodilohkosis}

Edellisissä esimerkeissä käsitellään lähdetiedostoa nimeltä
\koodi{teksti.tex}, mutta jos lähdetiedostoa ei anna komennolle
lainkaan, käännetään kaikki nykyisessä hakemistossa olevat
\koodi{tex}\-/päätteiset tiedostot.

Latexmk-ohjelmalle voi tehdä asetustiedoston, johon voi kirjoittaa omaan
käyttöön sopivat asetukset. Asetustiedosto sijoitetaan käyttäjän
kotihakemistoon nimellä \koodi{.latexmkrc}. Alla on esimerkki, mitä se
voisi ehkä sisältää.

\begin{koodilohkosis}
  $pdf_mode = 5; # 5=xelatex, 4=lualatex
  $xelatex = 'xelatex -interaction=nonstopmode %O %S';
  $lualatex = 'lualatex -interaction=nonstopmode %O %S';
  $clean_ext = 'snm nav xdv';
\end{koodilohkosis}

Ensimmäisen rivin asetus määrittää, mitä kääntäjää käytetään oletuksena.
Toisella ja kolmannella rivillä määritellään, millä tavoin Xelatex ja
Lualatex suoritetaan. Tässä esimerkissä oletus\-asetuksiin on lisätty
\koodi{non\-stop\-mode}, joka estää kaiken vuorovaikutteisen toiminnan.
Asetus on tarpeen ainakin silloin, kun kääntäjä käynnistetään toisesta
ohjelmasta kuten teks\-ti\-edi\-to\-ris\-ta eikä vuorovaikutus kääntäjän
kanssa ole mahdollista.

Neljännellä rivillä luetellaan kääntämisen aikana syntyvien
väli\-aikais\-tiedostojen päätteitä. Yleiset väli\-aikais\-tiedostot
(\koodi{log, aux, out} ym.) on \koodi{latexmk}\-/ohjelmalla jo tiedossa,
mutta tällä asetuksella mukaan voi lisätä muitakin.


\chapter{Merkintäkieli}
\section{Erikoismerkit}
\section{Komennot}
\section{Ympäristöt}
\section{Mitat}
\section{Laskurit}

\chapter{Dokumentin asetukset}
\section{Dokumenttiluokat}
\label{luku:dokumenttiluokat}
\section{Fontit}
\label{luku:kirjaintyypit}

Fontit ja niiden asettaminen on Latexissa melko monimutkainen
kokonaisuus, koska fonteilla on paljon ominaisuuksia ja niihin
vaikutetaan monilla eri asetuksilla ja abstraktiotasoilla. Aika monta
asiaa pitää ymmärtää, jotta voi tehokkaasti työskennellä fonttien
kanssa.

Fontti jo itsessään on moniselitteinen käsite, joka vaatii
typo\-gra\-fias\-sa usein täsmentäviä ilmauksia. Sana \emph{fontti} voi
tarkoittaa kokonaista kirjainperhettä, eli muutaman yhteensopivan
kirjainleikkauksen muodostamaa kokonaisuutta. Samaan kirjainperheeseen
kuuluu yleensä neljä eri leikkausta: tavallinen, kursiivi, lihavoitu ja
lihavoitu kursiivi. Joihinkin perheisiin kuuluu leikkauksia paljon
enemmänkin, kuten useita eri vahvuuksia. Joissakin puheissa sana
\emph{fontti} tarkoittaa vain yhtä kirjainleikkausta, ja silloin koko
perheeseen viitataan sanalla fonttiperhe.

Tässä oppaassa käytän \emph{fontti}\-/sanaa yleisnimityksenä Latexin
kirjaimiin liittyville asetuksille. Se tarkoittaa kirjainperhettä tai
siihen kuuluvaa yksittäistä leikkausta. Silloin kun merkitystä pitää
täsmentää, käytän suomenkielisiä nimiä kirjainperhe ja kirjainleikkaus.
Sen sijaan sanan \emph{kirjasin} jätän kokonaan pois, koska se
tarkoittaa vanhassa metalliladonnassa ja mekaanisissa kirjoituskoneissa
metallisen kirjakkeen päähän muotoiltua kirjaimen kohokuviota, joka
painaa mustejäljen paperille.

Kuten Latexissa yleensäkin myös fonttien kanssa kannattaa käyttää
korkean abstraktiotason komentoja, jotka piilottavat yksityiskohdat ja
teknisen toteutuksen. Latexin fonttitoiminnot on suunniteltu juuri
siihen: matalan tason fontti\-asetukset määritellään mieluiten vain
kerran dokumentin esittely\-osassa, ja dokumentin teksti\-osassa
käytetään pelkästään korkean tason komentoja.

\subsection{Fontin valinta}

Latexin fonttien perus\-toiminnot rakentuvat kolmen erityyppisen
kirjainperheen varaan: antiikva eli pääteviivallinen
(\textenglish{serif, roman}), groteski eli pääteviivaton
(\textenglish{sans serif}) ja tasalevyinen kirjoituskoneen kaltainen
perhe (\textenglish{typewriter, monospace}). Kuvassa
\ref{kuva:kirjainperhetyypit} on tässä oppaassa käytetyt kolme eri
kirjainperhetyyppiä. Leipätekstissä käytetään pääteviivallista,
otsikoissa pääteviivatonta ja koodi\-esimerkeissä tasalevyistä.

\leijukuva{
  {\rmfamily\addfontfeatures{Scale=7}Amf}
  \hfill
  {\sffamily\addfontfeatures{Scale=7}Amf}
  \hfill
  {\ttfamily\addfontfeatures{Scale=7}Amf}
}{
  \caption{Vasemmalla pääteviivallinen, keskellä pääteviivaton ja
    oikealla tasalevyinen pääteviivallinen kirjainperhe}
  \label{kuva:kirjainperhetyypit}
}

Kirjoituskoneen kaltainen tasalevyinen kirjainperhe on tässä tapauksessa
tyypiltään pääteviivallinen, mutta se voisi olla muutakin. Tasalevyisyys
on sen kirjainperheen tärkein määrittävä tekijä Latexin asetusten
näkökulmasta.

Kirjainperheet otetaan käyttöön Fontspec\-/paketin komennoilla seuraavan
esimerkin mukaisesti.

\begin{koodilohkosis}
  \usepackage{fontspec}
  \setmainfont{Linux Libertine O}[Scale=1]
  \setsansfont{Linux Biolinum O}[Scale=MatchLowercase]
  \setmonofont{Linux Libertine Mono O}[Scale=MatchLowercase]
\end{koodilohkosis}

Samalla voi määritellä lukuisia kirjainperheeseen sisältyviä asetuksia
kuten ligatuureja ja optisia kokoja. Tässä esimerkissä käytetään vain
\koodi{Scale}\-/valitsinta, jolla fontin voi skaalata haluttuun kokoon.
\koodi{Scale}\-/kerroin on desimaaliluku, ja oletus\-arvo on 1.

Esimerkissä peruskirjainperheen (\koodi{\keno setmainfont}) skaalaus on
1, eli sille ei tehdä mitään, ja koko valitsimen voisi jättää pois. Sen
sijaan kahdella muulla kirjainperheellä (\koodi{\keno setsansfont, \keno
  setmonofont}) käytetään ker\-roin\-ase\-tus\-ta
\koodi{MatchLowercase}, joka skaalaa fontin siten, että gemenakirjaimet
eli pienet kirjaimet ovat yhtä korkeita kuin peruskirjainperheessä.

Jos edellä kuvatut kolme kirjainperhettä eivät riitä, on
Fontspec\-/paketissa komennot lisäperheiden ja \=/leikkausten
määrittämiseen. Uusi perhe määritellään seuraavasti:

\begin{koodilohkosis}
  \newfontfamily{\hienoperhe}{TeX Gyre Schola}[...]
\end{koodilohkosis}

Komento \koodi{\keno newfontfamily} toimii samalla tavalla kuin aiemmin
esitellyt \koodi{\keno setmainfont} ym. komennot, mutta lisäksi
ensimmäinen parametri määrittää komennon, jolla kirjainperhe otetaan
käyttöön. Edellisessä esimerkissä luodaan komento \koodi{\keno
  hieno\-perhe}, joka kytkee päälle TeX Gyre Scho\-la \=/nimisen
kirjainperheen.
          
Jos ei tarvita kokonaista perhettä vaan yksi leikkaus riittää, käytetään
komentoa \koodi{\keno newfontface}:

\begin{koodilohkosis}
  \newfontface{\hienoleikkaus}{TeX Gyre Schola Bold}[...]
\end{koodilohkosis}

Edellä määriteltävä komento \koodi{\keno hieno\-leikkaus} ottaa käyttöön
lihavoidun (bold) kirjainleikkauksen perheestä TeX Gyre Scho\-la. Tässä
vaih\-to\-eh\-dos\-sa on se etu, että tietokoneen muistiin ladataan vain
yksi leikkaus, ei koko perhettä.

\subsection{Fontin koko ja rivikorkeus}

Fonttien koot on totuttu valitsemaan typo\-grafisen pistemitan avulla.
Esimerkiksi 10--12 pistettä on tyypillinen leipätekstin oletuskoko
teks\-tin\-kä\-sit\-tely\-ohjel\-mis\-sa. Piste on mitta\-yksikkö, jonka
pituus on 1/72 tuumaa eli 0,3528 millimetriä. Kirjainleikkauksen
pistekoko mitataan kirjaimiston ylimmän ja alimman kohdan välillä, ja
siihen luetaan mukaan ylä- ja alapuolella oleva pieni tyhjä tila, jonka
fontin suunnittelija on määritellyt.

Myös Latexissa koot voi määritellä pistemittojen (lyhenne pt) avulla,
mutta halutessaan ne voi unohtaa lähes kokonaan ja käyttää niin sanottua
suhteellista tapaa koko\-asetuksiin. Käsittelen suhteellisia ja
absoluuttisia koko\-ase\-tuk\-sia luvuissa
\ref{luku:fontti_suhteellinen} ja \ref{luku:fontti_absoluuttinen}.

Matalalla tasolla fonttien kokoon vaikuttaa Latexissa eräs yllättävä
asia. Nimittäin dokumenttiluokalle (luku \ref{luku:dokumenttiluokat})
voi antaa valitsimen, jolla koko asetetaan. Vaihto\-ehtoja on Latexin
normaaleissa dokumenttiluokissa vain kolme: \koodi{10pt} (oletus),
\koodi{11pt}, ja \koodi{12pt}. Dokumenttiluokan koko\-asetus vaikuttaa
myös sivun marginaaleihin, koska Latex pyrkii pitämään rivin
merkkimäärän lukijalle sopivana: yhdelle riville ei kannata latoa ihan
mahdottomasti merkkejä, koska luettavuus heikkenee.

Fontin koon määrittäminen dokumenttiluokan valitsimella kuuluu jo vähän
menneisyyteen, mutta voi sitä edelleen käyttää, jos se riittää ja sillä
saa halutun lopputuloksen. Yleensä kyllä jättäisin dokumenttiluokan
fontti\-asetuksen oletukseksi (\koodi{10pt}) ja käyttäisin koon
asettamiseen luvussa \ref{luku:fontti_suhteellinen} tai
\ref{luku:fontti_absoluuttinen} kerrottuja tapoja. Sivun marginaalien ja
muiden mittojen määrittämiseen on ohjeita luvussa
\ref{luku:sivuasetukset}.

Fontti\-asetuksiin kuuluu fontin koon lisäksi toinenkin mitta:
rivikorkeus (\koodi{\keno baselineskip}). Se on mitta rivin
peruslinjalta seuraavan rivin peruslinjalle. Fontin koko ja rivikorkeus
määritellään saman\-aikaisesti, koska ne ovat saman \koodi{\keno
  fontsize}\-/komennon parametreja. Esimerkki:

\begin{koodilohkosis}
  \fontsize{10pt}{12pt} \selectfont
\end{koodilohkosis}

Ensimmäinen parametri on fontin kokomitta ja toinen on rivikorkeus.
Mitta\-yksiköt voivat olla mitä tahansa Latexin mittoja, ja oletuksena
käytetään pistemittaa (pt), jos yksikköä ei ole mainittu. Komento
\koodi{\keno selectfont} on mukana, koska vasta sen myötä matalan tason
fonttikomennot tulevat voimaan. Korkean tason fonttikomennot (luku
\ref{luku:fontit_korkea}) suorittavat sen automaattisesti.

Rivikorkeus on vähintään sama kuin fontin koko, mutta yleensä se
asetetaan hieman suuremmaksi, jotta rivit eivät olisi liian lähellä
toisiaan. Esimerkissä \ref{esim:rivikorkeus} on kaksi erilaista
\koodi{\keno fontsize}\-/komentoa ja ladottu lopputulos.

\begin{esimerkki}
\begin{koodilohko}
  \fontsize{10pt}{12pt}\selectfont Tässä on pienehkö leipätekstin
  fonttikoko ja sitä hieman suurempi rivikorkeus. Rivikorkeus on
  liian pieni näin pitkille riveille.

  \fontsize{16pt}{25pt}\selectfont Tässä on melko suuri fontti ja
  reilu rivikorkeus. Rivit eivät tunnu kuuluvan enää yhteen.
\end{koodilohko}
\centering%
\parbox{.9\textwidth}{%
  \linespread{1}\addfontfeatures{Scale=1}
  \fontsize{10pt}{12pt}\selectfont Tässä on pienehkö leipätekstin
  fonttikoko ja sitä hieman suurempi rivikorkeus. Rivikorkeus on liian
  pieni näin pitkille riveille.

  \fontsize{16pt}{25pt}\selectfont Tässä on melko suuri fontti ja reilu
  rivikorkeus. Rivit eivät tunnu kuuluvan enää yhteen. } \vspace{3ex}
\caption{Fontin koon ja rivikorkeuden asettaminen ja vaikutus}
\label{esim:rivikorkeus}
\end{esimerkki}

Toinen tekstirivien peruslinjojen väliseen etäisyyteen vaikuttava asetus
on \koodi{\keno baselinestretch}. Se on desimaalilukukerroin, jolla
nykyinen rivikorkeus kerrotaan. Kerroin asetetaan helpoimmin komennolla
\koodi{\keno linespread}.\footnote{Toinen tapa: \koodi{\keno
    renewcommand\{\keno baselinestretch\}\{kerroin\}}}

\begin{koodilohkosis}
  \fontsize{10pt}{12pt} \linespread{1.3} \selectfont
\end{koodilohkosis}

Edellä oleva esimerkki asettaa fontin kooksi 10 pistettä ja
rivikorkeudeksi 12 pistettä. \koodi{\keno linespread}\-/komennolla
asetetun kertoimen vuoksi rivien peruslinjojen väliseksi etäisyydeksi
tulee lopulta 1,3 kertaa 12 pistettä eli 15,6 pistettä. Ei ole väliä,
kummassa järjestyksessä \koodi{\keno fontsize}- ja \koodi{\keno
  linespread}\-/komennot annetaan. Asetukset tulevat voimaan vasta
\koodi{\keno selectfont}\-/komennon jälkeen.

Käytännössä \koodi{\keno linespread} sopii rivikorkeuden yleistason
hienosäätöön, esimerkiksi dokumentin esittely\-osassa. Sitä ei
kannattane kovin paljon muutella, koska se vaikuttaa kaikkialla. Sen
sijaan \koodi{\keno fontsize}\-/komennolla määritetään rivikorkeus
tietylle fonttikoolle ja tiettyyn tilanteeseen.

\subsection{Korkean tason komennot}
\label{luku:fontit_korkea}

Latexissa on joukko korkean tason fontti\-komentoja, jotka on
tarkoitettu käytettäväksi sen jälkeen, kun matalan tason asetukset on
kerran määritetty. Taulukossa \ref{tlk:fonttimallikomennot} on komennot
kirjainperheen ja kirjainleikkauksen valintaan.

Komennot fontin koon valintaan ovat taulukossa
\ref{tlk:fonttikokokomennot}. Taulukko kertoo myös, mitä
fontin pistekokoa mikäkin komento tarkoittaa oletuksena. Oletus riippuu
Latexin dokumenttiluokkien (luku \ref{luku:dokumenttiluokat})
fonttikokovalitsimista \koodi{10pt, 11pt} ja \koodi{12pt}.

Kaikille korkean tason fonttikomennoille on olemassa myös samanniminen
ympäristönsä. Alla olevassa esimerkissä on kaksi fontteihin vaikuttavaa
ympäristöä sisäkkäin.

\begin{koodilohkosis}
  \begin{LARGE}
    \begin{scshape}
      Tämä teksti on isoa (LARGE) kapiteelia (scshape).
      Typografisesti varmaan aika typerää...
    \end{scshape}
  \end{LARGE}
\end{koodilohkosis}

\leijutlk{
  \begin{tabular}{llll}
    \toprule
    \multicolumn{2}{l}{\ots{Komento}}
    & \multicolumn{2}{l}{\ots{Merkitys}} \\
    \midrule
    \koodi{\keno rmfamily} & \koodi{\keno textrm\{...\}}
    & {\rmfamily perhe} & pääteviivallinen, antiikva, serif, roman \\
    \koodi{\keno sffamily} & \koodi{\keno textsf\{...\}}
    & {\sffamily perhe} & pääteviivaton, groteski, sans serif \\
    \koodi{\keno ttfamily} & \koodi{\keno texttt\{...\}}
    & {\ttfamily perhe} & tasalevyinen, kirjoituskone, typewriter \\
    \midrule
    \koodi{\keno mdseries} & \koodi{\keno textmd\{...\}}
    & {\mdseries leikkaus} & lihavoimaton, tavallinen \\
    \koodi{\keno bfseries} & \koodi{\keno textbf\{...\}}
    & {\bfseries leikkaus} & lihavoitu, bold \\
    \midrule
    \koodi{\keno upshape} & \koodi{\keno textup\{...\}}
    & {\upshape leikkaus} & pystyasento, tavallinen \\
    \koodi{\keno slshape} & \koodi{\keno textsl\{...\}}
    & {\slshape leikkaus} & kallistettu, slanted, oblique \\
    \koodi{\keno itshape} & \koodi{\keno textit\{...\}}
    & {\itshape leikkaus} & kursiivi, italic \\
    \koodi{\keno scshape} & \koodi{\keno textsc\{...\}}
    & {\scshape leikkaus} & kapiteeli, small caps \\
    \bottomrule
  \end{tabular}
}{
  \caption{Komennot kirjainperheen ja kirjainleikkauksen valintaan}
  \label{tlk:fonttimallikomennot}
}

\leijutlk{
  \begin{tabular}{lr@{}lr@{}lr@{}l}
    \toprule
    \ots{Komento}
    & \multicolumn{2}{c}{\ots{10pt}}
    & \multicolumn{2}{c}{\ots{11pt}}
    & \multicolumn{2}{c}{\ots{12pt}} \\
    \midrule
    \koodi{\keno tiny} & 5 && 6 && 6 \\
    \koodi{\keno scriptsize} & 7 && 8 && 8 \\
    \koodi{\keno footnotesize} & 8 && 9 && 10 \\
    \koodi{\keno small} & 9 && 10 && 10&,95 \\
    \koodi{\keno normalsize} & 10 && 10&,95 & 12 \\
    \koodi{\keno large} & 12 && 12 && 14&,4 \\
    \koodi{\keno Large} & 14&,4 & 14&,4 & 17&,28 \\
    \koodi{\keno LARGE} & 17&,28 & 17&,28 & 20&,74 \\
    \koodi{\keno huge} & 20&,74 & 20&,74 & 24&,88 \\
    \koodi{\keno Huge} & 24&,88 & 24&,88 & 24&,88 \\
    \bottomrule
  \end{tabular}
}{
  \caption{Fonttien oletuspistekoot dokumenttiluokkien valitsimilla
    \koodi{10pt, 11pt} ja \koodi{12pt}}
  \label{tlk:fonttikokokomennot}
}

\subsection{Koot suhteellisesti}
\label{luku:fontti_suhteellinen}

Dokumentin fonttien koot on helpointa määrittää siten, että asettaa
ensin peruskirjainperheen koon ja antaa muiden fonttien määräytyä
suhteessa siihen. Esimerkki \ref{esim:fontti_suhteellinen} selventää,
kuinka se tapahtuu. Alussa otetaan käyttöön dokumenttiluokka
\koodi{article} ja annetaan sille valitsin \koodi{10pt}, joka määrittää
fonttikooksi 10 pistettä. Se on dokumenttiluokan ole\-tus\-ase\-tus,
jota ei tarvitsisi edes kirjoittaa näkyviin. Esimerkin toisella rivillä
otetaan Fontspec\-/paketti käyttöön.

Peruskirjainperheen (rivi~4) koko skaalataan 1,4\-/kertaiseksi, eli
pistekooksi tulee 1,4 kertaa 10 pistettä eli 14 pistettä.
Normaalikokoinen peruskirjainperhe on ainoa, jonka pistekoko tiedetään.
Kaikkien muiden koot täytyisi selvittää laskemalla.

Pääteviivaton kirjainperhe (rivi~5) ja tasalevyinen perhe (rivi~6)
skaalataan samankorkuiseksi kuin perusperhe. Vertailukohtana ovat pienet
kirjaimet (\koodi{MatchLowercase}). Näiden kahden kirjainperheen
pistekokoa ei tiedetä. Se ei välttämättä ole sama kuin perusfontissa,
koska fonttien pistekoko mitataan ylimmän ja alimman kohdan välillä ja
koska fonttien mittasuhteet ovat erilaisia.

\begin{esimerkki}
\begin{koodilohko}
  \documentclass[10pt]{article} % 10pt on oletus
  \usepackage{fontspec}

  \setmainfont{Linux Libertine O}[Scale=1.4]
  \setsansfont{Linux Biolinum O}[Scale=MatchLowercase]
  \setmonofont{Linux Libertine Mono O}[Scale=MatchLowercase]
  \linespread{1.45}
\end{koodilohko}    
\caption{Fonttikokojen määrittäminen suhteessa peruskirjainperheeseen}
\label{esim:fontti_suhteellinen}
\end{esimerkki}

Viimeisellä rivillä oleva \koodi{\keno linespread}\-/komento on tärkeä.
Se asettaa rivikorkeuden kertoimeksi 1,45. Kertoimen täytyy olla
vähintään yhtä suuri kuin peruskirjainperheen skaalauskerroin (1,4),
jotta rivivälit ovat riittävän suuret. Näiden asetusten jälkeen
dokumentissa käytetään korkeamman tason komentoja fonttien valintaan,
esimerkiksi fonttikoon valintakomentoja \koodi{\keno small, \keno
  normalsize, \keno large} (taulukko \ref{tlk:fonttikokokomennot}).

Edellä kuvatussa suhteellisessa kirjainperheiden koon määrittelyssä on
sellainen ongelma tai kummallisuus, että Latex koko ajan luulee, että
peruskirjainperhe on normaalikokoisena 10 pistettä. Latexin matalan
tason fonttikomennot eivät tiedä kirjainperheen skaalauskertoimesta, ja
siksi esimerkiksi komentojen

\begin{koodilohkosis}
  \fontsize{10pt}{12pt} \selectfont
\end{koodilohkosis}

tuloksena ei todellisuudessa ole 10 pisteen fontti, vaan mukaan
lasketaan myös kirjainperheen skaalauskerroin. Tämän vuoksi \koodi{\keno
  fontsize}\-/komennon käyttö menee aika oudoksi. Parametrina annetut
mitat eivät pidä paikkaansa.

Jos korkean tason font\-ti\-koko\-komen\-to\-jen (taulukko
\ref{tlk:fonttikokokomennot}) lisäksi tarvitaan jotakin muuta kokoa,
voisi ehkä \koodi{\keno fontsize}\-/komennon sijasta käyttää
Fontspec\-/paketin tarjoamaa komentoa ja tilanteeseen sopivaa
skaalauskerrointa esimerkiksi seuraavalla tavalla:

\begin{koodilohkosis}
  \addfontfeatures{Scale=3.2} Poikkeuksellisen isoa tekstiä
\end{koodilohkosis}

Jos edellä mainitut kummallisuudet eivät häiritse eikä ole tarvetta
määritellä fontteja tarkasti tietyn pistekoon mukaiseksi, on
suhteellinen määrittelytapa todella helppo. Kaikki dokumentin fontit
määräytyvät perusfontin skaalauskertoimen kautta. Tämä tapa sopii hyvin
varsinkin dokumentin sisällön kirjoittamisvaiheeseen, jossa halutaan
vain nopeasti asettaa dokumentti suurin piirtein järkevän näköiseksi.

\subsection{Koot absoluuttisesti}
\label{luku:fontti_absoluuttinen}

Absoluuttinen fonttien koonmääritystapa tarkoittaa sitä, että koot
asetetaan tietyn kokoiseksi käyttämällä esimerkiksi pistemittoja ja että
kirjaimet myös päätyvät lopulliseen dokumenttiin juuri sen kokoisena.
Tämä tapa on myös teknisesti eheä, eli Latexin eri osat ovat samaa
mieltä siitä, minkäkokoisesta fontista on kyse. Näin ei ollut
suhteellisen tavan kanssa (luku \ref{luku:fontti_suhteellinen}).

Joskus yrityksen, oppilaitoksen tai muun tahon julkaisu\-ohjeissa
määritellään tarkasti, mitä fontteja käytetään ja mikä on leipätekstin
ja otsikoiden fonttikoko. Silloin tarvitaan tässä luvussa kuvattua tapaa
fonttien asettamiseen.

\begin{esimerkki}
\begin{koodilohko}
  \documentclass{article}
  \usepackage{fontspec}

  % Leipätekstiin samankokoiset fontit
  \setmainfont{Linux Libertine O}
  \setsansfont{Linux Biolinum O}[Scale=MatchLowercase]
  \setmonofont{Linux Libertine Mono O}[Scale=MatchLowercase]
  
  % Muualle sans ja mono ilman skaalausta
  \newfontfamily{\sffamilyabs}{Linux Biolinum O}
  \newfontfamily{\ttfamilyabs}{Linux Libertine Mono O}

  \linespread{1} % ei välttämättä tarvita

  % Kaikki tarvittavat fonttikoot ja komennot
  \renewcommand{\footnotesize}{\fontsize{10pt}{12pt}\selectfont}
  \renewcommand{\small}       {\fontsize{12pt}{14pt}\selectfont}
  \renewcommand{\normalsize}  {\fontsize{14pt}{17pt}\selectfont}
  \renewcommand{\large}       {\fontsize{17pt}{19pt}\selectfont}
  \renewcommand{\Large}       {\fontsize{20pt}{22pt}\selectfont}
  \normalsize % jotta tulee heti voimaan eikä vasta tekstiosassa
\end{koodilohko}    
\caption{Fonttikokojen määrittäminen pistekoon avulla}
\label{esim:fontti_absoluuttinen}
\end{esimerkki}

Esimerkistä \ref{esim:fontti_absoluuttinen} selviää perus\-ajatus.
Peruskirjainperhe (rivi~5) otetaan käyttöön ilman skaalausta (eli
\koodi{Scale=1}), minkä vuoksi koon voi jatkossa asettaa täsmälleen
kohdalleen \koodi{\keno fontsize}\-/komennolla. Samaa ei tehdä
pääteviivattoman eikä tasalevyisen fontin kanssa (rivit 6--7), vaan
käytetään skaalausta \koodi{MatchLowercase}, jotta tekstikappaleessa
kaikki kirjainperheet näyttävät samankokoisilta. Tässä menetetään
mahdollisuus määrittää näiden kirjainperheiden koko täsmällisesti
pistemitan avulla. Jos siihen on tarvetta esimerkiksi otsikoissa,
voidaan käyttää rivien 10--11 komentoja. Niillä luodaan uudet
kirjainperheet, jotka ovat käytännössä samoja mutta ilman skaalausta.

Uusien kirjainperheiden komentojen nimiksi on valittu \koodi{\keno
  sffamilyabs} ja \koodi{\keno ttfamilyabs} (vrt. \koodi{\keno sffamily}
ja \koodi{\keno ttfamily}), ja näillä komennoilla kirjainperheet
kytketään päälle. Jos esimerkiksi jonkin julkaisun vaatimuksiin kuuluu,
että otsikossa täytyy olla 16 pisteen lihavoitu Linux Biolinum~O
\=/kirjainleikkaus, voi esimerkissä \ref{esim:fontti_absoluuttinen}
olevien asetusten pohjalta antaa otsikolle seuraavat komennot:

\begin{koodilohkosis}
  \sffamilyabs\fontsize{16pt}{18pt}\bfseries
\end{koodilohkosis}

Esimerkin \ref{esim:fontti_absoluuttinen} riveillä 16--20 määritellään
uudelleen Latexin korkean tason komennot, joilla fonttikoot asetetaan.
Vähintäänkin täytyy määritellä komento \koodi{\keno normalsize} mutta
sen lisäksi kaikki ne, joita omassa dokumentissa tarvitaan. Tässä
esimerkissä normaali koko asetetaan 14 pisteen kokoiseksi ja muut koot
on ajateltu suhteessa siihen.

Jokaiselle fonttikoolle määritetään riveillä 16--20 myös oma
rivikorkeus, ja se on tarkoitus asettaa sopivaksi juuri kyseiselle
koolle. Rivikorkeuteen vaikuttaa myös kerroin \koodi{\keno
  baselinestretch}, joka asetetaan komennolla \koodi{\keno linespread}.
Sitä ei välttämättä tarvitse käyttää, koska kirjainperheitä ei ole
skaalattu ja koska rivikorkeus asetetaan aina \koodi{\keno
  fontsize}\-/komennolla. \koodi{\keno linespread} on kuitenkin kätevä
komento rivikorkeuden säätämiseen yleisesti kaikkialla.

Fonttikokojen määrittelyn lopuksi rivillä 21 suoritetaan komento
\koodi{\keno normalsize}, jotta se tulee heti voimaan. Dokumentin
esittely\-osassa voidaan käyttää fonttikokoon viittaavia mittoja
\koodi{em} ja \koodi{ex}, ja ne viittaavat nyt tähän kokoon.
\koodi{\keno normalsize}\-/komento suoritetaan kyllä myöhemmin
automaattisesti dokumentin teksti\-osan eli
\koodi{document}\-/ympäristön alussa.

Edellä kuvatun absoluuttisen koonmääritystavan etuna on se, että
kirjoittaja hallitsee fonttien kokoa ja rivikorkeuksia tarkasti ja että
julkaisuun saadaan juuri ne mitat, jotka halutaan tai vaaditaan. Tapa on
myös teknisesti eheä eli toimii Latexin sisäisen logiikan näkökulmasta
oikein. Haittana voi pitää sitä, että kaikki koot täytyy määritellä
erikseen (esimerkki \ref{esim:fontti_absoluuttinen}, rivit 16--20).

\section{Kieli}
\label{luku:kieliasetukset}

\subsection{Polyglossia}
\subsection{Babel}

\section{Sivu}
\label{luku:sivuasetukset}
\subsection{Marginaalit ja mitat}
\subsection{Ylä- ja alatunnisteet}



\chapter{Dokumentin rakenne}
\section{Kappale}
\subsection{Tasaus}
%\sloppy, \fussy
\subsection{Sisennykset ja välit}
\subsection{Riippuva sisennys}
\subsection{Lesket ja orvot}
\section{Korostus}
\section{Otsikot}
\section{Luetelmat}
\section{Taulukot}
\section{Kelluvat osat}
\section{Ristiviitteet}
\section{Alaviitteet}
\section{Tavutus}
\section{Lähdeluettelo ja -viitteet}
\section{Grafiikka}
\section{Laatikot}

\chapter{Matematiikka}

\chapter{Virittely}
\section{Päiväykset ja kellonajat}

\end{document}
