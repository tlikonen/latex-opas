% -*- TeX-master: "opas.tex"; -*-

\chapter{Valmistautuminen}

\section{Käsitteet ja nimet}

Se, mitä sanalla Latex yleensä tarkoitetaan, on monimutkainen
kokonaisuus erilaisia pieniä palikoita kuten tietokone\-ohjelmia,
makropaketteja ja asetustiedostoja. Yritän tässä luvussa selventää
joitakin perus\-asioita.

\subsection{Tex ja Latex}

Tex on tietokone\-ohjelma (\koodi{tex}), joka osaa lukea tietynmuotoisia
tekstitiedostoja ja latoa niiden perusteella tekstidokumentin, joka on
tarkoitettu ihmisten luettavaksi. Tex on myös tekstin ladontaan
tarkoitettu ohjelmointikieli. Se noudattaa sille annettuja ohjeita ja
latoo kirjaimia ja muuta tavaraa peräkkäin sivulle. Ihmisten
näkökulmasta Tex on hyvin tekninen ja matalatasoinen järjestelmä, eikä
sellaisten kanssa yleensä haluta olla missään tekemisissä. Siirtykäämme
siis eteenpäin.

Latex toimii korkeammalla abstraktiotasolla kuin Tex. Se on kokoelma
Tex\-/ohjelmointikielen makroja, jotka piilottavat monimutkaiset
tekniset yksityiskohdat ja toteuttavat varsin helppokäyttöisen
merkintäkielen, jolla dokumenttien rakenne ja ulko\-asua kuvataan.
Ihmiset siis kirjoittavat yleensä Latex\-/muotoisia dokumentteja, ja
Latex puolestaan huolehtii matalamman tason Texin käskyttämisestä. Latex
on myös tietokone\-ohjelma (\koodi{latex}), jolla lähdetiedoston voi
kääntää julkaistavaksi dokumentiksi.

\subsection{Xelatex ja Lualatex}

Ajan myötä mukaan on tullut sekalainen joukko muitakin ohjelmia, joista
kannattaa tässä yhteydessä mainita kaksi: Xelatex ja Lualatex. Ne ovat
erilaisia Latexin toteutuksia ja tietokone\-ohjelmia (\koodi{xelatex,
  lualatex}), joilla lähdedokumentti käännetään. Nyky\-aikana käytetään
näistä yleensä jompaakumpaa, ja esimerkiksi tämän oppaan lähdetiedosto
on käännetty PDF\-/tiedostoksi Xelatexilla.

Xelatexilla ja Lualatexilla ei ole käyttäjän kannalta suurtakaan eroa.
Jälkimmäinen sisältää Lua\-/nimisen ohjelmointikielen, ja sillä on
merkitystä joillekin makropakettien tekijöille. Jotkin makropaketit
eivät toimi Lualatexissa ja jotkin eivät toimi Xelatexissa. Xelatex
taitaa olla hieman paremmin tuettu, koska se on vanhempi eli ehtinyt
vakiinnuttaa asemansa paremmin.

Kääntäjäohjelmien toiminnassa on pieniä eroja. Vaikka Latex\-/dokumentit
yleensä kääntyvätkin molemmilla, saattaa valmiissa PDF:ssä näkyä pieniä
eroja, kun kääntäjää vaihtaa. Onkin ihan hyödyllistä kokeilla kääntää
omat dokumentit kummallakin ohjelmalla, koska se voi paljastaa huonoja,
epäyhteensopivia käytäntöjä omassa Latex\-/koodissa. Pienten erojen
vuoksi en kuitenkaan suosittele vaihtamaan kääntäjää viime hetkellä
ennen tärkeän tekstin julkaisua.

\subsection{Latex yläkäsitteenä}

Jotta kaikki olisi mahdollisimman sekavaa, sana Latex toimii myös
yleisenä nimityksenä tälle kaikelle. Se esiintyy ilmaisuissa kuten
''Toteutin dokumentin Latexilla'' tai ''Tämä artikkeli on tehty
Latexilla''. Ilmaukset sitten tarkoittavat suunnilleen seuraavanlaista:
Henkilöllä on asennettuna tietokoneelle Latex\-/jakelu (kuten Tex Live).
Hän on kirjoittanut teksti\-editorilla (kuten GNU Emacsilla)
tekstitiedoston, jossa on dokumentin sisältö ja Latex\-/makroille
tarkoitettuja komentoja. Sitten hän on kääntänyt eli ladotuttanut
tekstitiedostonsa PDF\-/tiedostoksi Latex-la\-don\-ta\-oh\-jel\-man
jollakin toteutuksella (kuten Xelatexilla).

Meille taitaa riittää vain Latexista puhuminen, mutta siitäkin on
mainittava vielä yksi asia. Latexin harrastajat tykkäävät käyttää
dokumenttiensa leipätekstissä logoja kuten \TeX{} ja \LaTeX{}. Usein
teksteissä näkyy myös logojen pohjalta mukailtuja kirjoitus\-asuja TeX
ja LaTeX. Mielestäni logojen eikä muodikkaiden kIRjoiTus\-AsuJen paikka
ei ole leipätekstissä, koska ne erottuvat tekstipalstasta liiaksi ja
tekevät siitä rauhattoman näköisen. Tässä oppaassa viittaan kaikkiin
nimiin kielenhuollon normien mukaisesti eli käytän tavallisia erisnimiä
kuten Tex ja Latex. Koodi ja komennot ovat siinä muodossa kuin ne
tietokoneelle annetaan, esimerkiksi \koodi{xelatex}.

\section{Asentaminen}
\label{luku:asentaminen}

Latex pitää tietysti asentaa tietokoneelle, jotta sitä voisi käyttää.
Miten edellisessä luvussa kuvattu sekava kokonaisuus saadaan ehjänä
omalle tietokoneelle? Onneksi muut ovat jo ratkaisseet sen ongelman aika
pitkälle.

Tavallisin tapa Latexin käyttöön\-ottoon on jonkin Latexin jakelupaketin
asentaminen. Jakelupaketti sisältää perus\-osien lisäksi paljon
makropaketteja ja niiden ohjekirjoja. Kaikkea ei koskaan tarvitse, mutta
kun yllättävä tarve tulee tai lukee vinkkejä verkkokeskusteluista, on
mukavaa huomata, että makropaketti olikin itsellä jo valmiina. Siksi
suosittelen kokonaisen jakelupaketin asentamista.

GNU/Linuxissa ja muissa Unix\-/tyyppisissä käyttöjärjestelmissä
käytetään yleensä Tex Live \=/nimistä jakelua. Se on todennäköisesti
saatavilla käyttöjärjestelmäjakelun pakettivarastoista. Esimerkiksi
Debianiin ja sen kaltaisiin järjestelmiin on asennuspaketti
\koodi{texlive-full}, joka asentaa kaiken helposti ja kerralla.

Windows\-/käyttöjärjestelmälle on saatavilla Tex Liven lisäksi Miktex ja
Protext. Mac OS \=/käyttöjärjestelmän kanssa käytettäneen yleensä
Mactex\-/nimistä jakelua.

\section{Ensimmäinen dokumentti}

Olemme varmasti jo puhuneet tarpeeksi, ja olisi hyvä päästä tekemään
jotain käytännöllistä. Tallenna esimerkin \ref{esim:ensimmainen} sisältö
teksti\-editorin avulla tiedostoon vaikkapa nimellä \koodi{teksti.tex}.
Käännä eli lado se PDF\-/tiedostoksi komennolla \koodi{xelatex
  teksti.tex}. Tuloksena pitäisi olla tiedosto \koodi{teksti.pdf}, jota
voi ihailla jollakin PDF\-/tiedostojen katseluun tarkoitetulla
ohjelmalla.

Tutkitaan esimerkkiä \ref{esim:ensimmainen} tarkemmin. Ensimmäisellä
rivillä määritellään dokumenttiluokka \textenglish{\koodi{article}},
joka on tietynlainen sivupohja tai asetusten kokoelma, jonka perustalle
aletaan rakentaa omaa sivua. Luokka \textenglish{\koodi{article}} on
tyypillinen valinta lyhyehköille teksteille. Lisätietoa
dokumenttiluokista on luvussa \ref{luku:dokumenttiluokat}.

Toisella ja kolmannella rivillä käytetään komentoa \koodi{\keno
  usepackage} ja niiden avulla otetaan käyttöön fontti\-asetuksia
hoitava Fontspec\-/paketti ja kieli\-asetuksista vastaava
Polyglossia\-/paketti. Kumpaakin tarvitaan melkein joka kerta
dokumenteissa, ja niihin palataan tarkemmin luvuissa
\ref{luku:kirjaintyypit} ja \ref{luku:kieliasetukset}. Sivun asetuksia
käsitellään luvussa \ref{luku:sivuasetukset}.

Seuraavilla riveillä asetetaan kieleksi suomi (\koodi{finnish}) ja
määritetään oletuksena käytettävä fontti (kirjainperhe). Latin Modern
Roman \=/fontin tilalle voi toki kokeilla muitakin. Fontin oletuskoko on
10 pistettä, mutta tässä esimerkissä se venytetään 1,3\-/kertaiseksi eli
13 pisteeseen. Riviväliin liittyvä kerroin asetetaan rivillä 7.

\begin{esimerkki}
\begin{koodilohko}
  \documentclass{article}
  \usepackage{fontspec}
  \usepackage{polyglossia}

  \setdefaultlanguage{finnish}
  \setmainfont{Latin Modern Roman}[Scale=1.3]
  \linespread{1.4}

  \begin{document}

  Minun Latex-dokumenttini!

  \end{document}
\end{koodilohko}
\caption{Ensimmäinen Latex-dokumentti}
\label{esim:ensimmainen}
\end{esimerkki}

Dokumentin alku\-osaa esimerkin riville 8 saakka kutsutaan johdannoksi
tai esittelyksi (engl. \textenglish{preamble}). Tässä osassa ladataan
tarvittavat makropaketit ja määritetään dokumentin asetuksia ja
taustatietoja. Riviltä 9 alkaa varsinainen teksti\-osa eli sivulle
ladottava sisältö. Se osa kirjoitetaan \koodi{document}\-/ympäristön
sisään eli riveillä 9 ja 13 olevien ympäristön aloitus\-/{} ja
lopetuskomentojen väliin.

Tällaisen merkintäkielen ja rakenteen avulla dokumentit kirjoitetaan.
Osa merkintäkielen komennoista tulee Latexin perus\-osasta ja osa tulee
erikseen ladattavista makropaketeista (Fontspec, Polyglossia jne.).
Komentoja voi luoda itsekin.

\section{Apuohjelmia}

\subsection{Tekstieditori}

Hanki kunnollinen teksti\-editori. Teknisesti ainoa vaatimus on se, että
editori osaa tallentaa UTF\=/8\-/muotoista tekstidataa, tai vähän
kikkailemalla pelkkä ASCII\=/merkistökin riittäisi. Käytännössä
editorissa olisi hyvä olla muitakin ominaisuuksia, ja niin sanotut
ohjelmoijien tai tehokäyttäjien teksti\-editorit ovat paras valinta.

TÄNNE VIELÄ TEKSTIN SYNTAKTISESTA VÄRITTÄMISESTÄ JA EDITORIN HIENOISTA
OMINAISUUKSISTA.

\subsection{Texdoc}

Latexin kirjoittajan täytyy silloin tällöin lukea ohjekirjoja. Vaikka
Latexin perus\-osat joskus oppisikin ulkoa, ei voi koskaan muistaa
kaikkien hyödyllisten makropakettien kaikkia ominaisuuksia. Myös uusia
makropaketteja ja ihmisten suosituksia tulee vastaan esimerkiksi
verkkokeskusteluissa.

Tex Live \=/jakelun (luku \ref{luku:asentaminen}) mukana tulee mainio
komentotulkissa toimiva komento \koodi{texdoc}, jolla voi hakea ja avata
omaan järjestelmään asennettuja Latex\-/aiheisia ohjeita. Jos vaikka
haluaa tutustua esimerkissä \ref{esim:ensimmainen} mainittuun
Fontspec\-/pakettiin syvällisemmin, tarvitsee vain komentaa
\koodi{texdoc fontspec}, ja paketin PDF\-/muotoinen ohjekirja avautuu.

\subsection{Latexmk}

Hyödyllinen ohjelma on myös Latexmk, joka helpottaa dokumenttien
kääntämistä. Nimittäin varsin usein Latex\-/dokumentit täytyy kääntää
useita kertoja ennen kuin PDF\-/tiedosto on valmis. Se johtuu siitä,
että dokumentit sisältävät usein ristiviitteitä eli viittauksia
dokumentin toisiin osiin. Latex ei saa viitteitä kohdalleen yhdellä
kääntämisellä, vaan ensin se kirjoittaa viittausten kohteet muistiin
väli\-aikais\-tiedostoon ja seuraavilla kääntökerroilla käyttää
väli\-aikais\-tiedostoa apunaan.

Kääntäjä huomauttaa tietokoneen käyttäjää, kun uusintakäännös on
tarpeen, mutta Latexmk\-/ohjelma käynnistää uusintakäännöksen ihan itse,
aina kun se on tarpeellista.

Alla on kääntämiseen esimerkkikomentoja. Ensimmäinen kääntää dokumentin
Xelatexilla ja jälkimmäinen Lualatexilla.

\begin{koodilohkosis}
  latexmk -xelatex  teksti.tex
  latexmk -lualatex teksti.tex
\end{koodilohkosis}

Seuraavista esimerkeistä ensimmäinen komento poistaa kääntämisen aikana
luodut väli\-aikaistiedostot (\koodi{log, aux, out} ym.), ja
jälkimmäinen rivin komento poistaa kaikki luodut tiedostot eli
väli\-aikais\-tiedostojen lisäksi myös valmiin PDF\-/tiedoston.

\begin{koodilohkosis}
  latexmk -c teksti.tex
  latexmk -C teksti.tex
\end{koodilohkosis}

Edellisissä esimerkeissä käsitellään lähdetiedostoa nimeltä
\koodi{teksti.tex}, mutta jos lähdetiedostoa ei anna komennolle
lainkaan, käännetään kaikki nykyisessä hakemistossa olevat
\koodi{tex}\-/päätteiset tiedostot.

Latexmk-ohjelmalle voi tehdä asetustiedoston, johon voi kirjoittaa omaan
käyttöön sopivat asetukset. Asetustiedosto sijoitetaan käyttäjän
kotihakemistoon nimellä \koodi{.latexmkrc}. Alla on esimerkki, mitä se
voisi ehkä sisältää.

\begin{koodilohkosis}
  $pdf_mode = 5; # 5=xelatex, 4=lualatex
  $xelatex = 'xelatex -interaction=nonstopmode %O %S';
  $lualatex = 'lualatex -interaction=nonstopmode %O %S';
  $clean_ext = 'snm nav xdv';
\end{koodilohkosis}

Ensimmäisen rivin asetus määrittää, mitä kääntäjää käytetään oletuksena.
Toisella ja kolmannella rivillä määritellään, millä tavoin Xelatex ja
Lualatex suoritetaan. Tässä esimerkissä oletus\-asetuksiin on lisätty
\koodi{non\-stop\-mode}, joka estää kaiken vuorovaikutteisen toiminnan.
Asetus on tarpeen ainakin silloin, kun kääntäjä käynnistetään toisesta
ohjelmasta kuten teks\-ti\-edi\-to\-ris\-ta eikä vuorovaikutus kääntäjän
kanssa ole mahdollista.

Neljännellä rivillä luetellaan kääntämisen aikana syntyvien
väli\-aikais\-tiedostojen päätteitä. Yleiset väli\-aikais\-tiedostot
(\koodi{log, aux, out} ym.) on \koodi{latexmk}\-/ohjelmalla jo tiedossa,
mutta tällä asetuksella mukaan voi lisätä muitakin.
